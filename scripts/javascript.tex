\chapter{JavaScript}
JavaScript 已經是目前 web GUI 上的公認語言,他跟 CSS 的結合可以實作很多 web
application,像 google map, google doc 等都很不可思議,除此外,他還是
mongoDB 的使用語言,JSON 表示法的鼻祖語法等等。在剛開始時很多資深 programmer
甚至不屑於使用 JavaScript ,隨著他支援非同步 request ,還有在 HTML 5 能控制
影音,現在連 server side script 都有人專門使用 JavaScript node.js。所以可能將
來 perl, python 都不用了,就從front end 到 backend end 都專門用 JavaScript, 
所以是很值得一學的script。不過他這語言很奇妙,跟傳統 c 或 script 都不像,蠻多
奇怪的語法。
\\\\
一樣不囉唆,直接進入核心。在 debian 上可以裝 spidermonkey-bin 這 package ,
裝起來後有個 smjs 可以執行。這是 Mozilla 的內部 javascript engine,所以也可以
去 mozilla 下載自己編譯,在 src 下有個 shell/js 可以執行。或者用 firefox 裡面的
tool 下有個 scratchpad 可以直接寫 code 測試。不過現在應該直接去下載
\href{https://nodejs.org/en}{node.js},就有 js 引擎 v8 + 延伸 server side 實做。
\\\\
Javascript 目前寫作習慣可以參考 AirBnB 的
\href{https://github.com/airbnb/javascript}{Javascript 寫作 convention}
,在系統程式跟 end user 尤其 GUI 程式中有個很大不同在變數命名習慣,系統
C 程式多半要求不要廢話,ansi C coding convention,GUI 程式多半要求寫的
落落長,匈牙利命名。所以 coding convention 是
\begin{itemize}
  \item indent 2 空白,網路上很多 windows 上開發的都用 tab, 但落落長。
  \item 匈牙利命名
  \item 即使是 block comment 也用 // 一行註解,這點有點百思不解,我猜可能
    是傳統 block comment 是給 jsdoc 寫文件用,所以故意這樣區分。
  \item 傳統 ANSI brace while for loop 跟 if 括號
\end{itemize}
python 的變數與 function 命名有點像 C ,也沒要求落落長與匈牙利命名,只有
class 名有規定,這是比較不一樣的。
\section{資料型別}
物件導向的 JavaScript 有以下型別。
\begin{itemize}
  \item Number 比較特別的有Infinity,100 / 0 會等於Infinity; NaN, 表示not a
    number。
  \item Boolean true/false是保留字。
  \item String "this is a string"
  \item Object \verb={ key : value }= 可以亂搞
  \item Array \verb=[ "e1", 2, {key3: value3} ]= 也是可以亂搞。
  \item Function \verb=function fn(arg1, arg2) { doSomething; }=
\end{itemize}
這些資料型別也同時是class的定義,跟c++很像有 new 這 operator 所以
\begin{verbatim}
x = new Number(1);
x = Number(1);
x = 1
y = new String("mystr");
y = String("mystr")
y = "mystr"
z = new Boolean(true)
z = Boolean(true)
z = true
\end{verbatim}
可以用delete x 清掉一個物件。用
\begin{itemize}
  \item typeof 測試哪種type, 還回一個小寫字串。
  \item instanceof 測試哪種instance, 必須大寫第一個字元。還回true/false
\end{itemize}
以上面為例,用 typeof x 時,會還回"number",但其中literal的 x = Number(1) 
跟 x = 1。用x instanceof Number時,會得到false,只有x = new Number(1),
才是真 Number 物件,這才會真的得到一個 true。所以 literal 所產生的東西跟
用 new 產生的物件是不同的。
\\\\
JavaScript 有一些值是很 confused 的 literal 保留字
\begin{itemize}
  \item NaN Not a Number, isNaN("string"), isNaN(Infinity)
  \item null 物件特別給一個 null 的值 。
  \item undefined 有宣告沒給值時。typeof xxx 會得到 "undefined",但卻沒有
    instanceof Undefined。
\end{itemize}
所以又來了
\begin{verbatim}
js> x = NaN
js> typeof x
"number"
js> x instanceof Number
false
js> x = new Number(NaN)
js> typeof x
"object"
js> x instanceof Number
true
js> typeof null
"object"
js> null instanceof Object
false
js> x = new Object(null)
({})
js> x instanceof Object
true
js> typeof xxx
"undefined"
\end{verbatim}
其中變數使用可以不用宣告就直接用,也可以特別用 var 宣告,差別在變數 scope,
但判斷式取值時,如果沒有給值或宣告,則會出現 exception error,主要在 var
宣告後,會自動給一個看不見的 'undefined' 值。
\begin{description}
  \item[x = null] \hfill \\
    可以, x 有 null的值。x 是 global variable。
  \item[var x] \hfill \\
    可以, x 沒有值但typeof x 是"undefined"的值。var 宣告 scope 在 function 內。
  \item[var x = null] \hfill \\
    可以, x 有 null 的值。
\end{description}
JavaScript 沒有所謂的 block scope,只有 global 變數,或者一個 function 裡面
的變數。任何在 function 內的 if, while ... block 宣告過了,就會存在。例如
\begin{verbatim}
function myfunc() {
  print(a);
  if (true) { var a = 1; }
  print(a);
}
\end{verbatim}

  \subsection{immutable與call by reference}
  跟 python 一樣,物件式語言,string 這個型態也是特別奇怪,是 immutable
  primitive,也就是沒有任何語法方法能去改變一個已經存在的string內容。所以當 
  x = "mystr" 或 function傳值時,x 不是像 primitive copy/pass value,因為那
  太沒效率,但也不是 by reference,因為我們沒法拿到 reference 後去改他。
  總之就是無法正確歸類string。
  \\\\
  邏輯的比較,call by value/reference 與 = 的 assignment 或 copy 有如下規則
  \begin{center}
  \begin{tabular}{ c | c | c | c }
    \hline
    Type   & Copied by & Passed by & Compared by \\
    \hline
    Number & value     & value     & value \\
    Boolean& value     & value     & value \\
    String & immutable & immutable & value \\
    Object & reference & reference & reference \\
    Array  & reference & reference & reference \\
  Function & reference & reference & reference \\
    \hline
  \end{tabular}
  \end{center}
  Copied by就是 assignment 時做的動作,所以物件object, array, function 都是
  reference 指來指去而已。

  \subsection{條件判斷與控制語法}
  一樣跟 perl, python 等要注意的有空字串,但由於他多了 null 與 undefined,
  NaN,所以我們也來測試。
  \begin{itemize}
    \item false/true 可,有保留字。
    \item null 可,傳回false
    \item NaN, 可,傳回false
    \item undefined, 可,傳回false。
    \item 0 可,傳回false
    \item \verb=' '=與\verb=" "=, 可,傳回false
    \item '0'  不可,不像 perl 會傳回false
    \item \verb=[]=,不可, 不像 perl傳回false
    \item \verb={}=, 不可, 不像 perl傳回false
    \item new Object(null) new Object(undefined), 不可,
  \end{itemize}
  由上可以看出,他其實是用物件的 reference 來作 true/false的比較,所以'0',
  \verb=[] {} new Object()=這是有 reference 值的。
  \\\\
  另外多了 === 這個判斷,這是說兩邊的東西不經過type 轉換還是絕對相等的。
  \begin{verbatim}
  js> a = 1
  js> b = "1"
  js> a == b
  true
  js> a === b
  false
  \end{verbatim}
  基本語法跟 C 一樣是
  \begin{verbatim}
  if () {
  } else if {
  } else {
  }

  condition ? true_return : false_return ; 
  \end{verbatim}
  也有跟 shell 一樣的 string switch,但卻是 C 語法
  \begin{verbatim}
  switch (var) {
    case 'string1':
    break;
    case 'string1':
    break;
    case 'string1':
    break;
    default:
  }
  \end{verbatim}
  while 是 C 語法
  \begin{verbatim}

while (condition) {
  do something;
}

do {
  do something;
} while (condition)
  \end{verbatim}
  for, foreach 的語法,也是
  \begin{verbatim}
for (i=0; i <10; i++) {
  do something;
}
for (key in vars) {
  do something;
}
for (elemnts of array) {
  do something;
}
  \end{verbatim}
  for 這邊要小心的是像 shell 的 in 其實只有傳回 key,只有 of 才會傳回 array
  element,另外如果是 array, 那物件還有 forEach, map, keys 等可以用。 以上的
  break; continue 跟 C 一樣。
  %各種 for 的使用,
  %https://stackoverflow.com/questions/9329446/loop-for-each-over-an-array-in-javascript
  \\\\
  有像 python Java 的 try catch finally,物件導向的程式,通常都是複雜
  層層資料結構,因此錯誤處理不會是以往的 return return ... return ,而是做一堆
  高階的事情,任何一出錯,就 catch throw error 處理,尤其是這種接近 end user
  的 GUI 程式,都一堆傳進 function 的參數,這可以用一個大 object 傳,然後用
  try catch 去設值檢查,比起一個個檢查 return 好。
  \begin{verbatim}
try {
  try {
    throw new Error("oops");
  } catch (ex) {
    console.error("inner", ex.message);
    throw ex;
  } finally {
    console.log("finally");
  }
} catch (ex) {
  console.error("outer", ex.message);
}

function myFunc(initObj {
  try {
    let m0 = initObj['m0'];
    let m1 = initObj['m1'];
    if (m0 < 0) {
      throw new Error('m0 less than 0');
    }
  } catch (e) {
    console.error(e.message);
    throw e;
  } finally {
    console.debug('finally ok');
  }
  ...
}
  \end{verbatim}

\section{基本 class 與 物件}
有以下一些特徵
\begin{description}
  \item[property] \hfill \\
    member data,用法跟一般變數一樣可以直接設值,book.title = "JavaScript"
  \item[constructor與this] \hfill \\
    定義一個function就可以是一個constructor
    \begin{verbatim}
    function Rectangle(w,h) {
        this.width = w;
        this.height = h;
    }
    \end{verbatim}
    用 new 就可以創一個物件,new Rectangle(1, 1),每個物件內會有一個自動的
    property, 叫constructor會等於建造的 function 名稱。每個 member function
    自動傳進自身物件作為第一 argument,自身物件在裡面用 this 表示。
  \item[inheritance與prototype] \hfill \\
    每個class有個內定的 property, 叫 prototype,所有掛在 prototype 下的變數
    都是所有 object 都看得到的,例如Circle.prototype.pi = 3.1415,
    要作繼承需用到 prototype, 一個object將會繼承所有prototype的properties。
  \item[associate array 與 property互換] \hfill \\
    obj.property 等於 obj["property"],這個assocaite array的好處是裡面的string
    是可以變換的。不像obj.property是固定的。
  \item[class變數與function] \hfill \\
    除了可以用prototype給變數,但像pi這種東西不需要每個object都有一份copy,
    所以用Circle.pi = 3.1415 則是class 變數,算是全域變數shared by所有object,
    同樣Circle.func()也是class function,這種function呼叫只能透過class.method()
    ,所以裡面不能有 this。
  \item[toString() valueOf()] \hfill \\
    每個物件都有toString跟valueOf,這兩個也可以用來做型別轉換,轉成String,
    或者Number。也可以overwrite這兩個讓他輸出有意義的字串或數字。parseInt
    與parseFloat也可以拿來轉string成數字。
\end{description}

\subsection{Array 與 Hash-associate array}
一些關於 array 的操作
\begin{itemize}
  \item a.length 自動就是 a 陣列的長度
  \item a.join(",") 產生一個字串以 , 分隔。
  \item a.reverse() a.sort() 顛倒與排列
  \item a.concat() a.slice() 連結與切割
  \item indexOf() lastIndexOf() 找到element的陣列index
  \item a.push() a.pop(), a.shift() a.unshift()
  \item a.splice(index,1) 砍掉某個 index,後面的 1 表示只砍一個
  \item forEach(Function(e,i)) 把每一個元素 e 一次一個傳給Function處理,i 是
    array 的 index。
  \item map(Function(e,i)) 這跟perl, python的從舊array 新建造array一樣功能的
    map。 Javascript map 跟 forEach 最重要的是他們都是非同步的,一呼叫
    Function 就返回,兩者差別是 map 會建造一個新 array return,forEach 就單
    純執行 Function。
  \item filter(Function(e)) 從舊 array 根據 Function 的 boolean return 
    建造新 array,如果 true, 那 element e 就留下,不然就剔除。
  \item keys() 傳回associate array的所有key陣列。
\end{itemize}
沒有double assocaite array 但有
\begin{verbatim}
js> a = [ [1,2,3], [4,5,6], [7,8,9] ]
[[1, 2, 3], [4, 5, 6], [7, 8, 9]]
js> a[1][1]
5

> a = [ 1 , 2, 3, 4, 5 ];
[ 1, 2, 3, 4, 5 ]
> b = a.map((e) => { return 2*e; } );
[ 2, 4, 6, 8, 10 ]
> c = a.filter(function(e) { return e % 2 });
[ 1, 3, 5 ]
\end{verbatim}

\subsection{Function}
function可以有 nested function定義,算是 class 的 method
\begin{verbatim}
var f = function Sample0(a,b) {
            this.x = a;
            function sample1(x, y) {
              print(x);
            }
            this.y = b;
        }
\end{verbatim}
function 有一些該注意的
\begin{itemize}
  \item arguments是一個特殊 property 物件,裡面有陣列\verb=arguments[]=
    ,是傳進來的參數陣列。
  \item arguments.callee 等於是被呼叫的function本身。
  \item arguments.callee.caller 指向上一層呼叫 function,
    arguments.callee.caller.name 是上層 function 名字。
    這通常用來作debug的工具用。
  \item f.length 是這個function物件期望的 arguments 數目。
  \item \verb=f.apply(o, [arg1, arg2, ...])= 這是用來使別的物件也能暫時擁有這個
    function。這樣一來 o 物件等於
    \begin{verbatim}
    o.newf = f;
    o.newf(arg1, arg2);
    delete o.newf;
    \end{verbatim}
  \item f.call(o, arg1, arg2, ... ) 等於是apply(),差別就只有不是傳陣列而已。
  \item f.bind(o, arg1, arg2, ... ) f 裡面的this,將會被o來取代。
\end{itemize}

\subsection{scope chain與closure物件}
closure function物件是指一個 nested function 使用了不在他local定義下的變數,
但卻延伸了這變數的生命週期。等於是包住一個外面變數,延伸了這個外在變數的生命
週期。例如
\begin{verbatim}
var foo = (function() {
  var foo='boo';
  return function() {
    print(foo);
  }
  foo = 'foo';
})();
foo();
\end{verbatim}
foo等於是那個print(foo);而已,可是foo的定義是在外面一層的function中,所以正
常講,應該真正執行時,是undefined的,但卻會跑出foo這字串來,所以他關住的是
這個foo變數,在上一層最後的值。這意義是 function 是跑在被定義地方的scope,
而不是在被執行地方的scope。這有點像 c 的static local變數。所以在recursive
或者一些namespace上可以用到。
\\\\
這變數的生命週期原則很重要的用在找尋變數的順序,是為尋找的scope chain,
在 browser 的 event callback 中很重要。最外面的當然就是global object。
如果沒有function,則最外面的code就是找global object。如果有function,
則回一層層找上去,以DOM來說就是會button, form, document, window, global
這樣找回去。

\section{字串與regex}
字串操作
\begin{itemize}
  \item s.length 字串長度
  \item s1 + s2 字串連結用 +
  \item s.charAt('x') 字元在字串的index
  \item s.match(regex)
  \item s.search(regex)
  \item s.replace(regex, "replaceStr") 替換成 replaceStr,這跟其他 regex
    工具很像,用( ) 包住想要搜尋的字串,後面代換字串一樣有
    \begin{itemize}
      \item \$1, \$2 ... 搜尋到的第一第二... 括號字串。
      \item \$\& 搜尋到字串
      \item \verb=$`= 表示搜尋到字串之前的所有字元。
      \item \verb=$'= 表示搜尋到字串之後的所有字元。
    \end{itemize}
  \item s.split(",") 還回一個陣列
  \item s.substr('mystr') 尋找子字串
  \item string template 使用 backquote,一般字串代換在物件導向 script 中,
    不像 shell 那樣好用, string template 跟 shell, Makefile 一樣可以用
    \$\{var\} 代換成 var 的字串,但字串不是用double/single quote,而是要用
    backquote,這是 ES6 以後才有的功能。
\end{itemize}
regex 與 pattern,pattern用法跟regular express介紹的一樣,用//夾住。一樣
有modifier, i 不分大小寫,g global,m表示multiple line。
\begin{verbatim}
var p = /Javascript/ig;
var o = new RegExp("\\d{4}, "g");
var s = 'Javascript';

p.test("javascript"); // true
o.test("1234");
s.search(/script/i);
s.replace(/(search1)alibuda(search2)/, "$2$1")

o.compile("newReg", "i") // don't discard regex object
\end{verbatim}
使用test()時,會還回true/false,使用exec()時會還回第一個match string。
以上是一般語言的觀念,在後來變革中,才有比較像本機 script 的用法,像模組
library 的使用留待新的說法一併解說。

\section{ECMAScript}
ECMAScript 是一個由 ECMA international 組織統整 Javascript, ActionScript,
JScript 的標準語言。在ES5 後,讓Javascript 變得更 powerful。ECMAScript
這個名字根據 Wikipedia 講法是有很多爭議。從這邊開始,ECMAScript 比較像
可以是本機跑的類似 perl, python ... 等 script 語言,這邊下載
\href{https://nodejs.org}{node.js} 來測試比較接近本機使用 script。
\\\\
不過 Javascript 的生態是由 GUI 的人主導,當初也是為了網頁而來,因此術語使用上
比較偏向 GUI 的人,例如 Events, main thread loop 等等,這些在系統程式工程
師來看就是把這些轉成底層術語而已,不要被這些名詞拖著走,主要是所有的 GUI 上
的 object 都是在執行畫圖的背景 process/thread,他們隨時在 main thread loop
監控下等待呼叫,這是一般低階 library 不會幫忙處理的。
\\\\
基本上 ES5, ES6 是比較重大的變革,也跟 perl python 一樣,最後都要像 C 一樣,
對於變數 scope, memory, object, multiprocess 等都要強化,才能寫大型軟體,
尤其是2015年後的 ES6 算是才比較定型了。 以下介紹我覺得比較重要的變革語法,在
\href{https://www.w3schools.com/js/js_versions.asp}{w3schools} 有快速的
改版語法蒐集
  \subsection{ES5 use strict 與 ES6 新 literal}
  就跟 perl 一樣,最後一定強制定義檢查所有變數的 scope 定義,script 原本宣傳
  的好處最後都是自己打臉自己。在js 檔前面宣告 \verb='=use strict\verb='=;
  後跟 perl 一樣 ,所有的變數都需要宣告,以決定他們屬於 global 或是 local
  變數 scope。只是這個是一個 string ,並不是語言的 statement,在 browser
  中使用,就只是個沒意義的 string 物件。舊的引擎看到不會出錯。
  var 宣告還是特別的,但 ES6 多了 let, const,並且有 block, function, global
  scope,只要在 function 內宣告的,就只能在 function 內用,只是 let, const
  特別可用在 while, for loop 的 block 宣告。
  \begin{itemize}
    \item var 一旦宣告,就都可以用,所以沒有光 block scope 效果。
    \item let 可用在 block scope ,一旦離開 block, 就失效。
    \item const constant 宣告,跟 let 有一樣效果。
  \end{itemize}
  例子
  \begin{verbatim}
'use stricts';
var astring = "string1";
const myconst = 100;
{
  let myvar = ['a1', 'a2'];
}

console.log(myvar);
  \end{verbatim}
  新的 literal 與資料型別
  \begin{itemize}
    \item Set 像 python 的 Set, 裡面的每個 element 是唯一的。
    \item Map 另一種 associate array (hash) 的 literal \\
      \begin{verbatim}
const fruits = new Map([
  ["apples", 500],
  ["bananas", 300],
  ["oranges", 200]
]);

fruits.get("apples");
      \end{verbatim}
  \end{itemize}

  \subsection{ES5 物件 accessors}
  一般物件語言的 accessors,get/set 的使用
  \begin{verbatim}
var person = {
  firstName: "John",
  lastName : "Doe",
  get fullName() {
    return this.firstName + " " + this.lastName;
  },
  get height() {
    return this.h;
  },
  set height(value) {
    this.h = value;
  }
};

person.height = 100;
console.log(person.height);
console.log(person.fullName);
  \end{verbatim}
  自動會有兩個 attributes 可以用,像存取 attributes 一樣
  ,要注意的是 function 名字,不能跟 this 的 attribute 名一樣。

  \subsection{ES6 class 與物件}
  物件導向的 class 語法終於在 ES6 中塵埃落定,這是非常重要後續 JS 軟體寫法
  ,在 strict mode 下執行
  \begin{verbatim}
class Rectangle {
  #height;
  #width;

  constructor(height, width) {
    this.#height = height;
    this.#width = width;
  }

  draw(...) {
    ...
  }

  memberf1() {
    ...
  }
}

class FilledRectangle extends Rectangle {
  #color;

  constructor(height, width, color) {
    super(height, width);
    this.#name = "Filled rectangle";
    this.#color = color;
  }

  draw(...) {
    ...
  }

  memberf1() {
    ...
  }

}
  \end{verbatim}
  有一個 construtor() 可以定義,繼承也是用 extends,以往用 prototype 寫法
  慢慢會被淘汰掉。member method 的寫法也不像 perl, python 要特別把 self
  寫出,他本身就自動有 this 這個自身物件可用,super 只能用在 constructor
  還有一定要在 this 之前使用。
  \\\\
  private member 現在有一種特別寫法是無須加上 let const 的宣告,在變數前面
  用 \# 符號,用上 \# 符號的變數,在 class 外使用會一開始解譯就 error,這
  會杜絕 API 使用者直接取用 attribute, 但繼承的 sub class 也是無法存取,
  所以要被大家繼承公用的變數不行用這 private member。
  \begin{verbatim}
#var = 'private_value1';
  \end{verbatim}
  \subsubsection{object oriented programming}
    就像在 perl, python 裡面說到,要 OO 一定要做到封裝,繼承,多型,一般
    shell 很難做到良好的封裝,就是沒有好的 struct 語法,c, perl 不好做繼承,
    python 雖有 overloading 但沒有好的 dynamic binding,一般解譯器形式的
    script 可能也不太需要 dynamic binding。Javascript 同樣沒有好的 overloading
    同一 class 內也不能有同名 method。一般要做 overloading,也跟 python
    一樣最後面那個參數必須是變化的 object,只是他語法上不像 python 那麼像 C
    ,只要單純的傳一個 object 就可以了
    \begin{verbatim}
    fun(a,b, opts) {
      ...
      if (opts['arg1']) {
        do_something_with_arg1()
      }
    }

    fun(1,2, {'arg1':value1, 'arg2': value2})
    fun(5,6, {'arg3': value3})
    \end{verbatim}
    物件導向程式中封裝繼承設計,有名的 composition 與 inherit 問題
    \\\\
    inherit (extend)
    \begin{verbatim}
class Fruit {
  #withSeed;
  constructor(withSeed) {
    this.#withSeed = withSeed;
  }
  peel() {
    console.log('No need');
  }
}  

class Apple extends Fruit {
  constructor(withSeed) {
    super(True);
  }
  peel() {
    ...
  }
}

class Banana extends Fruit {
  constructor(withSeed) {
    super(False);
  }
  peel() {
    ...
  }
}
    \end{verbatim}
    composition (contain)
    \begin{verbatim}
class Apple {
  #fruit = new Fruit();
  constructor(withSeed) {
    this.#fruit = new Fruit(True);
  }
  peel() {
    ...
  }
}

class Banana {
  #fruit = new Fruit();
  constructor() {
    this.#fruit = new Fruit(False);
  }
  peel() {
    ...
  }
}
    \end{verbatim}
    支持 inherit 的很多喜歡用 is a 跟 has a 來說明,但這也只是英語使用者的
    文字遊戲跟個人定義,並不是 100\% general,也就是 is a 的情況有的並不是
    永遠不變的強烈,例如 person, account, employee, engineer, manager 的關係,
    一個 person 有可能從 is an engineer 的角色跳到 is a manager 的角色,那這
    種由 person 衍生出的關係就不是那強烈, 另外使用 inherit 會自動有
    overloading, dynamic binding 的好處,但overloading 與 dynamic binding
    也不是只有好處沒有壞處,這會跟使用情境有關。還有 script 不太會用到
    dynamic binding。
    \\\\
    支持 compisition 的, 舉特例的時候也常跟使用情境有關,例如常有將來會怎
    樣怎樣的說法,有很多是在設計 design 中不見得花很大心力就避免的掉。
    \\\\
    支持 composition 的通常是說如果 subclass 與 superclass 強烈 couple 在
    一起,而且'將來'不會改動 superclass method 的才用 inherit, 如果不是tight
    together 那麼強,或者 superclass 會衍生出多種 type, is a, 的情況,那還
    是要用 composition, 以上面水果來說,其實有沒有子還有剝皮的情境太多了,
    將來的定義不確定性太高了,除了一些數學以外,在萬物範圍太多將來不確定,
    因此其實大多數情況都是屬於 composition 的情境,也認為將來在 code reuse
    上, 不會寫重複太多相同的 code 。
    \\\\
    其實兩種說法都可以找到漏洞,因此開始不必要太執著於哪一種
    \\\\
    關於 end user 使用的程式很煩是,資訊太多造成傳進function 的 parameters
    太多,這種 user 傳進的檢查,非常煩人,又因為不是強型態語言,所以太多
    function 如果太多參數,實在很難看,通常超過 4 個以上,那可以傳進物件 
    \{\} 或者 \verb=[]= ,然後用try catch 來捕捉可能的所有 error
    \begin{verbatim}
    try {
      ...
    } catch (e) {
      console.log(e.message);
    }
    \end{verbatim}
  \subsection{ES6 Iterator 物件與 generator function}
  這跟 python 的 iterable 物件類似,就是像 array, set, map 等等列舉物件有
  些新的好用操作
  \\\\
  ... 是把列舉的 elements 全部展開
  \begin{verbatim}
q1 = [ "Jan", "Feb", "Mar" ];
q2 = [ "Apr", "May", "Jun" ];
q3 = [ "Jul", "Aug", "Sep" ];
q4 = [ "Oct", "Nov", "Dec" ];
year = [ ...q1, ...q2, ...q3, ...q4];
  \end{verbatim}
  就會展開所有元素成新 array
  \\\\
  用 function* 宣告 generator function, generator function 會自動有兩個
  methods, next() 跟 return()。 generator 函數中使用 yield,表示每次呼叫
  next(),就會執行到一個 yield 的地方,例如
  \begin{verbatim}
function* myiter() {
  yield 1;
  yield 2;
  return 3;
}
i = myiter();
console.log(i.next().value);
console.log(i.next().value);
console.log(i.next().value);
console.log(i.next().value);
/* last one will be undefined */

function* foo(index) {
  while (index < 2) {
    yield index;
    index++;
  }
}

const iterator = foo(0);
console.log(iterator.next().value);
console.log(iterator.next().value)
  \end{verbatim}
  跟一般 function 不一樣的是,他回傳一個所謂 generator object,當 generator
  object 執行 member function, next() 時,他回傳兩個值
  \begin{itemize}
    \item value : 表示 next 的值。
    \item done : 是個 boolean 表示 next 是否結束。
  \end{itemize}
  要注意的是 Mozilla 支援跟 python 一樣可以是一個 class 裡面有提供 next()
  method ,而且回傳一個 iterator 物件。
  \begin{itemize}
    \item 必須是 extends Iterator 的物件
    \item 必須有 next()
    \item 必須回傳一個 iterator protocol 物件
  \end{itemize}
  用 class 宣告
  \begin{verbatim}
class myIter extends Iterator {
  next() {
    return {value: 1, done: true};
  }
}

iter = new myIter();
console.log(iter.next().value);
  \end{verbatim}
  但這個在 v8 引擎是不支援的,node.js 跑會出 error。

  \subsection{ES6 Rest arguments}
  一般 script 的好處是 function 的 arguments 數目是不定的,C 也有,但 script
  使用比較方便,之前說到 function 有一個特殊 arguments 變數,自動是一個
  arguments array 藏著所有 arguments,ES6 有一個 ... 的rest arguments
  \begin{verbatim}
function myFun(a, b, ...manyMoreArgs) {
  console.log("a", a);
  console.log("b", b);
  console.log("manyMoreArgs", manyMoreArgs);
}

myFun(1,2,3,4,5,6,7);
  \end{verbatim}
  \subsection{ES6 的模組}
  perl 用 PERL5LIB, python 用 PYTHONPATH, Javascript
  不限制在本機上,所以並沒有這種環境變數,反而是任意 url ,然後通常就是
  指到真的 js 檔。 在引用 library 上,,跟 python, Java 很像的用 import,
  以前使用 require() 跟 module.exports 來使用別的模組或 export 變數
  出去,但 ES6 引進類似 python Java 的
  \begin{itemize}
    \item import 一樣有 package 目錄, module 檔案分別,但使用上不像 python
      區分 package, module 的 import 不太一樣,一律用 module。
    \item export 一樣 export symbol。
  \end{itemize}
  傳統例子如下
  \begin{verbatim}
const req = require('./request');
const res = require('./response');

/**
 * Expose `createApplication()`.
 */

exports = module.exports = createApplication
exports.json = bodyParser.json
exports.query = require('./middleware/query');
exports.raw = bodyParser.raw
exports.static = require('server-static');
exports.text = bodyParser.text
  \end{verbatim}
  ES6 例子
  \begin{verbatim}
import {export1, export2} from "/modules/myexport.js";
import http from 'http';
 
export let name1, name2/*, … */; // also var
export const name1 = 1, name2 = 2/*, … */; // also var, let
export function functionName() { /* … */ }
export function* generatorFunctionName() { /* … */ }
export class ClassName { /* … */ }

export { name1, /* …, */ nameN };
export { variable1 as name1, variable2 as name2, /* …, */ nameN };
export { variable1 as "string name" };

export default expression;

export * from "module-name";
export * as name1 from "module-name";
export { name1, /* …, */ nameN } from "module-name";
export { import1 as name1, import2 as name2, /* …, */ nameN } from "module-name";
export { default, /* …, */ } from "module-name";
export { default as name1 } from "module-name";
  \end{verbatim}
  差別在於 require() 相當於 source,整個檔案引進,沒有 private/public 之分
  ,也不能只引進部份變數,另外 import 是 multithread 非同步引進。如果是多個
  js ,彼此不會 block 住。 不過在 Node.js 除了語法處理還有很多特別處理,
  也有辦法不 block 就是。
  \\\\
  ES6 import/export 語法中
  \begin{description}
    \item[Named import]: import \{ export1, export2 \} from "module-name";
      用物件 \{ 跟 \} 括號引進的 symbol, 表示 mjs 裡面一定要有 export1, 
      export2 這兩個一模一樣的名字 export,這應該是最常用方法,require 也
      是一樣用法。
    \item[Default import]: import defaultExport from "module-name";
      表示 mjs 裡面要有 export default defaultExport 這樣的名字存在。
      這也可以用來換名字 import defaultExport as alias from "module-name"。
    \item[Namespace import]: import * as name from "module-name";
      這個等於把 module-name 換成 name,然後所有 mjs 檔裡面 export 的變數名取
      用必須透過 name.xxxx 還有 name.default.xxxx 方法。
    \item[Side effect import]: import "module-name";
      這個等於以前的 require, shell 的 source,整個 include JS 檔跑一遍。
  \end{description}
  在 Linux/Unix 中,import 同目錄下的檔名要用 ./xxx.js ,在 Windows 中,可以
  只給 xxx.js 即可。

  \subsection{ES6 Arrow function}
  使用 => 表示一個 function的語法, 以前
  \begin{verbatim}
hello = function() {
  return "Hello World!";
}
  \end{verbatim}
  現在可以用
  \begin{verbatim}
hello = () => {
  return "Hello World!";
}

() => x 等於 () => { return x; }
  \end{verbatim}

  \subsection{ES6 Promise 物件與非同步}
  非同步平行處理是所有高級程式課題,從 OS, 資料庫,到 GUI 都是。只是 GUI 的
  比較接近終端使用者,他使用的術語有他自己那套 events ...等等。 multiprocess
  multithread 的用語都是寫 GUI 的那套思維。會把 I/O blocking 跟 fork/exec
  術語包覆起來。
  \\\\
  Javascript 一開始就是為了 browser 上的 GUI,因此設計上就是一個眼睛看不到的
  大單一 main thread 控制,裡面的函數執行都是平行非同步,不管做什麼都是先註冊
  一個客製觸發,然後馬上回到這個大 main thread, 所以呼叫回傳都不是得到想要的
  值,而是一個等待處理的物件,例如檔案系統的 stat(),照理應該要得到 stat 的
  structure,但 JS 不是這樣的,而是
  \begin{verbatim}
pobj = fs.stat()
....
pobj.then( (stat) => { console.log(stat.size) })
  \end{verbatim}
  所有的客製函數觸發都靠
  callback 註冊,然後發生 event 時呼叫,當所有註冊沒有結束前,JS main thread
  不會死掉, 這是跟傳統程式寫作很不一樣的基本觀念。

  \subsubsection{setTimout}
  傳統 JS 用 setTimeout(),相當於進階的 system call alarm(),設定一個時間後,
  做什麼事情。最常用的就是 setTimeout(functionRef, delay, param1, param2)
  \begin{verbatim}
setTimeout(() => {
  console.log("this is the first message");
}, 5000);
setTimeout(() => {
  console.log("this is the second message");
}, 2000);
tobj = setTimeout(() => {
  console.log("this is the third message");
}, 3000);
console.log("HeHe... I am parent process");
clearTimeout(tobj);
  \end{verbatim}
  setTimeout 不會 blocked,只是會在一個時間後執行某特定程式。

  \subsubsection{物件導向與Promise}
  在傳統 C 語言中,pass 一個 callback 進一個 function 會很難看的語意,
  \begin{verbatim}
func1(arg1, arg2, callback);
  \end{verbatim}
  最常見的就是加法,string + string 跟  5 + 5 ,兩個不同的資料型別,
  在外面想用同樣 function 呼叫,但內部需要用不同方法來做加法
  \begin{verbatim}
add(arg1, arg2, callback);
  \end{verbatim}
  但物件導向是
  \begin{verbatim}
result = arg1 + arg2
  \end{verbatim}
  適用於任何資料都可以,傳統 C 語法, 在物件導向程式中很難看,
  (說實話,物件導向很多都在做這些文字遊戲,就是
  語法的好看不好看,結果惹出更多新觀念出來加重程式師的痛苦)
  \\\\
  使用 Promise 非同步物件, promise 物件跟 setTimout 物件一樣會 fork 一個
  process 後,不等待像 waitpid() 中使用 WNOHANG,然後可以用 .then() method
  來處理成功或者失敗,在下一次的 event loop 之前得到 promise 物件的回傳值。
  他有幾個很常用的 methods, resolve, reject 跟 then
  \begin{itemize}
    \item Promise.resolve(value); 回傳一個 Promise 物件,裡面有個隱藏的變數,
      將來會傳給處理成功的 function 做參數。
    \item Promise.reject(reason); 回傳一個 Promise 物件,裡面有個隱藏的變數,
      將來會傳給處理失敗的 function 做參數。
    \item Promise.then(sucessCallBack, failCallBack);
    \item Promise.all([ promise1, promise2, ...]) 當所有 promise 陣列
      fullfilled 後,回傳一個 Promise 物件。
    \item Promise.any([ promise1, promise2, ...]) 任一 promise 陣列
      fullfilled 後,回傳一個 Promise 物件。
  \end{itemize}
  最簡單的
  \begin{verbatim}
const promise1 = Promise.resolve(123);

promise1.then((value) => {
  console.log(value);
});
  \end{verbatim}
  看個例子來了解一般非同步用法
  \begin{verbatim}
const promise1 = new Promise((res, rej) => {
  setTimeout(() => {
    res('foo');
  }, 3000);
});

promise1.then((value) => {
  console.log(value);
});

console.log(promise1);
console.log("I am parent");
setTimeout(() => {console.log(promise1);}, 4000);
  \end{verbatim}
  constructor 的參數是一個 executer function,constrcutor 直接執行它。 這個
  function 自動裡面有兩個參數, res 表示 resolve, rej 表示 reject method,
  設定了 resolve reject methods 後,等待 3 秒後,
  resolve 送出 foo 字串,這時,then 的 successCallBack 收到 resolve 送來的
  value 是 foo ,就把他印出來。要特別注意的是,這些都是非同步的,以都不會
  blocked 原本程式,所有的 then, all, any 都只是註冊一個 event handler,
  可以看到.then, console.log(promise1), I am parent 都沒有被 blocked,繼續往
  下走直到等待 4 秒的 setTimout 出現, 然後最後過了 4 秒,印出的 promise 物件
  字串不同。
  \\\\
  第一個console.log(promise1) 會印出目前promise 物件的 status,他有
  \begin{itemize}
    \item pending
    \item fullfilled
    \item rejected
  \end{itemize}
  fullfilled rejected 的條件是呼叫那個 res() 跟 rej(),當呼叫 res(obj)時,
  就是 fullfilled 了,當呼叫 rej(errobj) 時,就是 rejected 了。也就是先設定
  ,然後讓他自己跑,後面用一個 then 的程式碼,then 的第一 function 參數是
  fullfilled 的,第二 function 參數是 rejected 來處理成功失敗。
  \\\\
  EMACScript 現在還是繼續不斷出新的語法,但 ES6 是最重大的變化,使用 ES6
  是最基本的保障。
  \subsubsection{async function 與 await}
  如果 function 用 async function 來宣告,那他每次都會回傳一個"新的" Promise
  物件,自動 resolve 的 value 是 async return 的那個值,reject 的 error
  是 async function 裡面沒有被處理的 exception,這個 function 物件是為一個
  AsyncFunction 物件,也因此多工時可以使用 promise 或者 aysync function 兩種
  是一樣的意思。
  \begin{verbatim}
async function foo() {
  return 1;
}
  \end{verbatim}
  類似於
  \begin{verbatim}
function foo() {
  return Promise.resolve(1);
}
  \end{verbatim}
  當我說類似,表示他不是一樣的,因為他回傳不同新的 promise 物件
  \begin{verbatim}
function handle_s(value) {
  console.log(value)
}
async function foo() {
  return 1;
}
pobj = foo();
pobj.then(handle_s);
  \end{verbatim}
  會印出 1 來。
  \\\\
  使用例子來看強迫等待的 await 使用
  \begin{verbatim}
function resolveAfter2Seconds() {
  return new Promise((resolve) => {
    setTimeout(() => {
      resolve('resolved');
    }, 2000);
  });
}

async function asyncCall() {
  console.log('calling');
  try { 
    const result = await resolveAfter2Seconds();
  } catch (err) {
    console.log(err)
    process.exit(1)
  }
  console.log(result);
}

asyncCall();
  \end{verbatim}
  當呼叫 asyncCall 時,await 一個 Promise 物件,相當於 block 在這邊等待
  fullfilled 或 rejected,等於使用 wait() system call,本來呼叫 Promise 或者
  async function 時回傳一個 Promise 物件,但使用 await 時,會得到那個
  fullfilled 的值。then 不會 blocked 但 await 會 blocked/synchronized,所以
  用上 await 時,會像傳統 python script,error control 要用 try catch finally
  而不是用 promise 的 resolve/reject 或 callback (err,stdout,stderr) 方式。
  \\\\
  JS 沒有 sleep 函數,不會 blocked,要 sleep 就要自己用 await setTimeout
  做一個。
  \begin{verbatim}
await new Promise((r) => {setTimeout(r, 1000)});
  \end{verbatim}
  await 要在 async 裡面使用, 而且一定要相匹配的 asyn function 與 await,
  也就是 await 所在最內那個 function 一定要是個 async function,不然就會出錯。
  有些操作是一定要做 synchronous 的,例如檔案操作等等,所以一定要用 await。
  \\\\
  這能與 import() 一起來做 dynamic import,這跟 python 的 \_\_import\_\_
  一樣意思,能在程式碼中根據條件 import 別的模組。
  \begin{verbatim}
(async () => {
  if (somethingIsTrue) {
    await import("/modules/my-module.js");
  }
})();
  \end{verbatim}
  另外在 JS 語言中有內定的 Atomics 物件的 methods 能做 atomic 操作,算是
  在非同步中的重要工具,同時也能拿來做 blocked 的功能,例如 sleep。
  \begin{verbatim}
Atomics.wait(new Int32Array(new SharedArrayBuffer(4)), 0, 0, 1000);
  \end{verbatim}
  在一般一行一行同步邏輯中, 很多乍看之下覺得是多此一舉, 但基本上 GUI
  或者人機界面處理中都是非同步的,因此常用的非同步處理用簡單的界面函數跟
  術語處理掉,讓程式師簡單的使用非同步處理。

\section{Node.js 與 server side script}
跨平台說穿了,就是有沒有大神喜歡這個語言,為他在各個 OS 中用 C 再重寫一遍
編譯出一個新的執行檔而已。而所有功能其實都是靠 library 解決,就是有沒有大
家寫了一堆的 library 讓這語言連結而有一堆功能。所以 python, perl, ...
都能跨平台,Javascript 一樣只要有人寫出能解釋這語法的執行引擎,就能在各OS
跑 javascript 語法寫的邏輯程式。而隨著所有 OS 都要支援美國政府要求的 POSIX
標準,這樣的程式編譯早就問題不大。
\\\\
在以往Javascript 都只能在 broswer 上處理人機界面輸入錯誤處理,圖形處理等,
不像 perl, python 很早就有跟 System call 溝通的能力,例如 perl, python
都有提供 fork, exec, open, read/write 等等處理本機事物的library call。因
此 Javascript 就有大神幫他寫了 Node.js 這個 library ,提供了相對應的系統
能力,也因此用 javascript 可以寫跨平台的 server/ desktop GUI 程式了。最
早也有 \href{https://www.electronjs.org/}{Electron JS} 這 library ,這是
很多程式的後端 library,像 github, vsCode,我記得好像 line 還是 skype 
曾經嘗試用這寫 Linux 的程式過。
\\\\
Node.js 使用 v8 這個 javascript 引擎,並且多加上一些自己定義的環境處理,
javascript 引擎蠻多家的,不像 python , perl ... 只此一家別無分店,像 C 有
MS visual C, gnc c, clang ...等。node.js 利用已有 v8 引擎,再加上自己的 c/c++
程式然後 binding JS,提供新的 system call JS API,就像 perl, python 一樣。
當然別人一樣也能做,例如 Electron JS 的 fork
\begin{verbatim}
utilityProcess.fork()
\end{verbatim}
Node.js 的 fork 是
\begin{verbatim}
child_process.fork(modulePath[, args][, options])
\end{verbatim}
不過他們這 fork 跟 C 的已經差別很大了,只是故意名字類似,事實上跟原本C 呼叫
的含意都不一樣了。
GUI 也是一樣的道理而已,就是把 JS 的呼叫轉成 X 或 Windows 的呼叫。
現在有很多這種 JS 引擎,除了 node.js, electronjs, 還有像 bun, deno等等,
不過還是玩 node.js 比較齊全,
首先要 \href{https://nodejs.org/en/download}{下載 node.js}

  \subsection{npm 與 library, module}
  只要程式語言就會有 library 堆疊問題,就一定要有套件管理,所以 Node.js
  出了 npm 管理,npm 不只 package 管理,也是軟體 project 管理, release
  等功能於一身的管理程式。
  \begin{itemize}
    \item 用npm 啟動 project 管理與 package 管理
      \begin{itemize}
        \item mkdir proj
        \item cd proj
        \item npm init
      \end{itemize}
      就會建立一個 package.json 檔案,這個檔案是 node.js 最重要的檔,相當多
      的設定,文件有四個 Modules 開頭文件。由此開始,在 init 時,有一個 entry
      point ,他像是C 裡面的 header .h 檔案,任何project module 要 export 
      的變數,api 都在這裡,在 web application 的 framework 中,就是 app.js 
      。 當別人引用這個 library 時,會用 require() ,這個填入的就是 entry
      point 名字,然後有個相對應的 JS 檔,內定是 index.js,所以自己要新建
      這個 JS 檔。
      \\\\
      npm test command,主要是 package.json 裡面的 scripts 定義名字, 這個定
      義用 npm test, start, run, run-script 跑自己寫的 js 檔都是一樣的。
      必須自己往 package.json 裡面加上自己想要的命令,通常 unit-test 是要自己
      這邊加進去,內定就是 shell echo 字串離開而已。
    \item npm install <package-name>@<version>
    \item npm test
    \item npm run, npm run xxx
  \end{itemize}
  npm install 會產生一個 node\_modules 的目錄, node.js 在模組引用上
  比 JS 語法引擎會多做處理,例如之前說的 require, import 的差別,還有如果在
  node.js 裡面使用 require 會自動搜尋相對
  node\_modules 下的程式庫,這是一般 JS 引擎不會做的,也就是像 PYTHONPATH
  會自動有 node\_modules,所以這邊真的要小心 JS 跟 node.js 環境是不一樣
  的。 在 node.js 中 package.json 跟 node\_modules 就是最根本的兩個起始。
  \\\\
  JS 原本就沒有像 python 那樣分 package 目錄, module 檔案,node.js 這邊多了
  package 與 node\_modules package 目錄, 與實做 JS 模組檔,把 package 的資訊
  放進 package.json 檔案去,這樣的分層管理比較沒有 python 的模組命名問題。
  其他重要安裝命令,對於套件來說,無非就是搜尋,裝了什麼,dependency
  \begin{itemize}
    \item npm uninstall
    \item npm update
    \item npm search
    \item npm ls
    \item npm query
    \item npm publish
  \end{itemize}
  搜尋其實應該是去\href{https://www.npmjs.com/}{npmjs}網站搜尋,
  node.js 的內定模組是屬於內部 source 的一部分,定義在 lib/node\_modules 下的
  npm/node\_modules/builtins/index.js, 所以一個簡單的 8080 http server
  \begin{verbatim}
var http = require('http');

http.createServer(function (req, res) {
  res.writeHead(200, {'Content-Type': 'text/html'});
  res.end('Hello World!');
}).listen(8080);
  \end{verbatim}
  裡面的 http 模組是看不到 js source code,他是在
  \href{https://github.com/nodejs/node/tree/main/src}{c++} 裡面 implemented。
  所有模組的文件在\href{https://nodejs.org/docs/latest/api/}{nodejs 網站docs}
  \\\\
  而一個 module 裡面含有另外 module,就很簡單在這目錄下重複剛剛的步驟,或者
  直接是一個 JS 檔案放在 node\_modules 下。
  \\\\
  從 8.5.0 開始, node.js 支援 ES6 module 與傳統 module
  \begin{itemize}
    \item 使用 ES6 方法的 import/exports/class module 的方法 
      \begin{itemize}
        \item 副檔名用 mjs
        \item 在 package.json 加個 \verb="type" : "module"=
        \item 使用 node 命令時,使用 \verb=--=input-type module
      \end{itemize}
    \item 用傳統的為 common JS
      \begin{itemize}
        \item 附檔名用 cjs。在 node.js 中
        \item 在 package.json 加個 \verb="type" : "commonjs"=
        \item 使用 node 命令時,使用 \verb=--=input-type commonjs
      \end{itemize}
  \end{itemize}
  import, require 不能出現在同一檔案中,在 mjs, cjs 只能分別用 import, require
  的方法。
  \\\\
  最後除了 npm, 也有一個叫 yarn 的 package management,是大同小異,有人喜歡
  用這個。

  \subsection{檔案系統 File I/O}
  使用
  \begin{itemize}
    \item fs 模組
    \item path 模組
  \end{itemize}
  但這邊要特別注意的是很多檔案操作是必須 synchronized 的,必須是檔案狀態已經
  是某種狀態才能往下操作,例如當要 read/write 時,就是要等 open 處理完,才能
  往下走,unlink 也要有檔案才能砍掉檔案,
  因此 fs 模組裡面會特別提供一些 sync 的 API,Javascript 在使用上要小心非同步
  API,很多都是一直往下走得非同步,忘記這點,會讓程式處理檔案邏輯變得跟傳統使
  用 C python 不一樣。
  \\\\
  檔案系統 fs 模組有 promise 跟傳統 callback 兩種方法做錯誤處理, node.js
  不再有很麻煩的 open()/read()/write(),而是直接提供非同步的檔案高階操作,
  \\\\
  使用 promise 方法,API 都回傳一個 Promise 物件
  \begin{verbatim}
const { unlink } = require('node:fs/promises');

(async function(path) {
  try {
    await unlink(path);
    console.log(`successfully deleted ${path}`);
  } catch (error) {
    console.error('there was an error:', error.message);
  }
})('/tmp/hello');

const fsPromises = require('fs/promises');

// 可直接用 fsPromise.writeFile() 可是這會開開關關檔案
fsPromise.writeFile('/tmp/file1', 'content data', {'mode': 0o644, 'flag':'a+'})

// 用 open 傳回 fileHandle object
try {
  fsPromises.open('file1','w+',0o644).then((fileHandle) => {
    await fileHandle.writeFile('file1', 'my content')
  })
} finally {
  fileHandle.close() 
}
  \end{verbatim}
  使用 callback 方法
  \begin{verbatim}
const { unlink } = require('node:fs');

unlink('/tmp/hello', (err) => {
  if (err) throw err;
  console.log('successfully deleted /tmp/hello');
});

const fs = require('fs');

fs.writeFile('file1', 'my content', (err) => {
  if (err) {
    console.log(err)
  } else {
    console.log('write file successful')
  }
)

  \end{verbatim}
  簡單 path
  \begin{verbatim}
const path = require('node:path'); 
console.log(path.basename('/home/myhome/myfile'));
console.log(path.dirname('/home/myhome/myfile'));
  \end{verbatim}
  簡單 os
  \begin{verbatim}
const os = require('os');
console.log(os.cpus());
  \end{verbatim}
  這些 module 的 API 就跟 python 一樣,就是去查,就該有什麼就有什麼。

  \subsection{subprocess/thread}
  既然是跟系統連結,那自然就有系統的 multiprocess, filesysem 跟 IPC,Hmm...
  他的 API 基本上就是懂 system call 就應該看得懂,但 node.js 提供很多高階直接
  非同步呼叫,而不用自己從 fork/waitpid/exec 慢慢刻出來,例如上面講的 fs 模組
  就很簡單的 open,然後 write 都是非同步了,主要是 Javascript 本來就不是為了
  低階應用, 因此直接綁定平行處理為必備使用,掛在每一個物件 library 裡面了,
  並沒有像 perl,python 提供那麼低階類似系統呼叫的 fork/waitpid/exec 等呼
  叫,而使用
  \begin{itemize}
    \item process 模組設定或回傳 process 的 attributes,像 pid, uid ...。
      但 process 物件雖然有些操作仰賴 process 模組,但他是 EventEmitter 物件,
      自動跑一個 JS 檔,不用宣告就有的物件,代表目前跑的 process。
    \item child\_process 模組執行外部程式,不需要像 C system call 那麼麻煩,
    \item tty 模組 當 JS 被 node.js 執行並且發現在 tty 環境中,這會自動 init
      一個 process 物件, process.stdin 是 tty.ReadStream, process.stdout,
      process.stderr 是 tty.WriteStream 物件。
    \item worker thread 是 node.js 提供的 thread 方法,這跟 child process 的
      fork 都才有提供 IPC 的能力。
  \end{itemize}
  process 一般資訊
  \begin{verbatim}
console.log(process.pid);
process.stdout.write("write to the standard out of current process\n")
  \end{verbatim}
  簡單的 shell 外部命令使用
  \begin{verbatim}
const { exec } = require('node:child_process');
exec('echo $HOME | grep -i home', (error, stdout, stderr) => {
        console.log(stdout);
  }
);
  \end{verbatim}
  可以看出,物件導向語言很麻煩的是,所有執行結果都會封在一個物件裡面,要
  進一步處理,都要用他的物件裡面的物件,並且用他的 API methods 處理,這也是
  物件導向程式特性之一。總共有 4 個執行外部程式手段
  \begin{verbatim}
child_process.exec(command[, options][, callback])
child_process.execFile(file[, args][, options][, callback])
child_process.spawn(command[, args][, options])
child_process.fork(modulePath[, args][, options])
  \end{verbatim}
  其中只有 exec 的參數不用放在 array 中,也比較像在 shell 執行命令一樣,
  所以他其實是會先產生一個 shell 來執行,像 system(),其他用 arguments
  array 的,就沒辦法解釋 shell 的一些奇怪的字元。
  \\\\
  這些都是非同步執行, 當要執行 multi process 時, 這些都是會
  背景執行的, 就像 shell 執行命令加上 \& 符號一樣。 也就是傳統 C 用 fork
  然後在程式中, 自己拆開 child pid 然後 exec 這麼不直覺的事情, JS 就不用了,
  他完全都是自動非同步執行。
  \\\\
  exec, execFile 有 callBack 有三個參數處理 error,stdout,stderr 三個物件。
  \\\\
  spawn 跟 execFile 一樣,但由於他不像 execFile 只是存在小 buffer 裡,200kB
  而已, spawn 的 child process 的 output 是存在一個大 file stream 裡面,
  因此如果 output 很大要處理,用 spawn 才對。
  \\\\
  fork 只能執行 JS,無法執行 shell command,binary,
  \\\\
  這四個都會回傳一個 ChildProcess 物件,用 fork ,還會建立一個 ChildProcess
  channel 物件,可用來做 IPC message channel,以及相關 channel 操作 ,例如
  disconnect(), send(),kill(),所以 fork 是裡面唯一沒有 shell 這個選項的
  \\\\
  沒有 callback 的 spwan/fork stdout, stderr 處理要用 event 的 on
  \begin{verbatim}
// 用 spwan
const { spawn } = require('node:child_process');
const subprocess = spawn('ls', ['-l']);

subprocess.stdout.on('data', (data) => {
  console.log(data.toString());
});
subprocess.stderr.on('error', (err) => {
  console.error('Failed to start subprocess.');
});

// 用 exec
const { exec } = require('child_process');
const subprocess = exec('ls -l / | grep vmlinuz', (err,stdout,stderr) => {
    if (err) {
      console.log(stderr)
    } else {
      console.log(stdout)
    }
  }
)

// 用 execFile
const { execFile } = require('child_process');
const subprocess = execFile('node', ['--version',], (err,stdout,stderr) => {
    if (err) {
      console.log(${err})
    } else {
      console.log(stdout)
    } 
  }
)
  \end{verbatim}
  跟原本 Javascript 標準的 Promise  
  物件關係可用 util.promisify,取得後用原本 Javascript 的 async function
  \begin{verbatim}
const util = require('node:util');
const execFile = util.promisify(require('node:child_process').execFile);
async function getVersion() {
  const { stdout } = await execFile('node', ['--version']);
  console.log(stdout);
}
getVersion();
  \end{verbatim}
  使用一般 fork 來執行 JS 程式,fork 只能執行 Javascript
  \begin{verbatim}
const fork = require('child_process').fork;
const program = path.resolve('other.js');
const child = fork(program, [], {
  silent: true
});
console.log(child.pid)
  \end{verbatim}
  silent 是 true 表示不會顯示結果到 stdout,但用 fork 好處是 child, parent
  彼此可以用 message channel 溝通,在 IPC 再說明。
  \\\\
  exec 會執行 shell,其他則不會,用 spwan 來執行像 shell pipeline,on
  必須處理 data 與 close 兩個 event ,其實就是原本的 parent 的 close()
  呼叫與 child 的 read() 呼叫,沒有呼叫 close(), child 的 stdin read()
  是會 blocked 的
  \begin{verbatim}
const  {spwan} = require('child_process)
const ps = spawn('ps', ['ax']);
const grep = spawn('grep', ['ssh']);

ps.stdout.on('data', (data) => {
  grep.stdin.write(data);
});
grep.stdout.on('data', (data) => {
  console.log(data.toString());
});

ps.on('close', (code) => {
  if (code !== 0) {
    console.log(`ps process exited with code ${code}`);
  }
  grep.stdin.end();
});
grep.on('close', (code) => {
  if (code !== 0) {
    console.log(`grep process exited with code ${code}`);
  }
});
  \end{verbatim}
  spawn 中的 options 中有個 stdio 的 array
  \begin{verbatim}
const spawn = require('child_process').spawn;

const command = 'node';
const parameters = [path.resolve('program.js')];

const child = spawn(command, parameters, {
  stdio: [ 'pipe', 'pipe', 'pipe', 'ipc' ]
});
  \end{verbatim}
  其中重要的 stdio 參數設定是指跟 parent 的 stdin, stdout, stderr 的關係,
  是一個 array ,index 分別代表 fd 0, 1, 2, 3 ..... 裡面的字串值
  \begin{itemize}
    \item pipe 表示 child 的 0,1,2 到  parent 的 0,1,2
    \item overlapped 在 unix 系統中,這跟上面的 pipe 是一樣的。
    \item ipc 這跟 fork 一樣,是會建立 parent child 間的 message channel
    \item ignore 忽視 child 的這個 fd, 也就是 /dev/null 會等於 child 的 fd,
      例如 shell 中的 cmd > /dev/null 2>\&1 中需要導到 /dev/null 去。
    \item inherit parent 的 0,1,2 等於 child 的 0,1,2
  \end{itemize}
  使用 tty 模組會自動 init process.stdin, process.stdout, process.stderr
  ,這些其實是 tty.ReadStream 與 tty.WriteStream 物件,可以直接拿來用。
  \begin{verbatim}
'use strict';
import tty from "tty";
console.log("TTY width: " + process.stdout.columns);
  \end{verbatim}
  傳統 C 的 multiprocess 如果是 fork/exec ,那在 JS 就是用 exec/execFile
  就達到一樣效果, 如果像傳統 C 程式,從 fork 起,程式內變為 child 的做法,
  那就是 setTimeout , Promise 做法。
  \\\\
  而從 10.5.0 版本開始,也提供了W3C 制定的 Worker class 呼叫,就是當初
  browser 的 multiprocess,在 node.js 中放在 worker\_threads 這個模組中
  \begin{verbatim}
const {
  Worker,
  isMainThread,
  setEnvironmentData,
  getEnvironmentData,
} = require('node:worker_threads');

if (isMainThread) {
  setEnvironmentData('Hello', 'World!');
  const worker = new Worker(__filename);
} else {
  console.log(getEnvironmentData('Hello'));
}
  \end{verbatim}
  \_\_filename 是自身 js 檔名,所以他用 worker fork 了一個 child thread,
  並且能用 isMainThread 判斷。
  \\\\
  不管如何都不要跟 web application 扯上關係, web browser 是 web browser,
  這裡的 node.js 就是不折不扣的像 perl, python 的 script 語言。很多在講
  JS thread 的其實都是在說 browser 的 thread 能力。

  \subsection{IPC}
  GUI 中跟傳統 system call 不同的是,使用者都只是大 library 下的韭菜,跟
  系統中會隔著一個大的 main thread loop,由這個大 main thread 控管一切,
  派送不同的 event 與處理相對應的 callback,node.js 不提供 bsd socket,
  sysv 的 message qeuee, shared memory, semaphore。
  \\\\
  在古早jQuery widget set API 有提供一個 on method 來註冊 event 與 handler,
  Node.js 持續用這個 method 在 events 這個內定模組中,所有有 event 的物件
  都應該會繼承 events 這個 class 而自動有 on method。以下為 eventEmitter
  class 物件例子。Node.js 也提供 setImmediate 來馬上執行,不過 web 程式中
  ,現在是不建議使用這個 setImmediate 的。
  \begin{verbatim}
const EventEmitter = require('node:events');
class MyEmitter extends EventEmitter {}
const myEmitter = new MyEmitter();
myEmitter.on('event', (a, b) => {
  setImmediate(() => {
    console.log('this happens asynchronously');
  });
});
myEmitter.emit('event', 'a', 'b');
  \end{verbatim}
  process 這個物件本身就是一個 EventEmitter 物件,是自動就帶有的 global 物件。
  \\\\
  system call 中的 signal, pipe, fifo, socket, SVR4 msg queue, shared memory,
  semaphore 都是 IPC 手段,但僅有少數提供,大多數都隱藏進他們 GUI 程式師的思
  維去了。
  \\\\
  signal 處理,使用 process.on,process 物件
  \begin{verbatim}
process.on('SIGINT', () => {
  console.log('Receive SIGINT.');
});
 
function handle(signal) {
  console.log(`Signal Received is ${signal}`);
}
 
process.on('SIGINT', handle);
  \end{verbatim}
  除了 on 以外,還有 once 可以用來註冊一次性的 event handler。
  \\\\
  使用原本 process, child\_process 的 fork,在四種 child process 執行方法中,
  使用 spawn 可以給 stdio 的參數陣列來設定 parent 跟 child 的溝通關係,
  使用 fork 可以建立一個內定的父子溝通管道。還回的 object 中有 send()
  這個 method 可以用來送 message,彼此 process 裡面有 message 這個 event 來
  呼叫 handler,這個後面 node.js 的 C 實做在 Linux
  是用 domain socket,在 Windows 是 named pipe,這最後就是不要去管什麼
  pipe, fifo, shared memory 了,就統一用他的 pipe,在 Unix 系統骨子裡是
  用 dmoain socket 就是。
  \\\\
  child.js
  \begin{verbatim}
if (process.send) {
  process.send("Hello");
}

process.on('message', message => {
  console.log('message from parent:', message);
});

setTimeout( () => {
  process.exit(0)
}, 3000);
  \end{verbatim}
  parent.js
  \begin{verbatim}
const path = require('path')
const fork = require('child_process').fork;

const program = path.resolve('child.js');
const child = fork(program)

child.on('message', message => {
  console.log('message from child:', message);
  child.send('Hi');
});

console.log("parent pid: " + process.pid)
console.log("child pid: " + child.pid)
child.on('exit', () => {
  console.log("child exit: " + child.exitCode)
});
  \end{verbatim}
  兩邊都會 blocked 在 system call 的 read() 的等待中,才能互相送 message,
  child 必須自己 exit, 因為是 socket 程式,所以有 I/O,才會 close(),parent
  才會看到 EOF 繼續往下走,不然 parent 會一直 blocked 在 read() 中。
  Node.js 必須等到所有的 exit event 的 handler (listener) 被呼叫了才會結束。
  \\\\
  最後在新的 10.5.0 後可以用 worker thread,這裡面也有 message 功能,每一個
  建立的 worker thread 都有一個 MessagePort 這個內定 channel 跟 boss 溝通,
  所以他們都有一個postMessage() 這個 methods.
  \begin{verbatim}
'use strict';

import { fileURLToPath } from 'url'
import { dirname } from 'path'
const __filename = fileURLToPath(import.meta.url)
const __dirname = dirname(__filename)

import {
  Worker,
  isMainThread,
  parentPort,
  workerData, threadId,
  setEnvironmentData, getEnvironmentData,
  MessageChannel, receiveMessageOnPort, BroadcastChannel
} from 'node:worker_threads';

// 這些 global variable 是無法像 C/python threading 一樣使用的
let tcount = 0
const buffer = new SharedArrayBuffer(1);
const uint8 = new Uint8Array(buffer);

// new MessageChannel 會得到兩個 MessagePort 物件
// 使用 receiveMessageOnPort() 來得到 message
// 如果有一個 new Worker, 這會被執行兩次
const { port1, port2 } = new MessageChannel();
port1.postMessage({ hello: 'world' });
console.log(receiveMessageOnPort(port2));

if (isMainThread) {
console.log("######################################")
// worker 使用 BroadcastChannel 來 broadcast message 給所有 thread
  console.log("boss: " + threadId)
  // 使用 environment 與 workerData 傳遞 message
  setEnvironmentData('Hello', 'Use environment!');
  const worker1 = new Worker(__filename, { workerData: 'Use workerdata!' });

  // 內定有個 MessageChannel 產生 MessagePort 物件
  // parent 能使用 worker.postMessage 傳 message 給 child
  // 也能從 child 接到 message
  worker1.postMessage('Use MessagePort channel to worker and back to boss!');
  worker1.on('message', (message) => {
    console.log(message);
  });

  // BroadCast channel
  const bc = new BroadcastChannel('testing');
  let c = 0;
  bc.onmessage = (event) => {
    console.log("boss get msg from bc: " + event.data);
    if (++c == 2) bc.close()
  };

  bc.postMessage("boss post")

  const worker2 = new Worker(__filename);
  worker2.postMessage('thread 2')
} else {
  console.log("worker: " + threadId)
  // worker gets message from environment and workerData
  // Javascript 的 multithread 其實只是 multiprocess
  // 無法像 C/python 的 multithread access 同一份 global variable
  tcount++
  Atomics.add(uint8, 0, 1)
  console.log(getEnvironmentData('Hello') +
    ":" + tcount +
    ":" + Atomics.load(uint8, 0));
  console.log("child get workData: " + workerData);

  // javascripot 全靠 message passing 存取共同資料
  // worker 能使用 parentPort.postMessage 傳 message 給 boss
  parentPort.on('message', (message) => {
    console.log("thread: " + threadId + " " + message);
    parentPort.postMessage("thread: " + threadId + " toBoss " + message);
    parentPort.close()
  });

  // Broadcast.postMessage 只能由 worker 傳 message 給其他人
  const bc = new BroadcastChannel('testing');
  bc.postMessage('broadcast from every worker');
  bc.onmessage = (event) => {
    console.log("worker get msg from bc: " + event.data);
  };
  bc.close()
}
  \end{verbatim}
  但這要非常小心,跟正常 C fork 程式不一樣。
  \begin{itemize}
    \item multiprocess 但 global 的程式碼都會被執行,所以會有執行兩份,一直
      到 isMainThread 這個 branch 才會分開。
    \item 他這不是 multithread,global variable 不會被共同看見。
    \item 目前用 node.js 20.11.1 但 broadcast channel 完全沒有文件上說的
      broadcast 效果,只能 worker 往 boss 發出而已。
  \end{itemize}
  Javascript 當初並沒有想那麼多,只能用 message passing 做到憋腳的 multiple
  processing,然後後面用 message passing,完全是 GUI 人的想法。所以 worker
  thread 基本上還是用 message passing 方法來溝通兩個 thread (process)。
  任一個 message passing channel 就是 socket 程式,沒有 close() 就會 block
  住。
  \\\\
  要做到像
  c 真的 threading,使用 global 變數來讓兩個執行 function 溝通,是沒有辦法的,
  即使用 EMACScript 2017 年 global 的 SharedArrayBuffer ,也還是需要
  postMessage 丟來丟去,是非常沒有意義的做法。
  \\\\
  message passing 或者可以用外部 package 的 message 來做 IPC,
  兩個不相干 process 通常用 socket, install node-ipc 模組
  \\\\
  process 1
  \begin{verbatim}
  const ipc = require('node-ipc');

ipc.config.id = 'a-unique-process-name1';
ipc.config.retry = 1500;
ipc.config.silent = true;
ipc.serve(() => ipc.server.on('a-unique-message-name', message => {
  console.log(message);
}));
ipc.server.start();
  \end{verbatim}
  process 2
  \begin{verbatim}
const ipc = require('node-ipc');

ipc.config.id = 'a-unique-process-name2';
ipc.config.retry = 1500;
ipc.config.silent = true;
ipc.connectTo('a-unique-process-name1', () => {
  ipc.of['jest-observer'].on('connect', () => {
    ipc.of['jest-observer'].emit('a-unique-message-name', "The message we send");
  });
});
  \end{verbatim}
  他這是有 unix domain socket 也有 tcp socket 的,只是這個例子用最簡單
  message, 還有其他 message queue 工具,像rabbitMQ, redis ... 等,只是這
  些是外來的,包括 node-ipc 都是外來的模組。
  \\\\
  javascript 的學習也很豐富,主要還是要多做從做中學,不做,永遠都在 
  hello world。
  \href{https://github.com/lydiahallie/javascript-questions}{Javascript問答},
  還有一個跟 web application 有關的
  \href{https://github.com/wesbos/JavaScript30}{Javascript 30天挑戰},
  Javascript 關於web application 的應用,留待網路 API 再解說。

  \section{JSDoc}
  Javascript 有個 API 文件產生器 jsdoc,跟 pydoc, doxygen 類似的 comment
  寫法。先安裝
  \begin{verbatim}
npm install -g jsdoc
  \end{verbatim}
  然後在 JS 檔的 function, class 上方
  \begin{verbatim}
/**
 * This is a function.
 *
 * @param {string} n - A string param
 * @param {string} [o] - A optional string param
 * @param {string} [d=DefaultValue] - A optional string param
 * @return {string} A good string
 *
 * @example
 *
 *     foo('hello')
 */

function foo(n, o, d) {
  return n
}

jsdoc foo.js
  \end{verbatim}
  @param, @return, @example ... 這些稱為
  \href{https://jsdoc.app/#block-tags}{tags},
  用 \verb={}= 框起來的為資料型別 type。用 \verb=[]=框起來的是 optional的
  參數。 只要根據定義標準寫完,用 jsdoc 命令就能在 out 這個目錄下轉出 html
  文件。
  \\\\
  由於他設定太多,所以要用個 json 檔案做 jsdoc configuration 設定,如果有
  玩過 Doxygen ,這個就相當於 Doxygen 檔。
  \begin{verbatim}
jsdoc -c /path/to/conf.json
  \end{verbatim}
  type 例子
  \begin{verbatim}
type:
@param {string=} n 	Optional
@param {string} [n] 	Optional
@param {(string|number)} n 	Multiple types
@param {*} n 	Any type
@param {...string} n 	Repeatable arguments
@param {string} [n="hi"] 	Optional with default
@param {string[]} n 	Array of strings
@return {Promise<string[]>} n

typedef:
@typedef {Object} Song
@property {string} title - The title
@property {string} artist - The artist
@property {number} year - The year
  \end{verbatim}

  \section{nmp package 的產生}
  類似 perl, python 的 package ,但不一樣的是通常不會用檔案形式的 package,
  他的 package 有下面幾種情況
  \begin{enumerate}
    \item 一個目錄,藏著 package.json 之前說的那些檔案。
    \item 一個 tar.gz 檔,是第一形式的壓縮檔。
    \item 一個 URL 指向一個tar.gz 檔。
    \item 一個 <name>@<version> 字串,而這個字串是 public registry 註冊的。
    \item 一個 <name>@<tag> 字串由上一個字串 public registry 註冊 tag。
    \item 一個 <name> 字串,然後這個 public registry package 有 lastest 這個
      tag 註冊。
    \item 一個 git url 指向第一形式的目錄,在 package.json 的 "repository" 
      設定。
  \end{enumerate}
  這個是在 package.json 裡面可以指定的。基本上當說要建造 package,其實是要
  把自己的傑作送上 npmjs.org 註冊的,當然也可以自己 tar.gz 但就自己站在目錄
  上面打包就好。
  \\\\
  npmjs 維護一個網站與 registry 資料庫,可以有公開與私有的註冊,也有付錢註冊
  ,維護更多選項包含商業,付費等 package 維護。
  可以從單一 npm 命令產生。package 有
  \begin{itemize}
    \item private package(也叫scope package), package 名字用 @xxx/pkg-name 形式,
      "exports": "./index.js"。這種 package 是要註冊付費會員的。
      \begin{itemize}
        \item User-scoped private packages,只能自己存取或指定特定人存取。
        \item Organization-scoped private packages, 能由一組公司人員存取。
      \end{itemize}
    \item public package, package 名字用 pkg-name。
  \end{itemize}
  首先要先去 https://registry.npmjs.org 建立會員,然後要login
  \begin{itemize}
    \item npm login
    \item npm publish 這是 publish 一個 private package
    \item npm publish \verb=--=access public
  \end{itemize}
  以下為練習範例
  \begin{verbatim}
  ls . -R
  file-progress-bar.js node_modules package.json unit-test.js

./node_modules:
numfmt

./node_modules/numfmt:
index.js  package.json  unit-test.js
  \end{verbatim}
  \begin{itemize}
    \item 我這個叫 file-progress-bar 的模組下面還有一個 numfmt 的模組,被
      file-progess-bar.js 引用。
    \item 每個裡面有一個 unit-test.js 的測試檔,所以在 package.json 中,
      "scripts" 會有這個 unit-test.js entry。
    \item 我這裡面都用 ES6 module 的方法寫,所以 package.json 中會有
      "type": "module"
  \end{itemize}
