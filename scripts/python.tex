\chapter{python}
\section{準備}
python 是後來的後起之秀,主要是perl當初的目標只是要統一一個
unix script 介面,這當初的 script 只是單純的 sed/awk/shell 之類
的統一替代品,沒想到越做越複雜,語法變得很複雜,物件語法也不是很完整,
很多東西都是後來補東牆補西牆的加上去的。因此全新的 script 設計就萌芽了。
python 是個物件 script,語法設計考慮到很多已經屬於正統型態
語言所具有的特性,但又不失為一個 script 的簡便性。目前很多大公司的 project
都是用 python 去實作的。他有一些特點我想是 perl/shell 轉過來的人比較不太習慣
\begin{itemize}
  \item python的型態比 perl 等 strong 點,不太能任意變換,所以喜歡 C 之類的人
    上手 python 可能會簡單些,但如果習慣了shell,perl 之類的人就覺得他怪怪的,
    不太像 script ,在字串連接,目錄檔案測試上等習慣上他已經不像傳統的 script
    了。在跟環境的溝通上例如 shell backquote,不像 script 般的直覺,必須寫的像
    一般 C 一樣的 popen 之類的才能達到。如果熟悉用 c/c++ 來寫東西,那會喜歡
    python,如果從 shell/perl 過來的就頭大了。 主要是大部分的資料,都要盡量用 
    API 來獲得,而不是從系統 shell 命令來。
  \item python 的版本差異相當大,一樣是2.x的版本,內定的 function call 差異
    很大,版本號碼的定義不知道在哪裡,這算是我比較不喜歡的一點,在3.x
    以上的 API 變動更大。
  \item python本身是個物件語言,語言本身的算符定義可以說是有 overloading,
    例如=,到底是 assign 還是 copy,他會根據後面資料改變而沒有一致性,這也是從
    傳統語言過來的人要小心。
  \item regular express 的表現手法也比較偏c語言,沒有很方便的轉換方式。
\end{itemize}
但用久了,其實在物件導向的寫法中,python比perl簡潔,語法也正確多了。
\\\\
有了之前的經驗,我們不再用一般介紹,而直接改用進階的觀念介紹。
\subsection{工具與特性}
python 有她蠻不錯的特性
\begin{itemize}
  \item interactive 的環境,可以直接打 python 進入互動模式,像以前的 BASIC 一樣
    ,現在去 http://www.python.org 還有 shell 可以馬上線上玩。
  \item pyc byte code,跟傳統 script 不一樣的是 python 是以 byte code 來執行的。
\end{itemize}

由於python程式可以在 interactive 的環境跟 standalone 的環境執行,所以程式
可以透過\verb=__name__= 的值來區別目前是在什麼環境下,如果 \verb=__name__=
是字串 \verb=__main__= 那就是一般 standalone 的環境。
\\\\
python跟java很像的他也有個 virtual machine,負責解讀 python 所編譯的 bytecode
,他不像c那樣繁瑣有 assemble,compile,link 這些,但會把.py檔先編成.pyo, .pyc並
且存檔起來,下次執行時就會比較快,python內部的解譯器CPython才會讀進這些
bytecode,執行他。bytecode 原理就跟機器語言 assembly 很像,他把+ - * ...if else, 
while 等等語法先轉成虛假的 assembly, and or xor 等等,這種 bytecode 也只有他自己
讀得懂但有先特殊資料排列處理,能讓真的執行時,少掉一些動作而使得快一點。python
的 bytecode 並沒有宣佈說能跨平台甚至跨不同版本 python 的能力,所以每次都應該重來。
但一般使用者是感受不到的。

\subsection{coding convention}
python其實是跟unix c style的,只有一點點不同,所以寫系統c過來的人會很習慣。
\begin{itemize}
  \item indentation 4與expandtab,由於語法上block是由indentation來的,
        所以不建議tab與space混用。要就全用tab,然後告訴editor顯示幾格,不然
        就全用spaces。目前標準是空四格與space,vim可用set expandtab命令。
  \item 模組檔名 小寫
  \item Class名 開頭大寫,匈牙利命名。
\end{itemize}
其他常數,變數,函數名等都是傳統ansi c的方法。

\subsection{文件pydoc}
他這文件系統也很簡潔,很不錯,在interactive mode下看到的help()就是。
\begin{itemize}
\item 多行文字時,必須用三個quote """ 表示。
\item 第一行必須第一個字大寫開始,句點結束。這行就是標題"Capital letter
  and end with a period."
\item 第二行必須空白
\item 第三行開始Description,要小心的就是indentation而已。
\item module 的 description 起始於檔案最開頭,所以 \_\_init\_\_.py 
  的最上面必須是三個 quote 起始表示文件。
\item 每個method下開始的敘述就是help(method)時跑出來的。
\item 內定變數 \_\_version\_\_ 表示這個package版本, \_\_author\_\_ 表示
  作者, \_\_all\_\_ 表示public method與變數字串。 如果有用 \_\_all\_\_,則才會
  這裡面的變數會跑出document來。
\end{itemize}
範例
\begin{verbatim}
""" MyClass implementation. """
#vim: sts=4 sw=4 et :

MY_CONST = 0
NO_EXPORT = 1

class MyClass():
  """ MyClass is my class.

  MyClass is really my class
  """
  def __init__(self, arg1 = 1):
      """ Constructor:

      Constructor for MyClass
      """
      self.a = arg1

  def get_a(self):
      """ get_a:

      get_a return member data a
      """
      return self.a

__all__ = ['MY_CONST', 'MyClass']

if __name__ == "__main__":
  mo = MyClass(5)
  print "my a is %d \n" % m0.get_a()
\end{verbatim}
pydoc命令有以下選擇
\begin{itemize}
  \item pydoc module會跑出man page的help。例如 pydoc re 或 pydoc sys
  \item pydoc 路徑檔名或./myclass.py也可以
  \item pydoc -p 1234會跑一個在 localhost:1234的 http server,列出目前系統上所有API。
  \item pydoc -g 需要裝python-tk,會跑出一個GUI的API reference。
\end{itemize}

\section{模組與函式}
很多函式或者class定義的組合成一個檔案叫module,很多個module檔組合叫package。
有以下基本觀念用法
\begin{itemize}
\item 函式定義 def xyz(arg0 = 0, arg1 = 'init', arg2 = \verb=[]=):
	可以看出他有 default value 的觀念在裡面,除了 default value 外,我
	喜歡 python 的呼叫可以光指定變數給值,xyz(arg1='myvalue') 這樣表示 arg0,
	arg2 用 default value,arg1 用 user 指定的,這比 perl,C 的要方便多。
        但要特別特別特別注意的是,python的mutable argument default value 只會
        在第一次建造 method/function 時產生,所謂 mutable 是指 \verb=[] {} object=
        ,immutable 有 string, tuple 等。所以這搞了我很久,我看很多人大概都
        不曉得。這很笨的是每次都要用 \verb|xxx = xxx or []|,有 default value
        等於沒有。
\item 變化長度的參數數目def xyz(*args),在C裡面用...與va\_start, va\_end,
	在 perl 等 script 是比較簡單,愛怎麼亂搞都可以,python算是兩者間,
	呼叫上只能少不能多,要多的話要用 *args, **kargs 來作變化,*args 是用來
	給沒有 default value 的參數用的,**kargs 是用來給有 default value 用的,
	這其實有 c 的 pointer 味道了。一個假裝看成是 tuple 的 pointer,
        一個假裝看成是 dictionary的 pointer。
        \begin{verbatim}
        def test_var_args(farg, *args):
            print "formal arg:", farg
            for arg in args:
                print "another arg:", arg

        test_var_args(1, "two", 3)

        def test_var_kwargs(farg, **kwargs):
            print "formal arg:", farg
            for key in kwargs:
                print "another keyword arg: %s: %s" % (key, kwargs[key])

        test_var_kwargs(farg=1, myarg2="two", myarg3=3)
        \end{verbatim}
	用來傳參數時的用法
        \begin{verbatim}
        def test_var_args_call(arg1, arg2, arg3):
            print "arg1:", arg1
            print "arg2:", arg2
            print "arg3:", arg3

        args = ("two", 3)
        test_var_args_call(1, *args)

        def test_var_args_call(arg1, arg2, arg3):
            print "arg1:", arg1
            print "arg2:", arg2
            print "arg3:", arg3

        kwargs = {"arg3": 3, "arg2": "two"}
        test_var_args_call(1, **kwargs)
        \end{verbatim}
	*args這種用法主要還有可以讓參數進來的型態是變化的,不是死的,
	例如可以讓進來的參數是物件或者string,進來後再判斷用什麼方法
	去初始。這是overloading。還有就是子class繼承父class時的參數,也可以
        用這作彈性的interface。
        \begin{verbatim}
        def my_overriding(self, *args, **kargs):
        \end{verbatim}
\item 變數scope是跟C一樣的,自動就有local variable,不像shell,perl當初是
	沒有的,還要鱉腳的宣告才有。python有幾個scope,
        \begin{itemize}
          \item global變數 這像c的global宣告,只要在檔案內最外block定義就有,
            但不同的這只在這模組檔內有效,也可以在任何 block 地方用 global var
            宣告。
        \item class變數 這是所有instance都會共有的,在class宣告底下一開始的地
          方定義,例如
          \begin{verbatim}
          class MyClass():
              class_var = 1	
          \end{verbatim}
          存取時用 MyClass.class\_var

        \item instance變數 這是單一個instance有的,在class內通常用self.var來使用。
        \item block local變數 函數內變數宣告, for, while內變數
        \item 從function return 回來的,跟C一樣是copy一個新的物件在原呼叫
          context範圍內。
        \end{itemize}

\item pass by reference/value與assign/copy算符,
	基本上python完全是物件語言,如果是基本型態像是數值字串是pass by 
	value的,複合型態像物件,list等是pass by reference,在C中就是pass by
	address。
	\\\\
	python的型態是強型態語言,但是他的=這種assign不像perl/c的定義一致,
	由於沒有pointer與perl的reference,所以他的 = 其實是有算符overloading,
	當他碰到數值字串時,是創造一個新的數值字串空間並copy原來內容過來,
	但如果碰到物件或者list, tuple,arry等,卻是等於是把pointer指過去。
	總之python內的字串跟數字都把他當成基本
	型態,其他當成pointer,要改變物件或list內容只能用物件提供的method。
	\\\\
	所以在一個function中,傳數值或字串進來要真的去改變他的內容時,而不能
	用=這種assign物件的方法去改變物件內容。如果傳進來的是一般數值或者string
	,那就不好處理,只能用return的方式。而return回去的東西,也就是在原本
	的namespace中的一個新物件reference。
	\\\\
	但如果是傳進一個list,dict或者物件等,那就要小心,亂搞是會把原本傳進
        的東西給改變,只能用list(), dict()等另建list, dict來操作他。
        \\\\
        所以真正要會電腦語言,還是要從 C/C++ 學起,不然我看過有人講解這個
        ,又引進一堆新名詞,什麼deep,shallow copy ...blabla...,根本就不必要
        也胡搞瞎搞,所有電腦只有記憶體與值兩件事,其他都是XX。
\item 函式return
	如果 return 一個 function 內部物件,會傳回一個在原本namespace上的一個
        物件 copy。但如果 return 一個外部傳進去的物件,則回傳同一個 reference
        。
\item 模組檔名與定義class 建立一個檔名mymodule.py
        \begin{verbatim}
        class myclass(parent_class):
            def __init__(self):
                my_member_init...
            def mymethod(self):
                myimplement...
        \end{verbatim}
        \verb=__init__=是為資料init的自動呼叫函式。
\item import 模組
	把sys這個模組引進來使用
        \begin{verbatim}
        import sys
        sys.xx()
        sys.yy()
        \end{verbatim}
	上面自訂模組就是import mymodule。
\item from time import localtime
	從time模組引進localtime 這個symbol,在程式中可以直接使用 localtime
	也就是說只有 import 後面的字串才能直接使用,所以當 import time 時,要使
	用 localtime,還必須用time.localtime()才行。但只要用
	from time import *或者from time import localtime,就可以直接用
	\verb=localtime()=這樣呼叫。這其實是如果模組中有定義\verb=__all__=
	import只會import\verb=__all__=裡面的symbol,如果沒有會import所有不
	是underscore起始的全域變數。from time import *,會把namespace給毀了,
        除非確定完全沒有同名的symbol。
\item 重複import同一個模組,並不會使得模組中的程式碼被重複執行多次,如果想
	要重新執行模組,則使用imp.reload()函式
\item PYTHONPATH 與sys.path 跟 PERL5LIB 與 @INC 是一樣的道理的。在 python 中有
        自動本來就有的 standard 模組,這些是在 PYTHONPATH 中看不到的。但使用上
        卻還是要 import 進來,sys 這個模組就是這樣子的。如果 python 的
        safe-path 或 isolate mode 是 false,那麼只由被執行 script 的目錄會在
        sys.path 前面,所以 interactive mode 執行 script 目錄不會放在 sys.path
        前。
\item 如果使用python指令跑.py檔案,python xyz.py則\verb=__name__=在xyz.py內的值
	為\verb=__main__=,如果是import xyz,則\verb=__name__=在xyz.py內為
	模組名。
\item 用dir()來找出目前import的模組,用dir(sys)找出在sys下所有的函數名,用
	help(sys)列出sys模組的man page
\item 製作自己的package,從 PYTHONPATH 下有如下的目錄跟檔案結構
	\begin{verbatim}
	mypkg/
                __init__.py
                module1.py
                modeul2.py
                mysubpkg1/
                        __init__.py
                        submodule1.py
                mysubpkg2/
                        __init__.py
                        submodule2.py
	\end{verbatim}
	使用 package 跟 module 很像,也是 import mypkg 或者 from mypkg import
        module1, 在以前 from mymodule import * 會把所有的 class 或變數引用
        進來,但是在 from pkg import * 是不會引進所有模組,他只會引進
        \_\_init\_\_.py 裡的東西而已,其他的模組檔案還是要一個個引進,通常就
        是在\_\_init\_\_.py 裡面import。
	\\\\
	\_\_init\_\_.py 在 import package 時叫用,通常裡面設定變數 \_\_all\_\_
	與從這 package 下的模組 import,如果沒有這檔案,python 不會把目錄視為
	package 目錄。\_\_init\_\_.py 是 package ,不是 class 的。所以不用幻想
        可以呼叫i = package()然後會自動呼叫 \_\_init\_\_py 裡面的 \_\_init\_\_()
        來創造出一個新instance。
        \\\\
        同一 package 下的 module 檔,可以用 import .moduleName 來彼此 import,
        例如上面例子中的 module1.py 裡面可以用 import .module2 來 import 
        module2。如果要 import 上層目錄的就是 ..moduleName 類似相對路徑使用。
\item 目前的python錯誤檢查,如果編譯 py 檔出錯,其實可能卻是import的
  自己的 library 出錯,而不是 error message 所顯示的目前這個檔的錯誤。
\end{itemize}
package/module 的用法在於命名與 library 的組織規劃更有結構。但要小心規劃這些命名
,這有兩個使用方法,一種是傳統的 utility 的建立方法,也就是函式就只是很
單純的工具函式而已,但分門別類的收集在一起,python 很多內定的函式就是這
樣,例如 time.localtime(),這 localtime 可能只是單純的傳回 localtime 的功能
,而在 time.py 裡面有很多這樣的工具函式而已,裡面可以完全沒有 class 的宣告。
這很像 perl 但卻不像 perl 有 class 在裡面。這樣的模組用 import xxx 使用就可。
\\\\
另一種是帶有 class 的模組與想要物件建立,但這種的命名就有點麻煩,例如一個模組叫
device.py,那裡面有 Device 這 class時,在建立物件時變成dev = device.Device(),
這變成很重複的名稱是很不好用。
他這問題在於不像C++的命名空間是整個.h或.c且有個new()來產生物件,所以c++
裡面可以單純的用dev = new Device(x,y),或者perl的namespace是整個檔案跟著整個
class所以可以定義new,\verb|$dev = Device->new()|,這個解決方法可以用
\begin{itemize}
  \item 仍然用很醜的device.Device
  \item 在device.py內多定義一個global function叫new()。但每個都寫會抓狂。
  \item 用 import device.Device as Device。
  \item 用 package 方式,把 symbol 放在 \_\_init\_\_.py ,使用 import package。
  \item 用 from device import Device 藏在 \_\_init\_\_.py 內,但這破壞原本
    python import 的意思,或者只能讓使用者自己from device import Device。
    畢竟這是一個模組一個 class 才有的情況,用這樣可能是比較好的方法了。
\end{itemize}
如果沒有開 device package 目錄,會被迫把所有 device 的 class 定義在一個檔案內,
維護起來 就很沒有條理。所以把 Device 這 class 藏在 device/\_\_init\_\_.py 內,
外面用 device 目錄,裡面可以放各種device的模組檔,繼承 Device 這 class。
\\\\
如果PYTHOHPATH 裡不同名目錄,但下面的模組同名,則會根據 PYTHONPATH 先找到的先用
,第二個就不會import進來了。
\\\\
如果已經有別的模組或 package 同名了,例如 PYTHONPATH下 有兩個目錄,下面都有一
個叫 test 的目錄,也就是 package 名相同,但下面的模組名跟 class 名不同,則一樣
只會抓 PYTHONPATH 下先抓到的 package。所以要小心不要跟系統已有的 package 名相
同。並不會所有的全部放進同一個 package,ruby 可以這麼作,但整個 namespace 是蠻
亂的。
\\\\
如果已經有別的模組或package同名了,例如 PYTHONPATH 下有一個叫 device/power 
的目錄, 也就是 package 名,然後我在 device 這個 \verb=__init__.py= 中裡面有個
power() 的 global function,則這樣也不行。
\\\\
就是說 python 不會去區別變數名,函數名,class 名,模組名,套件名,只要是在同一
個 name scope 下,管你是什麼東西,通通不准同名。
\\\\
如果要動態的 import,那麼就只能用 \verb=__import__()=,用法是
\begin{verbatim}
import pkg
pkg = __import__('pkg')

import pkg.mod
pkg = __import__('pkg.mod')

from pkg import mod, mod2
pkg = __import__('pkg', ['mod', 'mod2'])
mod = pkg.mod
mod2 = pkg.mod2

from pkg.mod import func, func2
pkg_mod = __import__('pkg.mod', ['func', 'func2'])
func = pkg_mod.func
func2 = pkg_mod.func2

from pkg.mod import submod
pkg_mod = __import__('pkg.mod', ['submod'])
submod = pkg_mod.submod
\end{verbatim}
如果fromlist是空的,則只import pkg,如果fromlist不是空的import pkg.mod
最後都是return一個module物件。所以光是用\verb=__import__()=是沒有名字的,
必須得到那個module物件後,再用名字代換一次。

  \subsection{其他特殊形式用法}
    \subsubsection{Lambda form}
    python還有像C的inline的這種用於簡單邏輯的小函式定義,
    \begin{verbatim}
    f = lambda x, y: x if x > y else y
    \end{verbatim}

    可見lambda回傳的是一個function pointer的東西,或者就是函式
    物件,可以直接拿f來使用。在C中f可以丟來丟去,也就是作為
    callback function.當然一般def的函式也是一樣可以作為callback
    我猜這很像inline function,可能實作上沒有stack。
    \\\\
    \subsubsection{yield generator與iterator物件}
    函式裡面如果有 yield 語句,那他是一個 generator 函式,表示他回傳的是一個
    iterator 物件。python 的語法特性裡面除了物件導向外,有個很重要的 iterator
    的物件與 iteration 用法。 iteration 用法很多地方都會用,最重要是這種物件有
    next() 這個 method 可以呼叫,如果函式中有 yield,那這函式自動回傳一個
    iterator物件。
    \\\\
    \begin{verbatim}
    def myiter(mylist = []):
        for e in mylist:
            yield "element %s" % e

    iter = myiter(['a', 'b', 'c'])
    \end{verbatim}
    從現在開始有next()可以使用了,就可以 print iter.next()

  \subsection{基本的內建函數}
  基本函數中有一些跟物件,還有iterator有關的函數。
  \begin{itemize}
    \item hasattr(o,'m') 看是不是有這member data
    \item getattr/setattr/delattr,這等於o.x, 例如o.getattr("x")
      這參數是可以變化的。
    \item isinstance(o,C) 看是不是某個class的instance, 這必須是class的定義symbol。
    \item type(o) 這回傳變數的型別,分別有int, str, list, instance, classobj等,
      這會跟class定義時有沒有繼承 object 這廣義的class有關。
    \item id(o) 這就是像c的\&,傳回reference,在本機中就是記憶體位址。
    \item map(o,f) 把一個iterator物件內的物件一個個丟給一個函數處理。回傳一個list
    \item str()與repr() 當使用print object時,回傳出的代表此物件的字串。
    \item bin(o),int(o),float(o),str(o)可以做type casting
    \item vars(o) 會傳回一個list,裡面是物件的members。
  \end{itemize}
  注意不同的 class 定義方法,新的定義繼承 object 這廣義的class。
  \begin{verbatim}
>>> class Test(): pass
... 
>>> t = Test()
>>> t.__class__
<class __main__.Test at 0x7f682b5d2328>
>>> type(t)
<type 'instance'>
>>> class Test1: pass
... 
>>> t1 = Test1()
>>> type(t1)
<type 'instance'>
>>> t1.__class__
<class __main__.Test1 at 0x7f682b5d2390>
>>> class Test2(object): pass
... 
>>> t2 = Test2()
>>> type(t2)
<class '__main__.Test2'>
>>> t2.__class__
<class '__main__.Test2'>
>>> isinstance(t, Test)
True
>>> isinstance(t1, Test1)
True
>>> isinstance(t2, Test2)
True
>>> type(t1) == t1.__class__
False
>>> type(t) == t.__class__
False
>>> type(t2) == t2.__class__
True
>>> class Test3(object):
...     __class__ = 'alibuda'
...     pass
... 
>>> t3 = Test3()
>>> type(t3)
<class '__main__.Test3'>
>>> t3.__class__
'alibuda'
  \end{verbatim}

\subsection{模組安裝}
像perl有 Makefile.pl,python 他的 module 套件打開來目前是規定要有setup.py
\begin{verbatim}
# python setup.py install
\end{verbatim}
但這個安裝並不會順便安裝 dependency,所以像 perl 有 cpan,除了本身系統上的 
package 外,大量的 library 的安裝也有他定義的一套。目前為 pypi.python.org 
維護大量的模組, 比較久的是 easy\_install 這命令,但是他跟 cpan 不太一樣的是他
這是像 deb/rpm 的 binary 套件管理,不是 cpan 類似 BSD 的作法,所以他的套件是叫
egg 格式的python package。 但其實他也只是一個 zip 檔,所以可以用 7z 來看他。可
以用 easy\_install xxx.egg來安裝 。他裡面有 dependancy 的資訊,會自己去下載其他
egg 檔。目前 python3 換到更新的 pip這個安裝程式去。debian 系統上的
package 在 python-setuptools 與 python3-pip,其實也是用 python 寫的程式。
\begin{verbatim}
$ python3 -m pip --version
$ pip install setuptools wheel
$ pip uninstall
$ pip list
$ python3 -m pip install --upgrade pip setuptools wheel
\end{verbatim}
如果沒有權限裝在系統上,通常我們裝在家目錄裡面,並且使用全新自己裝的 library
,而不是用系統的,這叫建立一個虛擬環境 venv,必須先裝 venv 模組,python3-venv。
\begin{verbatim}
$ python3 -m venv mypythonenv
$ source mypythonenv/bin/activate
或者
$ python3 -m virtualenv mypythonenv
$ source mypythonenv/bin/activate
\end{verbatim}
從此使用的 python, pip 裝的modules 都相對於這個venv 目錄下。
\\\\
也有一種早期的方法就是用 target
\begin{verbatim}
$ pip install --target mypythonlib requests
\end{verbatim}
這樣就會把所需 library 裝到 mypythonlib,但這個是會跟系統上使用相同
python, pip,所以要多指向 PYTHONPATH 跟 PATH。

\section{基本控制語法}
block與indentation
與一般語言用 \{ \}當成block方法不同,python 用 : 與 indentation 來作為程式
block,
\\\\
if elif else
\begin{verbatim}
if xxx < yyy:
    actionx
elif yyy:
    actiony
else:
\end{verbatim}

沒有C的 ? : 但 python 2.5 後可以直接 if else寫在一行\\

\verb|x = 'value_when_true' if condition else 'value_when_false'|

for loop
\begin{verbatim}

for i in range(1, 10):
    print i

for c in "mystring":
    print c

for i in [1, 2, 3]:
    print i

for a,b in [(1,2), (3,4)]:
    print a,b

for key,value in {'k0':'v0', 'k1':'v1'}.items():
    print key, value
\end{verbatim}

for loop
\begin{verbatim}
for :
    break
    continue
    else
\end{verbatim}

while loop
\begin{verbatim}
while xxxxx:
    print 1
\end{verbatim}
for loop的用法很靈活,只要是iteration物件傳回,且可以同時設定給多個變數。
\\\\
資料型別與boolean, 其中要注意的還是
\begin{itemize}
\item 沒有定義變數 : 不行作為判斷,exception error跳出
\item 空字串	   : 可以作為判斷。
\item 空list       : 可以
\item 數值非零與零 : 可以
\item 字串非零與零 : 不可,會把他當成字串,這跟shell,perl不一樣
\item 特殊保留字   : 有True/False與None,其中None相當NULL
\end{itemize}

基本上python太像傳統C了,所以沒有定義變數是比較不允許的,但還是可以用
\begin{itemize}
  \item try: var
        except NameError: print "not existed"
  \item if 'var' in locals():
  \item if 'var' in globals():
  \item if hasattr(obj, 'objAttr'):
\end{itemize}
來測試。
測試上面沒有像 shell,perl 那樣豐富簡單的有 -e -f ...這樣的檔案測試,必須使用
os.path.isfile()這樣的函式,所以基本上 python 已經很不像傳統的 script 了。但是
有人卻比較喜歡這樣,認為一些 sciprt 的語法很難懂,尤其 perl 裡面很多奇怪的內定
變數。python的測試上還是只有傳統的C上的 \verb|==, >, <|這些,加上新語法,is,
is not 等等。== 與 is的差別在於 == 是對物件的值做比較,is是對物件的reference做
比較。所以 x == y 是比較其值而is 其實是id(x) == id(y)。
\\\\
一般說來 script 都不會給 case switch 這種東西,因為 if else 跟
associate array(hash, dictionary...)已經
可以建造自己的 case switch,沒有必要多加一個case switch語法。
\\\\
exception是一般物件導向設計時的錯誤處理,在perl中是要另外import特別的模組才有
的,可是在python中是屬於語言的部份。exception主要在於複雜的資料結構的處理中,
不要再一層層的回報上去,傳統的 error handling 在簡單的 return 處理,但系統越來
大後,一件工作其實是好多步驟完成,所以try except就是一大串的步驟執行,只要某個
執行發生錯誤就raise一個exception,算是軟體 error 中斷處理。語法為
\begin{verbatim}
try:
    statement
except SomeException as e:
    statement
except:
    statement
else:
    statement
finally:
    statement
\end{verbatim}
else是除了正常與想抓的exception外的所有情況,finally是最後用來作cleanup的動作
,一定會被執行。exception在python中也是一種物件,所有的exception都是
BaseException的子class,當抓到exception時,可以把這物件assign給變數e,
\\\\
自訂的exception class必須是Exception的子class。
\begin{verbatim}
class MyException(Exception):
    pass
\end{verbatim}
而在程式中可以用raise發出exception了,這個python process就會死掉,並且backtrace
\begin{verbatim}
if something_wrong:
    raise MyException, "Something Wrong"
\end{verbatim}

\section{literal與資料型別}
除了int, str, float, bin外有四種基本複合型別比較重要
\begin{itemize}
\item list = []  就是傳統的陣列
\item tuple = () 這是不可變化陣列內物件的陣列。
\item dictionary = \verb={}= 就是hash, (key,value) pair的陣列。
\item set = set() 沒有重複element的集合。
\end{itemize}
python把int,str,tuple稱為immutable types,表示這些都是不可變動。
所謂不可變動是沒有任何語法方法能經由物件reference 去改變一個已經存在的物件內容
。一般literal
\begin{verbatim}
n = 1

nb = 0b10111011

nh = 0xffabcd

s = "string"

multiline = """
多行的資訊可以用三個"前後包起來
這很像shell裡的
cat <<EOF
blablabl
blablabla
EOF
"""

list = [ 'no1', 'no2', 'no3', 'no4', 'no4' ]
list[-1] = 'no5'
list[3:]
list[1:4]
list[:] = []  # 表示全部清除
list.append('no6')
list.remove('no2')
del list[2:4]
other_list = list([ 1, 2, 3, 4])
[i+1 for i in other_list]

dictionary = { 'key1' : 'value1', 'key2' : 'value2' }
dictionary['key3'] = 'value3'
del dictionary['key2']
other_dict = dict(dictionary)

tuple = (['no1', 'no2', 'no3'])
tuple[2]

set = set(list)
\end{verbatim}
他這list要增加減少都不像perl那樣隨意可以加上去,而是都要透過 API 來增減。
\\\\
list中有很好用的slice,加上冒號就可以切出陣列中的一部份出來,範圍是第1
index跟第2 index - 1。使用負數的index可以從屁股拿出陣列來。這slice後來標準的
形式是\verb=[start:end:step]=,也就是那個step如果是2,那就每隔兩個元素挑出來
成為一個新陣列。如果step 是負的,start, end內定值是None,
在c語言中很喜歡考的一題是反轉string,在python中,只要
\begin{verbatim}
"hello world"[::-1]
\end{verbatim}
就能反轉一個string了。反轉一個sentence
\begin{verbatim}
' '.join("hello world".split()[::-1]
\end{verbatim}
也就是看到\verb=[::-1]=,就是把所有陣列反轉,如果是字串那就是字串變陣列反轉。
實在比c還方便。
\\\\
list literal中還有使用 list comprehension,也就是從一個list中 iterate 每個元素
,然後此元素可以經由運算得到新list的新元素。就像是perl 的map, 或python本身的map
一樣。
\begin{verbatim}
[ express for e in a_list ]
\end{verbatim}
像perl可以光用一個split就將一個string轉成想要的hash,python可以做類似的事情
\begin{verbatim}
strl = 'a:1,b:2,c:3'
dict(i.split(':') for i in strl.split(','))

d = { 'a':'1', 'b':'2', 'c':'3' }
",".join([i[0]+":"+i[1] for i in d.iteritems()])
\end{verbatim}
所以同樣從dictionary轉成一個string也是可以的
\\\\
這也可以看出string 只是 char 的list而已,所以對於substring, 或每個字元的
iteration,都可以用 list 的slice與for loop使用。
\\\\
另外tuple跟list是一般人比較容易混淆的,tuple跟list很像差別只有
\begin{itemize}
\item tuple的element是不能改變的,也就是原本只有3個元素,不能再加。
\item tuple沒有append, remove跟index這些操作method
\item tuple在內部處理是有事先sorting的,iterate處理會比list快。
\item tuple是不可變的,可以拿來作dictionary的key,list不行。
\end{itemize}

所以要作key/value時,只有tuple才行。
\begin{verbatim}
>>> a=1; b=2; c=(a,b); d={c:"testing"}
>>> a=1; b=2; c=[a,b]; d={c:"testing"}
Traceback (most recent call last):
  File "<stdin>", line 1, in <module>
  TypeError: unhashable type: 'list'
  >>> 
\end{verbatim}

一般script都有的整數與字串型別,但其實python中這些都是
物件,所以都是物件methods。也就是把所有變數都看成是C的pointer變數就對了。
但如果
\begin{verbatim}
n1 = n
s1 = s
list1 = list
dictionary1 = dictionary
tuple1 = tuple
set1 = set
\end{verbatim}
這時的 = 就有差別,對於數值與字串,他是創造新物件並且 copy原本內容,但其他型
態的就只是pointer指向而已。加上算符又不一樣
\begin{verbatim}
a = 10 + 10
b = 10 * 5
c = "1st string" + "2nd string"
d = "string" * 5
l = list1 + list + ['alibuda']
\end{verbatim}
這上面的 + * 都是物件的overloading方法。
運算中=是copy還是assign是非常重要的。在perl中由於一開始是沒有複雜資料結構,所
以一開始並沒有定義reference,然後有所謂的\verb=@var %var =這些型別,所以
使用這些型別時都是很單純的創造一個新物件然後copy內容,但是後來有了reference,
使用\verb=$var=,這時就跟原本的純量變數一樣用assign的了。在python中反而不是
這樣,python的=都是物件指向,只有literal上用上算符才是copy。
所以傳值進function,不想改變原本的物件內容又要操作他時,要小心先創造一個新物件
,然後再操作。所以perl是=還有算符都是固定定義,但他的資料型別跟pointer定義是
死的,這比較像傳統c,而python的=是動態變化的,會根據前後資料不同有不同定義。
對於list 與 dictionary要產生一個真的物件,而不只是新變數指向同一物件,要用
list() 與 dict()產生
\begin{verbatim}
a = [1,2,3,4]
b = list(a)

h0 = { 'key0' : 'value0', 'key1' : 'value1'}
h1 = dict(h0)
\end{verbatim}
沒有constant
因為直接定義就好了,何必多個保留字constant,但coding style還是遵循c的方法。
\\\\
字串concantation,不像shell, perl的處理反而比較像傳統C++的處理
\begin{itemize}
  \item +算符的overloading newstr = str1 + str2
  \item 不同型別的串接在C中都是用sprintf,在這邊有兩種方法
    \begin{itemize}
      \item 舊的形式比較像sprintf newstr = \verb='%s %d\n' % ('this is ', 1)=
      \item 新的物件形式str.format() newstr = \verb='{} {}'.format('this is', 1)=
      \item \verb='{:b}'.format(27)=表示27的binary表示法。或者用bin(27), 
        hex(27)轉換。
    \end{itemize}
\end{itemize}
可以看出來 \verb=''或者""=他已經是個string object,後面可以直接帶上method的。

\section{物件與class}
一個程式是不是物件導向最重要的是要作到
\begin{description}
  \item [封裝] \hfill \\
    abstraction, 良好的資料結構定義,這就是像一般c的struct宣告。
  \item [繼承] \hfill \\
    inheritance, 有父子關係的資料結構,c就沒有辦法作很好的繼承。這是變
       數的生命期與scope有更複雜的變化定義。在python中使用
       \begin{verbatim}
       class Son(Parent):
          def __init__(self, *args):
              xxxxx
              Parent.__init__(self, arg0, arg1)
       \end{verbatim}
       來定義一個class Son繼承自Parent,其中\_\_init\_\_=是初始化 function,
       其中Parent.\_\_init\_\_(self)為呼叫爸爸的init的方法。python 有新的 class
       叫 object,所有 class 只要宣告時是class xxx(object):這樣的話,就是新型態
       class,這樣的子 class 呼叫爸爸的 init 的方法改用 super([type, [object]]),
       他是回傳一個相對後面 object 的 type 的爸爸 class,所以呼叫要用
       super(Son, self).\_\_init\_\_(arg0,arg1)。 python 並不會自動對應參數
       並呼叫父 class 的 init ,必須自己額外用上述方法,使用 Parent 或 super 
       呼叫都可以。 python 3.x 後用 super 的方法不用那麼囉唆了,用
       super().\_\_init\_\_(arg0,arg1) 就可以。
       \\\\
       python 的繼承可以多個 class, class Son(Parent0, Parent1):
       搜尋相關 variable,methods 的順序是從左到右依次往上。
  \item [多型] \hfill \\
    polymorphism, 有幾種如 overriding, overloading 與 subtype(dynamic binging)
    \begin{itemize}
      \item 當子 class 跟父 class 的同名函式一模一樣時,稱為overridding,整個蓋掉。
      \item 當子 class 跟父 class 的函式同名但傳的參數不同時,稱為overloading,此
            時根據傳進來的參數型別或者個數自動使用相對應的呼叫。有名的如
            compare(),傳進去是 integer 跟 string,會自動用相對應的比較方法。
            這也叫 overloading。python 也沒有 overloading,只可以用判斷傳進的參數
            型態來作不同執行。但也就是說一個 class 下只能有一個函數名稱。
      \item subtype是需要有繼承的 class 才行的,也就是當呼叫時某函式時,傳進去
        雖是同一種父型別,這跟 overridding 有點像,c++ 的 virtual function 跟
        polymorphism 是不提供原始碼給人編譯時,只提供更高階 abstraction 的 class
        參數的API,就能自動呼叫到該有的同名函式呼叫,這在c++內為 dynamic binding
        。binding 是說一個 method 連結上一個 object 稱為 binding,在執行期才決定
        object 的 type 稱為 dynaminc binding,請看例子:
        \begin{verbatim}
class Animal:
    def __init__(self, name): # Constructor of the class
        self.name = name
    def talk(self):           # Abstract method, defined convention only
        raise NotImplementedError("Subclass must implement this method")
 
class Cat(Animal):
    def talk(self):
        return 'Meow!'
 
class Dog(Animal):
    def talk(self):
        return 'Woof! Woof!'
 
def talk(animal=Animal('myanimal')):
    animal.talk()

animals = [Cat('Missy'), Dog('Lassie')]
 
for animal in animals:
    print talk(animal)
        \end{verbatim}
    \end{itemize}
    不過執行上面程式是錯的結果,主要是 python 不是那麼強型態的語言,所以不支
    援 subtype,只要有 overriding 就夠了,這要在更強型態的 c++ 上才看得出來他
    的用途。
\end{description}
  因此 python 只有 overridding 跟 使用 *args **kargs 方法的 overloading,無法
  在同一 class 下定義兩個同名的 methods。
  如果沒有用上這些結構與方法,只是在那邊呼叫來呼叫去的程式,即使號稱使用了c++,
  python, ruby, java, blabla,其實一點都不是物件導向的程式,物件導向的程式適
  合複雜的資料結構的程式,像GUI或有關人的管理程式。但如果是平常的小作業,用上
  物件導向,那就是殺雞用牛刀。甚至像OS kernel這麼複雜的程式都只用C而已。不過話
  說回來很多語言到最後都是在語句的表達上作到像人表達的語法,在語句的表達上盡量
  對所有型別一致性,例如簡單的a + b這樣的運算可以通用所有的型別,不用像c一樣,
  兩個字串相加要spriintf()。不過有時候想想為漂亮的語句,光在那邊搞這些資料型別
  就昏倒了。
  \\\\
  嚴格說起來,python是沒有所謂整數字串型別的,python中,所
  有東西都是物件,連10, 0.5等等全是物件。只是他們的算符有特別overloading
  處理。所以在tuple(1,2,3) 中,無法改變t[0]的數值,但tuple ([1,1],[2,2],[3,3])
  中, 卻可以改變t[0][0]的數值。

  \subsection{reference}
  物件最重要的就是參照reference,就是在一個時空中,這物件能夠被使用者參考到,
  所以必須像身份證號碼一樣,本機實作上通常就是記憶體位址。在網路元件架構像
  CORBA, SOP,他必須是一個在這時空中唯一的一個號碼id。
  這有兩個語言上的標準問題
  \begin{itemize}
    \item 空物件 在C++中的空物件 就可以用NULL來表示。在python中以None表示。
    \item pass by value與pass by reference,這在前面說過了。
  \end{itemize}

  \subsection{constructor與initializer}
  python 的物件建立是用\verb=__new__(cls), __new__(cls)=將會建立一個 cls 的 instance
  ,如果還回一個 cls 的 instance,則繼續呼叫\verb=__init__(self)=,如果沒有還回一個
  cls 的 instance,\verb=__init__=將不會被呼叫。 特別需求在於建立一個父 class 的
  instance,但使用自己的 initializer , super(currentclass, cls).\verb=__new__(cls[, ...])=
  所以一般程式基本上用\verb=__init__= 就可以了。

  \subsection{overloading}
  function overloading 也是物件導向的重要特色,也就是我們希望物件的運算能夠像
  傳統資料用一些基本算符就能表達,物件導向有很多其實是在於語句的表達上,能
  夠簡單的讀懂與一致性。所以算符與 function 的 overloading 就必須給程式師更大方便。
  \begin{itemize}
    \item \verb=__add__= 取代 + 的算符
    \item \verb=__sub__= 取代 - 的算符
    \item \verb=__mul__= 取代 * 的算符
    \item \verb=__truediv__= 取代 / 的算符
    \item \verb=__eq__= 取代 = 的算符
    \item \verb=__str__= 當用print obj時,會去叫用 \verb=__str__= 這個function
    \item \verb=__repr__= 當用repr()時,會去叫用 \verb=__repr__= 這個function,
                        也就是光 print obj 時,會印出 \verb=__repr__= 結果,如果沒有
                        \verb=__str__= 則會用 \verb=__repr__= 代替。
  \end{itemize}
  但 python 的 overloading 是爸爸跟小孩的class 間而已,同一 class 內的 function
  是沒有 overloading 的,這很麻煩。

  \subsection{class method與變數namespace}
  每一個 class 都有一個\verb=__dict__= list,裡面就是 class 的變數,
  在宣告時,其實在 class 這行後馬上跟著的變數宣告就是 class 變數。
  當用 SomeClass.var1時,其實是SomeClass.\verb=__dict__.var1=,
  而一個 instance 的變數其實會先去找 instance 上的,找不到最後才會去
  找 Class 裡面的。 instance 上也有\verb=__dict__=,就是我們在
  \verb=__init__=裡面用的self.x時,這是x是在instance的\verb=__dict__=。
  member function 跟 member data 是無法區別的,也就是定義a.x跟a.x()是沒有意義的,
  會只有a.x。在同一 namespace 中,變數名跟函數名不能相同。
  另外還有個\verb=__all__=是另一個 list,只有在這裡面的才會 export 出去,這有點
  算 perl 的 Export 或者C的 extern。如果沒有\verb=__all__=,則內定是所有前面
  不帶\verb=_=的變數在使用 import 時會被帶進來。

  \subsection{accessors}
  物件導向的程式有一個特點是 private/public member 的 get 與 set,這
  會影響使用物件的外圍程式使用物件內部資料,傳統物件導向程式是盡量
  不允許直接取值而使用methods來取用,這有兩個好處
  \begin{enumerate}
  \item 保護內部資料不被任意取得,有的時候有些程式員,為了交出東西來,
    會直接進class裡面亂拿變數就來用,但會破壞原本程式的架構,前面還好,
    但隨著程式維護,結果後面會越改越亂,整個就亂掉了。
  \item 取用或傳值時,能夠對資料作進一步的檢查與錯誤處理,其實我們在寫程式,
    幾乎都在作這些使用者的錯誤處理,真正完成事情的code就那麼一行,但前
    面一大段都在錯誤處理檢查與return。當取值只能經由get/set這樣的呼叫時,
    get/set進來的參數可以作進一步的錯誤處理。
  \end{enumerate}
  accessors就是這些get/set成員資料的methods。
  使用物件導向的方法會有 function call 的 overhead,在電腦內部運算中
  使用 function call,其實要多好多步驟才能完成,跟只是單一取值的
  速度是有差別的,也就是\verb=obj.get_x()=跟obj.x取值是有 performance 差別的。
  但現在電腦很快,所以為了程式可讀性與維護性,這是需要作取捨決定的。

    \subsubsection{accessors建立}
    在perl中可以用特殊型別 typeglobs,來作到快速建立 accessors,並且 member function
    跟 member data 可以同名,但在 python 中卻無法作到 member data 跟 member function 同名,
    所以假設有個 member data, a,那就必須實作get\_a / set\_a / del\_a。

    \subsubsection{property函數}
    在已有的程式碼中如果已經使用了直接取用 member data,那怎麼辦呢,全部重寫嗎,
    或者有人就是這麼不長眼,或者 member data 有跟 member function 一樣的名字怎麼辦。
    我們只需改寫 class 並且在 class 最下面用 property 這個函數。
    x = property(fget, fset, fdel, doc) ,其中
    fget...fdel 就是存取 x 時get / set / del 對應的函數

\begin{verbatim}
class myclass():
    def __init__(self, a=1, b=2, c=3):
        self._a = a
        self._b = b
        self._c = c
    def geta(self):
        print 'I am a'
        return self._a
    def seta(self, value):
        print 'I am setting a'
        self._a = value
    def dela(self):
        print 'I am deleting a'
        del self._a
    
    a = property(geta, seta, dela, "This is for a's accessors")

if __name__ == '__main__':
    mo = myclass()
    print mo.a
\end{verbatim}
則 mo.a 的值實際上是由 geta 傳回來的,不再是直接拿 a,而 mo.a = xx 也變成是
呼叫 seta()。


    \subsubsection{descriptor class}
    當class有\verb=__get__ __set__ __del__=三函數時,此 class 稱為 descriptor class
    ,當 instance 有定義像\verb=self.__get__/__set__/__del__=時,那 instance 上帶有
    descriptor。
\begin{verbatim}
class my_descriptor():
    def __get__(self):
        return self.a
    def __set__(self, value):
        self.a = value
    def __del__(self):
        del self.a
class
    def __init__(self):
        a = my_descriptor()
\end{verbatim}
    descriptor的binding為當取用a.x時或者A.x,真正binding到的執行是
    \begin{itemize}
      \item direct \verb=x.__get__(a)=
      \item instance \verb=type(a).__dict__['x'].__get__(a, type(a))=
      \item class \verb=A.__dict__['x'].__get__(None, A)=
      \item super \verb=super().=
    \end{itemize}
    當用上 property 函數時,這個 class 其實就是一個 descriptor class,
    fget / fset / fdel 實際上是相對應於 \verb=__get__= / \verb= __set__= /
    \verb= __del__=,基本上在 python 裡面是不寫傳統的 get / set 這種東西,而是用
    property decorator, 而取值時就用原本的 obj.v 這樣 C 的方法就好,在語意上會
    簡潔多而把後面的事情交給 python 處理。
    \\\\
    總結就是當我們用 obj.a 時,他其實尋找 a 這個 member data 有很多順序的,
\begin{verbatim}
如果有__getattribute__(),就永遠先傳回這個,這恐怖,最好不要用
-> instance的__dict__ 如果有instance的__get__ 就再呼叫__get__傳回
-> class的 __dict__ 如果有class的__get__ 就再呼叫__get__傳回
-> 呼叫__getattr__() =
\end{verbatim}
    所以其實可以不必定義self.x,而光定義\verb=__getattr__=也可以。

    \subsubsection{不使用\_\_dict\_\_改用\_\_slots\_\_}
    在物件產生時,\verb=__dict__=同時也會被建立,如果建立大量物件但是有的物件卻
    只有1,2個變數需要設定,其他變數都沒用到的話,則用\verb=__dict__=來存變數名稱
    與值是很浪費記憶體的,這時可以用\verb=__slots__=這個 list,裡面是 member data
    名字,例如 obj.a, obj.b 則\verb|__slots__ = ['a','b'] |,這樣a,b在沒有設值前,
    都不會在記憶體中存在,存取a,b也會失敗。他是個 class 變數,宣告了他,這個 class
    就不會自動建立\verb=__dict__=,不過\verb=__dict__=是vars(o)傳回的值,要小心。

  \subsection{iteration與iterator物件}
  python 的特性就是物件,還有他提供很多跟資料 iteration 的語法工具。這是很特別的,
  由於在程式中,其實我們大部分都在處理這樣的東西,所以這是很 powerful 的。
  一個 iterator 的 class 必須要
  \begin{itemize}
    \item 實作next()與\verb=__iter__()=這兩個methods
    \item \verb=__iter__()=必須 return 一個 iterator object
    \item next 必須 return 下一個物件,結束必須 raise StopIteration 這個exception.
    \item 函數中若使用 yield,則此函數自動為 iterator generator。
  \end{itemize}

  \subsection{decorator}
  對於已經存在的function或class加工,稱為decorator,以@wrapper使用
  \\\\
  例如func1已經存在原本的code base中很久了,不想動他,而新的func2,func3
  都可以用到func1的功能則
\begin{verbatim}
def func3(arg):
    xxx
    return callable_function
def func2(arg):
    xxx
    return callable_function

@func3(arg)
@func2
def func1
    xxx
\end{verbatim}
  其中func3跟func2必須return一個callable物件,也就是function pointer,通常可以
  用lambda來作。則當呼叫func1時,會變成相當於多層的function呼叫,
\begin{verbatim}
func1 = func3(arg)(func2(func1))
\end{verbatim}
  所以我們常常看到人家在class裡面這麼寫
\begin{verbatim}
@property
def myfunc(self):
    self.xxx
\end{verbatim}
  這等於是把之前講的property函數用來作decorator,因為property的第一參數是get,
  所以這樣會把myfunc變成一個read only的member data。obj.myfunc會是由這個func
  return的值。
\\\\
  還有常見到
\begin{verbatim}
@classmethod
def myfunc(cls, arg):
    xxxx
\end{verbatim}
  這是說myfunc呼叫時,會傳進cls而不是self object,由於python的同 class中沒有
  overloading,這通常拿來作不同init的方法,只是不同函數名。最後要還回一個
  init instance
\begin{verbatim}
class Date(object):
    day = 0
    month = 0
    year = 0

    def __init__(self, day=0, month=0, year=0):
        self.day = day
        self.month = month
        self.year = year
    @classmethod
    def from_string(cls, date_as_string):
        day, month, year = map(int, string_date.split('-'))
        date1 = cls(day, month, year)
        return date1

date1 = Date('11', '09, '2012')
date2 = Date.from_string('11-09-2012')
\end{verbatim}
  既然class名稱是可以呼叫的,同樣decorator也能在class這麼運作,
\begin{verbatim}
@func
class class0():
\end{verbatim}
  表示
\begin{verbatim}
class0 = func(class0)
\end{verbatim}
  但這個func必須這麼定義
\begin{verbatim}
def func(cls)
    class class1():
        xxxx
        cls()
    return class1
\end{verbatim}
  他自動會pass一個class object, class0給這func,可以在裡面呼叫cls(),最後必須
  return另一個class。他不像classmethod還回一個instance,而是一個class物件。
  \\\\
  另外還有個@staticmethod,這就是把某個函數視為像c一般的static function,scope
  在同個檔案內的global function,他這個function的第一個參數,不像一般member
  function是self instance,而是實在定義的參數。不過我不太清楚完全的global
  function跟一個 class 內故意用@staticmethod 呼叫有什麼特別的差異。而且覺得好像
  脫褲子放屁,在import上的naming變得多此一舉,但可能在歸納整理上有好處吧。

  \subsection{with statment與context manager}
  with statement搭配函式的使用在於簡化一些函式的必要處理,最常見的例子是開檔
  ,有開檔就要有關檔,如果程式小,一下做完了那就沒什麼注意,但在長迴圈時,如
  果沒關檔,一直開檔就會把file descriptor開爆了,有很多類似的情形,
  在python內定的open中,就有實作這種處理,所以我們可以直接用 with open as f
  \begin{verbatim}
  with open("x.txt") as f:
      data = f.read()
      do something with data
  \end{verbatim}
  這樣即使使用者忘了關檔,或即使with下面的level很長很亂,只要一出with敘述,就自動
  會關檔。有兩種作法,一種為context manager class,一種為context manager generator。
  \\\\
  在class中,一個有實作
  \begin{itemize}
    \item \_\_enter\_\_ 進來函式做的事
    \item \_\_exit\_\_ 出去函式做的事
  \end{itemize}
  的class為context manager class,class可以被呼叫,\verb=__enter__ __exit__=
  為相對應的呼叫,例如lock/unlock, save/restore, open/close可以放在enter / exit
  中。例如
  \begin{verbatim}
class MyException(Exception):
    pass

class Sample:
def __enter__(self):
    return self
def __exit__(self, type, value, trace):
    print type
    print value
    print trace

def raise_exception(self):
    raise MyException, "Something Wrong"

with Sample as s:
    s.raise_exception()
  \end{verbatim}
  會看到如果member function出問題了,會自動跑去叫\verb=__exit__=,通常我們會在
  裏面放cleanup的code。也就是整段class為一個context,enter為init code, exit
  為cleanup code,無論context怎麼死的,都會執行cleanup。
  \\\\
  但以open為例子,我們只要function 不要class,至少with的時候我們想要小寫的
  呼叫,我們可以用python標準庫裏面的 contextlib 並用yield做 iteration generator。
  \begin{verbatim}
from contextlib import contextmanager
@contextmanager
def tag(name):
    print "<%s>" % name
    yield
    print "</%s>" % name

>>> with tag("h1"):
...    print "foo"
...
<h1>
foo
</h1>
  \end{verbatim}
  所以用contextmanager作為 decorator ,則yield前放\verb=__enter__=的code,
  之後放\verb=__exit__=的code,變得很簡單。

  \subsection{class的內定變數與相對應的內定函數}
  在 python 中有很多有用的內定函式對物件操作,例如 isinstance(), type()...
  這些其實都可以改變內部變數與一些函式,像之前講的 +-/*= 外作到 overloading 的。
  了解這些特殊的變數與函數也算是了解 python 對 class 的細部操作為何,對於
  進階的程式控制有更進一步的幫助。type() len() print()...相對應的內部實作
  +-/* ==
  \\\\
  變數名
\begin{verbatim}
__name__    : 物件名字,可能是function, class, module名
__module__  : 模組名
__dict__    : 所有class參考的symbol與值
__slots__   : 這個list阻止class自動建立__dict__,並且沒有設值之前不能取用
__file__    : 檔案名
__doc__     : 檔案內可被pydoc解讀的document文字
__bases__   : 所有super class的一個tuple
__class__   : 物件所屬class名稱
__all__     : 所有exoprt出去的symbol
\end{verbatim}
  特殊 method
\begin{verbatim}
__new__     : 物件的建造
__init__    : 物件的初始
__del__     : 物件的刪除
__repr__    : 物件的表示
__str__     : print(o) 物件時的表示
__call__    : 有__call__(),則 instance i 可以用i()傳入引數,呼叫__call__()
\end{verbatim}
  Accessors method
\begin{verbatim}
__getattr__ : o.member 的內部實作的最後呼叫
__setattr__ : 跟前面一樣意思
__delattr__ : 跟前面一樣意思
\end{verbatim}
  property descriptor
\begin{verbatim}
__get__     : 用在property descriptor
__set__     : 用在property descriptor
__delete__  : 用在property descriptor
\end{verbatim}
  context manager
\begin{verbatim}
__enter__   : 使用 with 時,一進來執行的code
__exit__    : 使用 with 時,不管如何出去執行的cleanup
\end{verbatim}
  其他
\begin{verbatim}
__len__     : 物件的長度,用len(o)時,呼叫這個函式。
__iter__    : 實作 class 為一個可以有 iterator物件
__getitem__ : 跟getattr()很像,但主要是可以用o['member']取回值。
__setitem__ : 一樣可以用 hash 設值
__delitem__ : 跟前面兩個一樣意思
\end{verbatim}

\section{I/O與系統溝通}
  \subsection{檔案系統 File I/O}
  file I/O 有以下
  \begin{itemize}
    \item open 類似fopen的模組,os.open則是類似system call,open
    \item os.path  處理有關path,例如abspath, exists, islink等等
    \item tempfile 處理temp檔用的,等於 c 裡面mktemp或tmpfile之類的呼叫
    \item shutil 提供像shell的檔案命令類似copy/move/delete的API。
  \end{itemize}

  一般的開檔存取使用open傳回的是file object
\begin{verbatim}
f = open('myfile', "w+")
f.read()
f.write("a line")
f.readline()
sys.stdout.write("haha")
\end{verbatim}
  而像一般system call的傳回file descriptor的是用os.open,這傳回fd
  ,並且os.write/os.read就是system call的read/write fd來操作。
  檔案的檢查,不像一般script的-e -f -x用法而必須用
  \begin{itemize}
    \item os.access()
    \item os.path.isfile() os.path.exists()
  \end{itemize}
  這其實就是一般程式的呼叫了,已經不像script了。

  \subsection{subprocess/thread}
  在與其他命令的合作上,python 又一次比較偏向c的作法,他並沒有像 backquote 這
  樣簡單的與系統溝通的方法,而必須用 fork/exec/popen 這種傳統的 system call 
  的觀念來跟其他命令溝通,很不方便。也就是如果要得到某些 infromation,例如 ip,
  檔案名等等,請不要用現有的命令,而盡量用c語言的作法,直接去 /proc, /sys 
  或者 system call 或者 ioctl 的方法拿,這是比較傳統的方法。當然要不就是有人
  寫好模組 API,去搜尋與使用。
  \\\\
  但在2.7以後也有一些新 function,但很不方便,因為2.x的API變化
  我覺得有點大,實在不是很好。有些系統還在2.4, 2.6下就不能用。
  \\\\
  os模組
\begin{itemize}
  \item os.execl
  \item os.spawn
  \item os.fork
  \item os.popen
  \item os.system 這是跟c的system()一樣的,return就是exit status,
\end{itemize}
  subprocess模組
\begin{itemize}
  \item subprocess.call 這跟os.system用法一樣。
  \item subprocess.check\_output("ls") 等於backquote,但這2.7以後才有。
\end{itemize}
  subprocess.Popen模組
\begin{itemize}
  \item supbrocess.Popen([cmd, arg1, arg2], shell=True)
  \item subprocess.Popen("/bin/cmd arg1 arg2") 這必須用絕對路徑
  \item subprocess.Popen.communicate()回傳tuple,(stdout資料,stderr資料)
\end{itemize}
  不過他這有點 confused 的就是 stdout 是 sys.stdout 呢還是 
  subprocess.Popen.stdout。
  \\\\
  shell與perl裡面的backquote
  \begin{verbatim}
import subprocess
p = subprocess.Popen(["ls", "-l"], stdout=subprocess.PIPE)
output = p.communicate()[0]

subprocess.call("ls -l", shell=True)
  \end{verbatim}
  2.7以後可以用
  \begin{verbatim}
output = subprocess.check_output("ls")
  \end{verbatim}
  或者可以用pexpect的run
  \begin{verbatim}
import pexpect
output = pexpect.run("/bin/ls")
  \end{verbatim}
  但其實這都不是正常我們想要的,因為那都只是類似exec的作法,命令後面無法隨心所欲
  加上自己想要的參數,或者像一般使用的命令,真正我們想要的是像面對terminal般執行
  shell命令,必須在一個 shell 環境下執行的,尤其shell的多命令 pipe 例如 
  \verb-output=`dmesg | grep hda`-,這就要在以上的 os.system, pexpect.run 
  等等使用上加上option來先產生shell環境再來執行命令
  \begin{verbatim}
import os
status = os.system("shell_cmd | sed '3' | awk '{print $2}'")

import subprocess
output = subprocess.check_output("shell_cmd", shell=True)

import pexpect
s = pexpect.spawn("/bin/bash --norc", timeout=10)
s.sendline("export PATH=$PATH:$HOME/bin")
  \end{verbatim}
  check\_output必須要有shell=True的參數,才會產生shell,pexpect要先spawn一個sh
  出來,然後帶著這個fork出的物件執行shell命令,只是pexpect的output會在expect物
  件內。
  \\\\
  使用svr4 system(), os.system, shell=True這些東西都會被警告說如果你帶入的字串
  沒有被控制的話,有可能會成為buffer overflow的漏洞,別人可以帶入一長串亂七八糟
  的字串,讓你的shell執行他的code,所以除非確定所有的輸入都在自己掌握中才使用。
  \\\\
  基本上python對於資訊的取得,不太用系統命令,要得到一些資料盡量直接用API跟系統
  拿,而不要用其他命令去拿,這在跨平台寫作也是有好處的,不會綁在某一平台命令上。
  所以如果要靠sed/awk/sort/uniq/find/...這些shell命令一堆pipe得到結果的方法,也
  都不要再用了。但像 dmidecode 這種東西,如果要很痛苦的去 /dev/mem 拿,這時就不知
  道怎麼取捨了。不然就是得等看看是否有人做了 package 讓我們用。

  \subsubsection{process與thread}
    有兩種作法,
    除了用os模組下的fork, exec等傳統system call的方式,但python提供比這種更high
    level點的multiprocessing模組,這跟thread的使用非常像,必須要指定一個function
    來讓一個Process物件啟動。而第2的參數是argument,必須是一個tuple,所以如果args
    只有一個參數必須用(arg1,)表示一個tuple。
    \begin{verbatim}
from multiprocessing import Process
def myfunc(i, v):
    v[0] -= 1
    print i, v[0]

if __name__ == '__main__'
    for i in range(10):
        v = [10]
        p = Process(target=myfunc, args=(i, v))
        p.start()
    \end{verbatim}
    除了Process外,還有一個threading內的Thread物件,用法兩者可以說是一樣的
    \begin{verbatim}
from threading import Thread
def myfunc(i, v):
    v[0] -= 1
    print i, v[0]

    for i in range(10):
        v = [10]
        t = Thread(target=myfunc, args=(i, v))
        t.start()
    \end{verbatim}
    兩者的差別就是傳進去的參數,在不同的process 中當然是各自獨立的,在thread
    間就是global variable,這正常來說要有lock機制來存取的。另外一點很重要的
    是python的文件中說明,python的thread實作是沒有,
    所以如果為了performance考量,python是建議用multiprocessing而不是用
    threading的。這點是跟c很不一樣的。

  \subsection{IPC}
  另外最重要常用到的就是IPC,傳統上UNIX系統上的IPC有pipe, fifo, socket, mmap,
  svr4 ipc有message queue, shared memory, semaphore,這些在python中的使用方法
  可以更簡便,另外thread也是後來常用到的多工手段,python在兩者提供的API讓使用
  者感到很一致,是很不錯的抽象實作。

  \subsubsection{signal}
  signal是最簡單的unix系統提供的軟體中斷,可以由系統或另一process發出signal給
  一個process,可以自己做signal的處理function並註冊之。只是這個handler必須要
  有兩個參數,一個是signum, 一個是function call frame
  \begin{verbatim}
# 2 args are required
def my_handler(signum, frame):
    print "\nEntering my territory"

signal.signal(signal.SIGINT, my_handler);
time.sleep(3)
  \end{verbatim}
  在三秒內趕快按Ctrl-C。

  \subsubsection{mmap}
  mmap本來只是檔案I/O的performance增進的方法,但python利用這種需要kernel參與
  的記憶體空間來做為不同process共享memory的方法之一,這很像system的shared memory
  的想法,但mmap主要是兩個有關的爸爸小孩間可以溝通,兩個無關的process就要用
  shared memory才行。
  \begin{verbatim}
import mmap
import os

# mmap used for ipc like shared memory
mm = mmap.mmap(-1, 13)
mm.write("Hello world!")

pid = os.fork()

if pid == 0: # In a child process
    mm.seek(0)
    print mm.readline()
    exit(0)
mm.close()

# real mmap usage as svr4
with open("hello.txt", "r+b") as f:
    # memory-map the file, size 0 means whole file
    mm = mmap.mmap(f.fileno(), 0)
    # now mm is a string array to be operated
    mm.readline()
    mm[5:] = "Hello\n"
  \end{verbatim}
    \subsubsection{pipe與fifo/queue}
    傳統上兩個process間的通訊可以用pipe與fifo來實作,這
    \begin{itemize}
      \item pipe是同一個process所fork的相關的process共用,通常是爸爸跟小孩兩個,
        一個寫一個讀。
      \item fifo相對於一個檔名,可以給很多process 只要open這個fifo檔就能一起用
    \end{itemize}
    \begin{verbatim}
from multiprocessing import Pipe, Queue, Process
import time
import random

def msgp(child):
    child.send([1,2,3])
    child.close()

def msgq(queue, i):
    time.sleep(random.randrange(0,3))
    print 'child input %s' % i
    queue.put(i)

if __name__ == '__main__':
    parent_conn, child_conn = Pipe()
    p = Process(target=msgp, args=(child_conn,))
    p.start()
    print parent_conn.recv()
    p.join()

    q = Queue(10)
    for num in range(10):
        p = Process(target=msgq, args=(q, num))
        p.start()
    time.sleep(20) # wait for all process return
    for num in range(10):
        print 'get all queue after children input'
        print q.get()
    \end{verbatim}
    即使在系統呼叫的open/read/write, pipe/fifo本身的sync問題是由kernel解決掉,
    在python 裏面也是thread safe,不需要sync的機制去存取這通道。pipe/fifo在c
    的系統中有大小限制的問題,目前是65535,也就是說如果一直寫,超過65535,則
    沒有人去讀他,再繼續write,就會 block。

    \subsubsection{synchronization}
    lock與semaphore是最基本的sync工具,大家要存取一個共同的記憶體時,一定要先
    取得存取權力,就是lock,或者這個共同的資源存取是有容量的,例如買票時,售票員
    只有一個,那就是lock,有兩個以上就是semaphore。有些更細微的通常會說,lock
    是不能等待的spin lock, binary semaphore是可有wait queue的。但最基本的用法
    是binary semaphoe是lock。在process中的共同記憶體通常就是shared memory,
    在thread間的通常就是global變數。先以threading這個模組示範
    \begin{verbatim}
from threading import Thread, Lock, Semaphore
import random
import time

common = 0
lock = Lock()

sema = Semaphore(3)

def inc_common(item):
	global common
	global lock
	time.sleep(random.randrange(3))
	lock.acquire()
	common = common + 1
	lock.release()
	print "thread id : %d common : %d\n" % (item, common)

def inc_resource(item):
	global resource
	global sema
	sema.acquire() # P()
	print "I am %d" % item
	time.sleep(10)
	print "increase the resource after 10 seconds"
	sema.release() # V()
	
if __name__ == '__main__':
	print "lock test\n"
	for i in range(10):
		Thread(target=inc_common, args=(i,)).start()
	time.sleep(10)
	common = 0
	print "semaphore test\n"
	for i in range(10):
		Thread(target=inc_resource, args=(i,)).start()

    \end{verbatim}
    semaphore的使用,有很多在限制的資源使用,例如資料庫的連線最多只有5個process,
    那就可以設定一個Semaphore(5),每次要使用資料庫連線時,都要先取得semaphore,
    這樣就可以把process資源的限制交給semaphore來幫忙就可以了。
    \\\\
    shared memory 由Array與Value兩個API完成。也就是跟mmap很像,所有kernel的記憶
    體的對應,在python中都由array, list來負責。這個就multi process示範了。
    \begin{verbatim}
import random
import time
from multiprocessing import Array, Value, Process

def smem(l, v, a):
    l.acquire()
    # multiprocess and the parameter are shmeme so get changed
    print "before shared v : %d" % v.value
    v.value = random.randrange(1, 10)
    print "after shared v : %d" % v.value
    time.sleep(1)
    l.release()

if __name__ == '__main__':
    smeml = Lock()
    smemv = Value('i', 0)
    smema = Array('i', [1, 2, 3])
    for num in range(10):
        p = Process(target=smem, args=(smeml, smemv, smema))
        p.start()
    \end{verbatim}
    這邊很重要的Array與Value的第一個參數是typecode,'i'表示signed integer,詳細
    typecode在python文件中有列出。另外value的值取出也很特別的使用v.value。類似
    message queue svr4或posix的message queue,其實可以用這來實作。
    \\\\
    condition variable是說當你獲得某全域變數的使用權時,卻發現要操作的條件不足,
    則此時必須把之前獲得的lock丟出,並且去等待,這裡有兩件事要同時完成,也就是
    丟出lock與sleep,所以要靠底層的幫忙才能做到atomic的操作,例如你要去考駕照,
    所以先到監理所排隊,好不容易排到你了(就是拿到服務人員的lock),卻發現你不滿
    18歲,所以讓出lock,乖乖必須去等到18歲(wait),等到18歲了,再通知(wake)你。
    通常標準作法就是
    \begin{verbatim}
cv = Condition()

def get_lock_and_do_something():
    cv.acquire()
    while condition_matched
        cv.wait()
    operate_on_global_resource():
    cv.release()

# 另一條process/thread要來打醒
def wake_up_all_sleeping():
    cv.acquire()
    try_to_make_condition_ok()
    cv.notify()
    cv.release()
    \end{verbatim}
    while loop的條件判斷是一定要的。也就是醒來後,還要再去檢查一遍條件是否又不
    行,如果在被打醒到檢查中間,條件又變不行,還要再進queue睡覺,等待下一次的wake。
    \begin{verbatim}
from multiprocessing import Process, Array, Value, Condition
import time

age = Array('i', [15, 19, 11, 17])
driver_license = Value('i', 0)
cv = Condition()

def go_to_dmv(id):
    global cv
    global age
    global driver_license
    cv.acquire()
    print "my id is %d and my age is %d" % (id, age[id])
    while age[id] < 18:
        print "my id is %d and I am %d not over 18, wait" % (id, age[id])
        cv.wait()
    driver_license = 'yes'
    print "I am #%d and %d years old so the driver license set to yes" % (id, age[id])
    cv.release()

def a_year_passed_by():
    global cv
    global age
    global driver_license
    inc_year = True
    while inc_year == True:
        cv.acquire()
        inc_year = False
        for i in range(len(age)):
              age[i] += 1
              if age[i] < 18:
                  inc_year = True
        cv.notify_all()	# it's random wakeup all threads
        cv.release()
        time.sleep(1)
        print "a year has passed"

if __name__ == "__main__":
    for i in range(len(age)):
        Process(target=go_to_dmv, args=(i,)).start()
    Process(target=a_year_passed_by).start()

    \end{verbatim}
    上面的共同資源是age這個list,所以有個看不見的lock會在cv產生時,也會產生,如果
    cv.acquire時的參數沒有給,就自動是這個看不見的lock在裏面。
    \\\\
    process pool是產生一堆固定數量的工人,裏面的工人會由python控制,當哪個工人
    有空時,python就會把一個job丟給他做,jobs的產生由map這個函式內丟一個list
    給他,每個element就是一個jobs,他很像原本內建的map函式,只是這裡用Pool物件時
    ,python會自己做平行處理,做完後所有的工作結果會放在list內傳回。
    \begin{verbatim}
import multiprocessing
def worker(x):
    return x*x

p = multiprocessing.Pool(processes=5)
result = p.map(worker, range(10))
print result
    \end{verbatim}
    以上可說是基本的python多工使用方法。

\section{string 與 regex}
字串在 python 來看就只是 char 的 array ,因此操作例如 length,就是原本
array 的操作即可。但 string 的 印出處理一直就是 script 的好處之一,pyhton
有兩個方法處理 format。一種類似 c 的 printf, 另外則是使用 python string
object methods。
\\\\
python的regular express,也是跟c的一樣,反而不像傳統 script 的語法
用 // 就好。regular express pattern 的表示法r""等於一般的 //,要用
\verb=\1\s\w= 這些 regular express 時要多加 \verb=r""=。類似的其他 string
object 像 bytestring 則要加 \verb=b""=,這在文件中要小心,他的 string 物件
有很多種。

\subsection{string format}
類似 c 的 print,但是用 \% 而不是逗號,後面要用 ( ) 做運算
\begin{verbatim}
print("%d" % (5 * 6))
\end{verbatim}
3.x 以後多了 string 物件 format(),以
\begin{verbatim}
'{}'.format(express)
\end{verbatim}
來建構單引號內新 string,主要是 \{\} 內的格式有很多新花樣,有三種形式
field\_name conversion 跟 format\_spec
\begin{verbatim}
{ [field_name] [! conversion] [: format_spec] }
\end{verbatim}

field name 一些例子,可以是 index, 也可以是 argument name
\begin{verbatim}
>>> '{2}, {1}, {0}'.format('a', 'b', 'c')
'c, b, a'

可以使用 argument name 指定

>>> 'Coordinates: {latitude}, {longitude}'.format(latitude='37.24N', longitude='-115.81W')
'Coordinates: 37.24N, -115.81W'
\end{verbatim}
format\_spec 基本
\begin{verbatim}
[[fill]align][width] [.precision][type]

fill  : 任意字元,可以 padding 在 align 前後的字元
align : > 向右對齊  < 向左對齊 ^ 中間對齊
width : 整個字串寬度
type  : 就是跟 C 一樣的 d, f, s, x ...

precision 就是數字 d, f, x ... 的精準度
\end{verbatim}
例子
\begin{verbatim}
不寫任何東西等於自動{:s}
>>> '{}'.format('string')
'string'

>>> '{:>30}'.format('right aligned')
'                 right aligned'

>>> progress=10
>>> '[{s:=>{n}}'.format(s='>]',n=progress)
'[========>]'

>>> '{:.1f}'.format(2.57 * 3.1415926)
'8.1'

用 # 會自動對 hex, octal, binary 前面補完
field_name + format_spec
>>> "int: {0:d};  hex: {0:#x};  oct: {0:#o};  bin: {0:#b}".format(42)
'int: 42;  hex: 0x2a;  oct: 0o52;  bin: 0b101010'

類似 date 命令的使用
>>> import datetime
>>> d = datetime.datetime(2010, 7, 4, 12, 15, 58)
>>> '{:%Y-%m-%d %H:%M:%S}'.format(d)
'2010-07-04 12:15:58'
\end{verbatim}

\subsection{比對}
\begin{verbatim}
import re
re.match("blablabla", string_to_search, re.IGNORECASE)
re.search(r"^2012-10-05.*", string_to_search)

>>> re.match("c", "abcdef")  # No match
>>> re.search("c", "abcdef") # Match
>>> re.findall("c", "abcdefcdef")
\end{verbatim}
match與search會只找到一個就return了,
match 跟search 的差別在於 match 是從頭開始,search是只要string裡面有substring
就算有match。可以copmile一個RE,來一直作比對,這基本上就是SVR4的regcomp
\begin{verbatim}
r = re.compile("^abc.*")
if r.search("abcdef"):
    print "got it"

regex = r"^(\d+)\.\(\d+)"
r = re.compile(regex)
\end{verbatim}
另外對於multiline string的處理,如果用上\verb=^ $=,則當加上re.MULTILINE這個
flag 給 search 時
\begin{verbatim}
re.search(r'^\w+', mstring, re.MULTILINE)
\end{verbatim}
會去跟每行都做一次\verb=^=的搜尋,但 match()還是只會 match string 的開頭。等於
search() 有MULTILINE 時,相當於很多string search,但match()沒有。
\\\\
search與match傳回來的是RE match object,所以還需要更進一步
的處理來抓出想要的部份。

\subsection{擷取}
\begin{verbatim}
m = re.search(r"(\w+) (\w+)", "first second")
m.group(0)
\end{verbatim}
是整個match到的字串所以是first second
\begin{verbatim}
m.group(1)
\end{verbatim}
回傳第一個match到的字串所以是first
\begin{verbatim}
m.group(1,2)
\end{verbatim}
回傳一個tuple,("first", "second")
\begin{verbatim}
m.groups()
\end{verbatim}
都回傳一個tuple,裡面的內容就是所有match到的,如果第二個
match不到東西,就是None.

\subsection{取代}
幾個function, sub,subn與translate
\begin{itemize}
  \item newstr = re.sub(\verb=r'\sAND\s'=, ' \& ', 'Baked Beans And Spam', flags=re.IGNORECASE)
  \item (newstr, n) = re.subn(\verb=r'(\w+) (\w+)'=, \verb=r'\2 blabla \1'=, 'Baked Beans And Spam')
  \item mystr.translate(string.maketrans('ab','AB'))
\end{itemize}
subn跟sub是一樣的,只是回傳時不是string而是一個tuple,第二個元素是代換的次數。
他這translate就相當於perl的tr///。

\subsection{其他處理}
\subsubsection{greedy regex的處理}
跟perl一樣,\verb=*+?= 都是 greedy 的 match,python也有 non-greedy 的用法,跟 perl
是一樣的用 \verb=.*? .+? .?? {m,n}? = 。
\begin{verbatim}
>>> import re
>>> s = '<html><head><title> TITLE </title></head></html>'
>>> re.match('<.*?>', s).group()
'<html>'
\end{verbatim}

\subsubsection{一些常用的regex}
\begin{description}
  \item [刪掉前後空白] \hfill \\
    這個用string.strip()就可以了
  \item [刪掉不是可印出ascii字元] \hfill \\
    \verb=re.sub(r'[\x00-\x1F\x7F-\xFF]+', '', line)=
  \item [最後一行] \hfill \\
    \verb|$last_line = re.search(r'([^\r?\n]+)$', mline).group(1)|
  \item [取得最後一個] \hfill \\
    \verb|$last = re.search(r'([^\s]+)$', s).group(1)|
  \item [把目錄加上斜線] \hfill \\
    \verb|$newdir = re.sub(r'(.*\w)$', r'\1/', dir)|
\end{description}

%\subsubsection{多行處理}

\section{setup.py的撰寫}
在程式最後是要 release 的,release 不是你 copy 給我,我 copy 給你就玩完了。python 的
relase基本的要寫 setup.py 讓別人能輕易的安裝。跟 perl 的 Makefile.PL 一樣有規矩的

\subsection{基本setup.py}
python 本身就有內定 distribution 的 package,可以根據這個內定來寫 setup.py
\begin{verbatim}
from distutils.core import setup
setup(
    name = 'mypkg',
    version = '1.0.0',
    description = 'Quark: Quality Running Kit',
    author = 'Cyril Huang',
    author_email = 'cyril@costco.net',
    url = 'http://www.python.org/sigs/distutils-sig/',
    scripts = [ 'src/bin/run.python' ],
    packages = ['quark', 'quark.device'],
    package_dir = {
        'quark' : 'src/lib/quark',
        'quark.device' : 'src/lib/quark/device'
        }
)
\end{verbatim}

\begin{itemize}
  \item 模組 使用packages與package\_dir來指定
  \item 可執行檔 使用scripts來指定
  \item 相依package 使用
\end{itemize}
其中packages是此目錄下所有的package,以上面例子來說,quark, quark.device
而且這表示這一定有\verb=__init__.py=在相對應的目錄下。他這distutils很笨
不會自動一直往下找,一定要特別每個都要寫在packages與\verb=package_dir=中
。scripts表示是一個可執行python script,最後會被裝在/usr/local/bin下。
命令如下
\begin{verbatim}
$ python setup.py --help-commands
$ python setup.py build
$ python setup.py install
$ python setup.py sdist
$ python setup.py bdist
$ python setup.py bdist_rpm
\end{verbatim}
會產生dist跟build這兩個目錄,最後的source跟binary package都在dist下。
install則會把所有該裝的裝到/usr/local/lib/python-x.x/下
sdist表示產生source distribution, bdist是binary distribution,
\verb=bdist_rpm=產生rpm。

\subsection{egg的產生}
在python安裝中有一種是easy\_install egg的binary package。
這個蛋蛋不是原本python基本內定的,必須額外裝python-setuptools這系統package,
裡面有setuptools這library。可能以後會進到python的裡面。跟內定setup.py很像的
也有setup()這個設定。最簡單的setup.py
\begin{verbatim}
#!/usr/bin/env python
#-*- coding:utf-8 -*-

from setuptools import setup

setup(
    name = ''
    version = ''
)
\end{verbatim}
然後用
\begin{verbatim}
$ python setup.py bdist_egg
\end{verbatim}
就會產生空的egg。7z l dist/UNKNOWN-0.0.0-py2.7.egg可以看出裡面的內容。
其實也就是多了\verb=bdist_egg=這個動作。
