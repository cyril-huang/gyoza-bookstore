\chapter{各語言語法大比較}
這個其實是我年紀大了有時記不住用的。
C, shell Makefile, 與perl, python 是我常用到的工具
  \section{變數}
  定義變數
  \begin{verbatim}
shell
var=

perl
$var =

C
int var =
char var =

python
var = 

javascript
const var =
let var = 

Makefile
var =
var :=
\end{verbatim}
使用變數
\begin{verbatim}
shell
$var
${var}

awk
var

perl
$var

C
var

python
var

javascript
var

Makefile
${var} or $(var)
  \end{verbatim}
  要注意的是
  \begin{itemize}
  \item shell的定義不加\$ 但要知道變數值時一定要加\$
  perl就比較一致 不管如何就是加\$符號
  C就是都不加
  \item shell的等號前後不可有空白\\
    var= var =
    是不一樣的。 perl,python與C就沒有限制,Makefile是定義與使用跟shell
    一樣,但是等號前後可以有空白,所以有四種情形請默記一下。
    \end{itemize}

  \section{陣列(array, list)與Hash literal}
    shell
\begin{verbatim}
array="e1 e2 e3"
\end{verbatim}
    perl
\begin{verbatim}
@array = (e1, e2 ,e2);
%hash = (key1 => val1,
         key2 => val2,
         key3 => val3);
$arrayref = [e1, e2];
$hashref = {key1 => val1, key2 => val2};
\end{verbatim}
C
\begin{verbatim}
int array[] = {1, 2, 3};
\end{verbatim}
python
\begin{verbatim}
list = [1, 2, 3]
dict = { key1 : val1, key2 : val2}
\end{verbatim}
要注意的是perl,很奇怪的用\verb=( )=不用\verb={ }=,python, c都想成reference,
shell 的串列型資料叫list裡面元素用space分開,最常用的場合
\begin{verbatim}
for var in $array; do
    cmd1
done
\end{verbatim}
array中的元素就是用space分開的資料。
perl有個很像的foreach
\begin{verbatim}
foreach $var (@array) {
  xxx
}
\end{verbatim}
javascript
\begin{verbatim}
a = [ 1, 2, 3 ]
b = { 'key1': val1, 'key2': val2 }
\end{verbatim}

\section{條件敘述}
shell
\begin{verbatim}
if test shell; then
    dosomething
elif test shell; then
    doother
else
    allright
fi
\end{verbatim}
perl
\begin{verbatim}
if (perl) {
  do_something;

} elsif (perl) {
  do_other;
} else {
  allright;
}
\end{verbatim}
C
\begin{verbatim}
if (c) {
  do_something;
} else if (c) {
  do_other;
} else {
  allright;
}
\end{verbatim}
python
\begin{verbatim}
if python:
  do_something
elif python:
  do_other
else:
  allright
\end{verbatim}
javascript
\begin{verbatim}
if (javascript) {
  do_something;
} else if (else) {
  do_other;
} else {
  allright;
}

\end{verbatim}
這邊要注意的是shell不需要括號,else if的寫法三個不一樣。條件式寫法,
shell沒有一定要用()包住條件式,只要test是某個可執行的敘述就好
\subsection{條件比較}
shell
\begin{verbatim}
  字串
[ "str1" = "str2" ]
  數值
[ num1 -ne num2 ]
\end{verbatim}
perl
\begin{verbatim}
  字串
if ($str1 ne $str2)
  數值
if ($num1 == $num2)
  \end{verbatim}
C
  \begin{verbatim}
  字串
if (!strcmp(str1, str2))
  數值
if (num1 == num2)
  \end{verbatim}
python
  \begin{verbatim}
  字串
if str1 == str2:
  數值
if num1 == num2:
  \end{verbatim}
javascript
\begin{verbatim}
  字串
if (a = 'str1')
  數值
if (a = 5)
\end{verbatim}
  要注意數值比較,字串比較,字串處理與邏輯比較。其實比較中,更進一步的像
  空array, 空hash, 空字串,0, 1等等是否能比較都是很重要的。

  \section{迴圈控制}
  \subsection{while迴圈}
  shell
  \begin{verbatim}
while test: do
  cmd
done
  \end{verbatim}

  perl跟c是一樣的,除了變數用法不一樣
  \begin{verbatim}
while (condition) {
  statment1;
}

unless (condition) {
    statment1;
}
\end{verbatim}

C
\begin{verbatim}
while (condition) {
    statment1;
}

do {
    statment1;
} while (condition)
\end{verbatim}
python
\begin{verbatim}
while condition:
    statement
\end{verbatim}
javascript
\begin{verbatim}
while {
    statement
}
\end{verbatim}

\subsection{for loop}
Bourne shell的for不是像C裡的for loop, bash有.
\begin{verbatim}
for var in element1 element2 element3...; do
    echo $var
done

bash 特有的

for (( i = 0 ; i < 10; i++)); do
    cmd1
    cmd2
done

\end{verbatim}
perl 有兩種for的用法
\begin{verbatim}
for ($i = 0; $i < 10; $i++) {
    statement;
}
這是像c shell的foreach,也就是B shell的for
foreach $var (@array) {
    statement;
}
\end{verbatim}
C的用法
\begin{verbatim}
for (i = 0, init = 1; i < 10; i++) {
    statement;
}
\end{verbatim}
python
\begin{verbatim}
for i in range(10):
    dosomething

for str in list:
    dosomething
\end{verbatim}
javascript
\begin{verbatim}
for (i=0; i< 10; i++) {
  dosomething
}

for key in array {
  dosomething
}

for element of array {
  dosomething
}
\end{verbatim}

\section{副程式}
shell
\begin{verbatim}
func1() {
    arg1=$1
    arg2=$2
}
\end{verbatim}
awk
\begin{verbatim}
function func1(arg1, arg2)
{
  xxx
}
\end{verbatim}
perl
\begin{verbatim}
sub func1(arg1, arg2)
{
  xxx
}
\end{verbatim}
C
\begin{verbatim}
int func1(int arg1, char *arg2)
{
          xxx
}
\end{verbatim}
python
\begin{verbatim}
def func1(arg1, arg2):
    xxx
\end{verbatim}
javascript
\begin{verbatim}
function fname(arg) {
  ...
}

(arg) => {
  ...
}
\end{verbatim}

\section{註解}
shell
\begin{verbatim}
# comment

: << BLOCK_COMMENT
alibuda
anyting is comment between BLOCK_COMMENT
BLOCK_COMMENT

\end{verbatim}
perl
\begin{verbatim}
# comment

=comment
multiline block comment
=
\end{verbatim}
C
\begin{verbatim}
/* comment */

/*
 * block
 * comment
 */

\end{verbatim}
python
\begin{verbatim}
# comment

"""
multiline block comment
"""
\end{verbatim}
javascript,不過 block comment 建議是用多個 \verb=//=
\begin{verbatim}
//

/*
 * ...
 */
\end{verbatim}

\section{提醒}
最後要提醒的是很多基本bit/byte處理,使用基本bash, od, printf, echo -ne, sed,
awk, perl, python, bc等工具就能做出以前c程式做了老半天才做到的效果。所以script
真是非常強大的日常處理工具。
\begin{description}
  \item [bitwise] \hfill \\
    \$((192 \& 255) 要不就是用perl, python的operator,例如
    python -c "print 0x\$\_ip1\_hex \& 0x\$\_subnet\_hex"
  \item [string to binary value] \hfill \\
    用echo -ne,例如
    echo -ne "\verb=\x55\xaa=" \verb=|= dd if=/dev/stdin of=\$bdev seek=510 bs=1 count=2
  \item [binary value to string] \hfill \\
    用 od 來得到 binary 值並且印出字串
    a=`echo -ne \verb="\x10"=`; echo \$a \verb=|= od -A n -t x1 -N 1
  \item [string format] \hfill \\
  \begin{itemize}
    \item decimal hex : 一般 printf 就能互轉\verb=printf "0x%x" 100; printf "%d" 0xab= 
      或者bash, \verb=echo $((0xa))=
    \item decimal octect : 一般printf , \verb=printf=
    \item binary string: 轉到binary比較難,只能靠 bc, perl, python這種,例如
      perl -e "printf\verb=(\"%b\",$decimal)"=, python -c "print bin(10)"
      轉回來bash有\verb=echo $((2#0101)=
    \item leading 0 : 這可以用awk, printf,反之要清掉leading 0 也可直接awk, printf。\\
      printf "0\%x" 12345, \\
      echo \$ip | awk -F . '\verb={printf("%d.%d.%d.%d",$1,$2,$3,$4)}='
    \item length : bash有簡單的表示法,strvar=alibuda; echo \verb=${#strvar}=,可以從屁股
      取回兩個字元,例如sector=0xabcd; char=\verb=${=sector\_hex: -6:2\verb=}=; echo \$char
  \end{itemize}
\end{description}
另外perl/python 中的array/list與hash/dictionary的處理,也是很常見的
\begin{description}
  \item [大小] \hfill \\
    perl 用\verb=$#=array,或者scalar(@array)\\
    python 用mylist.length()\\
    javascript 用array.length
  \item [搜尋] \hfill \\
    perl 可以 grep(/string/, @array);\\
    python 'string' in mylist\\
    javascript array.includes('string')
  \item [反轉] \hfill \\
    perl reverse(@array) \\
    python mylist.reverse()或者list(reversed(mylist))\\
    javascript 用array.reverse()
  \item [陣列切割] slice \hfill \\
    perl \verb=@array[2..3]= 只要index 2到index 3\\
    python \verb=[start:end:step] 第1 index與第2 index-1 以 step取slice=,-1表示從屁股開始\\
    javascript array.slice(1,3)
  \item [陣列連結] \hfill \\
    perl (@a, @b) \\
    python list1 + list2\\
    javascript array1.concat(array2)
  \item [字元切割 split] \hfill \\
    perl split(/,/, 'mystr1,mystr2');\\
    python str.split(",")\\
    javascript str.split(" ")
  \item [連結字元 join] \hfill \\
    perl @a = join(',', @array)\\
    python ",".join(mylist)\\
    javascript array.join('and') javascript 能連結字串
  \item [排序] \hfill \\
    perl sort(@array);\\
    python mylist.sort(); \\
    javascript array.sort()
  \item [iterate] \hfill \\
    perl map(func, @array) , 拿出key keys(\%hash), (\$key, \$value) = each(\%hash), while ((key,value) = each(\%hash))\\
    python x = iter\verb={[1,2,3]}= x.next(), 拿出 key dict.keys(), (key, value) = dict.items(), for key,value in dict.items(): \\
    javascript array.map(func), 拿出 key Object.keys(obj)
\end{description}

最後要說的是,不管哪種語言,說穿了就是
\begin{itemize}
  \item 所有東西都只是轉換轉換再轉換,並沒有什麼神奇的,跨平台等等都是。
  \item bit, byte, string, array, hash 表示法的轉換與使用。
  \item 一般性的 if else, while for loop 使用還有 condition 的情況為何?
  \item regex string 的搜尋,擷取,代換,與多行 string 的使用。
  \item 語法是否支援物件,如果是就有 function overloading, try catch等。
  \item module 的引用與被引用,與 library 管理。
  \item 有沒有 system call 相對應支援,對於多工,I/O, IPC 如何處理。
  \item 最後就是每個 script 語言都有相對應的 C/C++ binding , 但如果要寫
    有點用的大型程式,最後都要有跟系統溝通能力,都要會寫這種 c/c++ binding
    ,甚至要改進script 的 performance 都要去看原始 c/c++ code
\end{itemize}
是的,就這樣,要真的全部都會,就能寫點電玩軟體了。

\begin{thebibliography}{99}
  \bibitem{linux}Linux In a Nutshell, Ellen Siever, O'Reilly 
  \& Associates, ISBN 0596000251
  \bibitem{sed}Sek \& Awk, Dale Dougherty, et al, O'Reilly 
  \& Associates, ISBN 1565922255
  \bibitem{perl}Programming Perl, Wall, Christiansen \& Orwant,
  O'Reilly \& Associate, ISBN 0596000278
  \bibitem{perl}Perl Cookbook, Tom Christiansen, Nathan Torkington ,
  O'Reilly \& Associate, ISBN 1565922433
  \bibitem{kr}The C Programming Language, Brian W. Kernighan, 
  Dennis M. Ritchie, Prentice Hall, ISBN 0131103628
  \bibitem{manperl}man page of sed, awk and perl
  \bibitem{perl}{\href{https://perldoc.perl.org/}{perl docs}}
  \bibitem{python}{\href{https://docs.python.org/}{python docs}}
  \bibitem{python3}{\href{https://pymotw.com/3/}{python 3 modules}}
  \bibitem{javascript}{\href{https://developer.mozilla.org/en-US/docs/Web/JavaScript}{mozilla Javascript}}
  \bibitem{node.js}{\href{https://nodejs.org/docs/latest/api/}{Node.js API}}
\end{thebibliography}
