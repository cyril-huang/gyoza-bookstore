\chapter{sed}
    \section{簡介}
    sed是一種Stream EDitor,也就是餵給它一串資料流跟命令完成編輯的動作,
    這很適合自動化的scripts作業
    \begin{verbatim}
file -> stream -> sed -> stream -> file
    \end{verbatim}
    傳統的sed由於它必須有個in, 處理完後有個out所以不能同時改個檔案,
    必須先處理舊檔案後送出成新檔名,把舊檔殺掉,再把新檔改名成舊檔,
    又因為他內定輸出是standard out所以常看到的方法是
    \begin{verbatim}
$ sed 'sed_syntax' old_file > new_file
$ mv new_file old_file
    \end{verbatim}

    GNU sed常用的選項:
    \begin{itemize}
    \item -i 可以直接對檔案作處理。所以不需要上面的mv了。
    \item -e 多個sed命令的串接。
    \item -n 原本每處理一行,就會echo這行到螢幕上,加上-n不echo.
    \item -f 後面跟個sed命令檔。
    \end{itemize}
    sed通常是用來對字串的處理顯示,不是像一般editor來modify檔案的,
    而且記住是一行一行處理的所以空白, tab, 換行這些字元很重要。
    \\\\
    sed的一般式是
    \begin{verbatim}
$ sed [address],[address][!]command[args] file
    \end{verbatim}
    其中address,command,用單引號'xxx '包住。
    address就是上面regular express的東西,常用的命令有d(delete), s///(代換)
    一些簡單例子
    \begin{verbatim}
$ sed 's/yes/no/g' file               把檔案中yes換成no
$ sed '/save/!d' file                 把沒有save字眼的行幹掉
$ sed -e 's/\(.*\)\(#.*\)/\1/' xxx.sh 把#後面的註解幹掉
    \end{verbatim}
    其中-e是常用的一個option,通常是用在兩個以上的選項時。
    其中,由於\verb=( )=有特別意義所以需要反斜線escape一下。

    \section{符合條件的addressing}
    address是sed裡的特殊用語,用來設定符合條件的行,以便對它執行編輯命令,
    符合條件的條件指定方法有1. range(範圍) 2.regular express。
    \\\\
    sed的address除了之前regular expression介紹的
    1,3 . \$ \%這種表示法外  還可以直接用 /regex/找到要的行
    \begin{verbatim}
$ sed '2d'        file   找到第二行幹掉
$ sed '/delete/d' file   找到delete這個字幹掉整行
$ sed '/RE/,$d'   file   從有RE字眼的行到檔案尾通通幹掉 
$ sed '/^$/d'     file   幹掉空白行
$ sed '/START/,/END/!s/xx/yy/g' file > file1  把xx換成yy
    \end{verbatim}

    \section{pattern/hold space, 多regex條件與script檔}
    如果有兩個條件以上那可以加-e \\
    \verb=$ sed -e xxxx -e xxxx file= \\\\
    例如先將註解行變成空白行然後砍掉所有空白行 \\
    \verb=$ sed -e 's/\(.*\)\(#.*\)/\1/' -e /^$/d file xxx.sh = \\\\
    sed其實每次都把一行放到pattern space,然後對他操作
    執行結果再先放到pattern space這個地方,如果有 -e,
    就接著對pattern space的行再做script的執行,所以好幾個條件是有順序的。
    做完一行後,再來一遍,所以最通用的一般式應該是\\
    \begin{verbatim}
    [address],[address]{
            command[args]
            command[args]
            ...
    }

    或者
    [address],[address]{command[args];command[args];...;command[args];}

    \end{verbatim}
    每一行command相當於有個-e的作用會先把結果放到一個pattern space,
    可以寫成一個sed script file然後用-f 執行 \\\\
    \verb= $ sed -f script_file file =  \\\\
    或者每個command的最後面要放個分號;。
    另外sed有個hold space是個暫存東西的buffer,sed有些命令可以利用
    hold space與pattern space,很像vi裡面的yy dd 先放到一個看不見的
    buffer,再用p命令把它叫出來。

    \section{sed command命令}
    command有25個比較常用的
    \subsection{基本指令}
    就是先把這幾個編輯命令學好啦,字串的搜尋,擷取,代換是玩電腦所有的重點
    ,尤其是下面的 s 命令是所有最重要的命令。
    \begin{itemize}
    \item Append       a\verb=\=    找到address後append所有a\verb=\=
                                    後面所有字串
    \item Change       c\verb=\=    改變一段文字
    \item Delete       d     砍掉regex找到的字串那一行,記住是整行砍掉喔
    \item Insert       i\verb=\=    找到address後在前面insert字串
    \item Substitution s///  代換是最常用的,砍掉某字串也很常用s/string//g。
    \item Translate    y///  可以一個字元一個字元作不同的代換
    \end{itemize}
    \verb=a\= 命令與\verb=i\=命令,注意\verb=a\ i\= 
    都要換行才開始要加入的文字。如果要加的文字
    有換行要用\verb=\= 這個符號表示換行,如果要加入的文字有\verb=\=,
    可以用\verb=\\=來escape掉\verb=\=。不過這邊有很大的問題,
    也就是如果文字裡面同時有\verb=\= '怎麼辦。
    由於shell中的單引號 ' 沒辦法逃掉,這個可以另寫script檔解決。請看例子
    \begin{verbatim}
    cyril@gyoza:~$ cat last
    第一行
    最後一行

    cyril@gyoza:~$ sed 'a\
    > 中文試試看
    > ' last
    第一行
    中文試試看
    最後一行
    中文試試看

    cyril@gyoza:~$ sed '$a\
    > 中文試試看 故意超過一行看看不做任何處理時 超過一行時
    > sed會怎麼樣處理這樣的問題
    > ' last
    sed: -e expression #1, char 93: Unterminated `s' command

    cyril@gyoza:~$ sed '$a\
    > 中文試試看 故意超過一行看看不做任何處理時 超過一行時\
    > sed會怎麼樣處理這樣的問題
    > ' last
    第一行
    最後一行 
    中文試試看 故意超過一行看看不做任何處理時 超過一行時
    sed會怎麼樣處理這樣的問題
    
    cyril@gyoza:~$ cat address.txt
    台北市建國南路一段
    270號
    <Nick's Address>
    <Cyril's Address>
    台北市松江路
    一段10號
    <MiMi's Address>
    
    cyril@gyoza:~$ cat insert.sed
    /<Cyril's Address>/i\
    100 Gyoza Blvd\
    San Jose, CA
    
    cyril@gyoza:~$ sed -f inser.sed address.txt
    台北市建國南路一段
    270號
    <Nick's Address>
    100 Gyoza Blvd
    San Jose, CA
    <Cyril's Address>
    台北市松江路
    一段10號
    <MiMi's Address>
    
    \end{verbatim}
    \verb=c\= change這個命令跟\verb=a\ i\= 很像,只有一點要注意,
    就是如果要改變的文字跨很多行,則
    adress是個range 1,10這樣時,所有的range行都會消失,並且只有一行改變了。
    \begin{verbatim}
    cyril@gyoza:~$ cat last
    第一行  first line
    第二行  second line
    最後一行  last line

    cyril@gyoza:~$ cat change.sed 
    1,2c\
    第一行跟第二行都被幹掉了
    
    cyril@gyoza:~$ sed -f change.sed last
    第一行跟第二行都被幹掉了
    最後一行  last line
    
    \end{verbatim}
    然後如果要作用的字串變數中含有 / ,尤其是路徑名稱中常有這種字串,則 gnu
    sed 允許使用別的 dlimiter ,最常用的就是用 |
    \begin{verbatim}
    sed 's|search|replace|g'
    \end{verbatim}
    小寫變大寫
    \begin{verbatim}
    把第一行到第十行中的小寫變大寫
    $ sed '1,10y/abcdefghijklmnopqrstuvwxyz/ABCDEFGHIJKLMNOPQRSTUVWXYZ/' lower.txt
    \end{verbatim}
    代換的命令中有很重要的第二層處理,如果在一行內的字串吻合要代換的有很多
    個,那麼內定會用第一個吻合的換掉,如果想換掉特定的就在後面加上想要的
    第幾個,想全部換掉用g這個flag
    \begin{verbatim}
    cyril@gyoza:~$ cat last
    第一行  first line
    第二行  second line
    最後一行  last line

    cyril@gyoza:~$ sed 's/first/line/1' last
    第一行  line line
    第二行  second line
    最後一行  last line

    cyril@gyoza:~$ sed 's/first/line/2' last
    第一行  first line
    第二行  second line
    最後一行  last line

    cyril@gyoza:~$ cat lines
    第一行  first line
    第二行  second line
    很多行  multiple line line
    最後一行  last line

    cyril@gyoza:~$ sed 's/line/row/2' lines
    第一行  first line
    第二行  second line
    很多行  multiple line row
    最後一行  last line

    cyril@gyoza:~$ sed 's/line/row/g' lines
    第一行  first row
    第二行  second row
    很多行  multiple row row
    最後一行  last row
    \end{verbatim}
    \subsection{pattern/hold space的處理}
    %\include{space}
    \begin{itemize}
    \item l list pattern space的東西是甚麼
    \item h 把hold space裡的東西清掉,把pattern space的東西copy給hold space
    \item H 把pattern space的東西加在hold space東西後面
    \item g 把pattern space裡的東西清掉,把hold space東西拿回給pattern space
    \item G 把hold space的東西加在pattern space東西後面
    \item p 印出pattern space的東西
    \item x 交換(exchange)pattern space與hold space
    \end{itemize}
    l跟p的差別在對非ASCII的處理
    \begin{verbatim}
    cyril@gyoza:~$ cat last
    第一行  first line
    第二行  second line
    最後一行  last line

    cyril@gyoza:~$ sed '1,2p' last
    第一行  first line
    第一行  first line
    第二行  second line
    第二行  second line
    最後一行  last line

    cyril@gyoza:~$ sed '1,2l' last
    \262\304\244@\246\346  first line$
    第一行  first line
    \262\304\244G\246\346  second line$
    第二行  second line
    最後一行  last line
    \end{verbatim}
    這邊可以對pattern space有更深的了解,第一行進到pattern space
    看到p把第一行印出,接著pattern space變成output,所以sed又把
    他印一遍,接著第二行進來了,在pattern space裡,看到p把第二行印出,
    同理又有兩個第二行被印出,第三行由於p沒作用,直接是pattern space
    送到output。再看一個例子(抄出來的)
    \begin{verbatim}
    cyril@gyoza:~$ cat command.txt
    This describes the Linux ls command.
    This describes the Linux cp command.
    
    cyril@gyoza:~$ cat command.sed
    /Linux/{
    h
    s/.* Linux \(.*\) .*/\1:/
    p
    x
    }
    
    cyril@gyoza:~$ sed -f command.sed command.txt
    ls:
    This describes the Linux ls command.
    cp:
    This describes the Linux cp command.

    或者寫成一行
cyril@gyoza:~$ sed '/Linux/{h;s/.* Linux \(.*\) .*/\1:/;p;x;}' command.txt
    \end{verbatim}
    這邊看第一個h把第一行推進hold space,不過記住pattern space還保有
    原本第一行的句子。然後代換掉變成ls:,現在pattern space只有ls:,
    然後p印出ls:,最後把hold space跟ls:交換,所以最後pattern space內
    就是原本的東西,被sed當成output送出,cp是同樣的道理。
    \\\\
    所以沒什麼,只是兩個眼睛看不到的space在那邊換來換去,把內容在這兩個
    地方加加減減copy/paste就好了。如果熟悉vi的人應該很容易體會。
    \subsection{再進一步}
    \subsubsection{多行處理與迴圈}
    在原本的h, H, g, G, x裡面都只有對單一行處理,如果pattern match要跨行
    時,就捉襟見肘。下面的命令用來處理兩行以上的pattern match及pattern
    space, hold space的處理
    \begin{itemize}
    \item n   把pattern space東西送出,讀入下一行,所以下一行變成pattern
              space中的東西。
    \item N   讀入下一行跟目前pattern space的
              東西連起來變一行,但是裡面有個embedded換行符號。可以用
              \verb=\n=來表示,記住,只有這時才可以用\verb=\n=來表示。
    \item D   幹掉到第一embedded newline部份,與N合用喔。
              如果D在最底下被執行了,則pattern space東西保留下來,
              迴圈到script頂端再來一次。
    \item P   print到有embedded newline的一行,保留所有pattern space東西,
              如果P在最底下被執行了,則pattern space東西保留下來,
              ,迴圈到script頂端再來一次。
    \end{itemize}
    %\include{sed_loop}
    找到某一行的下一行
    \begin{verbatim}
    sed -n '/pattern/{n;p}' mytext.txt
    \end{verbatim}
    Join兩行,
    \begin{verbatim}
    /join/ {
    N
    s/\n/ /g
    }
    \end{verbatim}
    把多行空白行幹掉成一行空白,
    \begin{verbatim}
    /^$/ {
    N
    /^\n$/D
    }

    \end{verbatim}
    請看,找到空白行後,把下一行加進來,如果只有一行空白,空白行與
    下一行併成一行保留在pattern space,不過這時下個命令
    \verb=/^\n$/D=沒作用,pattern space全部送出,
    如果有兩行空白,所以Delete掉第一行空白,剩下一行空白,保留下來,
    這時因為D命另有作用了,所以並不送出pattern space的東西,
    繼續\verb=/^$/=,一直會有一行空白行留下來。這邊D與P都有迴圈的作用
    ,可以拿來作重複的處理。

    \subsubsection{flow control branch}
    這是用來轉移命令執行的就像是c裡的if else, while這些控制命令
    這要搭配:一起用。由:設定一個label,然後用b label跳到這個地方執行。
    \begin{itemize}
    \item : label, 設定label
    \item b label  branch 換到label地方執行,如果沒有label就通通不執行
            直接跳到script尾巴。
    \item t label  如果一個代換成功,test 測試條件,成立後跳到label,
            所以必須與s///命令一起用,s命令一定在t命令前。
    \end{itemize}
    多說不如看例子
    \begin{verbatim}
    $ cat join.txt
    join
    second line
    在C語言中反斜線代表字串的延續 \
    這一行跟上一行是連在一塊的

    $ sed '{
    :a 
    N
    s/\\\n//
    t a}' join.txt
    join
    second line
    在C語言中反斜線代表字串的延續 這一行跟上一行是連在一塊的
    
    \end{verbatim}

    \subsubsection{多檔處理}
    \begin{itemize}
    \item r 讀檔案 將檔案append到目前pattern space中
    \item w 寫檔案 將pattern space中的東西寫出到一個檔案
    \end{itemize}
    \begin{verbatim}
    $ cat insert.sed
    /<Cyril's Address>/i\
    100 Gyoza Blvd\
    San Jose, CA

    $ cat lines
    第一行  first line
    第二行  second line
    很多行  multiple line line
    最後一行  last line

    $ sed '/first/r insert.sed' lines
    第一行  first line
    /<Cyril's Address>/i\
    100 Gyoza Blvd\
    San Jose, CA
    第二行  second line
    很多行  multiple line line
    最後一行  last lin

    $ sed '/first/w firstline.txt' lines
    第一行  first line
    第二行  second line
    很多行  multiple line line
    最後一行  last line

    $ cat firstline.txt
    第一行  first line

    \end{verbatim}
    \subsubsection{其他}
    \begin{itemize}
    \item = 印出行號
    \item q 離開sed
    \end{itemize}
    \begin{verbatim}
    $ cat lines
    第一行  first line
    第二行  second line
    很多行  multiple line line
    最後一行  last line

    $ sed -e '/first/=' -e '/second/q' lines
    1
    第一行  first line
    第二行  second line
    \end{verbatim}
    
    \section{實用例子}
    configuration檔的分析與修改是很多system admin或系統整合, embadded系統
    軟體工程師要負責的,用shell/sed/awk其實都是短短幾行就解決掉的事情。
    \begin{verbatim}
    # comment blablabla
    [main]
    key0 = value0
    key1 = value1
    [section1]
    key0 = value0
    key1 = value1
    [section2]
    key2 = value2
    key3 = value3
    \end{verbatim}
    在很多configuration檔案中都會有類似上面的分類法,要找到特定section中的
    某個key的值,修改,新增,刪除等等很多人可能寫了幾百行的程式,但用sed,
    根本就是幾行內就完成的事。例如要找到main裏面是否有key0,有的話就改value10
    ,沒有的話就加value10
    \begin{verbatim}
if [ "`sed -n -e '/\[main\]/,/\[.*\]/p' $file | grep 'key0 *='`" ]; then
    sed -i -e "s/\(key0\).*/\1=value10/" $file
else
    sed -i -e "/\[main\]/akey0=value10" $file
fi
    \end{verbatim}
    五行code結束的事情,寫成shell function也不過9行的事情,我看過有人寫了上百
    行的perl在搞定。
    \begin{verbatim}
    ip_address {
            172.25.97.100
    }
    \end{verbatim}
    那如果像這種要更改的呢? 也完全沒有問題,一行就解決了。
    \begin{verbatim}
    sed -i -e '/ip_address/,+2c\ip_address {\n\t172.25.97.99\n' my.conf
    \end{verbatim}
    這比用 s/// 好的地方是因為 s/// 命令是作用在某內定範圍內,而這範圍
    一定要有內定結束標誌,而這只能選擇 newline,因此就無法跨行處理。
    多行刪除我們可以用範圍選擇 '/start line/,/end line/d' ,但 s 語法中沒有
    這種語法。因此多行處理用 s 有個標準模式,例如我們想換掉從 id 到 xxx 的文字
    \begin{verbatim}
sed '/id.*/{      # 如果找到 id ,開始玩
:b                 # label b
$!N                # 如果不是檔案最後一行就繼續N ,把下一行放進 pattern space
/xxx/!bb           # 如果沒找到想要的xxx,跳回 label b,繼續玩
s/.*/alibuda/      # 不然就用 s 代換掉所有 pattern space 東西變成 alibuda
}' infile
    \end{verbatim}
    \verb=$!N=不到最後一行繼續 N (把下一行放進pattern space), /xxx/!bb 不找到
    xxx 則跳到b,這裡關鍵用 "不" 來表示一個 while loop, while [ ! xxx ] 就
    怎麼樣,而且這個不 ! 是放在條件後面 -> (條件 !)。
    直到找到,就離開 loop block, 進行整行的 s/// ,因為整個 pattern space 已
    經被我們加了很多行進來,所以整個 s/.*/alibuda/ 是作用很多行。
    \\\\
    而 sed '/start line/,/last line/{s/.*/alibuda/}' ,用這種 range 的方法,
    會作用在每一行,而不是把所有行一次處理。 所以
    \begin{verbatim}
    /first line/{:b;$!N;/last_line/!bb;s/replace/something/}
    \end{verbatim}
    也是一種多行代換,但我想用 \verb='//,//c\'= 可能比較方便。
    \\\\
    找到一行與下三行
    \begin{verbatim}
    sed 's/find/,+3p'
    \end{verbatim}
    印出找到一行的上一行面
    \begin{verbatim}
    sed '/find/{x;p;d;}; x'
    \end{verbatim}
    找到一行與上三行,請 tail 幫忙
    \begin{verbatim}
    sed '/find/q' | tail -n 3 含有找到行
    sed -n '/find/q;p' | tail -n 3 不含找到行
    \end{verbatim}
    用 grep 找到一行的上下個三行,但有的 grep 是不行的
    \begin{verbatim}
    grep -A 3 -B 3 'find' myfile
    \end{verbatim}
    整個文件變一行,清除某個唯一的跨行字串,再變回來
    \begin{verbatim}
    tr '\n' '\0' < txt | sed 's/id//' | tr '\0' '\n'
    \end{verbatim}
    由於 sed 必須要知道什麼時候處理結束,因此 \verb=\n= 是 delimiter,
    代換一個多行字串的變數,基本上無法做到很漂亮的一行完成,
    sed 無法處理 \verb=\n=,必須先用 tr 把 \verb='\n'= 換成某個不可能出現字元
    再換回來。
    \\\\
    另外除了之前在shell外部命令講的幾個有用命另外,還有一些在coreutils裏面
    的常用命令
    \begin{description}
      \item [cut] \hfill \\
        可切割一行的輸出為多欄位的處理,沒有awk時可以用,例如:
        echo "f1 f2:f3 f4" \verb=|= cut -d " " -f 1
        這個 cut 好處是可以處理到 column 裡面的字元,不光是 column 而已。
        -c2 表示只取第二字元, -f 3-5 還可以表示只處理第三到第五個 field
        (column) 就好。-d, -c, -f 是常用選項。
      \item [sort] \hfill \\
        可以排序很多行的輸出,比較常用有 sort -u,可以刪掉 duplicate 的行。
        sort -k2.3 根據第2 column的第3字排序,-k3.1,3.2 表示先3.1再3.2排序,
        這個 sort 好處也是可以處理到 column 裡面的某個字元。可以排整數,
        排字串,排版本號等等。
      \item [printf] \hfill \\
        這是後來posix標準工具。可以用來做hex/dec轉換:
        printf "\%x" 12345 是 hex 輸出, printf "\%0x" 1 則有
        leading 0 的輸出格式,跟 c 是很像的。而如果要像 c 
        \verb=printf("%d",'a');=,則可以用 \verb=printf "%d" "'a"=
      \item [numfmt] \hfill \\
        做數字與人類可讀數字的轉換,例如4G -> 4000000000, 138611456 -> 133M
        numfmt --from=iec 125K, numfmt --to=iec 421384951344
      \item [od] \hfill \\
        用來處理 binary 檔案內容的。可以做 hex/dec 轉換,
        例如看第1 partition:  od -A n -t x1 -N 16 -j 446 /dev/sdc.
        echo "" \verb=|= od -A n -t x1 ,則會看到跑出 0a 表示換行。
        a=`echo -ne \verb="\x10"=`; echo \$a \verb=|= od -A n -t x1 -N 1 
        則會跑出 10。 -t 表示輸出 format, x1 表示一個 hex, -N 表示讀進幾個 byte
        。-j 表示跳到那個 byte 去。echo -ne 是 string 到 binary 值,od 是 
        binary 值回到 string, printf 是 string 對 string 的 hex,dec 轉換。
      \item [dd] \hfill \\
        用來處理device檔案或一般檔案的I/O。例如寫 mbr signature:
        echo -ne \verb="\x55\xaa" |= dd if=/dev/stdin of=\$bdev seek=510 bs=1
        count=2
      \item [seq] \hfill \\
        產生一連串的數字,跟 python 的 range 一樣,用來 loop 用的,例如:
        for i in `seq 3 20`; do echo \$i; done
      \item [tac] \hfill \\
        跟 tail 很像,但是倒著印出檔案內容,通常是從檔尾處理檔案時用,例
        如: tac myfile。
      \item [wc] \hfill \\
        計算字元,行數等用途的小計算器。例如: cat my file \verb=|= wc -l
    \end{description}
    基本上光 bash, sed + 這些 coreutils 的命令,就可以完成大部分的工作了。
    尤其很多系統 script 都是在叫用命令來完成,很多人號稱寫 perl, python
    script 結果不是去呼叫 API 而是在使用系統命令的話是一點意義都沒有的。如果為
    了簡單的搜尋修改檔案,在那邊 open, fopen, read, write的話,真的不如一行
    sed -i 就解決。光使用 echo -ne, od, printf, bc, expr, perl, python 等工具
    基本上就可以完成大部份工作。很多時候去interview 時,老闆根本就不要你會亂七
    八糟的東西,光真的會bit, byte, string 的處理,就已經非常了不起了。

    \subsection{binary 檔修改}
    在很久以前,我一直想不通怎麼 Linux 上沒有以前 DOS 下的 binary 編輯器呢?
    後來我終於想通了,因為二進位檔,根本就不是需要編輯的檔案,這種檔案都是要
    破解,修改,亂搞,只修改幾個 bytes ,幾個 bits ,根本不會有人從頭到尾用
    bits, byte 編輯檔案。而這種修改根本就是 sed 一行就解決的事情,所以根本就
    沒人大費周章的去寫這種軟體,又是以前錯誤的 DOS/Windows 使用習慣而來。
    \\\\
    以 Windows 的 EFI 啟動 bootx64.efi 為例,我們想要去改他的BCD位置,只要
    先抓到 \verb=\.B.C.D= 的位置,再用 xxd -r 去修改就完成了。
    請看例子,保險起見,印出上下共三行
    \begin{verbatim}
$ xxd /mnt/win10/efi/boot/bootx64.efi | grep -A 1 -B 1 '\\.B.C.D'

000e8080: 6500 6400 2000 3000 7800 2500 7800 0d00  e.d. .0.x.%.x...
000e8090: 0a00 0000 0000 0000 5c00 4200 4300 4400  ........\.B.C.D.
000e80a0: 0000 0000 0000 0000 4200 4f00 4f00 5400  ........B.O.O.T.
--
000ed310: 7400 0000 0000 0000 5c00 4200 6f00 6f00  t.......\.B.o.o.
000ed320: 7400 5c00 4200 4300 4400 0000 0000 0000  t.\.B.C.D.......
000ed330: 4c00 6900 6200 7200 6100 7200 7900 2000  L.i.b.r.a.r.y. .
    \end{verbatim}
    我們只想改第一個,把 B.C.D 改成 B.W.0
    \begin{verbatim}
echo '000e809a: 4200 5700 3000' | xxd -r - /mnt/win10/efi/boot/bootx64.efi
    \end{verbatim}
    這如果寫成 shell script
    \begin{verbatim}
bcd_line=$(xxd /w10/efi/boot/bootx64.efi|sed '/4200 4300 4400/q'|tail -n1|cut -f1-9)
echo $bcd_line|sed 's/4300 4400/5700 3000/'|xxd -r - /w10/efi/boot/bootx64.efi
    \end{verbatim}
    有 xxd 跟 sed 工具,誰還想用什麼亂七八糟的 binary editor
