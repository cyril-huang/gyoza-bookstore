\chapter{awk}
    \section{簡介}
    field的觀念是awk比sed好用的地方,相對於sed一行一行的處理,awk提供了
    一行內部欄位拆解的處理,還有更powerful的像C程式語法迴圈控制處理。
    通式可以是
    \\\\
    awk 'script' var=value files \\
    awk -f scriptfile var=value files \\\\
    其中var是自定的awk變數,value是給定的值。這可以用來跟shell變數溝通。
    \\\\
    另外像sed的regular express都拿一行行來作內定的輸入,再做比對,這樣很沒有
    彈性,awk允許用awk的一個變數來作regular express的輸入比對,因此像欄位也
    可以輕鬆的拿來比對。主要是用比對算符 ~ 來跟一個regular express比對,
    \\\\
    var \verb=~= /re/  \\
    var \verb=!~= /re/ \\\\
    這樣就很方便了,而且這比對也同樣會傳回true, false,可以當成一個condition。
    \\\\
    除了欄位處理以外awk還有更多好用的function可以呼叫,另外也有了if,while, 
    for與自定function的能力,這些比sed跟shell合併的能力又大上了許多,也是將來
    perl的基本功能的基礎。

    \section{script基礎}
    一個script由
    \verb=condition{procedure},condition{procedure},.....=組成。
    跟sed很像,找到什麼條件(pattern部份),做什麼動作(procedure部份),
    有一些內部的變數可以幫我們設定一些條件。condition可以是pattern regular 
    express或某個符合條件,這個請看邏輯運算符號與比對算符 ~ 。其中有兩個重要
    的內定condition, BEGIN跟END。一個script通常是這樣
    \\\\
    BEGIN \verb={xxx}= /re/\verb={xxx}= END\verb={xxx}=
    \\\\
    一般寫程式,一定要有init初始化一些值,就像人有出生年月日八字一樣,
    程式結束也要有處理善後的routine來處理。awk作為一個程式型態的script語言,
    提供了一個general的初始與結束的處理。
    \begin{verbatim}
    算算有幾個空白行
    $ awk 'BEGIN{x=0} /^$/{x++} END{print x}' scripts.tex
    \end{verbatim}
    一開始設定x=0後,BEGIN後的就不在執行,然後每讀一行進來,如果是空白行就
    x++,最後讀scripts.tex這檔案完到檔尾,印出x。
    \\\\
    要記住在上面的例子裡,awk的變數跟c是一樣的也就是沒有錢符號的,這邊跟shell
    perl等script比較不一樣。
    \section{變數與欄位處理}
    awk裡面很重要的兩個定義
    \begin{itemize}
    \item record 就是一行一行的資料
    \item field  通常由space或者tab鍵隔開的資料欄就是field
    \end{itemize}
    請看例子,並注意shell command line的單引號雙引號
    \begin{verbatim}
    $ cat /etc/mnttab
    /proc   /proc   proc    dev=31c0000     1022606134
    fd      /dev/fd fd      rw,suid,dev=32c0000     1022606198
    mnttab  /etc/mnttab     mntfs   dev=3380000     1022606209
    swap    /var/run        tmpfs   dev=1   1022606209
    swap    /tmp    tmpfs   dev=2   1022606211

    $ awk '/swap/{print $2}' /etc/mnttab
    /var/run
    /tmp

    $ awk '{print "Hello World"}' /etc/mnttab
    Hello World
    Hello World
    Hello World
    Hello World
    Hello World

    一二欄位中間是tab二三欄位中間是空白
    $ cat datafile
    99		98 3.5
    300		298 4.9
    498		493 5.9
    699		698 7.6
    900		748 9.0
    1200	703 9.6
    1500	651 10.4
    1698	627 10.8

    記住  這樣是從input兩個欄位輸出output一個欄位
    $ awk '{print $1 $2}' datafile
    9998
    300298
    498493
    699698
    900748
    1200703
    1500651
    1698627

    這樣是從input兩個欄位輸出output兩個欄位
    用print輸出欄位由逗號 , 分開
    $ awk '{print $1,$2}' datafile
    99 98
    300 298
    498 493
    699 698
    900 748
    1200 703
    1500 651
    1698 627

    這樣是從input兩個欄位輸出output一個欄位
    $ awk '{print $1 "," $2}' datafile
    99,98
    300,298
    498,493
    699,698
    900,748
    1200,703
    1500,651
    1698,627

    也就是說print $1$2 等於print $1 $2

    \end{verbatim}
    \subsection{內定變數}
    除了像C語法的自訂變數,x,y,z....。awk有一些內定的變數名。
    欄位上的內定變數
    \begin{verbatim}
    ARGC        :  number of arguments on command line
    ARGV        :  array containing the command line arguments
    ARGIND      :  
    $0          :  整行record
    $1....$n    :  第n個欄位(field)
    
    ENVIRON     :  一個associate array含有全部的環境變數
    ERRNO       :  最後system出錯的error number請看/usr/include/errno.h
    
    FILENAME    :  current filename
    FIELDWIDTHS :  list of field widths
   
    FS          :  field separator內定是空白或tab
    OFS         :  output field seperator內定是空白
    RS          :  record separator內定是newline
    ORS         :  output record seperator內定是newline
    NF          :  number of field	這常常也是陣列與loop要用到的length
    NR          :  number of record	這常常也是陣列與loop要用到的length
    \end{verbatim}
    其中FS, OFS, RS, ORS, NF, NR是常用到的請看例子
    \begin{verbatim}
    $ cat /etc/passwd
    list:x:38:38:SmartList:/var/list:/bin/sh
    irc:x:39:39:ircd:/var:/bin/sh
    nobody:x:65534:65534:nobody:/home:/bin/sh
    cyril:x:1000:1000:Cyril Huang,,,:/home/cyril:/bin/bash

    $ awk '{FS=":"; print $7}' /etc/passwd
    /bin/sh
    /bin/sh
    /bin/sh
    /bin/bash

    $ awk '{FS=":"; OFS=","; print $1,$6,$7}' /etc/passwd
    list,/var/list,/bin/sh
    irc,/var,/bin/sh
    nobody,/home,/bin/sh
    cyril,/home/cyril,/bin/bash

    NF NR可以拿來作condition用
    $1 $2....也可以拿來作condition用,搭配變數regular express match算符 ~

    $ cat datafile
    # This is the data file from experiment
    99      98 3.5
    300     298 4.9
    498     493 5.9
    # Failed data
    122     123
    699     698 7.6
    900     748 9.0
    1200    703 9.6
    1500    651 10.4
    1698    627 10.8

    $ awk 'NF == 3 && $1 ~ /[0-9]+/{print $3,$1,$2}' datafile
    3.5 99 98
    4.9 300 298
    5.9 498 493
    7.6 699 698
    9.0 900 748
    9.6 1200 703
    10.4 1500 651
    10.8 1698 627
    \end{verbatim}

    \subsection{awk變數與shell變數}
    awk 與 sed都是幫助shell programming的好工具,因此要跟shell變數溝通,
    通式中的 var=value,var就是awk變數,value就是shell變數的值。
    {\small
    \begin{verbatim}
    # fs_mounted $dev $mnt_pt $fs_type
    # This is shell script version of fs_mounted, check if the file system is 
    # mounted.
    fs_mounted()
    {
        _BINGO=
        [ $# -eq 3 ] || return $ERRNO_EINVAL
        _BINGO="`$AWK
          '{if (\$1 == _DEVICE && \$2 == _MNT_PT && \$3 == _FS_TYPE) print \$1}'
          _DEVICE=$1 _MNT_PT=$2 _FS_TYPE=$3 $MNTTAB`"
        [ "$_BINGO" ] || return -1
        return 0;
    }
    \end{verbatim}}
    當然也可以用shell單quote, double quote的"'"'"'的技巧來傳變數進去awk。
    \\\\
    shell script的執行與shell command中的awk對於一些保留字元處理跟sed一樣
    要小心點。還有sed的regular express中, + ? ( )需要在前面用反斜線escape,
    在awk裏面卻不用,所以shell, bash, sed, awk, perl, python都不太一樣,
    在單行使用時,要小心測試一下。
    \\\\
    這個例子裡面注意\$1 \$2 \$3,在AWK ' '單引號裡面的是awk的\$1\$2\$3,在
    外面\_DEVICE=\$1...是shell的 \$1 \$2 \$3。\_DEVICE \_MNT\_PT 等是awk的
    變數。所以awk的\verb=$1 $2 $3=有\verb=\=這個escape在前面。也就是說shell
    不處理它們並把這些送給awk處理。這在同時使用兩種script中常常要注意,例如
    php包javascript, php內用perl,perl包shell,....等等。
    \\\\
    gawk 也有 -v 選項設定awk 變數值。例如gnu awk 有個特別的IGNORECASE = 1時
    ,搜尋會不分大小寫,例如以逗號為區隔不分大小寫找出第八欄位是openssl
    \begin{verbatim}
awk -F , -v "IGNORECASE=1" '$8 ~ /openssl/ { print $1,$2; }' myfile
    \end{verbatim}

    \subsection{陣列Array與split函數}
    所有awk的array都是associate array,也就是hash,也就是帶有名字id的陣列。
    這個名字id就是hash的key。一般我們用1, 2 ,3 ...來當作陣列的id註標,在
    awk裡面允許我們用名稱來當id index來搜尋陣列。不過awk很聰明如果看到是數
    字型的index也可以像在C裡面用的一樣。
    \\\\
    注意!陣列從\verb=array[1]=開始,不是從\verb=array[0]=開始,跟c不一樣。
    我們可以指定\verb=x[1], x[2]...=的陣列照1,2,3 ...排列順序,
    但associate array的排列不是我們可以控制的。注意associate array的注標要用
    引號括起來,請看for loop的用法。
    \\\\
    可以用split(string, array, [sep])來得到一個陣列,sep沒有給就用FS的值當做
    這個string的欄位分隔,一個欄位就是一個陣列元素。
    \begin{verbatim}
    $ cat split.awk
    #! /bin/awk
    BEGIN{FS=":"}
    /cyril/{
        print $0
        split($0, afield)
        for (i = 0; i < NF; ++i) { 
            print i","afield[i]
        }
        
        delete afield[2]

        for (i = 0; i <= NF; ++i) {
            print i","afield[i]
        }
    }

    $ awk -f split.awk /etc/passwd
    lcyril:x:100:1::/export/lcyril:/usr/bin/bash
    0,
    1,lcyril
    2,x
    3,100
    4,1
    5,
    6,/export/lcyril
    7,/usr/bin/bash
    0,
    1,lcyril
    2,
    3,100
    4,1
    5,
    6,/export/lcyril
    7,/usr/bin/bash
    \end{verbatim}
    
    \section{基本控制語法}
    除了欄位的變數與特定用法外,awk比sed強的在於更接近程式語言的控制方法。
    awk的語法很像c,說實在awk蠻簡單的,有c的基礎很容易就會了,當然也可以先學
    awk再學c。
    \subsection{數學與邏輯算符}
    C的一般數學
    \begin{verbatim}
    + - * / % ++ -- ^ **(這是fortran的指數)
    \end{verbatim}
    C的一般指定
    \begin{verbatim}
    = += -= *= /= %= ^= **=
    \end{verbatim}
    C的一般邏輯
    \begin{verbatim}
    ! == < > <= >= != && ||
    ?: (c的cond ? result1 : result2 如果cond成立返回result1不然返回result2)
    \end{verbatim}
    condition的邏輯算符裡面有個比較特別的就是pattern match算符 \verb=~, !~= ,
    將來perl中很有用的類似算符\verb=~==。主要是regular express通常match的是
    一行行作為內定的比對,如果想要對某個特定變數做re match時,就要用到這個算
    符。
    \begin{verbatim}
    如果x是個阿拉伯數字
    if (x ~ /^[0-9]+/) {
         x += y
    }

    \end{verbatim}

    \subsection{if}
    沒什麼了不起就是c而已
    \begin{verbatim}
    if (condition) {
    } else if (condition) {
    } else {
    }

    \end{verbatim}
    注意else if跟c語法一樣

    \subsection{for loop}
    這邊的for有兩個,是c與bourne shell的綜合,perl也是一樣,不過perl用了
    c shell的語法。
    \begin{verbatim}
    for (i = 0; i < 10; i++) {
    }
    for x in xarray {
    }
    \end{verbatim}
    請看例子
    \begin{verbatim}
    $ cat gyoza.awk
    #! /bin/awk  heehee
    BEGIN {
    x[1]="I"
    x[2]="am"
    x[3]="a"
    x[4]="gyoza"
    y["I"]="I"
    y["am"]="am"
    y["a"]="a"
    y["gyoza"]="gyoza"
    for (i = 1; i < 5; i++) {
         print x[i]
    }
    for item in x {
         print item, x[item]
    }
    for item in y {
         print item, y[item]
    }
    }

    後面的gyoza.awk沒有用啦  我只是跑這個BEGIN而已
    $ awk -f gyoza.awk gyoza.awk
    I
    am
    a
    gyoza
    2 am
    3 a
    4 gyoza
    1 I
    gyoza gyoza
    am am
    I I
    a a
    \end{verbatim}

    \subsection{while}
    沒什麼了不起就是c而已
    \begin{verbatim}
    while (condition) {
        statment
    }

    do {
        statment
    } while (condition)

    \end{verbatim}
    for loop與while loop都跟c一樣,可以用break離開,continue不做這loop換到下個
    loop。

    \section{function}
    跟c不太一樣的,這是一種script,不用甚麼type要宣告,也沒有什麼
    pass by value, pass by address,變數是global的。
    \begin{verbatim}
    # 通式
    function fun(arg1, arg2) {
            arg1 = arg2
            xxxx
    }

    排序的例子
    $ cat sort.awk
    # sorting function
    function sort(Array, elements, temp, i,j)
        for (i = 2; i<= elements; ++i) {
            for(j = i; array[j - 1] > array[j]; --j) {
                temp = array[j]
                array[j] = array[j - 1]
                array[j - 1] = temp
            }
        }
        return 
    }
    # main routine, 一行一行來
    {
    for (i = 2; i <= NF; ++i) {
         grades[i - 1] = $i
    }
    sort(grades, NF-1)
    printf("%s: ", $1)
    for (j = 1; j <= NF - 1; ++j) {
        printf("%d ", grades[j])
    }
    printf("\n")
    }

    $ cat grade.txt
    西西  100 60 75 23
    米米  100 98 99 89

    $ awk -f sort.awk grade.txt
    西西: 23 60 75 100 
    米米: 89 98 99 100

    \end{verbatim}
    不過排序可以用sort這個command比較簡單,sort有一個選項-u,可以把資料
    裡面一樣的篩選掉,duplicate的會只剩一個,這也很方便。
    
    \section{其他函數}
    \subsection{字串處理-match與代換}
    字串上處理不像sed有s///這種代換,得用上一些內建函數。
    \begin{itemize}
    \item 傳回符合regex的在string的位置\\
    match(string, regex)
    \item 傳回子字串: 傳回在第m個字元往後數n個字元的子字串,n沒給就到底\\
    substr(string, m, [,n])
    \end{itemize}
    代換: 代換的幾個函數有的傳回值很特別
    \begin{itemize}
    \item 把string2裡面符合regex的第一個字串換成string1。成功傳回1。\\
    sub(regex, string1, string2)
    \item 把string2裡面符合regex的字串globally的全部代換成string1,
    等於sed的s///g。傳回整數表示代換數目。\\
    gsub(regex, string1, string2)
    \item 傳回一個字串為string2裡面符合regex的第n個字串換成string1,
    等於sed的s///nth\_match。但是string2沒有被改變。\\
    gensub(regex, string1, nth\_match, string2)
    \end{itemize}
    string2沒給通通是\$0,nth\_match可以用g或G表示globally(全部)。請看例子

    \begin{verbatim}
    $ cat match.awk
    #! /bin/awk
    BEGIN{FS=":"}
    /cyril/{
    print $0
    print match($0, /cyril/)
    print substr($0, match($0, /cyril/), length("cyril"))
    print gensub(/cyril/, "mark", 2)
    print $0
    sub(/^.*cyril/, "mark")
    print $0
    }

    $ awk -f match.awk /etc/passwd
    lcyril:x:100:1::/export/lcyril:/usr/bin/bash
    2
    cyril
    lcyril:x:100:1::/export/lmark:/usr/bin/bash
    lcyril:x:100:1::/export/lcyril:/usr/bin/bash
    mark:/usr/bin/bash
    \end{verbatim}
    這個例子用了length()這個函數,regex不用引號,有string的地方要用
    double quote括起來,用gensub不會改變string2的值,
    而且請看第三個print \$0,這邊還是有greedy的效應在,所以只剩下後面一截。

    \subsection{輸入輸出處理print(f)}
    輸出上來說print是很簡單的用法,不過有更好的format輸出,就像c裡的printf
    或者fortran一樣,不過script語言對於字串與數字的定義跟處理是會隨著不同
    語言有所不同這是很傷腦筋的。在awk裡面也是一樣,不同的awk會不太一樣,這
    要小心點。這也是所謂type strong語言跟type 不strong語言的差別。
    \begin{verbatim}
    printf()

    用法跟c函數一樣
    {printf("The sum on line %s is %d.\n", NR, $1+$2)}
    %s    表示一個字串
    %d    表示一個整數
    %n.mf 表示一個n個整數m個小數的浮點數
    %05d  表示為5位數整數,不滿五位數,則用0填滿
    \end{verbatim}
    但是在使用上有個很大疑問是\$1\$2這些field變數是字串還是數字,
    1234是字串還是數字,加括號的"1234"是字串還是數字呢。16進制的
    abcdef與"abcdef"又怎麼區別呢,還有由於awk變數是不需要init的,
    所以在code裡面忽然出現的abcdef是否代表了一個awk的變數呢,
    \begin{verbatim}
    seq 5 | awk 'BEGIN{var=5}{printf("%0"var"d\n", $1); }'
    \end{verbatim}
    gawk在後來做了區別,但是跟傳統的有了不一樣。
    \begin{verbatim}
    mawk
    echo ef | mawk '{ printf("%04d\n", "0x"$1) }'

    gawk
    echo ef | gawk --non-decimal-data '{ printf("%04d\n", "0x"$1) }'
    或者用上gawk特殊function strtonum
    echo ef | gawk '{printf("%04d\n", strtonum("0x$1"));}'
    \end{verbatim}
    這表示field 1是個hex轉成decimal後,不足10個digit的前面補上leading 0。
    在C的literal裡,0x在前面可以表示是一個16進制數字,可是原本的awk沒有這樣。
    而且你也不知道說0x或者x是否是個awk變數,複雜的資料結構對於這種script
    是很麻煩的。所以下面情況結果是差很大的
    \begin{verbatim}
    echo ef | mawk '{ printf("%04d\n", 0xef) }'
    echo ef | mawk '{ printf("%04d\n", "0xef") }'
    echo ef | mawk '{ printf("%04d\n", "0x"$1) }'
    echo ef | mawk '{ printf("%04d\n", "0x$1") }'
    echo ef | mawk '{ printf("%04d\n", 0x$1) }'
    echo ef | mawk '{ printf("%04d\n", 0x"$1") }'

    echo ef | gawk '{ printf("%04d\n", 0xef) }'
    echo ef | gawk '{ printf("%04d\n", "0xef") }'
    echo ef | gawk '{ printf("%04d\n", "0x"$1) }'
    echo ef | gawk '{ printf("%04d\n", "0x$1") }'
    echo ef | gawk '{ printf("%04d\n", 0x$1) }'
    echo ef | gawk '{ printf("%04d\n", 0x"$1") }'
    \end{verbatim}

    \subsection{亂數rand()}
    產生介於0 到 1之間的亂數
    \begin{verbatim}
    BEGIN {
    srand(systime())
    random_num = rand()
    print random_num
    }
    \end{verbatim}
    systime()傳回從1970, 1, 1到現在的秒數 srand()產生一個虛擬的亂數陣列以供
    rand()使用
    \subsection{系統system()}
    這個很像Unix的system()這個函數
    \begin{verbatim}
    BEGIN {if (system("ls -l")) != 0) {
               print "command failed"
           }
    }
    \end{verbatim}
    其他的像數學函數等等就不加以介紹了

