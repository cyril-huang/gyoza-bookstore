    \chapter{perl}
    相對於sed, awk, shell script 等工具,當初 Larry Wall 覺得要發明一種可以取代
    這些雜七雜八 script 泛通用型的 script 工具,比起所有的 script 還要 power,
    處理起regular express也更方便,解決了一些如greedy的問題。perl 最後跟
    system call 的連結與檔案, process,socket program, OO 等等的處理,讓他變成很
    強大的 script 語言。
    \\\\
    perl 跟其他不管 shell, TCL/expect...等 script 語言其實還是用 C 寫出來的,
    每一種 script 解譯器其實都還是一個執行程式,只不過這個解譯器懂得一些內定語法,
    會去解釋這些規則然後轉換這些script變數,控制迴圈等等變成自己的 c 變數跟迴圈
    (當然啦!其實最後還是都是 cpu 內的暫存器跟迴圈)來執行。就跟 shell, awk 一樣。
    \\\\
    perl 其實真的博大精深,man page 裡面的文件就相當的棒了。在/usr/bin, 
    /usr/lib/perl, /usr/share/perl 下也有很多很棒的人家寫的perl,在 debian 中,
    /usr/lib/perl 表示裡面的模組會跟機器有關例如 tty,/usr/share/perl 表示裡面
    是platform-independent 的模組。從裡面可以學到很多。
    \begin{itemize}
    \item man perl
    \item man perldata
    \item man perlop
    \item man perlretut
    \item man perlboot
    \item file /usr/bin/* \verb=|= grep -i perl
    \end{itemize}
    \section{基本語法與資料型態}
    \subsection{資料型別}
    以 c 來說,我以為只有兩種資料型別,一種是 primitive type 就是int, char, ...
    還有就是 pointer type 對記憶體的處理。perl 是種 script 語言,有三種基本型別
    \begin{itemize}
    \item scalar 字串數字物件等等 也是基本的變數使用
    \item array 就是標準傳統 array,也叫做 list,正式文件中都用 list.
    \item hash 就使用者角度看沒什麼,就是index 是有意義的字串(也就是 hash 的 key)
	  的array.
    \end{itemize}
    分別用
    \begin{itemize}
    \item \$var
    \item @var
    \item \%var
    \end{itemize}
    \subsection{literal}
    literal表示是一個定義集合,用來表示合法的表現方式。
    例如
    \begin{verbatim}
    $var = ""; $var = '';
    @var = ('a', 'b' , "c" , "d");
    $var[1] = 'a';

    %var = ('key1', 'value1', 'key2', "$replace_value2");
    %var = (key1 => 'value1', 
            key2 => "$replace_value2"
    );
    $var{'key1'} = 'value1';
    @var = (1..100);

    print $var[1];
    print $var{'key1'};

    use constant FALSE	=>	0;
    use constant TRUE	=>	1;

    \end{verbatim}
    這裡面的"value", 1..100 等表現方式就是 perl 為它的 scalar, array 定義的 literal
    也就是像在shell裡面的' "等這些意義,這邊有個規則就是單一變數永遠都用\$var
    來表示,所以array內的變數是\verb=$var[1]=或者\verb=$var{'key'}=
    其中單引號雙引號跟shell一樣,就是單引號內的東西不會做代換。\\\\另外我想
    perl 把string, array 等等物件的作法跟 C 的 struct一樣 C++ 的 class 一樣。
    assignment 一個物件變成是copy 所有member到新物件上去也不用理會記憶體問題。
    當然 copy 的方法效率比較慢,但是語法比較適合初學者吧。

    \subsection{quote}
    跟shell一樣,single quote, double quote也是有所不同的。
    single quote 也是一樣裡面的變數不做代換,double quote 一樣作代換還有backslash
    \verb=\=也是有特別意義,如果用上regular express 
    \verb=\w \s \d=這些東西如果是放在變數裡面,必需用 single quote,
    不然會得不到想要的結果。所以perl裡面也有eval可以用

    \subsection{String, Array and Hash處理}
    大部份的script好用就是不用管pointer,其中最重要的就是字串的處理了。這也是
    大部份寫一般程式的人要面對處理的。perl是這個中好手。
    陣列中我們最常要的是陣列長度 \$\#array 是最後一個index數字,所以長度是
    \$length = \$\#array + 1;
    \begin{itemize}
      \item length(string)	傳回字串長度
      \item join(string, @array)  把一個array用字串string作連結傳回新字串。
      \item split(regex, string)  把一個字串string用regex表示切成陣列或hash。
      \item keys(hash)		傳回一個hash中所有的keys的陣列。
      \item push/pop/unshift/shift陣列的pop/unshift
      \item delete		array/hash的delete
      \item scalar		傳回陣列長度。
      \item grep                搜尋陣列元素。
      \item map                 用一個陣列經由map運算變成另一個陣列。
    \end{itemize}
    其中split相當神奇,例如字串可以展成hash
    \begin{verbatim}
    my $string = "1:one;2:two;3:three";
    my %hash = split /[;:]/, $string;
    \end{verbatim}
    則hash會自動變成一個如下hash
    \begin{verbatim}
    { '1' => 'one', '2' => 'two', '3' => 'three' }
    \end{verbatim}
    用map可以轉換某個陣列到另一個陣列,原先陣列元素在map中變成\verb=$_=,加以
    操作後,就每個操作完結果是另一陣列之元素,例如
    \begin{verbatim}
    %chksums = (
        'file1' => '3f53afcba16980402f21f167ff826785',
        'file2' => 'ef5cf88a49c3a2910d17ac231fbfc48b'
    );
    $chksumstr = join(",", map { "$_:$chksums{$_}" } keys %chksums);
    print $chksumstr;

    file1:3f53afcba16980402f21f167ff826785,file2:ef5cf88a49c3a2910d17ac231fbfc48b,
    \end{verbatim}
    很是方便。

    \subsection{reference}
    物件導向中,有個重要的reference,他的廣義定義是指參照一個物件的索引,
    也就是這個物件的ID, 例如一個人物件的參照可以是一個身份證號碼,可以
    是名字等等。在程式理論中,就是要能代表這個物件就好。在一般local電腦
    的實作中,就通常是記憶體位址,代表一塊記憶的物件。在網路程式架構,
    CORBA,SOAP等等就可能只是在此系統下的一個數字號碼。
    \\\\
    在perl5.0後對物件或像array這種物件存取時提供了這樣的位址
    reference使得struct或巢狀的hash等複雜資料結構有了更好的表示方法。
    其實這實在很有趣,很多script就是當初不要pointer這樣的東西,因為對於剛學電腦
    的人覺得很麻煩,可是最後越來越複雜的程式就一定要有這樣的結構。
    \\\\
    用\verb=\=就像用c裡面的\verb=&=傳回一個像pointer的reference。
    \begin{verbatim}
    $scalar_ref = \$scalar;
    $array_ref = \@array;
    $hash_ref = \%hash;
    $func_ref = \&function;
    \end{verbatim}
    不同於傳統的array或hash用()建造方法,新的建造literal傳回一個reference
    \begin{verbatim}
    $array_ref = [1, 2 ,3];
    $hash_ref = { APR => 4,
                  AUG => 8 };

    $func_ref = sub { my $code = 1; }
    \end{verbatim}
    所以
    \begin{verbatim}
    $array_ref = [1, 2, 3];
    等於
    @array = (1, 2, 3);
    $array_ref = \@array;
    \end{verbatim}
    所以反操作的operator
    \begin{verbatim}
    @array = @{$array_ref};
    %hash = %{$hash_ref};
    \end{verbatim}
    只是要記住,\verb=@var或者%var=都是創造一個新var物件,=都是copy所有成員。
    reference取值的方法就像c裡面的struct pointer取法用\verb=->=
    \begin{verbatim}
    @a = ( [1, 2, 3],
           [4, 5, 6],
           [7, 8, 9]
         );

    $a[1]->[2] 跟 $a[1][2]是一樣的
    \end{verbatim}
    因為這是個array reference所以用\verb=$a->[]=的方法,如果是
    hash reference就要用\verb=$a->{}=,如果是function reference
    就要用\verb=$a->()=。所以callback function可以用傳reference來作。
    以下是從texi2html看來的code
    \begin{verbatim}
    $T2H_DEBUG = 0;
    $T2H_OPTIONS -> {debug} =
    {
        type => '=i',
        linkage => \$main::T2H_DEBUG,
        verbose => 'output HTML with debuging information',
    };
    \end{verbatim}
    所以
    \verb=T2H_OPTIONS=
    這個reference裡面有個debug這個變數同時又是個reference, 裡面又有
    type, linkage 和verbose這3個key,其中type, verbose是個字串,linkage又是個
    reference。
    \\\\
    由於reference只是一個數字,想要知道這個變數是什麼樣的資料型別可以用函數庫
    ref或者Scalar::Util::reftype來取得。傳回值是字串,"SCALAR", "ARRAY", 
    "HASH"與"CODE"。或者UNIVERSAL::isa來比較也可以。但是只能用在reference跟
    物件上,無法用在一般\verb=$ @ %=變數上。
    \\\\
    reference跟傳統的array, hash, string的assignment operator不太一樣,因為已
    經有pointer的概念了,不是物件概念。所以=的assignment,不能保證有copy。而是
    assign一個reference, 在implementation上就是記憶體位址。

    \subsection{內定變數與special literal}
    看man perlvar可以看到所有的內定變數不過有些比較重要常用的。這邊有的跟下面的
    function或者regex或者scope有關。先看看
    \begin{itemize}
    \item @ARGV 這是程式開始的args, 跟C不一樣的是\verb=$ARGV[0]=
	  是第一個argument.
    \item \$0	這是程式名
	    
    \item @INC	這是package的include path,在use, require, 這樣的library的搜
		尋路徑。
    \item @ENV	這是環境變數,例如取用時PATH時用\$ENV\{'PATH'\}
    \item @SIG	signal handler的陣列,指定時用\$SIG\{'INT'\} = "func\_name";
    \item STDIN	 標準輸入file stream
    \item STDOUT 標準輸出 stream
    \item STDERR standard error stream

    \item @\_	在函數中傳進來的parameters
    \item \$\_  超級重要的內定的輸入與RE pattern內定搜尋的地方,
	    例如一些莫名的像if(1..100)的控制值都是在測試\$\_。
	    很多時候一些operator在程式碼
	    裡面常會覺得到底在對誰作用阿,就是這個perl內部的內定變數,
	    例如print不寫print甚麼其實就是print \$\_ 。

    \item \verb=$|=  file buffer flush 的狀態。\verb=$|== 1;表示立刻寫入硬碟。
    \item \$\$	process id

    \item \verb=__FILE__=	    檔案名
    \item \verb=__PACKAGE__=	    PACKAGE名
    \item \verb=__LINE__=	    目前行數
    \end{itemize}

    \subsection{條件控制}
    比較字串用ne, lt, gt, eq, ge, le,比較數字用符號 \verb.!=, <, >, ==, >=, <=.
    \begin{verbatim}
    if (condition) { } elsif { } else { }

    while (condition) {}

    until (condition) {}

    unless {}(condition)

    for($i = 0; $i < $max; $i++) { print $i; }

    foreach $element (@array) { print $element; }

    while (condition) { last LABEL if (condition); }
    break LABEL ;
    \end{verbatim}
    基本上perl沒有switch case,主要是switch case可以用if elsif else達成,但
    最重要的因素是其實這可以宣告一個hash, 利用hash快速的取得相對應的處理。
    \\\\
    condition中比較重要的是
    \begin{itemize}
    \item undefined 變數的判斷 \hfill \\
    	可判斷,如果變數第一次出現在condition,傳回false
    \item undef常數的判斷 \hfill \\
    	可判斷,傳回false
    \item null的判斷 \hfill \\
    	不可判斷,沒有null,傳回true
    \item true/false的判斷 \hfill \\
    	不可判斷,沒有true/false,傳回true
    \item 空字串的判斷 \hfill \\ 
	可判斷,空字串傳回false
    \item 空白字元的判斷 \hfill \\
	可判斷,空字元傳回false
    \item 空陣列的判斷 \hfill \\
	可判斷,數字0傳回false
    \item 數字0, 1的判斷 \hfill \\
	可判斷,數字0傳回false
    \item 字串0, 1的判斷 \hfill \\
	可判斷,字串'0'傳回false
    \end{itemize}
    這也是大部份寫作script時要小心的,例如在javascript中的undefined就很頭大囉
    。主要是perl內部沒有null跟true/false的保留字,所以這些不能作為常數判斷。
    但undef, "", "0", 0等可以作為判斷。
    \\\\ 
    另外迴圈控制的 continue 跟 break 與 C 的用法不太一樣,可以靠 LABEL 跟 last, next
    的運用做出一些跳躍的 goto,不過亂跳還是蠻亂的。last 就是 C 的 break; next
    就是 C 的 continue。 也有 goto label 跟 C一樣使用。
    \subsection{operator}
    除了一些傳統或C標準的operator外perl有些operator比較奇怪的,可以看man perlop
    \begin{itemize}
    \item \verb@=~@	    REGEX找到後,assign operator
    \item \verb=~!=	    如果不等於這個REGEX
    \item \verb=<>=	    stream read, \verb=<STDIO> <$file_handler>=
    \item \verb=<<=	    這跟C的\verb=<<=一樣,perl也能處理bit.當然也有\verb=>>=囉
    \item .	    連結兩個字串
    \item ..	    range operator,例如(1 .. 100,'a'..'z','01' .. '10')
    \item ...	    跟..一樣,只是..運作是先測左邊再測右邊...不去測右邊的boolean
    \item \verb@<=>@	    分別還回-1, 0, 1如果左邊數值小於等於大於右邊。
    \end{itemize}
    quote operator
    \begin{itemize}
    \item q(string)	這是single quote \verb=''= 字串裏面變數不展開,其他不變。
    \item qq(string)	這是double quote \verb=""= 字串裏面變數會展開,其他不變。
    \item qx(command)	這是執行一個命令後,得到的string加quote
    \item qw(string)	這是用space split一個string後,全部加個single quote.
    \item qr	        這是quote regex,裡面是regex.
    \item \verb=``=	這跟shell的\verb=``=一樣,執行一個外部命令把結果傳回。
    \item m		pattern比對, match
    \item s		pattern代換, substitution.
    \item tr		truncate,去掉不要的字元
    \item \verb=<<EOF=  跟shell 很像的使用 \verb=<<= 方法來quote所有的字元。
    \end{itemize}
    
    \section{系統程式}
    系統程式是跟作業系統有關的I/O, multiprocess之類的,在perl中也通常用unix
    system call的方法在perl上implement類似的API而已。
    \subsection{一般檔案I/O控制}
    跟一般用shell的語法很像。
    \subsection{簡單的檔案I/O}
    Read
    \begin{verbatim}
    open my $file_handle, "< foo";
    $line_content = <$file_handle>;
    close $file_handle;
    \end{verbatim}
    Write
    \begin{verbatim}
    open my $file_handle, "> foo";
    print $file_handle "content";
    close $file_handle;
    \end{verbatim}
    所以就是shell的io導向的符號來決定開檔的屬性。很簡單
    然後\verb=<>=operator是拿出一行來。放在內定變數\$\_,如果沒有file handle的
    話,那就是從stdin拿一行出來。所以把檔案中有alibuda的行找出來的小perl。
    \begin{verbatim}
    $ cat find_alibuda.pl
    #!/usr/bin/perl
    while (<>) {
      print if m/alibuda/;
    }

    在shell 中執行
    $ perl find_alibuda.pl xxx.txt
    \end{verbatim}
    比較注意的是print時,不需要逗號來區隔file handle與內容。不過說實話,後來
    我用 shell/sed/awk 用習慣了,覺得這些開檔讀檔寫檔還真沒有shell方便。

      \subsection{檔案與process}
      perl也有fork,一般fork出的child process會繼承parent的一些特性,在這裡面最
      重要的就是file descriptor跟file offset是跟parent一樣的。這有兩個問題,一個
      是stdin/stdout/stderr被shared了,一個是file redirect的問題。在perl中可以跟
      shell 一樣很簡單的用backquote \verb=`cmd`= 執行命令,也有fork()可以用。
      \begin{verbatim}
$pid = fork()
if ($pid) {
    # parent
    builder::logger("forked child for package $package pid=$pid");
    $children{$pid} = $package;
} elsif ($pid == 0) {
    # child
    exit buildPackage_childMain($bt, $package);
} else {
    builder::printer("ERROR - fork child for package $package failed");
    $error_build = 1;
    goto cleanup;
}

while (waitpid(-1, 0)) {
    builder::logger("Return $? from build package $children{$pid} : $pid");
    if ($pid == -1) {
        builder::logger("All components built are done");
        last;
    } elsif ($?) {
        builder::printer("ERROR - build of package $children{$pid} failed");
        $error_build = 1;
        goto cleanup;
    }
}
      \end{verbatim}
      fork與waitpid的用法跟system call用法幾乎是一樣。

      
      \subsection{IPC}
      \subsubsection{mmap}
      mmap是最簡單的fork出的兩個相關process共同可以access一段memory內容。
      但mmap只能是有相關的爸爸小孩間的memory sharing。要真正不同process間的溝通
      ,使用 POSIX IPC,也可以使用SVR4 IPC,但後來POSIX出了比較light的API後,
      個人是比較喜歡POSIX的lock/unlock的方法,比較直觀。
      \begin{verbatim}
      use Sys::Mmap;
      mmap($foo, 0, PORT_READ, MAP_SHARED, FILEHANDLE) or die "mmap: $!";
      $foo = "alibuda";
      munmap($foo);
      \end{verbatim}

      \subsubsection{signal}
      signal handler使用就是把SIG這個hash填進handler名字。
      \begin{verbatim}
sub reaper {
      $waitpid = wait;
      $SIG{CHLD} = \&reaper;
}
$SIG{CHLD} = \&reaper;
或者
$SIG{CHLD} = "reaper";
      \end{verbatim}
      \subsubsection{posix IPC}
      \begin{verbatim}
      use POSIX::RT::Semaphore;
      $sem = POSIX::RT::Semaphore->init(0, 1);
      $sem->wait;   # down() / P()
        ... do something
      $sem->post;   # up() / V()

      if ($sem->trywait) {  # non-blocking try P()
        ... do something
        $sem->post;
      }
      \end{verbatim}

    \section{更power的regular expression}
    RE的3個主要功能 
    \begin{itemize}
    \item matching比對
    \item substitute代換
    \item extract擷取
    \end{itemize}
    \subsection{多一點點的新字元}
    \begin{verbatim}
    \d  數字        =  [0-9]
    \D  非數字      =  [^0-9]
    \w  字元        =  [0-9a-zA-Z_] 多一個底線_
    \s  空白        =  [\ \t\r\n\f]
    \W  非字元      =  [^\w]
    \S  非空白      =  [^\s]
    \b  定位字元    =  \b\b夾住words來區隔像cat category的搜尋
    .   除了\n的任何字元
    \Q	脫逃        = single quote的意思 一直到\E的字串都不是pattern
    \E
    \end{verbatim}

    \subsection{比對 - match}
    比對的重點在於有沒有比到,傳回是boolean,有下列幾個重點
    \begin{enumerate}
    \item 比對pattern m//  由m跟pattern的delimiter識別字元組成。m通常不寫,
	  delimiter識別字元通常是//。
    \item 如果光寫比對pattern在condtion內,自動比對\verb=$_=變數。
    \item 想要跟某個變數比對用比對operator \verb@=~ 跟 !~。$var =~ /regex/@。
    \item 可以改變delimiter,在比對pattern中直接改m@re@就可以用@當delimiter
    \item 可以加modifier在後面m//modifier,
	    \begin{enumerate}
	    \item modifier s表示 . 也可以代表\verb=\n=, s有single line的
	          意思。
	    \item modifier m表示 \verb="^" 跟 "$"=代表任何在頭跟尾巴的串接
	          下一行, m有multiline的意思。
	    \end{enumerate}
    \end{enumerate}

    使用例子
    \begin{verbatim}
$x = "There once was a girl\nWho programmed in Perl\n";

       $x =~ /^Who/;   # doesn't match, "Who" not at start of string
       $x =~ /^Who/s;  # doesn't match, "Who" not at start of string
       $x =~ /^Who/m;  # matches, "Who" at start of second line
       $x =~ /^Who/sm; # matches, "Who" at start of second line
\end{verbatim}
    常用搜尋
    \begin{itemize}
    \item 空白行 \verb=/^$/=
    \end{itemize}


    \subsection{代換}
    代換是比對到我們要的然後換成我們想要的字串。底下是幾個重點
    \begin{enumerate}
    \item 代換pattern s///  由s跟pattern的delimiter識別字元組成。
	  delimiter識別字元通常是//。
    \item 如果光寫比對pattern在condtion內,自動比對\verb=$_=變數。
    \item 想要跟某個變數比對用比對operator\verb#=~ 跟 !~。\$var =~ s/regex//#。
    \item 不保留原本變數字串,並替換一個新字串成為一個新變數。用(\$new) = (\$old =~ s///;)
    \item modifier
	    \begin{enumerate}
	    \item modifier g表示global,所有符合的都換。
	    \item modifier i表示不分大小寫。
	    \end{enumerate}
    \end{enumerate}

    例子
    \begin{description}
      \item[幹掉字串前後空白] \hfill \\
      \verb=for (@string) {s/^\s+//;s/\s+$//;s/\s+/ /g;}=
      \item[取代所有陣列裡面的元素] \hfill \\
      \verb=s/big/large/g for @all_element;=
      \item[不改變舊有變數將取代結果指給新變數] \hfill \\
      \verb|($new = $original) =~ s/big/large/g;|
    \end{description}

    常用的代換
    \begin{description}
      \item[幹掉空白行] \hfill \\
      \verb=s/^\s*\n//mg= 這邊要注意的是雖然\verb=/^$/=找得
          到空白行,但是\verb=s/^$//=的代換是不會換掉任何的特殊字元所
	  代表的意義。也就是說\verb=^$=怎麼樣就是不會被s換掉。
    \end{description}


    \subsection{擷取}
    擷取是比對到我們要的,把他抓下來變成一個新變數。
    \begin{enumerate}
    \item 用predefined 變數獲得的
	\begin{enumerate}
	\item \verb=$'=	在比對前面的字串
	\item \$\&	比對中的字串,這很像其他的script中的\&
	\item \verb=$`=	在比對後面的字串
	\end{enumerate}
    \item 用括號括起來的\$1\$2\$3,跟sed \verb=\1\2\3=是一樣的。
    \end{enumerate}
    例子
    \begin{verbatim}

    $x = "There once was a girl\nWho programmed in Perl\n";
    $var = ($x =~ /^Who/m);
    print "$&\n";

    會得到Who

    ($var1 $var2 $var3) = ($search =~ /re(re1)re(re2)re(re3)/);
    @$var = ($search =~ /re(re1)re(re2)re(re3)/);

    對應到括號中找到的3個字串分別是 $1 $2 $3,跟sed用\1 \2 \3是一樣的道理。
    \end{verbatim}
    例如比對ipv4的位址對不對
    \begin{verbatim}
    if ($ip =~ /^(\d{1,3})\.(\d{1,3})\.(\d{1,3})\.(\d{1,3})$/ &&
        ($1<=255 && $2<=255 && $3<=255 && $4<=255)) {
        print "correct";
    }
    \end{verbatim}
    常用的比對
    \begin{enumerate} 
      \item 最後一行 \verb|($last_line) = ($multiline =~ /([^\r?\n]+)$/);|
      \item 取得最後一個\verb|$a='this is my world'; ($x) = ($a =~ /([^\s]+)$/);|
      \item 把目錄加上斜線 \verb|$dir =~ s/(.*\w)$/$1\//g;|
    \end{enumerate}

    \subsection{變數與pattern}
    我們除了自訂的已知pattern外,其實常常會用一個變數來作pattern match,
    已知pattern的好處是他在perl compile time時就已經決定,如果是變數
    qr//可以有效的解決
    \\\\
    還有一種情形是我們不要變數裡面的字串是pattern,例如 . ? * \verb=[ ]=等字元,
    我們可以加上\verb=/\Q$variable\E/=,這樣則variable會先被轉成變數內的字串,
    然後這整個字串不會有pattern上的意義,例如*就是*。
    \subsection{greedy RE的處理}
    greedy and non-greedy的例子在介紹RE時提到,可是一般是沒有辦法處理的。
    例如
    \begin{verbatim}
    <t1>text</t1><t2>text</t2> <t3>text</t3>
    \end{verbatim}
    perl可以多加一個 ? 在一個 re後面表示non-greedy的RE搜尋。
    例如\verb=<.*?>= 這樣的RE就可以找到\verb=<t1>=而不會找到
    \verb=<t1xxxxx...../t3>=
    \\\\
    另外在\verb=[]=的字元使用也有greedy的限制,例如有可能出現以下的pattern
    \begin{verbatim}
    xxxxxxixx
    xxxxxxuxx
    xxxxxxixx,uxx
    \end{verbatim}
    那我們如果用\verb=/(.*)[ui].*/= 則第三種 pattern 的 \$1 會是xxxxxxixx,
    的greedy match.不會只是xxxxxx。

    \subsection{使用perl command line}
    而當系統上有perl時而sed/awk沒有支援非greedy時,可以使用 perl -e來做命令列的
    使用,或者很多情況下使用 perl 內部的豐富函式幫我們做些簡單快速的檢查,計算等
    例如密碼的crypt應用, 我們可以在 shell 中馬上傳參數叫出 perl 一行就搞定很多複
    雜的使用。比sed -e 多出 -ne, -pe 等用法。 
    \begin{verbatim}
    perl -e 'print "Hello World\n"'
    \end{verbatim}

    load 模組使用-M
    \begin{verbatim}
    perl -MLWP::Simple -e 'print head "http://www.example.com"'
    \end{verbatim}

    loop檔案使用-ne 與 -pe
    \begin{verbatim}
    perl -ne 'perl code' file
    
    其實是

    LINE:
    while (<>) {
       perl code
    }
    \end{verbatim}

    -pe 是每次會自動印出\$\_
    \begin{verbatim}
    perl -pe 'perl code' file

    其實是

    LINE:
    while (<>) {
      perl code
    } continue {
      print or die "-p destination: $!\n";
    }
    \end{verbatim}

    所以如果要跟shell作用,可以用
    \begin{verbatim}
    echo xxxx | perl -pe 's///'

    例如xml的取值,不管white space與"跟'的使用

    echo '<Property   oe:key="DNSIp"  oe:value =10.8.8.8/>' |
      perl -pe 's/.*Property +oe:key *= *['"'"'"'"]?DNSIp["'"'"'"']? +oe:value *= *['"'"'"'']?(.*?)['"'"'"'']? *\/>.*/$1/'
    \end{verbatim}
    所以使用 -pe 時,是把每次讀進來的\$\_,做s///,然後自動perl會印出在stdout上。
    \\\\
    跟 sed -i 一樣的 perl -i
    \begin{verbatim}
    perl -i -pe 's/\bPHP\b/Perl/g' file.txt
    \end{verbatim}

    \section{function,模組與OO}
    每一種語言工具一定都有提供副程式用法,perl也不例外,每個function都有
    arguments, return value 還有scope的問題,最後perl也有OO的解決方案的。
    \subsection{基本function}
    定義
    \begin{verbatim}
    sub subfunc
    {
        my $fh = 'xxx';
    }

    \end{verbatim}
    呼叫用 subfunc()或者 \&subfunc()。內定傳進來的參數用 \verb=@_=這個陣列來,所以
    \verb=$_[0]=...是第一個參數。
    \begin{verbatim}
    sub maybeset
    {
        my($key, $value) = @_;
        $Foo{$key} = $value unless $Foo{$key};
    }
    \end{verbatim}
    函數return可以是任何物件,array, string...等等在傳統上需要pointer來表達的。
    這跟C的struct很像。這也是C++的call by reference處理。如果
    \begin{verbatim}
    @var = func();
    \end{verbatim}
    不用考慮pointer問題,會把整個array的member一一填上相對應的值。這到object時
    ,還是一樣成立的(其實這邊的array就是一種新定義的object),不用考慮pointer的
    scope問題。這其實是C/C++的assignment。當然這比較耗時間,比起pass pointer填
    值,還回pointer沒有效率。所以如果真有performance考量,可以用
    call by reference,傳值時用\verb=\$var, \@arrary, \%hash=來傳值。
    \\\\
    而callback function由於一定是傳reference,所以就用reference的呼叫方式,
    \verb=func(\&callback, $param0, $param1) { $arg->($param0, $param1); }=
    一般函數的參考請看man perlfunc。

    \subsection{module, package, library}
    要將一堆寫好的函數兜成一個library檔案用package包好作成library。以下是個例子
    \begin{verbatim}
    package Foo;

    our $var, $bar;
    $var = 20;

    sub func1 {
        print $var;
    }
    \end{verbatim}
    其實package真正的講法是定義一個namespace(命名空間),這檔案下的變數宣告在這
    namespace下有效。如果是module的perl script,副檔名就用pm,這在
    /usr/share/perl, /usr/lib/perl/下面看到。
    \\\\
    對於程式的再利用,將已寫好的module包含進來,使用下面方法
    \begin{enumerate}
    \item do 'module.pl'
    \item require module\_name;
    \item use module\_name "name1 name2";
    \item eval xxx.pl; 當然也可以用 eval 像是shell裡的eval "xxx.sh"
    \end{enumerate}
    這裡面是根據@INC這個內定變數下的目錄再分類而已。do是等於eval "cat
    module.pl',即使之前有do過這檔案了,他會重複的執行,而且直接引用檔名,
    所以如果是想要include同目錄下的module,可以用do去蓋掉原本的定義,
    。require是比較像library的用法了,
    其中use等於BEGIN { require Module; import Module LIST; },BEGIN是指在
    compile time時,就會馬上被執行。如果已經有include了,不會include兩次。
    作為perl library檔案,必須return true,表示整個檔案import成功,所以通
    常會看到這種檔案最後是"1;"。現在比較多用require/use,但由於use其實是包在
    begin的,所以動態的eval內只能使用require,不能用use。當我們要
    程式去搜尋library時,可以把path加在@INC或者加在PERL5LIB這個環境變數中。
    用其他檔案中的變數與函數。有用double::或者單一quote'的用法。
    \begin{itemize}
    \item \$Foo::var
    \item \&Foo::func1()
    \item \$Foo'var
    \item \&Foo'func1()
    \end{itemize}
    這個還要跟Export搭配一起回答。請看下一章。

    \subsection{Typeglobs}
    在一個package內部的namespace中,名字有可能重複,例如\$foo, \%foo, @foo
    \&foo,這在perl內部是以特別型別所謂typeglobs來存取所有同名的不同資料型別
    資料。是以*foo來表示的。也就是*foo可以同時是上面四種型別的代表。因此*foo
    可以代表任意的一種資料型別變數,甚至也拿來當作constant用,
    \\\\
    Constant
    \begin{verbatim}
    *TRUE = \1;
    *FALSE = \0;
    \end{verbatim}
    Filehandles
    \begin{verbatim}
    \*STDERR

    \end{verbatim}
    Class Accessors
    \begin{verbatim}
for my $field (qw(member1 member2 member3)) {
    my $slot = __PACKAGE__ ."::$field";
    no strict "refs";

    *$field = sub {
        my $self = shift;

        $self->{$slot} = shift if @_;
        return $self->{$slot};
    }
}
    \end{verbatim}
    上面的 accessors 是 class 內有 member data, member1, member2, member3,
    如果使用了 class\verb=->=member1(),則只會傳回 member1 的值,如果有參數,
    就會設值。
    \subsection{變數scope與多檔結構化程式}
    正常的script是應該沒有變數scope等問題的,不過perl越來越強大,也就有了這樣
    的考量。也就是變數在副程式裡面應該是彼此看不到對方才對。正常perl一起來內定
    的目前檔案甚麼都不寫就是main這個內定package。perl提供了
    \begin{enumerate}
    \item my 宣告在一個subfunction {}, loop block {}, 一個perl檔案(package), 
	    及一個eval內的私有變數。這其實就是C裡面的auto變數,是用另一個stack
	    的。所以會藏在block內。這像C的static變數 
	    這很像C的static變數,只屬於某一個file的scope.
	    例如counter之類的
	    \begin{verbatim}
            for my $i (1, 2, 3) {
                some_function();
            }
	    \end{verbatim}
    \item local 只是很多人被local這個名字搞混了,其實想要像c那樣的變數scope應
	    該用的是my這個東西。local只是一個不屬於任何package的暫時global變數。
	    好像在繞口令。這是歷史的產物,也就是當初寫perl時是很隨性的,不用
	    宣告的,為了同一個檔案中的namespace,在perl4.0時出現了local這樣的
	    block scope變數,等到後來需要更模組化的變數scope時,在5.0以後
	    my取代了local.
    \item our 這個宣告變數是跨檔案的也就是跨package libarary,這很像C中的
	    extern,如果這個變數要在別的package改變了也有效。那就要用our.
    \end{enumerate}
    當要嚴格的規定變數的scope的時候,我們會加上
    \begin{verbatim}
    use strict;
    \end{verbatim}
    這樣的開頭,強制請perl幫我們檢查所有的變數要規定屬於那一個package。
    也就是所有的變數都要明確掛上my來。底下為傳統local用法,這是沒有用
    use strict;的時候。有用use strict;時,沒有local。
    \begin{verbatim}
    $my_var = "my_var\n";
    sub func {
            local($my_var) = "func_local_my_var";
            print "$my_var\n";
    }
    &func;
    print $my_var;
    會印出
    func_local_my_var
    my_var
    表示block裡外不受影響。
    \end{verbatim}
    新的use strict;要求一定要用my來作namespace的區隔。不能甚麼都不宣告的
    使用一個變數,用local時一定要指明是那一個package的變數。
    \begin{verbatim}
    use strict;
    package mainpkg;
    my $my_var = "my_var";
    local $other::my_var;
    sub func {
            my $my_var = "func_local_my_var";
            print "$my_var\n";
    }
    &func;
    print $my_var;
    \end{verbatim}
    加上別的檔案的使用,使得變數scope多了file(package)的使用scope. 這時要用
    our宣告變數才能別的檔案可以更改。my反而只能算是static的file scope了。
    \begin{verbatim}
    main.pl
    use strict;
    package mainpkg;
    my $my_var = "my_var";
    our $our_var;
    sub func {
            our $our_var = "main_our_var";
            print "$our_var\n";
    }

    $other::other_my_var = "assigned other my var in main package\n";
    $other::other_our_var = "assigned other our var in main package\n";
    &other::other_func
    print $other::other_my_var;
    print $other::other_our_var;

    other.pm
    package other;
    my $other_my_var = "other_my_var";
    our $other_our_var = "other_our_var";
    sub other_func {
        print "$other_my_var\n";
        print "$other_our_var\n";
    }

    執行main.pl
    結果是
    other_my_var
    assigned other our var in main package
    assigned other my var in main package
    assigned other our var in main package

    \end{verbatim}
    從這邊可以看出來
    \begin{enumerate}
    \item   用別的package的function,只要直接用package::func就可以了,可是別的
	    package的變數卻不可以直接拿來用。
    \item   但是如果別的package的變數是用our宣告的,就可以直接\$package::var用。
    \item   如果是my宣告,則別的package無法\$package::var拿來用
    \end{enumerate}
    \subsubsection{Export}
    不過每次都要這樣\$package::var, \$package::func用太麻煩了,我們想要跟C
    一樣直接使用他的名稱就好。這時必須把這個package的變數名給export出去。
    我們必須用標準的內部Exporter package, 加上EXPORT這個內定的array。
    \begin{verbatim}
    main.pl
    package mainpkg;
    use strict;
    use other;

    my $my_var = "my_var";
    our $our_var;
    sub func {
        our $our_var = "main_our_var";
        print "$our_var\n";
    }

    $other_my_var = "assigned other my var in main package\n";
    &other_func
    $other_our_var = "assigned other our var in main package\n";
    &other_func


    other.pm
    package other;
    use strict;

    BEGIN {
        use Exporter;
        our (@ISA, @EXPORT, @EXPORT_OK, %EXPORT_TAGS);
        @ISA = qw(Exporter);
        @EXPORT = qw($other_my_var $other_my_var &other_func);
    }
    my $other_my_var = "other_my_var";
    our $other_our_var = "other_our_var";
    sub other_func {
        print "$other_my_var\n";
        print "$other_our_var\n";
    }

    執行main.pl
    結果是
    other_my_var
    other_our_var
    other_my_var
    assigned other our var in main package
    \end{verbatim}
    可以看出來用my宣告的就是static的file scope。不能在其他的package更改值。
    BEGIN {}這可加可不加。這是perl一開始就執行的code。before compile time.
    因為我們用了use strict;所以還是要宣告一下our @ISA, @EXPORT這些東西。
    ISA, EXPORT, EXPORT\_OK ...這些變數是perl package的內有變數。
    @ISA表示is a "exporter",就是說other是Exporter下的一個子package,
    如果你要用到的某函數不在目前的package定義中,那麼他會自動去@ISA中的package名
    中找看看有沒有。EXPORT這個陣列放的就是要export給一般user namespace用的
    symbol名。ISA的性質可以是用來作物件繼承用的。

    \subsection{物件導向}
    要區別一個程式是不是OO,有很重要的封裝(encapsulation), 繼承(inheritance),
    多形(polymophism)三個條件。C也能做到簡單的結構化struct,但是就是因為沒有
    這三個東西的語法,因此不能作為OO的語言。其實perl也不是適合做到真正OO,只是
    後來用 package 類似結構的寫法還有一點點OO的語法子模組而已。\\\\
    perl的複雜資料結構本來就要用reference來表現所以class後來就用類似的宣告了,
    而其實OO的寫法中的implementaion就是不同的變數範圍來實作的。所以在perl中
    就要用package來進一步實作。傳統的package內的函數呼叫可以是
    package::func(@argv),新的class的函數呼叫多了package\verb=->=func(@argv);
    這等於 package::func(package, @argv);這是為了來作 constructor 用的。
    \\\\
    所以
    \begin{itemize}
    \item 一個物件一定是一個 perl reference,一個 perl reference 不一定是物件。
    \item 一個物件的 member function 所傳來的第一個參數是 class 名字。
      javascript 也是這樣。
    \end{itemize}
    因此為了要特別告訴 perl 說這是一個物件 package,則我們必須在 member function
    告訴他。
    主要要使用
    \\\\
    bless(Reference, ClassName);
    \\\\
    這個函數。這樣就是說某個變數reference是屬於某個ClassName下的。
    那這樣就宣告了某個參數是一個物件了。
    \\\\
    底下我們用一般人與系統上的使用者來作示範。
    \subsubsection{基本class宣告}
    \begin{verbatim}
檔案: Person.pm
package Person;
use strict;

sub new {
    my $self = {
        name => '',
        tel  => ''
    };
    bless($self, $_[0]);
    return $self;
}

sub name {
    my $self = shift;
    print $self->{name}."\n";
}
1;


檔案: Member.pm
package Member;
use strict;
use Person;

our @ISA = 'Person';

sub new {
    my $self = Person->new();
    $self->{name} = 'userName';
    $self->{tel} = 'telephone #';
    $self->{id} = 'uid';
    $self->{passwd} = 'passwd';

    bless($self, $_[0]);
    return $self;
}

檔案: main.pl
package Main;
use Member;
my $member = Member->new();
$member->name();

結果:
userName
    \end{verbatim}
    呼叫就是很簡單,\verb=c->f()=,在OO設計哲學裡面,
    其實不願意別的工程師直接使用變數,
    即使在簡單的得到某一個member的值,也都要用透過function call,也就是interface
    例如get\_xxx()然後return這個值。這樣的好處是程式將來對於拿值有限制時,上層
    程式不用改,直接改interface裡面的implementation就好。不過很多的function
    call是有執行效率問題的。不過執行效率與code可讀性本來就是有trade off的。
    \\\\
    Constructor就是\verb=Class->new()=;不過如果想要用\verb=$obj->new();=則必須用
    通用的class宣告。
    \begin{verbatim}
sub new {
    my $this = shift;
    my $class = ref($this) || $this;

    my $self = {};
    ...
    bless ($self, $class);
    return $self
}
    \end{verbatim}
    上面的\$self所以是個hash reference,然後自己往裡面填member data.就這樣。
    所以取用member就是跟hash一樣,\verb=$obj->{member1} $obj->method(@arg)=,
    其中bless這個function是用來告訴perl, 前個reference是後面這個class的一個 
    member.因為\verb=Class->method(@argv);=其實是
    Class::method(ClassName, @argv); 所以第一個參數就是class名。
    \\\\
    ref這個函數作用在物件上是會傳回Class名字的,所以constructor不管是那種方式
    呼叫都保證\$class是ClassName.
    \\\\
    基本上直接呼叫父class的函數名是不大好的,如果繼承很長,那想要一層層找上去
    ,應該要用\verb=$class->SUPER::func() 或者 $self->SUPER::func()=這樣子。
    \\\\
    script不用考慮甚麼memory leak,所以destructor就省了。construtor其實只是個
    function call而已愛叫new, spawn...隨便。
    \\\\
    constructor並沒有像C++所謂的default constructor,所以子class的constroctor
    必須額外的呼叫父class的constructor,通常就是\verb=parnet::new($self, @_);=。
    \\\\
    assignment operator跟C的struct或者C++的一樣,是copy member,其實這也是一般
    script的array, string..等等的作法。也就是reference的作法,把相對應參數的值
    copy 一份給 = 的左邊。沒有pointer問題,但是implement script的人有garbage 
    collection的問題。
    \\\\
    繼承我們看到是用原本package的@ISA來達到的,所以最後呼叫name()時,他會去
    Person上叫。 不過perl畢竟不是用來作OO的,這樣的OO實在很鱉腳。而且說實話
    也不用這麼複雜。
    \\\\
    所有的class其實都會自動繼承一個叫UNIVERSAL的爸爸,裡面有兩個member function
    很有用,一個是isa('MyClass'),也就是如果這個object是MyClass這個package,
    則會回傳True。一個是can('method'),如果這個object有method這個funcation call
    則回傳True。
    \\\\
    class變數的實作當然就是用our啦,可以把他Export出去,也可以另作一個class 
    method來存取class變數。class method的呼叫就是前面一般
    package::method的呼叫方法了。
    \\\\
    所以基本上Perl的所有型別的Literal可以用下面來表示
    \begin{verbatim}
my $h = { a => 1 };
my $a = [1,2,3];
my $s = \3;
my $f = \&somesub;
my $g = \*STDOUT;
my $o = bless {}, "MyClass";
    \end{verbatim}

    \subsection{CPAN的perl module安裝}
    這邊就是去 CPAN http://www.cpan.org,
    那抓下來用,這邊要注意的有一點是如果這個 perl module 是需要
    c compiler 來編譯的。那會跟原本編譯 perl 的 c compiler 有很大關係。
    當你用 perl Makefile.PL,其實他是會去抓原本 compile perl 的 cc 使用 options.
    perl 會記得當時他是怎麼被建造的,在他以後的模組都要照樣被建造。
    \\\\
    通常是在 solaris 上的 perl 用solaris 的 cc compile 的,結果 solaris 系統上的
    perl 記得一些 compiler 的選項,而不同 compiler 上的選項有的有,有的沒有,
    通常最常死掉的是
    \begin{verbatim}
    -KPIC -xdepend
    \end{verbatim}
    不管你裝了 gcc 在 solaris 機器也沒有用。像 rrdtools, 或一些模組用到 c 的
    compiler 的 driver 就會出現這樣錯誤。
    \\\\
    所以解決方法是
    \begin{enumerate}
    \item 買一個solaris cc
    \item 重新用gcc compile一個新的perl,通常去www.sunfreeware.com拿比較快。
    \item 改Makefile \verb=把cc -> gcc, -KPIC -> -fPIC,= 
		     \verb=-xdepend -> -xc or -xc++ =or 不寫
    \end{enumerate}

    模組安裝很簡單只要去www.cpan.org找,找到了通常就是解開後

    \begin{verbatim}
    $ perl Makefile.pl
    $ make
    $ make test
    $ make install
    \end{verbatim}
    就可以了,會裝到/usr/local/lib/perl ...下。
    \\\\
    另一種可以用cpan這隻程式來管理安裝cpan的模組,會自動找出dependancy
    當然這也跟rpm一樣,當初寫這些模組的人有把相關資訊放在裡面,大家遵守
    cpan的規定,所以cpan就可以根據這些資訊裝起所有的東西,他這跟bsd的port
    比較像,是要download, compile, test, install的。先用系統上的package
    裝起cpan。然後下cpan命令
    {\scriptsize
    \begin{verbatim}
    cyril@i7:~$ cpan 
    Terminal does not support AddHistory.

    cpan shell -- CPAN exploration and modules installation (v1.960001)
        Enter 'h' for help.

        cpan[1]> i /XML::Twig/
        Going to read '/home/cyril/.cpan/Metadata'
          Database was generated on Wed, 26 Sep 2012 11:19:03 GMT
          Module  < App::RSS2Leafnode::XML::Twig::Other (KRYDE/rss2leafnode-65.tar.gz)
        Module  < Test::Valgrind::Parser::XML::Twig (VPIT/Test-Valgrind-1.13.tar.gz)
        Module  < Test::Valgrind::Parser::XML::Twig::Elt (VPIT/Test-Valgrind-1.13.tar.gz)
        Module  < Test::XML::Twig        (SEMANTICO/Test-XML-0.08.tar.gz)
        Module  < XML::Twig              (MIROD/XML-Twig-3.41.tar.gz)
        Module  < XML::Twig::Elt         (MIROD/XML-Twig-3.41.tar.gz)
        Module  < XML::Twig::Entity      (MIROD/XML-Twig-3.41.tar.gz)
        Module  < XML::Twig::Entity_list (MIROD/XML-Twig-3.41.tar.gz)
        Module  < XML::Twig::XPath       (MIROD/XML-Twig-3.41.tar.gz)
        Module  < XML::Twig::XPath::Attribute (MIROD/XML-Twig-3.41.tar.gz)
        Module  < XML::Twig::XPath::Elt  (MIROD/XML-Twig-3.41.tar.gz)
        Module  < XML::Twig::XPath::Namespace (MIROD/XML-Twig-3.41.tar.gz)
        12 items found

        cpan[2]> install XML::Twig
    \end{verbatim}
    }
    使用cpan 所裝的 library 跟系統 deb/rpm 裝的會在不同地方,以我的 debian 為例子,
    他是放到 /usr/local/lib/perlxxx 下面的,但是系統的是放到 /usr/lib/perlxxx
    下面,所以如果有需要新版的某 library ,那要小心 cpan 跟系統管理是兩套的,
    必須小心將來搜尋時是以想要的 path 為優先。

    \subsection{Makefile.PL的寫作}
    必須要使用\href{http://search.cpan.org/~mschwern/ExtUtils-MakeMaker-6.64/lib/ExtUtils/MakeMaker.pm}{ExtUtils::MakeMaker}
    \begin{verbatim}
    use strict;
    use ExtUtils::MakeMaker;

    WriteMakefile(
        NAME              =>  'Quark',
        EXTRA_MODULES     =>  [
                              { NAME => 'Quark::Device'},
                              { NAME => 'Quark::Device::Ipmi},
                              { NAME => 'Quark::Device::Power}
                              ]
        VERSION           =>  '0.0.0',
        PREREQ_PM         =>  { 'Expect' => '1.2',
                              },
        EXE_FILES         =>  [ 'src/bin/run.pl' ],
        ABSTRACT_FROM     =>  'src/lib/Quark/Calc.pod',
        AUTHOR            => 'Cyri Huang <cyril@costco.net>',
    )
    \end{verbatim}

    \begin{itemize}
    \item 模組 使用NAME跟EXTRA\_MODULES來指定
    \item 可執行檔 使用EXE\_FILES來指定
    \item 相依package 使用PREREQ\_PM來指定
    \end{itemize}
    \section{coding sytle與Plain Old Document}
    傳統C有ansi C與k\&r C的coding style,後來Linux kernel也
    有自己的coding style.在perl上也有大家follow的coding style。
    一般說來要點有
    \begin{itemize}
      \item indentation 是4。
      \item sub funtion的 \{在sub 名字後,不像C的另起一行。
      \item class 名字第1字元大寫
    \end{itemize}
    POD是perl提供的文件系統,秉持著越簡單隨性就好的設計哲學,
    另外由於在這文件內的文字都不會被解譯器執行,所以也是可以
    用來comment整段文字的。不需要用很多個\#。
    只要是開頭是等號=加上一個id,那後面的文字就被認為是pod文件,
    這個id是任意的。然後可以透過perl提供的translator程式,
    轉換成我們想要的文件。目前 style sheet轉換程式有
    \begin{itemize}
    \item pod2html xxx.pm \verb=>= xxx.html;
    \item pod2latex xxx.pm \verb=>= xxx.tex
    \item pod2man xxx.pm \verb=| nroff -man | less=
    \item pod2text xxx.pm \verb=| less=
    \item pod2usage xxx.pm
    \end{itemize}

    目前perl認得的特定的id能夠被轉換程式識別的有
    \begin{itemize}
    \item =head1 =head2 ...像html的\verb=<h1><h2>=
    \item =over \# =item =back 用來作list的, over的\#是 indentation 數字, back表示結束。
    \item =cut 表示結束一段pod。
    \item =for TRANSLATOR =begin TRANSLATOR =end TRANSLATOR
    \end{itemize}
    目前perl建議的標準格式如下。indentation 也是perl的4,
    \begin{verbatim}
=head1 NAME

name of this package

=head1 SYNOPSIS

	use package_name;
	package_name->new();

=head1 DESCRIPTION

the description of package

=head1 AUTHOR

=head1 METHODS

=over 4

=item
...

=back

=head1 BUGS

=head1 SEE ALSO

=head1 COPYRIGHT


    \end{verbatim}

    注意的是
    \begin{itemize}
    \item 每一個=識別id跟文字都要空一行來區別。
    \item 只要前面是空格,tab就可以是verbatim文字。
    \end{itemize}
