\chapter{testing 測試}
測試其實是開發過程中最重要的一環,程式寫的再爛,只要有最後測試嚴格把關,
最後修修補補的成品可以動,就可以出產品,因此東西是不是能出去,應該是
QA 講了才算數。
\\\\
一般所謂的測試就是打個東西進去系統,看系統的反應合不合我們所期望的。
測試可以根據不同狀況與定義有以下分類。
\begin{itemize}
  \item unit test : 通常是DE自己寫的function 子函數的輸入參數與return測試,
    尤其API必須保證良好,別人才能用。
  \item sanity test : 由於完整測試通常相當大,這sanity表示前面基本一定要過的
    測試,通常用在要commit code前的automation測試。
  \item function test : 單一功能測試。
  \item feature test : 這是整個承諾客戶的大功能測試。
  \item system test (integration test) : 整體系統測試,包括 booting, 
    initial,重開機,回復,升級等等測試。
  \item security test : 安全性測試。這有的公司的 guideline 非常複雜。例如
    password的儲存,encript,library FIPS,網路安全等等。
  \item end to end test : 更大的系統測試,通常在網路服務產品中,兩端中間會
    經過很多子系統,例如switch, NAS等,必須兩端在 error handling, 
    都能保持穩定功能。
  \item regression test : 以前的測試做完後,新release有可能去改到舊code造成
    過去功能失效。所以必須對過去的功能重新測試,這通常會做成automation。
  \item stress test : 通常是產品的極限值測試。
  \item performance test : 一些速度,時間等的量測比較,尤其要跟競爭對手比較
    時的數據。
  \item UX test : 使用者習慣,UI 經驗測試。
  \item blackbox test : 測試者像白痴一樣,不管細部,就單純根據feature測試,
    通常end user的使用測試, 
  \item whitebox test : 測試者必須了解code的 if, else,等condition的細部,例如
    code path test 必須寫程式去仔細的測試這些condition碰到的反應。
\end{itemize}
當然大分類下還可以有細部測試分類。但手動manual測試,自動automation測試是最
大分類,一般測試者要準備的東西
\begin{itemize}
  \item test plan
  \item test case
  \item test report
  \item test statistics
\end{itemize}

test plan的 template 例子
\begin{center}
\begin{tabular}{|c|c|c|c|c|}
  \hline
  \hline
  Test case Id & description & category & pass criteria & Pass/Failed \\
  \hline
  t1001 & Test 1 description &  feature/vlan test & be good & passed \\
  \hline
  t1002 & Test 2 description &  regression/v1.0 1.1 & be good & failed \\
  \hline
\end{tabular}
\end{center}
傳統測試都是手動 manual 的,但產品愈來愈複雜,測試項目會變得很多,尤其
regression 很多都是重覆性質的,因此會把手動的步驟用 script 寫成自動跑的
測試。現在美國大公司都把這種繁多,成熟的測試丟到印度去做。其實台灣可以在
這邊搶到一些軟體生意,只要把測試整個流程的產線做出,就跟硬體一樣在軟體
供應鏈中找到生意機會。他同樣包含了文件,程式,軟工,只是美國大公司的重要
經理人現在都被阿三佔住,所以管理階層做決策時,是只會丟到印度去的。
印度公司為了搶這些測試生意,軟體公司都巨大的可以,還會通過CMMI Level 5
的認証。
\\\\
另外現在test plan, report等都會整合到網頁上去,也是 DevOps 的一部份,
從開 project 的 project 時間控管,requirement 文件,design 文件,
prototype, source code 控管,bug 控管,build, release, testing, 使用文
件,客戶issue 控管等等都有。
很多傳統的手寫文件(還是電腦文件)方式都不再用了。
