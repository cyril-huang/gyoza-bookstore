\chapter{程式碼分析}
程式寫完不是就馬上要上火線了,
通常 code 要進 respository 也要層層負責的 team leader 來做眼睛 review
的動作,主要是菜鳥寫code的盲點,在資深工程師的經驗下能很快被 review 一遍。以前
我們review 時是印出來,上台後大家品頭論足一番,例如你這為什麼用 int 不用char, 你這
為什麼用unsigned 等等。現在很多使用線上review 的維護軟體。
\\\\
做完人為的review後,通常要做自動化程式碼分析,分為 static 與 dynamic 分析
,static 指的是不執行電腦程式的條件下,進行程式分析的方法,dynamic 就是在程式
執行時的程式分析。static 分析主要是工具要對語法能做分析,在C 中,unix 系統在
很久以前就有lint 這個小工具,這小工具也是各家公司都可以不一樣的實作。
這些程式分析最後也越做越大,很多公司就靠這種分析工具能成為公司,例如 coverity。
\\\\
以上通常現在會利用這些工具寫出一些自動化script 檢查,放在整個軟體產出流程的
線上,例如 git push 前要先做 static/dynamic analsys,沒過就不允許進到 repository
,那這些需要在git server 那邊有hooking script。後來我們小組用 gerrit 做 code 的
review 與commit 的管理,再後來我們公司後來用github 與 atlassian 公司做的 stash
( 也叫 bitbucket ) 來做code 的這些管理, 這後面管理會越包越大包含 bug 跟 build
還有測試的連結。
\\\\
他主要是遠端要開一個相對於local 端的遠端branch, 好處是每次 push 後有個 
\emph{pull request} 的動作,然後 email, diff, review, reject, commit...
都在上面完成。 像 github 也有類似的功能, 整個流程有人稱為 git-flow。

  \section{lint}
  lint是最基本的 static analyser,很多公司有自己的 lint,有些 lint 甚至會把公
  司的 coding style 規則給放進去,我們以前跟HP合作,我們的 code 要進去,都要先
  通過HP自家的 lint 才行。不過後來愈來愈鬆散,我覺得後來的人寫code不像早期的人
  那麼嚴謹與要求。 opensource 我們使用簡單的 splint 做 C static 分析。
  其他語言也有相對應的 lint 程式。例如 java 的jlint等等,
  我們來寫一段 code, 用到 malloc, strdup, open, socket 等會用到記憶體,file
  descriptor的而沒有 free 掉時會怎樣。
  \begin{verbatim}
#include <stdio.h>
int main()
{
        myprint();
}

$ splint -I /path/to/my/header myprint.c
  \end{verbatim}
  出來的結果像這樣
  {\scriptsize
  \begin{verbatim}
root@dsuen-lnx:/home/chuang# splint myprint.c 
Splint 3.1.2 --- 20 Feb 2009

myprint.c: (in function main)
myprint.c:4:2: Unrecognized identifier: myprint
  Identifier used in code has not been declared. (Use -unrecog to inhibit
  warning)
myprint.c:5:2: Path with no return in function declared to return int
  There is a path through a function declared to return a value on which there
  is no return statement. This means the execution may fall through without
  returning a meaningful result to the caller. (Use -noret to inhibit warning)

Finished checking --- 2 code warnings
  \end{verbatim}}
  基本上是c code的嚴謹寫法的分析師,剛開始寫c code的菜鳥可以用這來訓練自己。

  \section{Valgrind}
  dynamic分析裡面,有個 IBM (以前rational software)的 purify 很有名,不過那個要
  錢啊,我們公司還有用coverity,這也是要錢的。 opensource 裏面有個valgrind,
  還不錯用。valgrind 是必須要有執行檔才能做分析檢查,而且也必須gcc 有-g才行。
  \\\\
  他有很多的子工具來做特別的檢查,
  \begin{itemize}
    \item memcheck 最基本的 memory leak 檢查
    \item cachegrind 
    \item callgrind 函式呼叫歷史的紀錄,檢查呼叫
    \item helgrind, drd, massif, lackey, none, exp-sgcheck, exp-bbv, exp-dhat
      其他的分系檢查
  \end{itemize}
  使用如下
  \begin{verbatim}
valgrind --tool=memcheck ls -l
  \end{verbatim}
  就可以檢查ls -l命令的memory leak問題
    \subsection{memcheck}
    動態分析中最重要就是memory leak 的檢查,在寫 daemon 程式是非常重要的,
    memory leak在很多地方
    \begin{itemize}
      \item malloc / free 這是最基本的記憶體管理
      \item opendir / closedir 有很多人忘了還有其他的open/close 也是
      \item shmget 當然,更不要說跟kernel有關的IPC,share memory 的allocate。
        這種沒有弄好是會把系統給玩壞掉的,不光是kill process 就了結。
    \end{itemize}
    基本 default 的檢查就是 memcheck,其中 error message 是顯而易見的,例如
    Invalid read of size 4, Syscall param write(buf) points to uninitialised
    byte(s),Invalid free 等等。除了valgrind 自己的定義會有下面結果,
    \begin{verbatim}
==23752==    definitely lost: 0 bytes in 0 blocks
==23752==    indirectly lost: 0 bytes in 0 blocks
==23752==      possibly lost: 0 bytes in 0 blocks
==23752==    still reachable: 1,836 bytes in 1 blocks
==23752==         suppressed: 0 bytes in 0 blocks
    \end{verbatim}
    這會跟--leak-check這個選項有很大的關係。根據valgrind文件,他認為有
    下列9種pointer 可能產生問題。
    \begin{verbatim}
     Pointer chain            AAA Leak Case   BBB Leak Case
     -------------            -------------   -------------
(1)  RRR ------------> BBB                    DR
(2)  RRR ---> AAA ---> BBB    DR              IR
(3)  RRR               BBB                    DL
(4)  RRR      AAA ---> BBB    DL              IL
(5)  RRR ------?-----> BBB                    (y)DR, (n)DL
(6)  RRR ---> AAA -?-> BBB    DR              (y)IR, (n)DL
(7)  RRR -?-> AAA ---> BBB    (y)DR, (n)DL    (y)IR, (n)IL
(8)  RRR -?-> AAA -?-> BBB    (y)DR, (n)DL    (y,y)IR, (n,y)IL, (_,n)DL
(9)  RRR      AAA -?-> BBB    DL              (y)IL, (n)DL

Pointer chain legend:
- RRR: a root set node or DR block
- AAA, BBB: heap blocks
- --->: a start-pointer
- -?->: an interior-pointer

Leak Case legend:
- DR: Directly reachable
- IR: Indirectly reachable
- DL: Directly lost
- IL: Indirectly lost
- (y)XY: it's XY if the interior-pointer is a real pointer
- (n)XY: it's XY if the interior-pointer is not a real pointer
- (_)XY: it's XY in either case
    \end{verbatim}
    分別
    \begin{itemize}
      \item definitely lost: 這是顯而易見的memory leak
      \item indirectly lost: 
      \item possibly lost:
      \item still reachable:
      \item suppressed:
    \end{itemize}

    \subsection{callgrind}
    之前的gprof 是 performance 的profile,callgrind也是類似的會紀錄整個呼叫歷史
    的profiler cache 跟branch prediction 的 profiler,
  \\\\
  valgrind算起來已經很不錯用,也相當的複雜了,對於基本的單一c/c++ program的
  初期檢查是還不錯的工具。但現在有的執行環境是很複雜的,例如網路程式,或者
  更複雜的多daemon多平台交互工作的環境測試就比較麻煩。
  \\\\
  使用lint或valgrind基本上還是要automate這個檢查,然後用
  script 掃出有問題的程式碼,所以最後還是要 script 寫作檢查。
  另外,這是對user space的code,kernel寫作的偵測就不行,kernel的技巧將另闢專
  欄介紹。

  \section{code review}
    code review是人的眼睛去做邏輯的分析,主要是借助有經驗的工程師,根據過去的
    經驗傳承錯誤的早期避免,以前都是印出來的,現在把這套管理程序也上線,
    code review的工具通常會伴隨SRM配套一起使用,就自己裝自己的。以前大公司內
    部都有專門寫這些工具的部門,但現在opensource也都有這些實作了。而且愈做越大
    越好。
    \subsection{reviewboard}
    reviewboard是個蠻有趣的線上code review管理。通常code完成分析,並且能編譯做完
    unit test後,是要給組裡的大老們聞香一下,大老們說可以了,有蓋印章了,出事有
    人扛了,那才可以check in。現在這種流程也能在網路上完成。所以要先裝apache或
    lighttpd,也要先裝資料庫mysql或postgreSQL.
    \\\\
    reviewboard是python寫的,所以他的package形式是python的蛋蛋,fedora的epel
    repository也有人包裝好rpm。如果是debian/ubuntu/mint
    \begin{verbatim}
# apt-get install pytohn-setuptools
# easy_install -U setuptools (升級setuptools)
# apt-get-install python-dev
# apt-get install memcached
# easy_install python-memcached (不在debian package)
# apt-get install patch
# easy_install ReviewBoard (不在debian package)
使用mysql的人要裝python跟mysql的binding API
# apt-get install python-mysqldb (或 easy_install mysql-python)
使用postgre的人要裝python跟postgresql的binding API - psycopg2
# apt-get install python-psycopg2 (或 easy_install psycopg2)
    \end{verbatim}

    基本上 code review現 在都跟 source control 工具組合在一起了,他只是part of
    整個流程中的一部份,所以單一reviewboard這東西只有光code review而已,必須整合
    source control在GUI管理流程中才行,這後來有像github, bitbucket, gitlab...等
    更強大的web GUI整合管理介面,reviewboard 現在已經沒人用了,除了reviewboard,
    還有一個gerrit的前身叫 Rietveld的,就不介紹了,正常公司整個使用會用 
    github/bitbucket + jenkins 統合整個開發流程,現在通常屬於 devop pipeline 
    的工作範圍,這留待後面 DevOps 說明。
