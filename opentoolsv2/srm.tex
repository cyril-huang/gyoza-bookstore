\chapter{版本控制系統}
SRM(software revision management) 版本控制是對 source 的歷史紀錄及開發實驗的重
要工具。早期有光純local的RCS, 後來延伸的client/server CVS,有的公司內部是很
厲害的,像我們以前的版本控制是自家工程師自己寫的,後來最有名的商用是rational
clearcase 等等,隨著時代進步,很多限制與缺點都慢慢改進,目前業界用很大的是
subversion 與 git。曾經在網路看到台灣人在問這東西可以用在2,3000人的公司嗎?
基本上現在全世界的軟體業已經全部轉到 opensource 上來,
現在 embedded 系統不是 BSD 就是 Linux。幾萬人的公司在用的工具全是 opensource
工具。
\\\\
版本控制中說穿了就是graph/node理論以及多人同時處理一個檔案的data sync問題,
各家有各家的想法跟實作。基本操作都差不多,但最重要的一個觀念是版本控制不是
備份,也不是雜七雜八亂七八糟 binary的東西都丟上去,只有最原始人用editor產生
的東西,還有很多人一天到晚亂check in。那種版本紀錄就跟沒有沒兩樣,一點也沒
有意義。
\\\\
branch與merge是版本控制中最重要的部份,沒了branch/merge,那就沒有 SRM 的意義。
有人說這樣不怕檔案丟失,如果同一顆硬碟,那是一樣的,如果不是同硬碟,沒有遠端
專業大手筆的RAID, SAN, backup 等手段,也是無
意義的備份而已。以我們公司來說整個NFS系統是24小時都在運轉備份的,我們的svn
git server 都是24小時無數的 SAN/RAID/cluster的storage的。當有這些東西時,那
code 正確小心的進入 SRM 才是正確的作法(每次 code 要進去,一定是大家 review 
過的才行,如果 review 過後出問題,那 senior 的工程師就不該領那份薪水,以前
是連一個 space 空錯了,都是打回去重寫的)。
\\\\
合併重要的觀念就是以兩點的delta中產生的變化與某支中的source進行合併。
其中有兩個重要名詞
\begin{itemize}
  \item rebase/promote 傳統上爸爸到小孩稱rebase, 小孩回爸爸稱promote,
    這應該是從clearcase來的, 算是正統srm/scm業界的標準辭彙。
  \item sync/reintegrate 這相當於rebase/promote,svn提出這樣的辭彙。但有點
    不太一樣的解釋。主要在 svn 的版本編號與model有點不太一樣,沒有 rebase 
    / reparent的東西,但一樣有爸爸到小孩這件事,而且他用svn merge這樣的命令。
  \item rebase/merge 這是git提出類似rebase/promote的辭彙。
\end{itemize}
也就是說 merge 這個名詞在標準 SRM 中是很籠統的,他的定義,語法,作用其實是跟著
不同工具而有所不同,這也是一般初學者搞不清楚的地方,以為merge merge就那麼一回
事。然後看了一堆文件還是一頭霧水,svn merge 跟 git merge 跟 clearcase 講的 merge
根本就不一樣。所以如果在以下文章中我說merge,有可能我說的
就是籠統的merge (合併),請不要搞混。在真實的 implementation中,每個SRM提出的
方法或許不同,或有他實作的特殊方法,但但但...最重要的最終目的,都只是要做到
rebase/promote這兩件事。不要被其他拉哩拉雜的語彙給騙了。
\\\\
merge back的觀念在於
只要public出去server的source,就無法undo歷史,就必須merge back,並且產生一個
新歷史紀錄。所以local怎麼亂搞都無妨,但只要svn commit/git push有對外歷史紀錄,
想反悔,那就只能merge back。這都是對外歷史紀錄,必須保留。
\section{subversion}
  他有一些特點
  \begin{itemize}
    \item client / server subversion是一般的client/server,所以要有svn server
      才行。
    \item 版本號碼是所有branch同一個version space,也就是不管在哪個branch上,
      只有一個獨一無二的版號。cvs的版本號是per file的,但 svn 的版本號是
      per project
    \item 能處理binary,雖然binary檔不該進srm,但現在很多網頁的圖檔屬於binary
      檔,很多人偷懶,當然也有很多白爛的工程師亂搞,正常 binary 檔應該在 file 
      system 裏面。
  \end{itemize}
  \subsection{server建置}
  以apache +SSL 當作溝通的server,apache 需要裝dav svn模組。
  裝三個packages, libapache2-svn apache2 subversion。3個設定: httpd, dav\_svn
  mod\_ssl
  \\\\
  httpd.conf要load的模組
  \begin{verbatim}
  LoadModule dav_module         modules/mod_dav.so
  LoadModule dav_svn_module     modules/mod_dav_svn.so
  \end{verbatim}
  dav\_svn.conf檔案的設定
  \begin{verbatim}
  <Location /svn>
      DAV svn
      SVNParentPath /var/lib/svn

      AuthType Basic
      AuthName "Subversion repository"
      AuthUserFile "/etc/apache2/dav_svn.passwd"

      <LimitExcept Get Propfind Options Report>
         Require valid-user
      </LimitExcept>
      SSLRequireSSL
  </Location>
  \end{verbatim}
  authentication/authorization 用的是 apache 內部的,所以 dav\_svn.passwd 必須以
  htpasswd 去建立。最好是把 ssl enable 起來,所以在 debian 上面是把 default-ssl
  與 mod\_ssl 給 enable 起來。
  \begin{verbatim}
cd /etc/apache2/sites-enabled
ln -s ../site-available/default-ssl .
cd /etc/apache2/mods-enabled
ln -s ../mods-available/ssl.* .
  \end{verbatim}
  這裡面其實是有很多東西在裡面了,ssl 的 key, csr, crt 已經 openssl 建立在/etc/ssl
  主要就是mod\_ssl 必須要 load,然後 dav\_svn.conf 的 SSLRequireSSl 是強制性一定要
  用ssl來連線。
  \\\\
  repository 建立必須找到相對位置,由於我們在 dav\_svn.conf 裡面使用了 SVNParentPath
  ,所以我們所有的 repository 都要在/var/lib/svn/ 下面另建目錄。
  \begin{verbatim}
# su www-data
$ svnadmin create /var/lib/svn/project0
  \end{verbatim}
  這個目錄下的所有人必須是 apache server 有權可以讀寫的。這個會 mkdir 也會在
  /var/lib/svn/project0 下面建立一些檔案。
  \\\\
  假設我們站在一個 source tree 叫 myproj目錄裡面。
  \begin{verbatim}
svn import https://server/svn/project0
  \end{verbatim}
  這樣就把project丟進遠端的repository
  \begin{verbatim}
svn list https://server/svn/project0
svn co http://server/svn/project0 xxxx
  \end{verbatim}
  會在自己這邊的xxx目錄下建立project0的repositroy
  % hook script
  \subsection{基本使用}
  基本命令都很簡單,多玩幾次就差不多了。
  \subsubsection{初始}
  \begin{itemize}
    \item svn import
    \item svn co
  \end{itemize}
  \subsubsection{更動}
  \begin{itemize}
    \item svn ci
    \item svn add
    \item svn del
    \item svn update 這會抓下server上最新的並跟working做自動merge。
  \end{itemize}
  \subsubsection{訊息}
  \begin{itemize}
    \item svn info
    \item svn status
    \item svn status -q 表示不要顯示沒有在 svn 控制下的檔案。
    \item svn log -g 用-g表示所有branch上有關這檔案的紀錄通通出來。
    \item svn log -v -r 1234 這個commit的log與所有檔案list。
    \item svn log -r {2013-8-10}:{2014-1-10}
    \item svn log \verb=--=stop-on-copy 表示在branch的頭就停下來。
  \end{itemize}
  status裏面的訊息面,前面7個字元是有意義的,常用就第1字元
  \begin{itemize}
    \item ? 表示這檔案不在 svn 控制
    \item A 表示這是新加檔案
    \item C 在第1字元表示檔案conflicts,在後面第4字元表示是tree conflicts。
    \item D 表示被刪除
    \item M 在第1字元表示這檔案內容被更動了,第2字元表示property被更動了。
  \end{itemize}
  \subsubsection{比較}
  \begin{itemize}
    \item svn diff 比較working與server上版本差異。
    \item svn diff -r 1234:5678 比較兩版本。
    \item svn blame 抓出每一行的check in者是誰。或者叫作annotate
  \end{itemize}
  \subsubsection{反悔}
  \begin{itemize}
    \item svn revert 本地端還沒check in時,都可以用revert反悔。
    \item svn merge -r 1234:1233 但只要有過版本歷史,就一定要再merge back,
      創造一個新的歷史紀錄與check in。通常merge回減一個號碼。
  \end{itemize}
  \subsubsection{遠端}
  \begin{itemize}
    \item svn list https://mysvn.com/branches 列出遠端所有branches
    \item svn copy https://mysvn.com/branches/b1 https://mysvn.com/branches/b2
      開branch
    \item svn move 改名
    \item svn del 刪除
  \end{itemize}
  \subsubsection{其他}
  在working下面其實有個.svn這個目錄,裏面放了很多資訊,有些操作是針對這目錄的
  \begin{itemize}
    \item svn export 跟co很像,但沒有那個.svn跑出來。
    \item svn cleanup 清除目前.svn內的暫時狀態,例如某檔的lock。這通常是跑一
      跑某命令,然後忽然按ctrl-C的結果,就需要cleanup。
    \item svn switch branch2 轉換到另一branch 去,如果有沒commit 的,會出error
  \end{itemize}
  另外每個檔案有所謂的 property,賦與一些屬性的key/value後,svn可以因為這
  property的設定而做一些功能。主要 property 的長相為 "svn:xxxxxx"
  \begin{description}
    \item [\$Id] \hfill \\
      使用 RCS/CVS的老玩家
      \begin{verbatim}
svn propset svn:keywords "Id” myfile.c
svn propget svn:keywords myfiles
svn proplist myfiles
      \end{verbatim}
    \item [submodule] \hfill \\
      使一個目錄下能連到別的repository
      \begin{verbatim}
mkdir submodule
svn add submodule
cd submodule
svn propedit svn:externals 'subdir https://otherproj.com/xxx' .
svn ci
      \end{verbatim}
      則svn update時會把otherproj.com/xxx下面的東西放到submodule/subdir進來
  \end{description}
  \subsection{branch/merge}
  開 branch 非常簡單。就像在 copy 一樣,這是 svn 開始讓很多人喜歡的。隨便你開。
  \begin{verbatim}
svn copy https://myproject/branches/b1 https://myproject/branches/b2
svn co https://myproject/branches/b2
cd b2
svn info
  \end{verbatim}
  svn 的 branch 只是一個紀錄,所以可以亂開。
  \\\\
  根據svn 的文件上來說,他認為的 merge 有四種:
  標準sync, cherrypick, reintegrate, 2url。但無時無刻不要忘了,所謂merge就是
  任兩點的差異跟某base做graph/node的運算。上面所說的四種,都是一樣。
  \\\\
  在文件中有新feature要開feature branch,然後 feature branch 的sync,就是爸爸
  對小孩的sync,就是從開始小孩branch到目前爸爸最前面這兩點對目前的 working 
  directory做merge。reintegrate 是sync...sync...幾次後,最後完成 feature 了,要回去
  爸爸那邊了,由於之前已經 sync 過了,所以爸爸小孩已經有重複相同的 code
  ,小孩已經不能merge回去,而要做reintegrate。這就是 rebase/promote。
  svn 的標準文件中說到b1 b2 兩個 branch,要先把爸爸b1 sync 到小孩 b2,然後再
  reintegrate回爸爸
  \begin{verbatim}
cd b2
svn merge --accept postpone https://myproject/branches/b1 . > ../merge.log
resolve the conflicts
(compile and test ...)
svn resolve --accept working
svn ci

cd b1
svn merge --accept postpone --reintegrate https://myproject/branches/b2
(compile and test ...)
svn ci
  \end{verbatim}
  除了傳統上的 sync/reintegrate,
  那爸爸小孩可不可以亂merge來merge去呢?基本上如果按照文件上所說,是沒有這樣
  說法的,但不要忘了"merge只是兩點間的delta 往某支source做運算",只要避免掉
  重複的可能性,就可以merge。所以只要爸爸從來沒有往下sync過,小孩是可以直接取
  兩點 merge 回爸爸的。同理爸爸到小孩也可以取兩點做,所以
  \begin{verbatim}
svn merge -r 1234:5678 https://myproject/branches/b2 . > ../merge.log
  \end{verbatim}
  是永遠存在的,但千萬要記住每次的兩點是哪兩點,還有版本號碼下一次不要忘了要
  多加一號。不要遺漏也不要又重複了,當然不加兩點,內定就是目前到最開始切出
  branch那一點。即使是爸爸已經往下sync過了,那也可以取爸爸的最後sync那點與小孩
  最後的那點的兩點差距往爸爸做2 URL merge。另外svn 的sync很聰明,不寫兩點時,
  他在mergeinfo裏面會自動記住目前的sync點是哪一個,說穿了也是 rebase。
  \\\\
  那有一種情況是兩隻branch是兄弟關係,sibling branch merge也是可以,甚至可以
  指定兩個URL點來對working directory做merge,所以任兩支branch的merge就是使用
  2 URL merge。只要你確定沒有重複code,沒有少掉code。不要忘了只是兩個點的delta
  來對某source做運算。
  \begin{verbatim}
svn merge https://myproject/b1@1234 https://myproject/b2@5678 . > ../merge.log
  \end{verbatim}
  以一個例子來說
  \begin{verbatim}

   123  a  456  b 789  c 901  d  1000
   -------------------------------              m
    \       \        \     \      \
     \       \        \   e \950 f \ 1010 g
      \       \        \-----v------v---------- n
       \       \         x
        \-------v------------------------------ o

  \end{verbatim}
  這例子裏面的 m n都已經 release 了,n比m晚release,然後希望n 往o branch倒東西
  ,但 n = a + b + c + d + e + f + g, 而o = a + x 。當 svn merge n o 的時候,
  相當於我們要得到 a + b + c + d + e + f + g + x 所以相當於我們要得到
  delta = b + c + d + e + f + g 進到 o 去,所以這有很多種可能,必須先checkout
  o,並且cd 到o去
  \\\\
  2 URL merge
  \begin{verbatim}
  svn co https://proj/o
  cd o
  svn merge https://proj/m@456 https://proj/n . > ../merge.mn
  \end{verbatim}
  他也能分兩段來做
  \begin{verbatim}
  svn merge -r 456:789 https://proj/m . > ../merge.m
  svn merge https://proj/n . > ../merge.n
  \end{verbatim}
  或者2 URL
  \begin{verbatim}
  svn merge -r 456:901 https://proj/m . > ../merge.m
  svn merge -r https://proj/m@901 https://proj/n@HEAD . > ../merge.mn
  \end{verbatim}
  或者
  \begin{verbatim}
  svn merge -r 456:1000 https://proj/m . > ../merge.m
  svn merge -r 1010:HEAD https://proj/n . > ../merge.n
  \end{verbatim}
  分段做的用意還是在減少code的變化太大所造成的conflicts太大,如果 code 都還蠻
  straight forward,那一次2 URL merge也可以。
  \\\\
  使用2URL的merge還能在原本reintegrate時,這是Corey認為很棒的mergeinfo產生方法
  ,從爸爸再拉一條出來,然後用小孩跟新拉出的branch,用2 URL merge到爸爸的
  working去,而不要用reintegrate。
  \begin{verbatim}
                                     -o
                                    /
                                   /
   -----------------------------------  m
                  \     \      \
                   \     \      \       
                    \-----v------v n
  \end{verbatim}
  本來n sync完後做reitegrate回m,並且放棄n了,但不做reintegrate,改從m拉出一條
  o來,此時做2 URL merge
  \begin{verbatim}
  svn co m
  svn merge n@xx o@yy .
  \end{verbatim}
  其他的選項應用
  \begin{verbatim}
  svn merge -r HEAD:1234 從HEAD到版本1234做反悔merge
  svn merge --dry-run 試試看就好了,不會真寫進working directory
  svn merge --accept postpone 所有的conflicts先使用postpone。
  svn resolve --accept working 解決改變檔案conflict狀態,才可以check in。
  \end{verbatim}
  當使用postpone時,所有conflicts檔裡會有
  \begin{verbatim}
  <<<<<< .working
  xxx
  =====
  yyy
  >>>>>>>
  \end{verbatim}
  working部份就是working directory下面原本的,conflicts有兩種
  \begin{itemize}
    \item text conflicts 這是真的發生conflicts了
    \item tree conflicts 這是在某branch被砍了,但在另一branch卻還存在,所以這
      通常要檢視一下到底怎麼回事,通常就是接受 merge 結果 \verb=--=accept working
  \end{itemize}
  svn log -g跟mergeinfo有很大關係,可以用svn mergeinfo
  .來看出目前總共的merge編號。

\section{git}
git是很神奇的,他直接紀錄檔案內容的變化,不對檔名檔案系統的東西作紀錄。
版本編號完全不是流水號,版本編號只是個sha識別號,編號跟編號之前沒有相對關係,
顛覆以前的想法。另外傳統上svn的server只要掛掉了,
就大家不用做事了,所以git使用local與server混用的distribution srm,local
間可以互相亂merge,不須透過server,但又可以有server來做統一的發佈與管理。
(雖說如此,但我好像沒看過我同事間在互相 merge,呵呵,主要我們都用 github
bitbucket,這些東西用法又不像原始git)
\\\\
他有一些相關定義與名詞:
\begin{itemize}
  \item objects 一個repository內的所有組成元素都叫object,都zip在.git目錄下
  \item blob 一個檔案內容物件
  \item tree 一個包含blob或tree的物件
  \item commit 每次commit產生的物件,git的branch只是一個指向commit物件的alias
  \item tag 貼上tag產生的物件
  \item refs (reference) 一直會改變的指向commit object的指標。HEAD, branch, 
    commit。以前svn的HEAD, PREV等都是。
  \item index (staging area)
\end{itemize}
這些東西放在.git內在git操作時,是內部使用的抽象定義。
\\\\
他等於有三層的地方在放source。working, local repository, remote
repository,但其實不只這樣,他還有一些暫存區可以亂放暫時的source。所以他的反悔
的使用,有很多不同層次,抓到這個重點,剩下的命令就簡單了。一些名詞定義
\begin{itemize}
  \item track/untrack 檔案在git的控管之下叫tracked
  \item modified/unmodified 檔案在track下有沒有被改動
  \item staged 檔案指被git add/modify後的狀態,commit,會進local的database。
  \item working directory 目前的工作目錄。
  \item origin/master master分別是遠端與local端的內定主branch名。遠端的branch
    使用(remote)/(branch) 這樣的名字樣式。所以 origin 是一個 remote 名字。
    remote 的使用在於他可以亂指到任一 url 去,從那個 remote 下再去分 branch。
\end{itemize}
所以檔案狀態層次為
\begin{verbatim}
tracked->staged->local->remote
\end{verbatim}
另外很重要的一點就是版本編號的方法,cvs 的版號是per file的,svn的版號是per
project,log 散在每個branch上 ,git的版號是per project,但會每個branch都有一個
指標說有沒有包含這個版號。
  \subsection{server建置}
  git有很多種遠端protocol使用
  \begin{itemize}
    \item git 缺點是他沒有authentication功能,必須搭配ssh,使用內定port 9418
    \item ssh 缺點是一定要有帳號才能存取,對read only不方便,但通常公司內用這種。
    \item https 缺點是速度比git protocol慢。不過read only都用這種。
  \end{itemize}
  建立server(可以三種都用,或只用一種也可)
  \begin{enumerate}
    \item 加一個新帳號, git:git (使用在git或ssh連線上):\\
      除了加git帳號外,也要把login shell換成git-shell這隻,然後要特別給一個目錄
      git-shell-commands 755的讀寫權,最後 ssh 可以把所有 developers 放到同 
      group 去, 通常是把 ssh-keygen 產生的 public key 放到
      /home/git/.ssh/authorized\_keys
      \begin{itemize}
        \item \# useradd -m -d /home/git git
        \item \# usermod -s /usr/bin/git-shell git
        \item \# mkdir /home/git/git-shell-commands
        \item \# chmod 755 /home/git/git-shell-commands
      \end{itemize}
    \item 建立repository目錄:\\
      應該安裝git就會有的,要注意的是要先變身git, 使用bash\\
      \begin{itemize}
        \item \# su -s /bin/bash - git \\
        \item \# mkdir /var/lib/git
      \end{itemize}
    \item 建立project目錄與初始化:\\
      \begin{itemize}
        \item \# su - git
        \item \# mkdir -p /var/lib/git/myproj.git;
        \item \# cd /var/lib/git/myproj.git;
        \item \# git \verb=--=bare init
      \end{itemize}
    \item git daemon (git protocol, 可以在兩個人間使用吧):\\
      如果只想用ssh方式, 這沒有一定要跑。\\
      \begin{itemize}
        \item \# git daemon \verb=--=user=git \verb=--=group=git \verb=--=reuseaddr \verb=--=base-path=/var/lib/git/ /var/lib/git/
      \end{itemize}
    \item ssh server 這有兩種作法來讓遠端寫入。 就是developers都在git group
      或者大家放 public key 到 /home/git/.ssh/authorized\_keys,所以如果是放
      public key 方法時,遠端的 repository 名字通常都是
      ssh://git@xxx/var/lib/git/myproj.git。如果不用git@則會每次git命令都要輸入
      遠端密碼很煩。
    \item https server:\\
      這必須修改httpd.conf檔案,與建立httpasswd。這跟svn 建立ssl方法一樣。
  \end{enumerate}
  git檔案相對位置是base-path設定,ssh是整個filesystem,http就是DocumentRoot了。
  uid/password的管理用ssh是可以快速方便用NIS系統上的,如果用https,最好還是用
  上ldap統一管理,不然就要用httpasswd很麻煩的維護。ssh有兩種方法,一種是只使用
  git這個帳號然後每個人public key 都放到.ssh/authorized\_keys,一種是把
  developers都放到同一個group,git,然後打開 group權限。初始也可以從一個已經有
  .git的repository來。
  \begin{verbatim}
  git clone --bare myproj myproj.git
  \end{verbatim}
  然後copy到/var/lib/git, 接著把source放上去,類似svn import 自己sources,
  假設有個目錄myproj裏面有 source code了。
  \begin{verbatim}
  $ cd myproj
  $ git init
  $ git remote add origin ssh://git@10.0.2.2/var/lib/git/myproj.git
  $ touch README
  $ git add README
  $ git commit README
  $ git push origin master
  $ git remote -v
  $ git branch -a
  \end{verbatim}
  在還沒有add, commit, push前,使用git branch, git remote 是看不到任何東西的,
  必須做了commit, push 才會有local的master與remote的origin名字跑出來。
  同伴client可以clone或者init做一個 git clone /var/lib/git/myproj.git,或者
  \begin{verbatim}
  $ git init
  $ git remote add origin /var/lib/git/myproj.git
  $ git pull origin master 遠端的origin拉下來成為local的master branch
  \end{verbatim}
  總之git \verb=--=bare init是用來建立一個respository,然後local用init建立working
  後把remote 指向這個repostory進行pull/push, 這個指向可以是
  \begin{itemize}
    \item 看得到的檔案系統。例如\\
      git remote add origin /var/lib/git/myrpoj.git
    \item 遠端 protocol url 新增與更改,例如\\
      git remote add origin ssh://git@srm.com/var/lib/git/myproj.git\\
      git remote add otherRemote https://srm.com/git/myproj.git\\
      git remote set-url origin git@github.com:cyril-huang/init-scripts.git \\
      git remote show origin
  \end{itemize}
  別人就可以clone與pull了。總之remote add, 加一個遠端branch 名字到自己這端
  master 來就是。
  可以同時 push 到多個 url 去,只要 set-url 加上 push 選項
  \begin{verbatim}
git remote set-url --add --push origin https://mygithub.com/myproj.git
git remote show origin
  \end{verbatim}
  這樣當 push 時,會同時 push 出去。
  \\\\
  在myproj.git目錄下會看到
  \begin{verbatim}
  branches  config  description  HEAD  hooks  info  objects  refs
  \end{verbatim}
  比較想去改的應該是hooks裏面可以放hooking script, 去掉裏面檔名的.sample
  就會被呼叫,呼叫的時間點定義在
  \href{https://git-scm.com/docs/githooks}{githooks}。這些時間點就是後
  來 gerrit, github, bitbucket 他們拿來做code review, merge 等動作的時間點所
  hooking 的 script。
  \subsection{基本使用}
  git的使用跟svn有很大的差異,主要在架構定義上就不一樣,所以連改動,checkin
  等步驟也都不同,由於他很多不同state,不像 cvs 轉 svn簡單,必須從頭改變想法。
  主要差異是多了staged state,使用add與reset來使檔案進入與脫離staged state。
  reset命令其實是在把指向HEAD的指標換來換去,不同的選項在於檔案跟HEAD的狀態
  是否完全變化而已。
  \\\\
  一開始先定義自己在 \$HOME/.gitconfig
  \begin{verbatim}
[user]
	email = develop@gyoza.com
	name = Developer
[core]
	editor = vim
[merge]
	tool = vimdiff
[http]
	sslverify = false
[pull]
	rebase = true
[color]
	ui = true
[alias]
	co = checkout
	ci = commit
	lg = log --oneline --decorate --all --graph
	d = difftool
[diff]
	tool = vimdiff
[difftool]
	prompt = false
  \end{verbatim}
  主要是 email, name 這樣才知道是誰 commit 了。
  \subsubsection{初始}
  server的初始在上面說明過了,本地的初始, git init, git clone, git remote add
  三種方法
  \begin{itemize}
    \item git \verb=--=bare init 上面講的初始一個server repository
    \item git init 初始一個本地目錄
    \item git clone https://mygit.com/myproj.git 拉下一個遠端project的default branch
    \item git remote add origin ssh://git@srm.com/var/lib/git/myproj.git\\
          git fetch origin otherBranch\\
          git pull\\
  \end{itemize}
  \subsubsection{更動}
  \begin{itemize}
    \item git checkout 這是local端的轉換某支branch,如果原本local下檔案有改變
      ,這些改變的檔案在新branch也是保留的,可以在新branch裏面commit。如果要
      回復原本的改變也用git checkout myfile, 等於svn revert myfile.
    \item git add file 把檔案加到index,準備下次commit時進去,每次更動檔案,
      都要做這步。這跟svn add想法是非常不同。
    \item git rm file 移除檔案
    \item git commit 這是local端的check in
    \item git commit -a 不需要多git add的動作,等於svn ci
    \item git push 到遠端去,這才是真正 publish 東西給全世界知道。之前的
      commit local 端,愛怎麼亂搞都沒問題,但一旦 push 出去就要被公審了
      變成以前 svn commit 其實是 git push。
    \item git clean -xdff 清掉所有不在 tracked 的檔案,這通常是清掉 build
      後一堆亂七八糟東西產生,還我漂亮拳。
  \end{itemize}
  \subsubsection{訊息}
  \begin{itemize}
    \item git status
    \item git log \verb=--=name\verb=-=only
    \item git log \verb=--=stat \verb=--=summary
    \item git log \verb=--=pretty=oneline
    \item git log \verb=--=since="2 weeks ago" \verb=--= myfile
    \item git log master \verb=--=not \verb=--=remotes=*/master
    \item git log --graph --oneline origin/branch1..branch2 找出兩個branch 間差異的 commit
    \item git reflog 這是看所有HEAD的refs的log,可以與git reset合起來亂搞。
    \item git show object 一個general的object資訊命令。
  \end{itemize}
  \subsubsection{比較}
  \begin{itemize}
    \item git diff 是staged 檔案跟目前 working space 的不同
    \item git diff master..experiment myfile
    \item git diff 61786..e79f06 myfile
    \item git diff HEAD..HEAD\verb=^= 等於 svn diff -r HEAD:PREV,HEAD\verb=^^=
      表示目前頭的前兩個commit。
    \item git diff \verb=--=cached
    \item git diff \verb=--=staged
    \item git apply ../my.diff 可以把一個 diff 檔,apply 到整個 source tree
  \end{itemize}
  有個比較複雜的找到遠端 branch 的 分歧點或者 parent branch 是誰,
  \begin{verbatim}
  git fetch origin '+refs/notes/*:refs/notes/*'
  git merge-base origin/master origin/a_branch
  git notes --ref=master_ref show b092f9c219cb60ccb8c3e264252718506f05d367
  git notes --ref=a_branch_ref show b092f9c219cb60ccb8c3e264252718506f05d367
  \end{verbatim}
  \subsubsection{反悔}
  這是最重要的,所有的歷史紀錄是版本控制的最重要功能,所以 commit log 的反悔,
  合併,新增,刪除就是操作的重中之重。 反悔有個很重要的觀念是local端自己可以
  亂搞反悔,遠端除非只有你自己玩的branch ,不然就不能隨意反悔造成別人的log
  歷史大亂。 最基本的是
  \begin{verbatim}
  git commit --amend
  \end{verbatim}
  把這次的 commit 修補成上次的 commit,這樣commit log 不會增加。
  \begin{itemize}
    \item git checkout \verb=--= myfile 這等於svn revert myfile。
    \item git checkout master \verb=--= myfile 這等於checkout master上的myfile
    \item git reset 使檔案脫離staged狀態,反悔git add的動作。但reset動作是所
      有檔案,單一檔案請用git checkout。
    \item git reset HEAD\verb=^= 把HEAD推回到目前HEAD的上一個,等於反悔git commit。
    \item git reset \verb=--=soft e1f34 推回到某一點。
    \item git reset \verb=--=hard 使所有檔案回到完全沒改狀態,等於svn revert -R *。
    \item git reset \verb=--=hard HEAD@\verb={1}= reset是來設HEAD的,亂設HEAD來undo之前reset。
    \item git commit \verb=--=amend 雖然commit了但使用上一個 commit log做修改。
    \item git revert 這等於svn merge back, 跟svn revert不同,這也會自動commit
    \item git restore \verb=--=staged
  \end{itemize}
  其中要非常小心 git reset 尤其是 hard,不要輕易使用,會把所有沒 commit 的
  東西清掉,如果是新檔案,則全部會被 delete 掉,如果發生這種慘事,這可以
  \begin{verbatim}
git fsck --lost-found
  \end{verbatim}
  在 .git/lost-found 找回來,例如
  \begin{verbatim}
git show .git/lost-found/other/048b54843c14fa344ecb11a70860a53acb3bff24
  \end{verbatim}
  可以看到被殺掉的檔案。
  \\\\
  也可以使用rebase來回到想要的原本那點,例如回去兩個commit點
  \begin{verbatim}
  $ git rebase -i HEAD^^  past 2 commits
  \end{verbatim}
  多個commit的回復解決,例如回復到4個commit前
  \begin{verbatim}
  $ git rebase -i HEAD~4  past 4 commits

  則會叫出你的editor

  pick 01d1124 Adding license
  pick 6340aaa Moving license into its own file
  pick ebfd367 Jekyll has become self-aware.
  pick 30e0ccb Changed the tagline in the binary, too.

  # Rebase 60709da..30e0ccb onto 60709da
  #
  # Commands:
  #  p, pick = use commit
  #  e, edit = use commit, but stop for amending
  #  s, squash = use commit, but meld into previous commit
  #
  # If you remove a line here THAT COMMIT WILL BE LOST.
  # However, if you remove everything, the rebase will be aborted.

  pick 就是保留原本commit與commit log
  把 pick 改成以下字串
  edit 會amend這個commit log
  squash 這個commit log不會出現了
  或者只是單純移掉那行就可以把那個commit 與commit log拿掉
  \end{verbatim}
  遠端要反悔要先確定沒人pull, merge你的remote branch, 然後local先反悔,遠端
  再用push \verb=--=force
  \subsubsection{遠端}
  \begin{itemize}
    \item git remote add b1 https://my.com/origin
    \item git remote -v
    \item git branch -r 看遠端有什麼branch, git branch -a所有branch
    \item git clone 只抓下master,複製一個遠端repositroy過來。
    \item git pull 抓下遠端並做merge,類似 svn update, git pull \verb=--=rebase
      會順便做 rebase,把頭頭 HEAD 指向最新地方,這也是通常預設使用。
    \item git push 類似 svn ci, 算是省略的git push (remote) (branch),
      或更精確的 git push (remote) (refs/heads/branch:refs/heads/branch)
      真正的意思是把local branch放到遠端 remote 的 branch。git 的版本只是一
      堆指標(refs) 指來指去而已,然後有個特殊指標 (head)。
    \item git push origin :b1 這反而是刪除遠端origin下的一個b1 branch, 或者用
      git push origin \verb=--=delete b1
    \item git fetch 抓下遠端所有branches資訊,但不做merge。 git fetch -p 會順便
      砍掉已經遠端被delete 掉的 branch,應該說這很像 package 管理,遠端跟 local
      端都有一份資料, fetch 就像 apt, yum update 一樣。
    \item git checkout -b local\_dev origin/remote\_dev,把遠端remote\_dev這
      branch抓下並且在local產生local\_dev branch,然後跳到local\_dev。
    \item git push --set-upstream origin feature/alibuda 表示目前local已有一個
      feature/alibuda的branch,然後在遠端從origin開一個feature/alibuda,並且
      讓local去track遠端同名的branch, 則當 git pull 時就會抓下兩邊同名並rebase
    \item git pull --rebase origin somelabel 這是 rebase 到遠端某個已設定 label
      的 branch
    \item git remote purge origin; git fetch -p 會sync遠端跟local的資訊,刪掉
      local端那些已經不在遠端的branch
  \end{itemize}
  基本使用命令都在操作一個內定refs的指標上,如果想要變化其他的branch, remote
  , stage, working的地方,只要變化這個refs後,就可以用類似的操作。單純working
  , local跟remote的操作。像 svn 有HEAD PREV一樣,HEAD指向某個branch的最新版。
  \begin{verbatim}
  HEAD^     前一版本,等於svn的PREV
  HEAD^^    前兩版本
  HEAD~5    HEAD前5個
  tag^      上一個tag
  \end{verbatim}
  有個很重要的 git stash 是
  如果作到一半,有人要你修某個 bug ,但你原本作的一半可能已經無法 build,或者寫到一
  半不完整,但又不想放棄,這時用 stash 可以先把亂七八糟的working directory先丟到
  一個地方,他就會回到原本 clean 的 HEAD上,改完bug後,我們可以 commit,這時用
  stash pop或apply就會再拿回來並且自動merge fix,但這不是開一個 branch ,
  這只是暫時跳到一個 clean 狀態。
  \begin{verbatim}
  git stash save
  git stash list
  git stash pop
  git stash apply stash@{0}
  git stash clear
  \end{verbatim}
  另外之前的 .gitconfig 可用 git config 命令加減
  \begin{verbatim}
  git config --global user.name "John Doe"
  git config --global user.email johndoe@example.com
  git config --global core.editor vim
  git config --global merge.tool vimdiff
  設定alias命令
  $ git config --global alias.co checkout
  $ git config --global alias.br branch
  $ git config --global alias.ci commit
  $ git config --global alias.st status -s
  設定顏色
  git config --global color.diff auto
  git config --global color.status auto
  git config --global color.branch auto
  \end{verbatim}
  但我想大家都是直接改 .gitconfig 吧,他裡面跟其他像 .vimrc 一樣是能定義變數
  ,例如
  {\scriptsize
  \begin{verbatim}
  [alias]
    lg = log --graph --pretty=format:'%Cred%h%Creset 
      -%C(yellow)%d%Creset %s %Cgreen(%cr) %C(bold blue)<%an>%Creset' --abbrev-commit --date=relative
  \end{verbatim}}
  當用 git lg 時,
  {\scriptsize
  \begin{verbatim}
* 8deb85dec4f - TunnelMismatch :Auto-committed by the workflow: PAW (8 days ago) <Alibuda Huang>
* 63d8dc33c06 - Syslog does not appear on console (8 days ago) <Aburamisi Honda>
* 443078ed1e1 - Sort and Display Wlan based on IDs and not name. (8 days ago) <Albert Einstein>
* 841e05a7feb - Adding the remaining parameters for the security features (8 days ago) <Indra Gunawan>
  \end{verbatim}}
  這個 git log 還是相當複雜,可以好好研究一下。
  \begin{verbatim}
  git log git log -n1 --pretty=format:'%h %ad %s (%an)' --date=short 看日期
  git log --author ${USER} --oneline 只看自己的 commit
  git log --follow [file] 印出檔案的所有 commit history
  \end{verbatim}
  除了 .gitconfig 有個 .gitignore 相當重要是列出不想被 git 處理的檔案名稱
  ,有個\href{https://github.com/github/gitignore}{github/gitignore}
  彙整了大家的 template,可以去查直接套用。
  \\\\
  其他有用的
  \begin{verbatim}
  git blame   跟svn blame一樣,一行一行知道是誰什麼時候改的,責任歸屬。
  git 
  git archive 跟svn export很像,拿掉.git這個隱藏目錄幫忙打包。
  git archive --format=tar.gz --prefix=git-1.4.0/ v1.4.0 > git-1.4.0.tar.gz
  git archive --format=zip --prefix=git-docs/ HEAD:Documentation/ > git-1.4.0-docs.zip
  git archive -o latest.zip HEAD
  \end{verbatim}
  \$Id 基本上在 git 上是沒有用的,因為一長串的sha1 id,不是流水號, 沒有意義。
  submodule 使用一樣能串香蕉的掛上外部另一個 repository 變成子 module。
  \begin{verbatim}
  git submodule add https://myproj.com/submodule
  git commit 
  git push
  git pull
  \end{verbatim}

  \subsection{worktree}
  在 clearcase 中有個 work space 之類的東西,也就是整個 source 會 copy 到一個
  目錄下作維護,本來實在想不透為什麼要有這畫蛇添足的東西,後來 git 也支援了,
  這東西是到大公司後才了解的,提出 request 也應該是用這習慣的人,這主要是因為
  開發軟體不光是基本 source 而已,例如 ctags 一旦創造,他只針對目前下面 source
  ,你一旦變化 branch ,則 ctags 就沒用了,另外很多人用了複雜的 IDE 軟體,這些
  IDE 軟體有他們本來的 framework ,還有 configuration 檔,這些檔也是一旦換
  branch 等等的,也都變不能用。build make 會根據日期來,所以也是一樣,擁有一個
  完全乾淨 source 開發是減少頭痛的小問題釐清快速方法之一。
  \begin{verbatim}
git clone --no-checkout ssh://github.com/cyril-huang/gyoza.git
不要真的 checkout source 而是光 clone 資訊就好。
git clone -b v171.throttle ssh://github.com/cyril-huang/gyoza.git local.v171.throttle
checkout 一個遠端 branch 到 local 端的目錄來
git log -n${NUMBER} --oneline
看一下 log

cd gyoza
git fetch -p
git worktree add -b ${BUG_ID}.${PARENT_BRANCH} ../${BUG_ID}.${PARENT_BRANCH}
從 publish 的 branch 再開一條要工作的私有 worktree 到外面的bug_id.parent_branch 目錄
公司的branch merge 規定中的branch與檔名規則有${BUG_ID}.${PARENT_BRANCH}
git worktree list
git branch
從此到外面那個 worktree 下工作
git worktree remove -f ../bug_id.parent_branch
git branch -D ${BUG_ID}.${PARENT_BRANCH}
  \end{verbatim}


  \subsection{branch/merge}
  local branch,開很大,不用錢。但兩個branch要有從屬關係時,他的觀點跟svn是完
  全不同,git的branch只是一個指標指到一個commit點,branch只是一個commit point
  的alias而已。在某branch commit時的紀錄上,你能說一個branch 包含某個commit,
  不像svn 是一個commit是在某branch 裏面完成。checkout時,HEAD轉到那隻branch的
  最後那個點上而已。基本上可以是沒有從屬關係的,就是很單純很多sanpshot 的點轉
  來轉去與紀錄,就叫branch。所以當開出一條branch時,如果你改動了local某檔,
  然後又開branch,則還是會在任一branch上看到改動,這是從svn等srm過來的人很不
  適應的,一定要commit了,才會在不同的branch有不同的結果。git不允許working 
  directory存在沒有commit 的東西,然後做switch,要這樣只能
  \begin{itemize}
    \item git commit
    \item git stash save
  \end{itemize}
  一般的branch命令
  \begin{description}
    \item [git branch -v] \hfill \\
      看local所有branch
    \item [git branch -r] \hfill \\
      看遠端所有branch
    \item [git branch -a] \hfill \\
      看所有branch
    \item [git branch branch1 master] \hfill \\
      從master開出一條branch1
    \item [git branch b1 \texttt{--track} origin/startpoint] \hfill \\
      開出一條 b1 然後是跟遠端 origin 下的 startpoint branch 做連結。
    \item [git branch -d branch1] \hfill \\
      刪除一個完成工作的branch。
    \item [git branch -D branch1] \hfill \\
      強制刪除沒有回到爸爸的branch。
    \item [git checkout branch1] \hfill \\
      轉換到branch1
    \item [git remote update origin \texttt{--prune}] \hfill \\
      表示update目前 local 下的遠端branch資訊
    \item [git show-branch] \hfill \\
      看目前的 branch 的情況,如果前面有*,表示目前所在 branch,在\verb=---=
      之後是每個  branch 的 commit 紀錄,如果是 + 表示一般 commit - 表示 merge
      commit
    \item [git branch \texttt{--merged}] \hfill \\
      看所有已經merged過的branch,出現在list上的, 就可殺掉了。
  \end{description}
  在 git 中,我們還是要問任意兩個 branch 可以隨便亂合併嗎?如果有從屬關係
  ,可以隨便爸爸來小孩,小孩回爸爸嗎? 如果是 sibling branch,要怎麼合併,
  才不會因為同時 rebase 了爸爸,而重複的 code 造成 git 的迷惑? 另外很重要的
  就是這些 commit, merge 等等都有commit log,這些commit log的整理與管理才是
  做source control最重要的部份
  \\\\
  第1,爸爸來小孩,稱為 rebase,小孩去找爸爸為 promote 是標準定義,在 git 中,
  跑到小孩那去 rebase 爸爸
  \begin{verbatim}
  git checkout child
  git rebase papa 一般小孩rebase
  git rebase --onto papa child grandchild 這是非常powerful的rebase
  \end{verbatim}
  跑到爸爸那去 merge (promote) 小孩
  \begin{verbatim}
  git checkout papa
  git merge child
  git merge --no-commit child 先不要自動commit,等會我自己做
  git merge --squash child  不要產生commit log的promote。
  \end{verbatim}
  第2,git 的 merge(promote),有以下選擇,也包含了commit log的整併
  \begin{itemize}
    \item straight merge 一般標準 git merge,會有傳統 merge log。
    \item fast-forward 如果開出的 branch 改動,但原本的爸爸沒有改動,則
      promote 時,會把爸爸的 HEAD 也指向新的 commit 點。因為只是
      forward HEAD點,所以稱 fast-forward。沒有傳統merge log。
    \item squashed commit 使用git merge \verb=--=squash,不會有原有
      branch commit log。 但會把所有 branch commit log 擠成傳統一個 merge log。
    \item cherry-pick 這跟 svn 一樣能夠蜻蜓點水式的合併。
  \end{itemize}
  第3,git 的rebase 跟 merge 命令就像 svn 的 merge 一樣是可以亂跑的,也就是
  git rebase merge 命令是籠統合併的意思。但你在看文件例子時,你可以看到建議的
  方法是照標準做的,這跟 svn 的標準文件說法是一樣的。但合併是 delta 變化與 local
  working產生新結果的行為,只要是有兩點的 delta 就能做合併。所以在小孩那不一定
  要用 rebase, 也能用 merge papa,就是跟 svn 同樣道理。
  \begin{verbatim}
  git checkout child
  git merge papa
  git checkout papa
  git rebase child
  \end{verbatim}
  那這兩者有什麼不同嗎? svn 的 sync 後,他會自動記住現在是哪一點 sync 過來的,
  rebase 也是一樣的道理,只是 svn 的 sync 過後,分歧點不會變,git 的 rebase 分歧點
  會改變。也就是 svn merge 如果不用 -r p1:p2 的話,他自動又是整支 branch 的 delta
  ,但 git 因為分歧點會變,也沒有-r p1:p2 這種選項,所以永遠都是目前整支的 delta
  。在實作上面,他們定義是
  \begin{itemize}
    \item rebase : 選擇某一分支的改變,將所有改變重新在另一分支試著重複一遍。
      分歧點改變。commit log變成自己的一部份。這點很麻煩,等於改動原有 commit
      log。所以做 rebase 要非常小心,出錯反悔很麻煩,因此已經 publish 外面 central
      server的紀錄,是不准做rebase的。
    \item merge  : 選擇兩點 delta,一點是 HEAD 一點是共同祖先點,跟 working 運算,
      分歧點不變。commit log 仍然在別支上。所以reset時的前一個會跟用單純 git log
      看到的前一個不一樣。當用 git reset \verb=--=hard HEAD\verb=^=時,是會跳回沒有
      merge 前的狀況,而不是那個所有commit log的前一個而已。
  \end{itemize}
  git的統一rebase/merge 流程,不分爸爸小孩的任兩支branch b1, b2與解決conflict的流
  程為
  \begin{verbatim}
  git checkout b2
  git rebase b1

  vi myconflicts
  git add myconflicts
  git rebase --continue (可以用git rebase --abort回到最原始狀態)

  git checkout b1
  git merge b2
  \end{verbatim}
  每次rebase 後等於把某一支 branch 的更動往另一支送,並在另一支的歷史上加入
  log,rebase 跟 merge 都有可能 conflicts,所以那個 vi 到 git rebase \verb=--=continue
  的循環有時可能會很多loop。最後那個 merge 的作用只是在於 b1 HEAD 的擺正,及要不
  要 merge log 的選擇而已
  \begin{itemize}
    \item git merge b2 : 內定fast-forward,沒有多merge log,只是把b1的HEAD
      擺到最前面
    \item git merge \verb=--=no-ff b2 : 不做ff, 多一個merge log,b1看到所有b2的
      commit log。
    \item git merge \verb=--=squash b2 : squash,多一個merge log, b1看不到所有b2
      commit log,只看到最後merge log。
  \end{itemize}
  那最後那個 merge 可不可以不做呢?可以,就是 b1 HEAD不擺正,繼續 commit 新變化時,
  再一次造成分歧而已。但如果是不做 rebase 單純做 merge (promote)回去呢?也是一樣
  的道理,就只是如上面定義的差別而已。這兩個結果應該要一樣
  \\\\
  既然最後結果一樣,那 rebase 跟 merge 對使用者有啥意義?基本上的差異是
  \begin{itemize}
    \item 不同的選項只是對於 merge log 的多寡,還有因應那麼多的 state,跳到某特定
      state 的情況而已。
    \item rebase 的 upstream branch是否含有目前本身這支的 commit log 的選擇而已。
  \end{itemize}
  也就是類似 svn mergeinfo的整理。說實話,這沒有哪個好哪個不好,純文字的話,
  主要 branch 基本上看起來是一樣的 log。當然我實在不曉得一堆人開的那些自己所謂幾
  十支branch 到底有沒有用,還有每次 commit 是否是自己精心的 commit,還是又只是亂
  commit 一通。備份??沒有review??
  \\\\
  再來看之前兄弟 merge 的 svn 例子,n o都是從m拉出來的,之間各有 sync 到那
  branch,在n release 後,想要n 到 o。
  \begin{verbatim}

   123  a  456  b 789  c 901  d  1000
   -------------------------------              m
    \       \        \     \      \
     \       \        \   e \950 f \ 1010 g
      \       \        \-----v------v---------- n
       \       \         x
        \-------v------------------------------ o

  git checkout n
  git rebae m
  git checkout o
  git rebase m

                          ----------n
                         /
                        /
  ----------------------            m
                        \
                         \
                          -----------o
  \end{verbatim}
  其中 git rebase \verb=--=onto newbase upstream branch這東西
  \begin{itemize}
    \item newbase 新base
    \item upstream remote的一個branch
    \item branch 目前工作的branch
  \end{itemize}
  意思是說找出 branch 跟 upstream 的共同點,然後 apply 到 newbase 去。
  這也可以用在兄弟branch的合併,跟svn 2 URL merge 一樣是常用到的技巧。
  \begin{verbatim}
  git rebase --onto o m n
  \end{verbatim}
  rebase並不是什麼神奇的東西,一樣是一堆的 commit 差異到另一支 branch上 的
  手段而已。碰上很複雜的 conflicts 一樣是沒輒。
  \\\\
  解決了上面兩個最重要的 local branch/merge 問題後,那遠端呢?
  \\\\
  遠端 branch/merge 可分兩種,一種是 distribution developer 之間 merge,一種
  就是像svn server那樣的 branch/merge。基本上如果使用git,沒有developers
  之間的merge,那是沒有意義的,local自己亂玩,怎麼玩的log都跟真正在外面
  server的無關,不過話說回來,我們公司用github, bitbucket的感覺也只是多了
  code review 與自動 trigger build 的 SVN 加強版而已,我很少看到同事間的互相
  merge。
  \\\\
  可以直接開一條新branch newb1與remote有直接關係,而不是跟 local 開出。
  \begin{verbatim}
  git checkout -b newb1 --track origin/b1
  \end{verbatim}
  其實tracking是 default 等於
  \begin{verbatim}
  git checkout -b newb1 origin/b1
  \end{verbatim}
  這newb1 稱為 tacking 或upstream,git push \verb=--=set-upstream 時用的是一樣。
  只是後面跟的參數有點不一樣,origin/b1 跟 origin b1
  \begin{verbatim}
  git push --set-upstream origin b1
  \end{verbatim}
  遠端branch使用跟local一樣,唯一要注意的是
  rebase 會更動歷史紀錄,rebase 就像 svn 的reintegrate一樣,準備要放棄原
  有的小孩,另外創建新的爸爸紀錄。所以如果這 branch 是 public 的遠端 central
  server上的 branch,那rebase就不能亂做,因為會打亂遠端歷史紀錄,而同伴還要
  pull一遍,然後造成一堆困擾。但local就可以亂搞,
  所以常看到人家說git pull \verb=--=rebase,就是 rebase 自己目前 local branch,但紀錄
  卻只有一個。
  \\\\
  commit log 與 history 的維護,跟svn mergeinfo 的維護一樣,都在保持資料的正確性
  ,其中又是以 branch/merge 的資訊維護最重要與麻煩。合併並不是單純自己 branch
  的合併,最重要的是共同 team branch 的合併,這通常是由branch owner,也通常是
  某組的 lead 負責把大家所有的 effort 往回倒。

  \section{branch/merge policy}
  除了工具的幫忙外,管理最重要的還是人的思維,如何先用腦袋plan好,再利用工具
  的優勢來處理是為組裏面的管理policy。
  \\\\
  基本上很多使用例子會跟你使用情境有關,例如有沒有再繼續commit,有沒有
  double commit,有沒有從屬關係,有沒有已經sync,有沒有需要維護合併,
  不是單一爸爸小孩就結束了,那種太簡單了。
  \\\\
  當然rebase的時候,如果整條branch拉太長了,整個conflicts的解決還是很痛苦的
  ,這跟有沒有rebase沒有關係。即使是rebase, merge一樣跟svn有很複雜的conflicts
  需要解。但主要是組裡的人寫作的能力,寫作能力太差基本上就是幾十號人在那邊瞎
  做,真正有戰鬥力的小組是不會超過5個人的,幾十號上百人的產品都是維護的多,但
  其實 code 做久了就是很多不同使用情境造成後面邏輯混亂的維護。
  \\\\
  不同的branch有不同的merge/branch policy,branch的使用與分類
  \begin{itemize}
    \item master/trunk/mainline 穩定的code line, 必須永遠保持可build,可run。
    \item release 這通常是已經準備要出去的,但這很麻煩的是在我們公司內部還有一
      層,也就是team與team之間,正式對外都有release/QA drop 的release。非常
      討厭。
    \item maintainence release後的維護branch
    \item hotfix 快速修補bug用的,所以在git通常很快並會fast-forward。但通常也根
      本不開這樣branch。
    \item feature 開發feature用的branch
    \item double commit code freeze後的處理,但我想現在這很難,大公司的 code
      freeze 維護,一般小 team 還是亂搞,只有真正世界級軟體,歐美日維護的軟體
      才相當嚴格, 現在寫code不像以前那麼嚴格,尤其很多印度人中國人做的管理,
      相當 ... 。
  \end{itemize}
  branch owner應該是組裏面的資深leader,他必須負責這個branch的policy推動,解決
  conflicts,保證code的乾淨正確,merge回mainline的事務。而標準動作的policy
  \begin{itemize}
    \item rebase的policy
    \item promote的policy
  \end{itemize}
  \subsection{Release 與 branch}
  release 的管理其實跟很多有關,除了 code 品質,也跟 marketing, 對內對外等有關
  ,通常我們是 branch 中有
  \begin{verbatim}
  v1.1.1
  v1.1.1-throttle
  feature/feature-id.branch
  bug/bug-id.branch
  \end{verbatim}
  throttle 是節流閥的意思,一個 release 或者任何管理就像水流一樣,一直在變動的
  ,要有控制調整流動的速度,一旦進入 throttle 狀態 branch ,就表示不能再隨意
  大量的 commit,這會進入 preRelease 狀態,alpha, beta ...,與真正 release
  freeze 狀態, 從此除非有重大修改,會跟主 branch 脫節,獨立成為自己維護 branch
  除非有重大 security, show stopper crash 等 commit ,不同獨立維護 branch 都要
  double commit,到處 commit。
  \\\\
  我們公司組織中有 developer team (dev, DE) ,跟很接近DE 的 development test
  (devTest) QA team,外圍的 testing 還有很多層,regression, system test,
  automation test 等等組織,每一層測試時,也都要有統一界面,不然誰曉得你測試
  的發出 bug 的根據是哪一個,因此連 release 都有分對內對外,對內,nightly
  build ,devTest, regression test 等等的都會不同,因此才會有公司針對這種管理
  特別做管理工具軟體,大公司很多其實內部是自己做的,因此大軟體公司經營的 cost
  很多都是這種品質要求的管理 cost,人的 cost 。這跟亞洲還沒吃飽飯只要有飯吃
  的管理思維是很大差距的。不過後來因為 cost down ,很多都外包印度,中國等地方
  ,造成品質其實都是往下降的。
  \\\\
  最後每次 release 的版本號,<version>-<release> 都要加上 git tag。version
  與 release 是不一樣的事情,source code 有 source code 的 version,binary
  release 相對 source version 的 release,然後還有紀錄
  的 tag。
  \section{結語}
  基本上沒有所謂哪一種SRM工具好過哪一種的說法,因為所有的SRM其實道理都是一樣的
  ,你用svn碰到的問題,在git一樣會碰到,網路上的教學都是最初淺的,真正6, 7支
  branch 由幾十人以上同時玩,並且是publish出去時,如果release plan的branch很亂
  ,不管你是哪一種SRM,通通都是沒有用。原因在於真正有用的log message是只有
  對外publish釐清問題與追究責任歸屬時,而一旦是上到central server的log,那機制
  都是一樣的。而對外release branch同時有很多支,問題無法單純用SRM工具解決。不管
  svn, git 我都碰過那種最後亂到 conflict 無法解決,只能重新來一遍的 branch。
  \\\\
  還有不要再亂commit了,因為那只是好像變成backup,而不是SRM,一堆無意義的log,跟
  沒有history log是一樣的。 我碰到的例子是開了4支sub sub sub branch, 因為要
  release給QA team, 要 developement,結果到處fix bugs,到處double commit, 
  最後還要求某支孫子回到爺爺 merge, 而且常有 branch 開了好幾個月要求merge的,
  這種情形就只能cherrypick,如果還亂commit,根本就無法做下去
  。總之工具不是萬能的,不要再有用上什麼工具就天下無敵的迷思。
  \\\\
  除了自己架設外,網路上也有很多免費的服務,像 github, bitbucket, gitlab,
  sourceforge, google code
  等,其實像github,每個月幾美金,就能有自己私人的repository管理,一般小公司其
  實可以把source丟上去,省下自己維護server, backup, 24x7這種煩人的事。
