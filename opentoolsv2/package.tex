\chapter{套件包裝}
package tools是release的包裝工具,很多人都在抱怨Linux很難安裝,其實是不懂得
Linux的安裝機制,只要你懂得運作原理的話,就可以順利的把想要的檔案自動裝到特
定的地方去。在unix的世界裡面,Solaris, HP-UX, 等也都有一套套件管理的辦法,
像Solaris有pkg tools,只要用pkgadd, pkgrm都可以劈哩啪啦就裝好或移除一個軟體,
而且裝到哪裡去,其實也可以知道。當然也可以自己寫給INSTALL script,不過這樣比
較不好,改版管理沒有中央集權會很混亂。套件製作原理只是寫一堆資訊檔,還有安裝
/移除的script,顧名思義,當install/remove套件時,這些scripts會被自動叫起來。
\\\\
建立package的原則是, 
\begin{itemize}
  \item 寫一個資訊檔,
  \item build source code原始程式碼到一個假的 root目錄,這個假的 root,在
    一般 source 的 ./configure 命令中,用 prefix 來表示,因此變數通常用
    prefix, DESTDIR 在 Makefile 中。
  \item 用工具把這個假root directory建立 binary package檔。
\end{itemize}
通常在原始程式碼社群中不僅要 build binary package也要build source package,
這個source package包含了包裝者對原始碼的更動或patch,主要因為open source 
community的package包裝者與source coding的人往往不是同一人。而是因為
每個distro都會從upstream拿到最原始放出的source code,然後自己根據自己
distro的決定,加上自己的patch後,編出適合自己某個版本的binary。binary裏面都
有特別的參照跟string寫進去。所以不是亂拿source編譯一通就可以安全無虞的在舊
系統亂跑。
\section{基本使用}
目前Linux兩大種類package, debian/ubuntu/mint等使用deb型式套件,
redhat/centos/fedora 等使用rpm套件。 套件資訊有幾個地方
\begin{itemize}
  \item 套件本身 - 套件本身的資訊像是版本,檔案列表,相依性。
  \item 系統上 - 系統上的套件訊息,安裝與否,相依性。藏在/var/lib/dpkg
    或/var/lib/rpm下
  \item 套件集中地 - 一堆套件但不在目前的系統上,像是url, 套件列表,安全檢查。
\end{itemize}
  \subsection{基本命令}
  debian/ubuntu, redhat/centos/fedora等基本cli,分別為dpkg與rpm,但後來有更多
  工具例如apt-get(通改為 apt), yum(已經改為 dnf了)來處理更大的 repository
  集散中心。
  \\\\
  對於一個套件檔案查詢
  \begin{itemize}
    \item 一般資訊\\
      dpkg --info\\
      rpm -pqi 
    \item 有哪些檔案\\
      dpkg --contents\\
      rpm --pql 
  \end{itemize}
  對於系統上套件資訊查詢
  \begin{itemize}
    \item 一般資訊\\
      dpkg --status \\
      rpm -qi
    \item 裝了沒\\
      dpkg --status\\
      rpm -qa | grep pkg-name  or yum list installed\\
    \item 裝了什麼東西\\
      dpkg --listfiles pkg-name\\
      rpm -ql pkg-name
    \item 什麼套件有這檔案\\
      dpkg --search "/bin/ls"\\
      rpm -qf `which ls` or yum whatprovides "/bin/ls"
  \end{itemize}
  對於套件集中地,必須有更上一層的工具apt與yum來處理,目前configuration檔在
  /etc/apt/sources.list 與 /etc/yum.repo/。apt與yum都能處理dependency,也就是
  例如我們想裝gnome,gnome有好多其他相關packages, libraries都要裝,他會根據
  package裏面的層層訊息,全部裝起來。
  \begin{itemize}
    \item 更新資料 - 這是把local與遠端的資料同步一次\\
      apt-get update\\
      yum update
    \item 安裝\\
      apt-get install\\
      yum install
    \item 移除\\
      apt-get remove\\
      yum erase
  \end{itemize}
  以網路安裝的集散地,例如 http://mirrors.kernel.org 下面有
  http://mirrors.kernel.org/debian 跟 http://mirrors.kernel.org/centos,
  除了 CLI 的工具外,debian/ubuntu 上有好用的curses based 的 aptitude,GUI上,
  debian/ubuntu/mint 的 synaptic, redhat/centos/fedora 的 packageKit。
  \\\\
  前面在 build 的時候,曾經講過大致上的版本號碼規則,傳統opensource, linux 
  kernel , library有版本的基本規則, current.revision.age-patch 的編號原則,後來
  deb rpm檔的規則也大致是如此xx.yy.zz-vv,只是可能名稱不一樣,有一個
  https://semver.org/ 定義了major.minor.patch-preRelease+buildMetadata 這樣的版
  本格式,也就是可能在沒有正式release 前,也能用這規則在每天的 nighly build。
  例如 1.0.1-alpha+20171203.amd64 ,preRelease就是alpha, beta這種名字,
  buildMetadata就是build的日期,或流水號。基本上版本高於1.0.0後就是正式對外API的
  release,我們可以用這原則建立自己的版本號碼。

\section{產生deb}
基本準備工具要安裝 devscripts, dpkg-dev,這裏面有很多命令,所以裝起來還蠻多的。
一個deb檔其實只是個 archive 格式的檔,可以用 ar 轉出來
\begin{verbatim}
$ ar vx xxx.deb 會產生 data.tar.gz (這名稱不一定,有可能gz, bz2, xz ...)
$ tar zxvf data.tar.gz
或者
$ dpkg --contents xxx.deb
$ dpkg -x xxx.deb .
\end{verbatim}
就會看到原本的假 root 目錄結構是怎麼樣的。有需要的話,
大部分的文件可以在\href{https://www.debian.org/doc/}{debian doc} 找到,
有些debian的基本規定與\href{https://www.debian.org/doc/debian-policy/}{policy}
希望遵守, 這有點像寫code的convention,希望大家遵守,我們可以先略過,以後
熟悉了,再回來仔細看。

  \subsection{由distro提供檔案建造}
  如果想要debian的source重新在自己機器上編一個新的package,則到 debian 或
  ubuntu 的 pool 目錄裡面的套件分類下載三個檔案
  \begin{itemize}
    \item xxx.orig.tar.gz
    \item xxx.debian.tar.gz (或者xxx.diff.gz)
    \item xxx.dsc
  \end{itemize}
  把他們放在一起,然後用
  \begin{verbatim}
  dpkg-source -x xxx.dsc
  \end{verbatim}
  就會解開,並且apply那個 diff,或者解開xxx.debian.tar.gz到一個新目錄,
  這個目錄下應該會有diff,通常是一個debian目錄,藏著package build所需,
  或者用命令幫你下載
  \begin{verbatim}
# apt-get source xxx
# apt-get build-dep xxx
例如 openssl
# apt-get source openssl
  \end{verbatim}
  則會下載3個檔, xxx.orig.tar.gz, xxx.debian.tar.gz(xxx.diff.gz), xxx.dsc
  並自動用 dpkg-source 命令幫你解開,進到source上, 最簡單的跑
  \begin{verbatim}
# dpkg-buildpackage -us -uc
  \end{verbatim}
  或者devscripts 套件下的命令
  \begin{verbatim}
# debuild -us -uc
  \end{verbatim}
  這樣應該就會在上一層目錄產生由這個xxx.debian.tar.gz 所定義的packages 了。
  通常我們用apt-get build-dep就會順便裝完應有的dependancy,
  如果編譯這個package時,他那個Build-Depends沒有寫好,需要其他的package,
  這時會出錯,錯誤訊息會有所需套件名稱,所以需要一一先裝好,就應該可以了。
  不過這有的很可怕,因為有的package會建立文件,有的會連latex, docbook 都給你
  抓下來, 所以可以先不用apt-get build-dep 只跑dpkg-buildpackage看看怎麼回事。
  \begin{verbatim}
# fakeroot debian/rules binary 在上面目錄產生
# fakeroot debian/rules build
  \end{verbatim}
  但有很多情況是有的版本跳躍的太厲害了,則編譯的dependancy packages 已經不一樣
  了, 也就是有的新版本patch 無法在舊系統上很容易的編譯出來的,這時只能自己去改
  debian 目錄下的檔案。其中重要的檔案有debian/rules, debian/control,有些新版
  本的 source code 無法在舊系統下重新編譯,通常要來改這兩個檔,rules 其實就是
  Makefile, 裡面通常有 ./configure 的呼叫,control 是 package 的資訊,通常會有
  其他相依libraries的定義,有時相依 package 名字變了,也要到此更改。

  \subsection{自己產生debian binary 套件}
  最簡單的就只是把已知位置的 binary,文件檔在將來的系統上,放到一個假的 root
  目錄,,然後工程師要在假root tree上準備一個DEBIAN目錄,基本上裏面有以下
  檔案,
  \begin{itemize}
    \item control   套件資訊檔
    \item preinst   四個shell script檔會被dpkg自動呼叫做該做的事
    \item postinst  可以看/var/lib/dpkg/info下的範例
    \item prerm
    \item postrm
  \end{itemize}
  最最最基本的就只有control這個檔, 可以man deb-control 了解細節,最基本的欄
  位有5個欄位, 例如
  \begin{verbatim}
Package: helloworld
Version: 0.9.9
Maintainer: Alibuda Huang <alibuda_huang@gyoza.net>
Architecture: amd64
Description: hello world
  \end{verbatim}
  首先我們建造自己的假root, fakeroot,把想放的結果先放到fakeroot下。
  \begin{verbatim}
mkdir -p helloworld/usr/bin/
mkdir -p hellowrold/etc/
touch helloworld/usr/bin/helloworld
touch helloworld/etc/helloworld.conf
  \end{verbatim}
  然後把剛剛的control檔放到一個大寫DEBIAN的目錄下,並且用dpkg-deb命令建造。
  \begin{verbatim}
mkdir -p helloworld/DEBIAN
cp control helloworld/DEBIAN/control
dpkg-deb --build helloworld .
ar t helloworld_0.9.9_all.deb
dpkg --info helloworld_0.9.9_amd64.deb
dpkg --contents helloworld_0.9.9_amd64.deb
  \end{verbatim}
  這樣就有個deb了,用ar t去看他內容。基本上helloworld這個目錄就是假的 root 。
  \\\\
  後續要知道的是
  \begin{itemize}
    \item Architecture 可以用 all, 或者 dpkg-architecture 去產生的結果。
    \item Description 的第一行是少於 60 字元的 summary,第二行起每行要空一格
      才是完整敘述。
  \end{itemize}
  所以正常從source make install就要裝到這目錄來,make install 要有 prefix 或
  者 DESTDIR 變數寫在 Makefile 裡面。
  \begin{verbatim}
make prefix=helloworld install
make DESTDIR=helloworld install
  \end{verbatim}

  \subsection{由自己 source 產生 package 檔}
  發現之前用 apt source xxx 下載的 debian source 的 control 目錄是在
  xxx/debian,但怎麼我們自己做的變成大寫的DEBIAN,原來大寫的DEBIAN 是給binary
  package 用的,如果要從一個 source 重新build起,要包括 source package 就要用小
  寫的 debian,這是比較標準的作法。
  \\\\
  這種標準作法是要產生 source package 的,要產生 binary package 要用 DEBIAN
  目錄,兩者都要產生就要兩個目錄都要做,而因此有 dpkg-gencontrol 命令會去讀
  debian/ 下的檔案建造DEBIAN control。
  \\\\
  所以我們要建造從source 開始build起的,這就比較複雜,首先要在 source tree 的
  最上面建造一個 debian 目錄,裡面放著所有建造 deb 套件的資訊,通常source 上
  最少要有 Makefile 來建造binary 檔, 不過我們先看別人寫的,來做解說。
  \\\\
  當用ar 解開 binary deb檔時其中有個control.tar.gz,裏面就
  藏著 control檔,或者我們用apt-get source xxx, 解開那個 xxx-debian.tar.gz,
  或去網路 debian/ubuntu/mint 的pool下的xxx-debian.tar.gz 也可以。
  以openssl為例的control檔,可以去解開
  \begin{verbatim}
  openssl_<version>-<release>.debian.tar.xz
  \end{verbatim}
  \begin{itemize}
    \item control 這是 source control 檔,雖然都是 control 檔,要注意的是跟
      前面 binary control 檔是不同的。兩者的完全解說可從 policy 下的文件找到。
      https://www.debian.org/doc/debian-policy/ch-controlfields.html,或者
      man deb-control, deb-src-control
    \item rules     這是 Makefile ,是 debuild/fakeroot 會來呼叫的 target 定義
    \item conffiles configuration檔,在upgrade時,不想被蓋掉的檔。
    \item copyright debian系列蠻重視版權的,這個也有特別格式。
    \item changelog debian系列蠻喜歡 changelog 的
    \item patchs 加上自己的改良或內定所做的patch,現在多為目錄了。
    \item source 這目錄裏面有個format表示你的source是哪一種型式。先使用
      3.0 (native),後面再解說。
  \end{itemize}
  相對應的package檔有相關的副檔名 .dirs, .install, .docs ...
  \begin{itemize}
    \item .dirs
    \item .install 將來放到系統去的binary檔。
    \item .docs 將來裝到/usr/share/doc下的檔案,可以比較自己的跟這檔案內容。
    \item .preinst .postinst .prerm .postrm 現在改成副檔名的4個 scripts.
    \item po 多國語言檔目錄
    \item gpg key檔 debian package用gpg 非對稱鑰匙簽名。
  \end{itemize}
  這先暫時不管,因為大部份都大同小異只是照抄,最重要的就只需要
  control, changelog 跟 rules, source/format, 這4個檔來從source建立我們的套件,
  所以先看control
  \begin{verbatim}
Source: openssl
Build-Depends: debhelper (>= 10), m4, bc, dpkg-dev (>= 1.15.7)
Section: utils
Priority: optional
Maintainer: Debian OpenSSL Team <pkg-openssl-devel@lists.alioth.debian.org>
Uploaders: Christoph Martin <christoph.martin@uni-mainz.de>
Standards-Version: 3.9.8
Vcs-Browser: https://anonscm.debian.org/viewvc/pkg-openssl/openssl
Vcs-Svn: svn://anonscm.debian.org/pkg-openssl/openssl/
Homepage: https://www.openssl.org/

Package: openssl
Architecture: any
Multi-Arch: foreign
Depends: ${shlibs:Depends}, ${perl:Depends}, ${misc:Depends}
Suggests: ca-certificates
Description: Secure Sockets Layer toolkit - cryptographic utility
 This package is part of the OpenSSL project's implementation of the SSL
 and TLS cryptographic protocols for secure communication over the
 Internet.
 .
 It contains the general-purpose command line binary /usr/bin/openssl,
 useful for cryptographic operations such as:
  * creating RSA, DH, and DSA key parameters;
  * creating X.509 certificates, CSRs, and CRLs;
  * calculating message digests;
  * encrypting and decrypting with ciphers;
  * testing SSL/TLS clients and servers;
  * handling S/MIME signed or encrypted mail.

Package: libssl1.1
Section: libs
Priority: important
Architecture: any
Multi-Arch: same
Pre-Depends: ${misc:Pre-Depends}
Depends: ${shlibs:Depends}, ${misc:Depends}
Breaks: salt-common (<= 2016.3.3+ds-3)
Description: Secure Sockets Layer toolkit - shared libraries
 This package is part of the OpenSSL project's implementation of the SSL
 and TLS cryptographic protocols for secure communication over the
 Internet.
 .
 It provides the libssl and libcrypto shared libraries.

...
.....
  \end{verbatim}
  這裏面不一樣的是第一行要從 Source: 這個欄位開始,並且他這裏面定義了很多
  package 定義,相比之前練習的 DEBIAN/control, 是他會一次產生不同binary package,
  第1 package 叫 openssl 應該只放使用者工具,第2 package 叫 libssl1.1 裏面放
  程式庫,以此類推libssl-dev,是寫程式人才需要用到的,裏面有.h header檔等等。
  \\\\
  然後 Version 不能用了,必須改用 Standards-Version ,Source 跟 Package
  這一大段落必須用空白行隔開。
  \\\\
  Package 段落中的 Description
  \\\\
  其中 Section與Priority是比較不知道怎麼填的,他們是 debian 的分類定義,
  debian 有個debian-policy這個package, 裏面定義了debian套件的規範,
  但也不用真裝他, 網路文件 http://www.debian.org/doc/debian-policy/ ,可以
  看到定義。 但目前 Section 在debian的網路集散地中 
  http://packages.debian.org/stable/ 也可以看到,每次版本可能不同, 然後實
  際分類名字是那個目錄名。例如Web Software的分類,你去看html link的目錄是
  web而已,所以Section: 要填上 web,不是Web Software. 而除非你的package是一
  定在標準安裝中一定要裝的,不然 Priority 目前都是 optional,
  \begin{verbatim}
Section:  base
          admin
          doc
          libs
          editors
          devel
          utils
          x11
          ...
          web
Priority: required  表示沒這個package, 系統就起不來了。
          important 表示unix-like的系統上一定會有的。
          standard  表示什麼都不選的情況下,在文字模式下一定要用到的。
          optional  前三者以外的都用optional
          extra     不再用了,改用optional
  \end{verbatim}
  Depends 就是套件需要別的套件名稱與版本,將來有的安裝工具會自動把有相依的套件
  一起裝起來,所以這個Depends也要寫好,彼此用逗號分開。
  \begin{itemize}
    \item Recommands: 程式中用了比較好的套件,但aptitude, apt-get 會自動裝這個,
      dpkg不會自動裝。通常 front end 會問你要不要裝。
    \item Suggest: 不裝也行的套件,front end 不會問要不要裝。
    \item Conflicts: 有衝突的其他套件。
  \end{itemize}
  另外還有 Build-Depends, Pre-Depends 比較重要,
  \\\\
  不過 openssl 的欄位是用變數 \verb=${shlibs:Depends}= 這種型式,這應該是
  要有 autoconf 才行
  \begin{itemize}
    \item \verb=${shlibs:Depends}= 表示自動計算 share libiraries 的相依性。
    \item \verb=${perl:Depends}= 表示自動計算 perl libraries 的相依性。
    \item \verb=${misc:Depends}= 表示自動計算
  \end{itemize}
  一般很複雜的像 perl, python, LaTeX, Nodejs 等等工具的 library 相依也相當複雜
  ,所以都會有自動計算 script 幫忙。
  目前我們
  可以先用自己知道有哪些套件是相依的,把他寫死的方法,先寫在裏面。變數的方法是
  工具產生的,比較複雜,下面再討論。所有的欄位可以 man deb-src-control 文件了解,
  \\\\
  再來是changelog,最簡單的一個單位就是五行
  \begin{verbatim}
<pkgname> (<version>-<release>) unstable; urgency=medium

  * change description

 -- Alibuda Huang <alibuda@gyoza.net>  Sat, 8 Mar 2014 18:30:51 -0800
  \end{verbatim}
  那個unstable 是debian 版本code name的定義,先不管他用這個就好,也可以用
  別的,那個urgency 就是緊不緊急,也先用這個。有一件事就是這種source如果是最
  原始的source檔,那版本號碼只能是<version>不能有<release>版號。source code
  是沒有 release 的,只有 binary package, distribution 才有 release. 然後每行
  的前面空格要完全一樣,第一行沒空格,description 前要兩格,最後行空一格,
  email 日期中間要兩格,日期要這種格式才接受,要一模一樣格式,不然會 warning。
  \\\\ 
  而真正啟動的其實是靠 rules 這個檔,因為他就是 Makefile 而已,但他必須是個可
  執行檔chmod 755,第一行必須用 make -f 像script那樣,他會被 debuild 或
  dpkg-buildpackage呼叫使用
  \begin{verbatim}
#!/usr/bin/make -f
  \end{verbatim}
  他的 target 有些policy規定
  \begin{itemize}
    \item clean: 相當於我們熟悉的make clean。(必須要有)
    \item build: 相當於我們熟悉的make 或者make all. (必須要有)
    \item build-arch: 建造arch有關如amd64 的 targe。(必須要有)
    \item build-indep: 建造arch無關的如文件target. (必須要有)
    \item install: 以前熟悉的make install, 安裝想要檔案到 debian 目錄. (Optional)
    \item install-arch: 安裝arch有關檔案到 debian 目錄. (Optional)
    \item install-indep: 安裝arch無關檔案到 debian 目錄. (Optional)
    \item binary: 產生所有binary packages, 就是我們用DEBIAN做出的xxx.deb,
      是build-arch, build-indep的合體 (必須要有)
    \item binary-arch: 建立binary package xxx.deb在上層目錄. (必須要有)
    \item binary-indep: 建立arch無關的binary package xxx.deb在上層目錄. (必須要有)
    \item get-orig-source 如何拿到upstream 原始 source 檔 (Optional)
  \end{itemize}
  也就是rules檔必須有這些target, 只要用上 source 的 debain/control,
  這些 target 都是由 dpkg-buildpackage 驅動, build 跟 binary 又語焉不詳很像,
  因此要好好看 dpkg-buildpackage manual 步驟跟設定。
  所以這些 target 到底要產出什麼呢?
  \\\\
  在1.16.2版本前,是只有build, binary, install,所以dpkg-buildpackage
  命令是會試著 make 這3個target, 後來就把這3個又拆出 -arch 與-indep。
  當 debian 支援不同的 arch, 或者與 arch 無關的 package 時呼叫,
  其中我們以前熟悉的make , 變成make build, 然後make binary就是之前
  dpkg-deb \verb=--=build 所產生的 binary deb 檔。
  \\\\
  所以文件上雖這樣寫但我後來測試結果是只要有 build, binary 就可以了。
  而且其實 binary 會跑真的 make clean, ... make install 到產出 deb,
  所以真正也不需要 build 去做事,只需要 binary 就可以。
  \\\\
  可以把最上面 source tree 的 Makefile 先 copy 成 rules 進來成 debian/rules
  ,最上面加上
  \begin{verbatim}
#!/usr/bin/make -f
  \end{verbatim}
  然後假設我們原本 default: 這個 target 就是我們最原先上面的 default target,
  這樣 dpkg-buildpackage 就會來叫用了。然後binary 要產出 fakeroot 的
  DEBIAN/control 與xxxx.deb 檔
  \begin{verbatim}
binary: default
build:
  \end{verbatim}
  這樣就有了必須要有的 target。 另外他有兩個環境變數可被設定傳給 rules 與
  dpkg-buildpackage
  \begin{itemize}
    \item CURDIR /path/to/package-version 等於是 topdir of source
    \item DESTDIR 如果只有單一binary package 那是 debian/binarypackage/ 如果
      多個 binary package 那就是 debian/tmp/ ,所以最後假 root 在這變數內,
      dpkg-buildpackage 接下來會來這邊找有沒有 binary,這個我通常設成
      \$(CURDIR)/.DEBFAKEROOT,跟 rpm buildroot 一樣意思。
  \end{itemize}
  在debian目錄上一層就是source的最上層跑
  \begin{verbatim}
dpkg-buildpackage -rfakeroot
  \end{verbatim}
  就會在 debian 目錄上一層,也就是 source tree 最上面,建造出一個標準的 
  package 了。這 dpkg-buildpackage -rfakeroot, 其實是去叫其他程式一步步的
  執行fakeroot debian/rules clean, dpkg-source ..., 所以如果
  碰到哪步驟壞掉了,就重跑他,看怎麼回事。
  \\\\
  我以一個簡單的pam-gyoza做例子, source tree 的目錄名稱要是 
  <pkgname>-<version>, 也就是通常mypkg-1.0.0 這樣子的目錄。
  \begin{verbatim}
cyril@develop:~$ tar zcvf pam-gyoza-1.0.0.tar.gz pam-gyoza-1.0.0/
pam-gyoza-1.0.0/
pam-gyoza-1.0.0/Makefile
pam-gyoza-1.0.0/pam_gyoza.c
pam-gyoza-1.0.0/debian/control
pam-gyoza-1.0.0/debian/rules
pam-gyoza-1.0.0/debian/changelog
pam-gyoza-1.0.0/debian/source/format
  \end{verbatim}
  關鍵 rules 檔範例,等於最外面的 Makefile 加上 build, binary 兩個關鍵的新
  target,等於是為了建造 deb 所產生的新替代最上層 Makefile
  \begin{verbatim}
#!/usr/bin/make -f
SUBDIRS = src
DESTDIR = $(CURDIR)/.DEBFAKEROOT

binary: default
	$(MAKE) prefix=$(DESTDIR) install
        mkdir -p $(DESTDIR)/DEBIAN
        dpkg-gencontrol -P$(DESTDIR)
        dpkg-deb --build $(DESTDIR) ..
build:

default:
	for subdir in $(SUBDIRS); do
		$(MAKE) -C $$subdir
	done

install:
	install -m 755 -D src/alibuda $(prefix)/usr/bin/alibuda

uninstall:
	rm $(prefix)/usr/bin/alibuda

clean:
	for subdir in $(SUBDIRS); do
		$(MAKE) -C $$subdir $@
	done
  \end{verbatim}
  跳到 source 上,執行dpkg-buildpackage -rfakeroot,產出 deb 檔,其中
  dpkg-deb 的產出目錄要小心,dpkg-buildpackage 內定會來找這個 deb 檔,會
  要產生一個 buildinfo 檔,如果找不到 deb ,內定找的目錄就是上一層,
  那就會出現 error。
  \\\\
  有幾個比較重要
  \begin{itemize}
    \item --rules-file 找到 debian/rules
    \item --buildinfo-file=filename 指定 buildinfo 檔輸出在哪裡?
    \item -T 只跑想要的 target ,除錯時,例如跑 -T binary-arch, build-indep
    \item -nc 不要做 make clean,這樣下次 binary 跑時應該會略過 make install.
    \item --no-sign 不要 gpg sign 任何的 .dsc, .buildinfo, .changes 檔案。
  \end{itemize}

  \subsection{一次搞定的 debhelper 工具}
  在之前的debian/rules,感覺非常奇怪,他跟自己在source上的Makefile的關係實在
  好像重覆寫一遍一樣的東西,這不對,而且其實真正的opensource Makefile 是用
  autoconf 去產生的,所以這兩者間的關係與使用一定不是這樣。然後很笨的是要寫
  兩個幾乎一樣的binary control, debian/control 跟 DEBIAN/control。去看別人的
  debian/rules, 只需要簡單的 dh 命令就好,他會自動在source tree的 最上層做
  make 的動作,也沒有 debian DEBIAN 兩個目錄, 並且單一source多binary 
  package 的產生,也用副檔名的方式產生。所以應該我們弄錯了,不是這樣用的,
  \\\\
  而如果想要將套件提交debian community或者自己公司要造一個網路update的套件庫,
  最好是有用gpg key簽名保護的套件,而control與rules的建造也能用工具幫我們做好
  一個template,我們去改就好,不過這最好能寫autoconf產生configure的能力才是,
  基本上走到這一步,裡面的build, 要非常的opensource才行了。
  \\\\
  主要工具是\verb=dh_xxxxx=命令群,通稱 debhelper suite,用 aptitude 或
  apt-cache search 搜尋 dh-.* package, 會看到這些命令群套件,但只要裝
  debhelper 這個套件就可以了。
  \\\\
  dh 命令群有一個 dh-make 套件,他可以建造 debian template 檔案,不裝也可以
  , \verb=dh_make=與 dpkg-depcheck兩個是最初始使用,他會
  建造完整的template在debian 目錄下, 是非常嚴格的種種規定,玩起來有點痛苦,
  以剛剛那個 pam-gyoza做例子。
  \begin{verbatim}
cyril@develop:~$ tar zcvf pam-gyoza-1.0.0.tar.gz pam-gyoza-1.0.0/
pam-gyoza-1.0.0/
pam-gyoza-1.0.0/Makefile
pam-gyoza-1.0.0/pam_gyoza.c
cyril@develop:~$ cd pam-gyoza-1.0.0
cyril@develop:~/pam-gyoza-1.0.0$ dh_make -f ../pam-gyoza-1.0.0.tar.gz
Type of package: (single, indep, library, python)
[s/i/l/p]?
Email-Address       : cyril@develop.gyoza.net
License             : blank
Package Name        : pam-gyoza
Maintainer Name     : cyril
Version             : 1.0.0
Package Type        : single
Date                : Sat, 10 Mar 2018 18:14:35 -0800
Are the details correct? [Y/n/q]
Done. Please edit the files in the debian/ subdirectory now.
  \end{verbatim}
  上面選項先選s (single), 這樣就會幫我們創造出 debian 目錄,與他下面的資訊
  檔 template, 然後有以下步驟
  \begin{itemize}
    \item 先準備gpg 非對稱的鑰匙,跟 ssh-keygen很像,不過這好像也非必要。
  \begin{verbatim}
gpg --gen-key
gpg -a --output ~/.gnupg/YOUR_NAME.gpg --export 'YOUR NAME'
gpg --import ~/.gnupg/YOUR_NAME.gpg
  \end{verbatim}
    \item 準備好source, 用<pkgname>-<version>.tar.gz 把你的source做成tarball
    \item 這其實主要是 source 目錄一定要用<pkgname>-<version>這樣格式,而且
      pkgname 不能有底線 \verb=_=。
    \item 跳到source 的最上層, 用
  \begin{verbatim}
dh_make -e alibudal@gyoza.net -f ../<pkgname>-<version>.tar.gz
  \end{verbatim}
      這叫 debianized
    \item 用 dpkg-depcheck -d ./configure 產生好的control檔,這就是會根據
      autoconf 找出Depends 相依package,填上變數的方法。所以要用這,必須你的
      source 是有autoconf的,所以也不是每個人都適用。
    \item 改control, rules, copyright ....
    \item 因為我們是tarball source ,所以改source/format 為 3.0 (native) .
    \item 改 changelog 裡面的版號, 如果是 3.0 (native) 只允許 x.y.z 這樣
      version 號碼。
    \item 最後用 dpkg-buildpackage -rfakeroot 產生package
  \end{itemize}
  其實這是大同小異,差別就是我想是rules這個檔, 這個檔變得我們看不懂的東西。他
  內定就只是去叫dh \$@ 這個東西,所有的建造都由dh這個命令完成,最簡單的 rules
  \begin{verbatim}
#!/usr/bin/make -f
%:
	dh $@
  \end{verbatim}
  這裏面會跑很多
  debhelper 的內定script, 例如 debian/rules clean 是去跑3個script
  \begin{verbatim}
dh_testdir
dh_auto_clean
dh_clean
  \end{verbatim}
  debain/rules build
  \begin{verbatim}
dh_testdir
dh_auto_configure
dh_auto_build
dh_auto_test
  \end{verbatim}
  完整的步驟相當多script, 需要去看文件,但所以我們可以改變他,需要建造 
  類似\verb=override_dh_auto_xxxx= 這樣的target ,override 表示用自己的
  ,不要用 dh 的target ,例如
  \begin{verbatim}
override_dh_auto_configure:
        ./configure --prefix=mybuildroot
override_dh_auto_build:
        $(MAKE) prefix=mybuildroot install
  \end{verbatim}
  這樣dh命令就不會跑原本Makefile的build, install。而會跑我們的
  override target。這時再去
  看openssl下的debian/rules 就大概看得懂他在幹嘛了。
  \\\\
  這邊要注意的是make install 時,DESTDIR與prefix這兩個我們傳進去的變數,
  這兩個就是假的根目錄, 我們在make install時,如果不是login as root,
  一般user就要設定這變數。 這在redhat rpm 裏面也是一樣的。
  \\\\
  要全部看懂還是要了解所有
  \verb=dh_=xxx scripts在幹嘛,然後把他custom 自己要的。
  \\\\
  最後 .install 檔有如下的格式,可以把要的檔案裝到將來系統上
  \begin{verbatim}
path/to/file/relative/to/source/root path/to/install/relative/to/system/root
例如
*.so  lib/x86_64-linux-gnu/security/
  \end{verbatim}
  所以debian的package相當複雜,從最簡單的自己的 binary 到最後要 submit 給
  debian community 有這三種做法,
  \begin{itemize}
    \item 第1種最簡單就自己放binary檔案到假的 / , 然後造DEBIAN/control,用命
      令是\verb=dpkg-deb --build=,只需要最基本dpkg套件。
    \item 第2種就是自己建造debian/control debian/rules ....,然後用命令
      debuild 或者 dpkg-buildpackage -rfakeroot,需要dpkg-dev套件,或者
      基於dpkg-dev之上的wrapper命令群,devscripts套件。
    \item 第3種用 debhelper 套件,用\verb=dh_make= 命令建造 debian 目錄,
      與下面的 template 檔案群, 然後自己改下面檔案, 最後也是用
      dpkg-buildpackage -rfakeroot 建造。
      可以 man debhelper 得到更多資訊。需要debhelper套件。
  \end{itemize}
  最後一種方法還有兩個檔很重要
  \begin{itemize}
    \item debian/source/format 這有"1.0" "2.0" "3.0 (native)" "3.0 (quilt)"
      表示source的格式為何?1.0表示source tarball 是orig.tar.gz + diff.gz 組成
      或者單一tar.gz,也叫native format 也就是tarball 是最原始從人家那拿回來
      的。 2.0表示會有一個debian/patches的目錄藏著patches,不過這目前已推薦改
      用3.0 (quilt),3.0 (native) 是1.0的延伸,但1.0允許x.y.z-r 這樣版本號,
      但3.0 (native) 不允許這樣,只允許 x.y.z ,所以用 3.0 (native) ,
      changelog 要小心裡面的版號。另外不只是.gz而是bz2, xz ...都可以,
      3.0 (quilt) 表示有一定要有.orig.tar.xx
      .debian.tar.xx,還有其他patches等檔案,3.0 (git) 表示是git source tree
      要定義git remote, 通常用1.0, 3.0 (git) 處理我們一般公
      司用途,3.0 (quilt) 就是非常嚴格的source處理方式了。man dpkg-source 有
      完整說明怎麼抓 source 與處理。
    \item debian/compat 這是debhelper的版本 compatibility, 可以man debhelper
      。目前有 10, 9, 7 等等,不同版本,有的script動作會不一樣,例如版本7 的
      \verb=dh_clean=會讀 debian/clean 所列的檔案然後砍掉他們,但以前的不會。
  \end{itemize}
  最基本的觀念跟說明
  https://www.debian.org/doc/manuals/debmake-doc/ch05.en.html

  \subsection{debconf}
  以上是基本的 debian package 建造方法,而 debconf是一個用來處理deb package
  的configuration database與UI系統的關聯工具,原本
  deb套件本身已經有preinst, postinst, prerm, postrm這樣的script來供套件包裝者
  使用,但這樣的套件script對於user interface是沒有統一的(通常只是簡單的 shell 
  script read 或者乾脆只有安裝 /etc/ 下的 configuration sample 檔,然後自己要在
  安裝完後去改),debconf 對外提供一致的 front-end,來得到 user 的 input,以得到
  user 想要的設定,也就是安裝過程中,configuration 檔也完成了,所以不見得
  package 要支援 debconf,只有支援 debconf 的 package 安裝時才會彈出問答式 UI。
  front-end binding有dialog, readline, gnome, python 等等。其中 dialog 就是我們
  使用text安裝時,看到的那個安裝畫面 widget API。gnome就是有 GUI 的。不像
  anaconda 是集中只有他們一家處理,這也是為什麼 debian 系列的 package 安裝軟體
  這麼多樣,很容易發展新的。
  \\\\
  可以用
  \begin{verbatim}
$ debconf-get-selections
  \end{verbatim}
  來得到目前系統上有debconf的package當初安裝時的選擇。這得出來的格式也是無人
  職守的自動安裝時,餵給apt系統的選擇格式,也就是debian/ubuntu的preseed.conf
  相當於redhat/centos的ks.cfg。
  \begin{verbatim}
<owner> <question name> <question type> <value>
  \end{verbatim}
  他的
  \begin{itemize}
    \item owner, 一般就是package名字,一個代表id,debconf-show --listowners 
      可以列出目前系統所有owner,d-i 表示debian-installer。自動安裝的
      owner是d-i。這個owner表示是這個id用debconf的命令來跟debconf溝通,
      debconf對內控制一個相對於owner的程式,對外使用某個一致的front-end
      UI,使用某個question name來得到user input的value,然後這個owner 從
      debconf拿回這個值後,由package的script來做configuration檔。一般
      package不見得要有debconf的支援,除非他想要有一個漂亮的UI來得到user input。
    \item question name, 一個 question 的代表 id, 這個是 owner 自己定義的,
      在d-i下沒辦法一下知道全部,只能慢慢來。下面 example 是debain網站上所給
      的。可以用 debconf-show d-i看目前 d-i 的 question name跟value。
    \item question type, 這個question 的data type,string的話那value就是字串。
    \item value, 這個question的答案。也是我們要填的值。例如 netcfg/hostname
      要填入這台機器名字。所以程式會來操作 question name 跟 value。
  \end{itemize}
  由於debconf 不是每個 package 都支援的,debconf 的資料庫也獨立於 dpkg 的,所
  以當每次要使用 debconf 的功能時,要先 preconfigure 這個 deb 檔
  \begin{verbatim}
# dpkg-preconfigure nis_3.17-32_amd64.deb
  \end{verbatim}
  平常 apt 安裝工具會自動執行這個 dpkg-preconfigure,所以一般人沒有感覺。如果之
  前的question的value想要重新改變,可以重新執行
  \begin{verbatim}
  # dpkg-reconfigure nis
  或
  # dpkg-reconfigure debconf => 這會重新問所有支援debconf的套件問題
  \end{verbatim}
  也就是當你不想直接去改/etc/xxxx.conf時,可以叫dpkg-reconfigure 有個漂亮的
  UI來幫你改。
  \\\\
  debconf會控制一個程式,這個程式的stdout會是debconf的命令(protocol),並且
  stdin會是front-end的result code。我們常見的debian/ubuntu安裝程式
  debian-installer 最後就是去跑
  \begin{verbatim}
debconf -o d-i /usr/bin/main-menu
  \end{verbatim}
  main-menu也是C程式。他裏面定義了跟debconf互動的命令與steps。主要是printf時
  要跑出需要的code。
  支援debconf的package(owner)必須提供
  \begin{itemize}
    \item templates : 上面的question name只是id, 這才是真正user看到的文字定義檔。
      使用templates可以用多國語言,debconf會自動去找到需要的文字。
    \item config    : 這是跟debconf溝通的程式。可以是shell, perl, c .... 只要遵守
      stdout跟stdin的規則,shell script有helper API,c就要自己照規則寫了,算是跟
      CGI很像的東西。這也是debconf會去控制的程式。以debian安裝程式來說來說就是
      /usr/bin/main-menu,apt工具的話,會解開deb檔,然後debconf -o mypkg config。
  \end{itemize}
  在debian的一般包裹裏面,解開control檔案
  \begin{verbatim}
dpkg --control xxx.deb
  \end{verbatim}
  如果在DEBIAN目錄下有templates與config兩個檔案,那這個package就是有debconf的支援的。
  如果沒有就是沒有debconf的支援。可以下載tzdata的deb檔,並且看他怎麼定義多個
  question name與data type是select的使用。
  \\\\
  templates檔例子
  \begin{verbatim}
Template: hostname
Type: string
Default: debian
Description: unqualified hostname for this computer
 This is the name by which this computer will be known on the network. It
 has to be a unique name in your domain.
  \end{verbatim}
  hostname表示要設定的key, string表示資料格式,debian是內定值。如果是select type
  我們就有選項可以選,例如以下的多國語言例子
  \begin{verbatim}
Template: tzdata/Areas
Type: select
Choices: Africa, America, Antarctica, Australia, Arctic, Asia, Atlantic, ...
Choices-eu.UTF-8: Afrika, Amerika, Antartikoa, Australia, Artikoa, Asia, ...
Choices-fr.UTF-8: Afrique, Amérique, Antarctique, Australie, Arctique, ...
Description-eu.UTF-8: Eremu geografikoa
 Hautatu bizi zaren area geografikoa. Konfigurazioko hurrengo galderek ...
Description-fr.UTF-8: Lieu géographique :
 Veuillez choisir le lieu géographique où vous êtes situé(e). Les questions ...
  \end{verbatim}
  config通常還是寫 shell script居多,debconf有個helper, 
  /usr/share/debconf/confmodule ,所以一上來就先include 然後使用他裏面的api
  \begin{verbatim}
  #! /bin/sh
  set -e

  . /usr/share/debconf/confmodule
  db_version 2.0
  db_capb backup
  \end{verbatim}
  他的api說穿了就只是protocol命令前面加上個db\_而已。與UI 溝通的protocol在
  \href{https://www.debian.org/doc/packaging-manuals/debconf\_specification.html#AEN106}{debconf的protocol}
  可以看到。可以用
  \begin{verbatim}
$ debconf-communicate
  \end{verbatim}
  來玩,跟系統上的資訊來溝通看看。例如我先用debconf-get-selections得到
  \begin{verbatim}
ntfs-3g ntfs-3g/initramfs       boolean true
# Boot loader configuration check needed
linux-base      linux-base/disk-id-manual-boot-loader   error   
# Use Control+Alt+Backspace to terminate the X server?
d-i     keyboard-configuration/ctrl_alt_bksp    boolean false
  \end{verbatim}
  這時表示ntfs-3g這個套件裏面有設定一個 question 叫ntfs-3g/initramfs 他的值是true,
  d-i表示安裝系統時,有個question是keyboard-configuration/ctrl\_alt\_bksp。
  然後我們用
  \begin{verbatim}
chuang@cyril-lnx:~$ debconf-communicate 
  \end{verbatim}
  \parbox{\textwidth}{debconf: DbDriver "passwords" warning: could not open /var/cache/debconf/passwords.dat: Permission denied}
  \begin{verbatim}
get ntfs-3g
10 ntfs-3g doesn't exist
get ntfs-3g/initramfs
0 true
  \end{verbatim}
  這個get就是跟UI的protocol命令,10或者0就是UI回給config時的error code, 在
  confmodule裏面的db\_get就是相當於get命令。所以很簡單的不需要操心UI是什麼,
  只是單純的定義 question, 處理user回給的值,在config中就可以把這些值寫入
  /etc/xxx.conf了。除了get命令外當然有set等多種命令,可以用在config裏面,
  所以如果不寫shell script,就要自己在stdin, stdout上實作這些命令與回傳值而已。
  當把templates, config放到DEBIN目錄下,並且用dpkg-create後產生的套件就可以有
  漂亮的UI來對話了。
  \\\\
  最後是一般以前Solaris的習慣會把公司檔案放在/opt下,但debian不建議這樣而是
  在/usr/share/company下,這是不一樣的地方。

\section{產生rpm}
製作rpm檔, 準備工具需要安裝rpm-build這個工具,他的package資訊install/remove 
scripts 不像debian的是額外檔案,而是通通要寫在一個spec檔案裏面。spec檔有其格式
,最後用rpmbuild命令根據spec檔的資料跑一遍build, 做出binary檔捆成一個rpm檔。
\href{http://www.rpm.org/max-rpm/index.html}{rpm檔格式的完整說明}可以在此
找到。
\\\\
一個 rpm 檔可以用rpm2cpio把他內部檔案轉出來。
\begin{verbatim}
rpm2cpio mypackage.rpm | cpio -vid
rpm2cpio mypackage.rpm | xz -d | cpio -vid (有的現在是用xz壓縮的)
rpm -pq --scripts mypkg.rpm
rpm -pqf --queryformat '%{name}\n%{summary}\n%{version}\n%{release}' mypkg.rpm
\end{verbatim}
建造rpm有很多的參數與內定變出,可以用
\begin{verbatim}
rpmbuild --showrc
\end{verbatim}
找出來,rpm內部的變數用 \verb=%{var}= 來存取,定義都在/etc/rpm, /usr/lib/rpm,
兩個重要的目錄\verb=%{_topdir}= 與\verb=%{buildroot}=,由於以前內定值都是在
/usr/src/redhat 要在 root 權限下,所以都要自己再設定才能跑,或者把他們設到
ramdisk上會跑快一點,現在新的內定值設在\$HOME/rpmbuild
\begin{verbatim}
rpmbuild -E %{_topdir}
rpmbuild -E %{buildroot} (又叫RPM_BUILD_ROOT)
\end{verbatim}
目前只要設\verb=%{_topdir}= 就好,\verb=%{_topdir}=是一個目錄,所有
建造rpm的暫時目錄檔案都會在這裏面,而\verb=%{buildroot}=則是安裝時會呼叫make
install 他會裝東西到根目錄去,buildroot就是假的root,相當於deb的fakeroot,如
果沒有設定的話,目前 centos 7 buildroot自動是
\begin{verbatim}
%{_topdir}/BUILDROOT/%{name}-%{version}-%{release}.%{_arch}
\end{verbatim}
當使用rpmbuild的時候,要先指定一個spec檔名,內定會去\verb=%{_topdir}/SPECS=
下面找,找不到再到目前目錄下面找,然後用spec檔裡面的source指定的檔名去
\verb=%{_topdir}/SOURCES=下找,最後make, install, rpm, srpm 檔放到下面目錄
\begin{verbatim}
%{_topdir}/BUILD
%{_topdir}/RPMS
%{_topdir}/SRPMS
\end{verbatim}
當然直接跑rpmbuild的時候也可以直接指定
\_topdir, \_sourcedir, \_specdir, 如果自己指定,那麼
\begin{verbatim}
rpmbuild -ba --define='_sourcedir /var/mysrc' --define='_specdir /var/myspec' myrpm.spec
\end{verbatim}
通常\verb=%{_topdir}=會讓他等於環境變數RPMBUILD,然後我們也可以設定自己常用的
在~/.rpmmacro, 老一點的rpmbuild不會幫你建造那些有的沒的需要自己先建好。新的
會自動創造RPMS, SRPMS, BUILD ...這些目錄。
\begin{verbatim}
mkdir -p ~/rpmbuild/{RPMS,SRPMS,BUILD,SOURCES,SPECS,tmp}

cat <<EOF >~/.rpmmacros
%_topdir   %(echo $HOME)/rpmbuild
%_tmppath  %{_topdir}/tmp
EOF
\end{verbatim}

  \subsection{由distro提供檔案建造}
  如果已經有人提供src.rpm了,我們想重新在我們系統上build則用
  \begin{verbatim}
$ rpmbuild --rebuild mypkg.src.rpm
或
$ rpm -i mypkg.src.rpm
$ cd $RPMBUILD/SPECS
$ rpmbuild -ba mypke.spec
會在
$RPMBUILD/RPMS/x86_64
得到一個binary rpm
  \end{verbatim}

  \subsection{自己建造rpm套件 }
  最簡單主要必須提供一個xxx.spec檔,跟一個 source 檔,這可以只是假 root 所捆起
  來的 tarball ,source 目錄用<pkgname>-<version> 這樣,這跟 spec 檔裡的Source
  欄位要一樣。 在CentOS 7 新的 \_topdir 內定是設在 ~/rpmbuild
\begin{verbatim}
cd ~/rpmbuild
mkdir -p helloworld-0.9.9/usr/bin/
mkdir -p helloworld-0.9.9/etc
touch helloworld-0.9.9/usr/bin/helloworld
touch helloworld-0.9.9/etc/helloworld.conf
tar zcvf helloworld-0.9.9.tar.gz helloworld-0.9.9
mv helloworld-0.9.9.tar.gz SOURCES 
\end{verbatim}
  去~/rpmbuild/SPECS下面編個 hello.spec
\begin{verbatim}
Summary: mini rpm
Name: minirpm
Version: 1.0.0
Release: 1
License: GPL+
SOURCE : %{name}-%{version}.tar.gz

%description
%{summary}

%prep
%setup -q

%install
rm -rf %{buildroot}
mkdir -p %{buildroot}
cp -a * %{buildroot}

%clean
rm -rf %{buildroot}

%files
%defattr(-,root,root,-)
%{_bindir}/*
%{_sysconfdir}/*
\end{verbatim}
  最基本的6個欄位與一些macro如\%description 一定要有,然後Name, Version的值可以
  用\%{name} \%{version}取回。用
\begin{verbatim}
$ rpmbuild -ba hello.spec
\end{verbatim}
  以前的rpm可能default \_topdir 在/usr/local/redhat, 或者xxx.spec一定要放在
  \_topdir/SPECS下面,這時可以儘量滿足這些條件再下命令。或者不用copy tar.gz
  到SOURCES下,直接在目前目錄有tar.gz 與 spec檔
\begin{verbatim}
$ rpmbuild -ba --define "_sourcedir `pwd`" hello.spec
$ rpm -pql ~/rpmbuild/RPMS/x86_64/helloworld-0.9.9-1.el7.centos.x86_64.rpm
\end{verbatim}
  這只是最簡單的假root 加上想放進的binaries所編出的binary rpm檔,同樣正統作法
  要從 source 開始編譯到產生binary source兩種package檔,他跟deb不一樣的是不分
  大小寫的debian目錄,而是完全由一個spec檔做完。
  \\\\
  我們一樣拿人家複雜的來看, 以openssl為例
\begin{verbatim}
%define _unpackaged_files_terminate_build 0

Release: 1

%define openssldir /var/ssl

Summary: Secure Sockets Layer and cryptography libraries and tools
Name: openssl
Version: 1.0.1o
Source0: ftp://ftp.openssl.org/source/%{name}-%{version}.tar.gz
License: OpenSSL
Group: System Environment/Libraries
Provides: SSL
URL: http://www.openssl.org/
Packager: Damien Miller <djm@mindrot.org>
BuildRoot:   /var/tmp/%{name}-%{version}-root

%description
The OpenSSL Project is a collaborative effort to develop a robust,
commercial-grade, fully featured, and Open Source toolkit implementing the
Secure Sockets Layer (SSL v2/v3) and Transport Layer Security (TLS v1)
protocols as well as a full-strength general purpose cryptography library.
The project is managed by a worldwide community of volunteers that use the
Internet to communicate, plan, and develop the OpenSSL tookit and its related
documentation.

OpenSSL is based on the excellent SSLeay library developed from Eric A.
Young and Tim J. Hudson.  The OpenSSL toolkit is licensed under an
Apache-style licence, which basically means that you are free to get and
use it for commercial and non-commercial purposes.

This package contains the base OpenSSL cryptography and SSL/TLS
libraries and tools.
%prep

%setup -q

%build
%install
rm -rf $RPM_BUILD_ROOT
make MANDIR=/usr/man MANSUFFIX=ssl INSTALL_PREFIX="$RPM_BUILD_ROOT" install

# Make backwards-compatibility symlink to ssleay
ln -sf /usr/bin/openssl $RPM_BUILD_ROOT/usr/bin/ssleay

%clean
rm -rf $RPM_BUILD_ROOT

%files
%defattr(0644,root,root,0755)
%doc CHANGES CHANGES.SSLeay LICENSE NEWS README

%attr(0755,root,root) /usr/bin/*
%attr(0755,root,root) /usr/lib/*.so*
%attr(0755,root,root) %{openssldir}/misc/*
%attr(0644,root,root) /usr/man/man[157]/*

%config %attr(0644,root,root) %{openssldir}/openssl.cnf
%dir %attr(0755,root,root) %{openssldir}/certs
%dir %attr(0755,root,root) %{openssldir}/misc
%dir %attr(0750,root,root) %{openssldir}/private

%post
ldconfig

%postun
ldconfig
\end{verbatim}
先注意的是 Summary到Packager這一段的東西,就望文生義,把該填的資料填上。
GROUP很像debain的Section分類用的,在/usr/doc/rpm/GROUP找的到。
BuildRoot:這個就是假的root,等下在\%build與\%install時,有個
\$RPM\_BUILD\_ROOT,這個內定環境變數就是這個設定值。 Requires:是手動設定
dependency。rpm可以自動幫你找 dependency,不過也可以用Requires強制手動指定。
BuildRequires是要build這個package的dependancy.
\\\\
另外常用到的sections有
\begin{itemize}
  \item \%description
  \item \%prep
  \item \%build
  \item \%install
  \item \%clean
  \item \%files
  \item \%pre
  \item \%post
  \item \%preun
  \item \%postun
\end{itemize}
    
\%prep這邊是prepare準備build的動作,通常是你download的source code需要做一
些你自己做的patch, 要作untar等動作。例如
\begin{verbatim}
%prep
rm -rf $RPM_BUILD_DIR/cdplayer-1.0
zcat $RPM_SOURCE_DIR/cdplayer-1.0.tgz | tar -xvf -
\end{verbatim}
其中\$RPM\_SOURCE\_DIR \$RPM\_BUILD\_DIR是rpm跑起來時設定的環境變數。
不過由於這樣的動作幾乎已經是公式化的動作,所以有人把他寫好成一個
macro \%setup, prep下有兩個常用的macro,\%setup與\%patch,可以只給下面的
\%prep就好。
\begin{verbatim}
%prep
%setup -q
\end{verbatim}
\%setup -q會根據Source:的位置來得到source與解開,patch是用來做source patch
用的。 \%build就是當你要從source來build binary時要打的命令。
通常就是
\begin{verbatim}
configure --prefix=$RPM_BUILD_ROOT
make
\end{verbatim}
不過由於rpm是跨平台的,在/usr/lib/rpm/rpmrc有一些定義了的變數,
make可以這樣比較有彈性,
\begin{verbatim}
make CFLAGS="$RPM_OPT_FLAGS" LDFLAGS=-s。
\end{verbatim}
通常RPM\_OPT\_FLAGS是目前distribution系統上認為怎樣的gcc flags設定比較
符合他的distribution的設定。
\\\\
\%install就是我們常在source上用的
\begin{verbatim}	
make install
\end{verbatim}	
因為之前已經./configure \verb=--prefix=,\%clean通常是把我們剛剛創造的假的root,
\$RPM\_BOOR\_ROOT給清掉。其實在\%build與\%install可以光打我們熟悉的
configure; make; make install這樣就好,
\%prep \%build \%install \%clean會真的run一次我們熟悉的從source code編譯成
binary的步驟,而且如果我們沒有用個假的root的話,
會真的裝東西到目前系統上,通常這是不好的作法,最好還是設定假root,這通常是
用\verb=--prefix=。
\\\\
\%file這個設定就是將來檔案要放到哪去的,
也就是Buildroot內的相對應將來真正系統上的檔案。也就是說前面的
./configure --prefix=xxx ; make; make install;就是從
\verb=%{_topdir}=/SOURCES/pkgname-ver.tar.gz解開後,跑進去做事,然後因為
prefix設到\verb=%{_tmppath}=/xxx那邊去,所以就裝到那個假的root,也就是
\verb=%{_tmppath}=/xxx下,其中\verb=%{_tmppath}=通常是/var/tmp,
有些內定的變數可以使用,跟autoconf 的 Makefile 其實是很像或一樣的,例如
\begin{itemize}
  \item \verb=%{_prefix}= 表示將來的/usr
  \item \verb=%{_exec_prefix}= 表示將來的/usr
  \item \verb=%{_bindir}= 表示將來的/usr/bin
  \item \verb=%{_libdir}= 表示將來的/usr/lib
  \item \verb=%{_includedir}= 表示將來的/usr/include
  \item \verb=%{_sbindir}= 表示將來的/usr/sbin
  \item \verb=%{_sysconfdir}= 表示將來的/etc
  \item \verb=%{_datadir}= 表示將來的/usr/share
  \item \verb=%{_mandir}= 表示將來的/usr/share/man
  \item \verb=%{_sharedstatedir}= 表示將來的/var/lib
\end{itemize}
這些是rpm package裝時,就設定在裏面的,可以用rpmbuild -E \%{macro} 例如
rpmbuild -E \%{dist} 可以看出dist這 macro 的值, 可能是.el6, .el7等等或者
rpmbuild \verb=--showrc=來看出來目前系統
內定變數有什麼。內定變數有的跟外面環境變數是一樣的,例如\verb=%{buildroot}=與
RPM\_BUILD\_ROOT,另外也能自己定義\verb=%define pkg_name %(echo $PKGNAME)=
,這樣就能把環境變數變成rpm.spec下的變數macro。而原本的Name,也有相對應的
\verb=%{name}=,也就是Summary到Packager這段的變數設定後,有相對應的小寫變數名
可在後來使用。
\\\\
最後把假的root當成將來要裝到系統上真的root的檔案在\%files下告訴rpm系統。
則他會幫你把這樣的資訊寫到binary package.
\%doc裡的檔將來會被裝到/usr/share/doc/\$NAME-\$VERSION-\$RELEASE,
\%attr是設定安裝時檔案的權限,
\begin{verbatim}
%attr(mode, user, group)
\end{verbatim}
mode如果不指定0755而只指定-的話表示, 不寫\%attr話用\%defattr,或者用root的權限。
如果有用BuildRoot的話,rpm自動知道要到相對應的假 root,下去把~fakeroot/usr/bin/
根據\%files指定的檔放到package的正確/usr/bin/。 
\\\\
rpm的pre- post- install與remove 不像solaris或debian的另寫真正的額外shell script檔,
而是把shell script含進spec檔內的\verb=%pre %post %preun %postun=下。
\\\\
寫完後到存到\verb=%{_topdir}=/SPECS下並且給命令
\begin{verbatim}
# rpm -ba libgyoza-2.0.2-1.spec
或
# rpmbuild -ba --rmsource $RPM_SPEC_DIR/libgyoza-2.0.2-1.spec
\end{verbatim}
就會把該做的事與檔案放進去package,-ba是binary跟source都會build.
rmsource rmspec是會清掉 SPECS 跟 SOURCES 下的相關檔案。

  \subsection{Upgrade需注意的情況}
  在rpm中有一個rpm -U的命令是用來upgrade新版本的,可是在rpm的
  pre/post/preun/postun的執行順序跟rpm -i的不一樣,所以要非常小心
  rpm -U其實是執行rpm -i 跟 rpm -e,但有如下規則
  \begin{enumerate}
    \item 先執行新版的pre
    \item 裝檔,就是\%files裡面的檔案
    \item 執行新版的post
    \item 執行舊版的preun
    \item 砍掉那些沒有被新版蓋掉的檔案
    \item 執行舊版的postun
  \end{enumerate}
  而這邊卻發現一個問題,如果我pre/post作的一些事情,例如symbolic link,
  本來在-e時,應該要刪掉的,但upgrade時卻執行-i -e就把原本-i作的事情
  給毀了,所以rpm又提供判斷的方法,傳給pre/post/preun/postun的第一個
  參數分別有0,1,2的值,以shell來看就是
  \begin{itemize}
    \item pre/post     \$1 = 1 表示-i \$1 = 2表示-U
    \item preun/postun \$1 = 0 表示-e \$1 = 1表示-U
  \end{itemize}
  但其實不光只是這樣,rpm -U 她可以用--force來安裝版本號碼比她小的rpm,
  結果又有如下的反應,如果新版的號碼
  \begin{itemize}
    \item 大於舊版 照原本的六規則走
    \item 等於舊版 只安裝新的pre/post,不執行舊preun/postun與砍檔案
    \item 小於舊版 照原本的六規則走
  \end{itemize}
  很神奇,但真實世界應該是不會這樣,這是因為我們在develop/test cycle
  時才會碰到這種莫名其妙的版本號碼關係。

\section{產生pkg}
如果你的系統是opensolaris的話,可以產生pkg檔。Solaris的scripts
\begin{itemize}
  \item preinstall   主要是做軟體安裝前環境的檢查例如需要什麼OS幾版以上等等
  \item postinstall  主要是做軟體安裝後的設定例如sendmail DNS的設定
  \item preremove    主要是移除軟體前的環境例如有別的套件需要這個軟體(dependency)
  \item postremove   主要是清除一些這個套件產生的東西例如log檔等等
\end{itemize}
不過這我很久沒做了,也不知道是否有更新的方法。
Solaris的package因為不是open source世界的,比較簡單,不用去注意一堆規則。
他主要的工具是
\begin{itemize}    
  \item \verb=#= pkgmk
  \item \verb=#= pkgtrans
\end{itemize}
要寫的資訊檔為
\begin{itemize}
  \item prototype  最重要的資訊檔,包括檔案的對應關係像\%file在rpm中的關係都在
    這裡。
  \item copyright  版權聲明
  \item pkginfo    這是很像你用rpm -qi得到的package資訊。例如pkg名,製造商等等。
\end{itemize}

\section{結語}
基本上不同 distribution 的 package 看來有共同的資料結構
\begin{itemize}
  \item name
  \item major
  \item minor
  \item patch
  \item release
  \item summary
  \item description
\end{itemize}
而且要在 <name>-<major>.<minor>.<patch> 目錄裡面放這次要 release 的 source
code,要產生一個<name>-<major>.<minor>.<patch>.tar.gz 的 tarball ,與最後
<name>-<major>.<minor>.<patch>-<release>.xxx 的 package 名,只是有的名字
有些限制,可能底線跟 hypen 的差別,當然可能還有更多,但我想這些是最基本的。
\\\\
而產生 package 有幾個層次
\begin{itemize}
  \item 下載 ditribution 提供的 upstream 與他的 diff 檔案等利用工具產生。
  \item 由已知的 binary 等檔案,包到一個假 root,由基本工具產生。
  \item 由 source 真的 make, make install 等步驟,由工具產生
  \item 需要真的 submit 到 distribution 去維護,這還要了解 distribution 的需求
        例如像 license, 文件,簽章等等龜毛的規定。
\end{itemize}
除了系統package外,很多library的安裝也非常複雜,因此perl CPAN Makefile.PL 方法
, python egg蛋蛋pip, nodejs npm 這些語言也另外提供他們的library套件管理,也
有他們自己的方法或格式。但這裡不處理這些格式, 只處理系統package。另外像LaTeX
也是一樣,也因為太複雜,有自己包裝的方法格式,任何有 library 形式的軟體都需
要有套件管理。
\\\\
而我們以上所介紹的也只是基本的而已,因為一個 source 可能根據不同使用情況,而
編成很多package, 例如有的library 的header檔只有開發軟體才需要,run time是
不需要的,所以常看到libxxx libxxx-dev 這兩種套件但都是從同一source編出來的,
所以單一source多重packages的做法,還有多國語言的package做法等等都有規則要遵守。
\\\\
新型態的套件管理還包含了收費,評價,社群,像Apple, Android 的 store,現在
ubuntu 也做這種事,Linux 新的套件管理 store 像 snap store, flatpak 兩者競爭
很厲害,因為是很大商機,主要是現在 Linux 成主流並且 library 的標準化也很完備
,都能盡量做到二進位檔能隨裝隨跑。
\\\\
在unix世界中其實原始碼的變化是非常的快的,也就是Windows的系統其實是個黑箱子,
windows的程式為什麼漏洞那麼多就是他很多程式庫不敢隨便upgrade,很多程式一直
編譯的是舊API的library,所以好像很多程式一直在更新版本,但都是based在舊api
上。 很多windows的管理者來到unix中的壞習慣就是什麼都要最新版的二元碼,
這是根本錯誤的,unix的版本號碼是有意義的,不是隨便就upgrade上去,有漏洞一定
要跟著 distro 出的改正版本,不是自己亂編譯後放上去,尤其是ssl或很多相對應的
library。
