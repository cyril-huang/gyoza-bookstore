\chapter{除錯工具}
寫完程式編譯完成了,跑出來的結果卻不是我們要的,所以一定又有bug了,
基本上最簡單的除錯或者有效的還是用印出來的logger,一般logger是起始project第一
個要做的API,通常就是用以下工具
\begin{itemize}
  \item fprintf : 不要忘了因為printf到 standard output是buffer I/O,所以有時程
    式先死掉了裡面的buffer沒有印出來,因此記得印到standard error。
  \item logger shell 標準logging命令
  \item syslog 能夠遠端logging並且使用標準logging priority與logging facility
  \item printk Linux kernel的fprintf
  \item 各語言的logger library,例如寫 java的人喜歡用的log4j等等。
\end{itemize}
運用以上工具在包裝自己想要的wrapper是大部份project的logger API。
這些API,在編譯,測試,除錯時是要用到的,但有些priority 像 debug message在發行
的code是沒必要出現的,所以編譯時也有分別。
\begin{verbatim}
內部測試時
gcc -DDEBUG myfile.c

真正發行時
gcc myfile.c
\end{verbatim}

\section{gdb基本}
如果要用debug工具,gcc編譯時不要忘了加-g這個參數,但是會讓你的執行檔
肥一點。一般用除錯工具要做
\begin{itemize}
  \item 載入程式(load)
  \item 設中斷點(break point)程式跑到這裡會停下來
  \item 開始跑程式(run)讓程式停在中斷點上
  \item 慢慢追程式(step, next ...)
  \item 檢查一些變數值(examine)
\end{itemize}
*nix 上的debug工具是gdb, gdb不僅能debug C/C++/Objective-C
還包括了Fortran, Java, Module 2 還有Ada。gdb有很多frontend, Kdevelop, ddd等。
只要有debug, trace程式的觀念,GUI的debugger非常容易上手。所以就不介紹,只
專注在gdb CLI的使用上。gdb是個命令列模式的交談(interactive)除錯器,跟
telnet或其它的unix交談式程式一樣有個提示符號,然後要下命令
  \begin{verbatim}
      
(gdb)COMMAND
      
  \end{verbatim}
  不要忘了gcc編譯時要加 -g 參數, 基本gdb命令
  \begin{verbatim}
      
檔案處理
========
file a.out                 載入可執行檔a.out
path                       告訴gdb obj code在那
directory                  告訴gdb source code在那裡

SHELL
=====
shell ls                   就會執行ls了
cd xxx                     不過用shell的方法跟Makefile一樣喚起sub shell而已
                           要真的cd到目錄要用cd

中斷點(Break point and watch point)處理
=======================================
break                      設定中斷點 
clear                      清除中斷點
delete                     清除中斷點
disable                    暫時使中斷無作用
enable                     使中斷再作用
condition                  進一步設中斷點的條件 如果條件為true則中斷
commands                   如果中斷了則執行commands與end中的一連串gdb命令
.....
end
      
      其中
      中斷點可以用source code的行數來代表(這些資訊藏在ELF格式
      裡的.line這個section裡),也可以用中斷點的流水號來表示
      
br                         在目前位置設中斷點
br 100                     在100行中斷
br func1                   在func1中斷
br +100                    目前位置+100行中斷
br *0x08048123             在這位址中斷
br file.c:100              因為如果是多個c檔案時指定file.c
tbreak                     同break的寫法 不過中斷一次後 此中斷點就失效
br 100 if (var == 5)       條件中斷 後面跟著c語法的條件判斷式
br 100                     在第100行中斷並且執行command...end中的gdb命令
commands
  silent
  printf "x is %d\n",x
end
break String::func1        C++ Function Overloading的中斷 String是class

clear 100                  清除中斷點  後面跟著行號或函數名
clear func1

delete 5                   清除5號中斷點  後面是中斷點流水編號
disable 3                  暫時使3號中斷點沒作用  後面是中斷點流水編號
enable 2                   使2號中斷點作用  後面是中斷點流水編號

condition 3 (var > 3)      設3號中斷點的條件 如果條件為true則中斷
condition 3                清除3號中斷點的條件

程式執行
========
set args xxx               給執行程式參數xxx,就是main裡的**argv            
run                        開始跑程式
continue                   中斷後繼續跑
next                       往下跳一步c程式 如果有副程式 執行完整個副程式
step                       往下跳一步c程式 如果有副程式 追進副程式
until                      跳離一個while for迴圈
nexti                      往下一步CPU組語的指令(Instruction)執行完整個副程式
stepi                      往下一步CPU組語的指令(Instruction)追進副程式
until                      執行到source code的行數比目前的大
                           如果目前所在行是loop的最後一行就會跳離loop

程式變數值(data)處理
====================
print var                  看var的值
print &amp;var             印出var的位址(其時這就是C 啦)
print *var                 印出*var值 var是pointer
display var                display會每次step, next時都會印出值來,print只印一次
print (var=value)          設var的值為value
                           其實print 可以只用p代替 很多指令都可以簡寫代替
p/x                        /x表示印hex值
                           /u表示unsigned digit
                           /d    signed digit
                           /t    二進位值
                           
                           /是列印的選項 在Solaris上的adb也有相似形式
x/3uh 0x8048012            印出記憶體
                           其中
                           3表示看3個
                           u      unsigned digit(跟上面p命令一樣意義) 
                           h      halfword就是2bytes(bhwg分別是1248bytes)

GDB內定變數(跟程式變數不一樣喔)
===============================
一些gdb方便的變數(convenience variable)
$_                         用x命令所得到的最後一個位址
$__                        用x命令所得到的最後一個位址的值
$_exitcode                 程式離開的code就是用exit時的code

CPU暫存器(registers)
$pc                        program counter就是目前cpu指到的執行位置啦
$sp                        stack pointer

訊息觀看與設定
==============
info                       得到一些program debug資訊
                           info break
                           info frame
                           info display

                           info program
                           info share
                           info registers
                           
show                       得到一些系統(OS, CPU Arch), GDB資訊
                           show args       (系統傳進來的argv[0],argv[1]...)
                           show os         (OS是什麼)
                           show endian
                           show prompt     (gdb的提示符號)

list                       看原始碼
                           list x  從第x行的source code印出,x不寫從目前行印出
	                   list *addr  秀出addr所在source code的行
                                       可以先用info program找出目前PC的值
                                       再用list *addr
                           search REGEXP 在目前source code做RE搜尋

disas                      想看machine code用這個

whatis var                 告訴我var的資料型別是啥 int, char or double
ptype var                  告訴我var的資料型別是啥 這用來看struct用的

set                        設定gdb, 系統的控制變數值(這些變數不是program內的)
                           set listsize xx  設定要看xx行source code
                           set $pc xx       把PC設到 xx
                           set convenience可以自己設變數

help                       可以得到命令HELP

程序與副程式(process and sub-function)
======================================
backtrace(bt)2             程式執行到這裡前的兩個副程式,2不寫則列出全部
frame        2             選擇2號frame跳過去  2不寫就列出現在執行到那裡
up           2             往上走2個副程式
down         3             往下走2個副程式
return       expression    不要玩了,回到上一層呼叫的routine去並return一個值
finish                     繼續玩完一個選擇的stack frame(副程式)

kill                       砍掉child process
signal       procss-id     送signal給process
attach       procss-id     debug一個已經在記憶體跑的process
detach       procss-id     釋放attach的process脫離gdb的控制
  \end{verbatim}
  其中每次程式呼叫副程式時,原本的執行的世界的東西(變數值啊等等)必需先保存起來,
  然後再跳到新世界(將要執行的副程式)這就是stack,每叫一個sub routine就等於進到
  一個stack frame
  \begin{verbatim}

(gdb)frame 2 

  \end{verbatim}
  就是選擇2號frame,而0號frame就是目前在執行的副程式,1號是呼叫0號的副程式,以
  此類推,finish搭配frame這個命令來用
  \\\\
  所以bt這個命令很重要,可以追回之前叫了那些function來到目前的地方。
  通常在命令列也有類似的追蹤system call的程式,因為system call很重要,
  在Solaris上我們可以用
  \begin{verbatim}  
$ truss prog1
  \end{verbatim}  
  在Linux上
  \begin{verbatim}  
$ strace prog1
  \end{verbatim}  
  來看現在程式到底叫了甚麼system call導致他毀掉。
  \\\\
  attach, detach必需在有支援process 的環境,因為有的沒記憶體保護OS,或embadded 
  system沒有支援,另外也要有能力送signal給process的環境才行,這主要可以來debug
  deamon或做multiprocess的除錯      
  
  \subsection{訊號(signal)處理}
  除了可以送signal給程式外孩可以指定如何處理signal
	
  \begin{verbatim}
handle signal keyword
  \end{verbatim}
  signal是下面其中一個
  {\small \begin{verbatim}
(gdb) shell kill -l
1) SIGHUP   2) SIGINT   3) SIGQUIT  4) SIGILL   5) SIGTRAP
6) SIGABRT   7) SIGBUS   8) SIGFPE   9) SIGKILL 10) SIGUSR1
11) SIGSEGV 12) SIGUSR2 13) SIGPIPE 14) SIGALRM 15) SIGTERM
16) SIGSTKFLT 17) SIGCHLD 18) SIGCONT 19) SIGSTOP 20) SIGTSTP
21) SIGTTIN 22) SIGTTOU 23) SIGURG  24) SIGXCPU 25) SIGXFSZ
26) SIGVTALRM 27) SIGPROF 28) SIGWINCH  29) SIGIO 30) SIGPWR
31) SIGSYS  34) SIGRTMIN  35) SIGRTMIN+1  36) SIGRTMIN+2  37) SIGRTMIN+3
38) SIGRTMIN+4  39) SIGRTMIN+5  40) SIGRTMIN+6  41) SIGRTMIN+7  42) SIGRTMIN+8
43) SIGRTMIN+9  44) SIGRTMIN+10 45) SIGRTMIN+11 46) SIGRTMIN+12 47) SIGRTMIN+13
48) SIGRTMIN+14 49) SIGRTMIN+15 50) SIGRTMAX-14 51) SIGRTMAX-13 52) SIGRTMAX-12
53) SIGRTMAX-11 54) SIGRTMAX-10 55) SIGRTMAX-9  56) SIGRTMAX-8  57) SIGRTMAX-7
58) SIGRTMAX-6  59) SIGRTMAX-5  60) SIGRTMAX-4  61) SIGRTMAX-3  62) SIGRTMAX-2
63) SIGRTMAX-1  64) SIGRTMAX  
\end{verbatim}
}
    keyword是
  \begin{description}
    \item [nostop] 當接到這個signal時GDB 不要停止你的程式
    \item [stop] 當接到這個signal時GDB 停止你的程式
    \item [print] 當接到這個signal時GDB 印出這是什麼sginal
    \item [noprint] 當接到這個signal時GDB 不印出這是什麼sginal
    \item [pass] GDB pass這個signal給你的程式 也就是你的程式有能力處理這個siganl
    \item [nopass] GDB 不會讓你的程式看到這個signal
  \end{description}
	
  \subsection{multi-thread與multi-process除錯}
  multi-thread mutli-process除錯最討厭的是程式邏輯不再是一步一步,
  而是會有時這個process or thread執行到一半,時間到就被排程到後面去了,
  造成輸出的沒有前後關係,所以講到thread/process,就一定要講到OS的Scheduler
  \\\\    
  一個thread就是一個sub-routine在同一個process image下,所以常聽教科書說 stack 
  pointer不一樣 (所以thread的text執行位置還是跟process一樣,
  不過stack區域每個thread有自己的保存區,原本process的是在最下面)
  programe couter(就是目前CPU應該指到的執行位置)不一樣,
  可以同時access變數(所以此變數為global變數),
  其實只要說是可以fork subroutine的就是thread,
  看一下前面Linux的執行image, 然後真的寫個multithread程式就懂了,
  以pthread而言
  \begin{verbatim}
pthread_create(&t_id, &t_attr_obj, sub_func_name, &arg);
  \end{verbatim}
  這就是一個thread的建立方法,每次呼叫這個就會把一個執行的單位(context)放到
  OS的scheduler後面去。等到一個個的執行一直到這個context了,就會被執行了。
  \begin{verbatim}
pthread_join(t_id, &status)
  \end{verbatim}
  相當於waitpid()會block住的等著這個要返回的thread,status是個指標的指標**status。
  其中作為thread的副程式如果要還回一個值,通常要還回一個指標型的位址,指著一串的還回值。
\begin{verbatim}
void * func_thread(void *arg)
{
       xxxxx
       xxxxx
}
\end{verbatim}
 arg就是在create的傳進去的arg,sub\_func\_name就是這個func\_thread副程式名。
 return必須return一個pointer。(compiler有的不會complain傳回不是void *,
 其他有的compiler就會complain)
 \\\\
  multi-thread程式中比較常見的錯誤是如果caller thread不等forked thread,
  也就是不join, 而傳遞local變數像在用一般程式傳遞, 因為local變數如果副程式走完了
  stack也消失了, 而scheduler的單位是thread, 此時forked thread往往得到錯誤的值,
  thread的除錯跟傳統的很像,只是要追蹤thread編號
  \begin{verbatim}
info threads                   跟info frame很像看thread號碼(pthread_create的ID)
thread xxx                     跳到xxx號thread
break 13 thread 2              執行第二號thread時在第13行中斷
break frik.c:13 thread 28 if bartab > lim
  \end{verbatim}
  break 的用法跟一般一樣,不過多一個thread的字眼其餘是一樣的
  multi-process 用gdb其實沒什麼, 只能在fork後的小孩放個sleep,
  讓child進到sleeping狀態停住, 然後再開一個gdb用attach的方法把小孩叫進來,
  在不同的process(就是不同的gdb啦所以要另外喚起一個gdb)中切換trace,
  每走一步每改一個想要的值,這相當於把多個process放到gdb來控管,
  以下面程式做例子
  \begin{verbatim}
#include <stdio.h>
#include <unistd.h>
#include <sys/types.h>
#include <sys/wait.h>

int func2 ()
{
        int c;
        static int d;

        return 2;
}
static int func1 ()
{
        static int b;
        int *c;
        int d;
        func2();

        return 1;
}

int main()
{
	int a, pid, status;
        tid_t &tid;

	if ((pid = fork()) < 0)
                printf("fork error");
	else if (!pid) {  /* child */
                sleep(100);
                a = 1;
                printf("I dont want to be a zombie");
	}
	else {
	        printf("I dont want to be a orphaned");
	        waitpid(pid, &status, 0);
                a = 100;
	}

        pthread_create(func1);
        pthread_join()
	return 0;
}
  \end{verbatim}
  先開一個gdb並且用file命令來load進a.out在parent直接run
  \begin{verbatim}
GNU gdb 19990928
Copyright 1998 Free Software Foundation, Inc.
GDB is free software, covered by the GNU General Public License, and you are
welcome to change it and/or distribute copies of it under certain conditions.
Type "show copying" to see the conditions.
There is absolutely no warranty for GDB.  Type "show warranty" for details.
This GDB was configured as "i686-pc-linux-gnu".
(gdb) file a.out
Reading symbols from a.out...done.
(gdb) run
Starting program: /home/cyril/tmp/a.out
  \end{verbatim}
  由於我們放了waitpid(),爸爸會停下來
  再開一個gdb也把a.out load進來,並且用shell ps -aef|grep a.out來看
  child的process id並且把它attach進來
  \begin{verbatim}	
cyril    26689 26678  0 12:20 pts/1    00:00:00 /home/cyril/tmp/a.out
cyril    26690 26689  0 12:20 pts/1    00:00:00 /home/cyril/tmp/a.out
cyril    26691 26687  0 12:20 pts/5    00:00:00 ps -aef
(gdb) attach 26690
Attaching to program: /home/cyril/tmp/a.out, Pid 26690
Reading symbols from /lib/libm.so.6...done.
Reading symbols from /lib/libc.so.6...done.
Reading symbols from /lib/ld-linux.so.2...done.
0x400b2081 in nanosleep () from /lib/libc.so.6
	
	0x400b2081是libc裡的sleep這個function,我們不要理它,所以我們下
	
(gdb) return
	
	讓它回到我們main裡來
	
(gdb) info program
Using the running image of attached Pid 26690.
Program stopped at 0x400b2081.
It stopped with signal SIGSTOP, Stopped (signal).
(gdb) return 
Make selected stack frame return now? (y or n) y
#0  0x80486b8 in main ()
    at debug.c:45
45                  sleep(100);
(gdb) s
Single stepping until exit from function nanosleep, 
which has no line number information.
46                  printf("I dont want to be a zombie");
(gdb) s
47              }
  \end{verbatim}
  可以看到小孩process在我們掌控下了,接下來就可以step, next...來追它了
  \\\\
  通常multi-thread中很麻煩的就是有dead lock,
  \begin{verbatim}
# ps -aef | grep apa
root      2189     1  0 Mar04 ?        00:00:59 /usr/sbin/apache2 -k start
www-data 21759  2189  0 06:25 ?        00:00:00 /usr/sbin/apache2 -k start
www-data 21801  2189  0 06:25 ?        00:00:00 /usr/sbin/apache2 -k start
www-data 21802  2189  0 06:25 ?        00:00:00 /usr/sbin/apache2 -k start

# gdb /usr/sbin/apache2 21759

(gdb) info threads
(gdb) interrupt -a
(gdb) thread apply all bt
  \end{verbatim}
  interrupt -a是把所有threads停下來,然後去抓所有threads的backtrace,就是這樣
  而已。其他的追steps的方法就跟上面是一樣的。
  \\\\
  同樣在後面multi-thread的程式中,我們可以來觀察thread與thread stack的記憶體
  在Linux中一般程式可以print一些變數或函式名試看看,例如printf() cos(),或者
  程式的動態變數,static變數, 對照一下之前的"linux下的ELF image",
  看一下thread的stack在0x0 ~ 0x08048000,
  奇怪這片這麼大片的記憶體幹什麼去了, 原來是thread拿去玩了,
  一般傳統的subroutine呼叫的stack變數從0xbfffffff往上長起,
  thread frame從0x8048000這邊開始長起
  \\\\
  client/server程式除錯, 寫過socket程式的人都知道有個無窮迴圈在那邊等待request,
  這部份應該跟用multiprocess的方法一樣,只不過變成兩台機器的multiprocess
  或multithread程式除錯	
  
  \section{release image的debug}
  在使用gcc -g後,整個的obj檔變得非常的大,在真正release時,我們是會用strip
  把這些多餘的information拿掉的,這些對客戶沒有用但一旦客戶發生問題時,又是
  非常重要的debug info與symbol tables的資訊,我們必須先把他拿出來,
    \begin{verbatim} 
$ gcc -g test.c
$ objcopy --only-keep-debug a.out a.out.debug
$ strip a.out
$ objcopy --add-gnu-debuglink=a.out.debug a.out
    \end{verbatim} 
    用objdump -s a.out會發現他多了一個新section,裡面有一個a.out.debug的值。
    這個a.out.debug就是相對於a.out而只藏有debug info與symbol table的檔案。
    假設a.out在絕對路徑/usr/bin/下,當用gdb /usr/bin/a.out時他會根據下列順
    序來找a.out.debug
    \begin{verbatim} 
/usr/lib/debug/.build-id/ab/cdef1234.debug
/usr/bin/a.out.debug
/usr/bin/.debug/a.out.debug
/usr/lib/debug/usr/bin/a.out.debug. 
    \end{verbatim} 
    其中build-id這是gcc編譯時,多加--build-id的給的,如果沒有就沒有。上面假設
    build-id是abcdef1234,所以取前面兩碼為目錄名,然後後面id加上.debug檔名。
    gdb有一些設定path的命令來讓gdb抓到a.out.debug檔案與source。
    \begin{verbatim} 
set debug-file-directory 內定就是/usr/lib/debug
set directories 內定是$cdr目前目錄。
    \end{verbatim} 
    要小心的是用debug-file-directory時,他其實是會從這個目錄下去找絕對路徑名的。
    例如當你gdb a.out時,其實是gdb `pwd`/a.out,gdb會嘗試著去找
    debug-file-directory/`pwd`/a.out.debug,而不是debug-file-directory/a.out.debug
    。如果binary很複雜的動態連結自己的so檔。
    \begin{verbatim} 
solib-search-path     : 設定so library的搜尋目錄
solib-absolute-prefix : 設定so library絕對路徑的prefix
    \end{verbatim} 
    通常比較大條的都是客戶那邊發生 core dump, 
    C處理不好的話很容易記憶體跨出範圍, 或者系統毀了(panic), 這時都會產生
    一個core dump, 就是毀掉的瞬間記憶體內部的內容會搬到一個檔案core,
    core file 包含了程式的read/write的memory部份, 也就是程式躺在記憶體時的狀態,
    executable只是一個可執行檔也就是程式躺在硬碟時。 gdb可以根據這個檔來除錯,
    只是這時的target是core 或exec 不是remote。
    通常這可以拿來做系統毀掉時的debug
    \begin{verbatim}	
gdb a.out core        命令列上給executable與core file
target exec a.out     跟file命令不一樣的是exec不load symbol table 
                      只loadTEXT與initialized DATA這時程式只可run
                      無法看source code與檢查任何變數

target core core      core file 
    \end{verbatim}
    用下面程式做例子
    \begin{verbatim}
#include <stdio.h>
#include <string.h>
int func2()
{
        char *x = 0x0;

        *x = 1;
        printf("%s", x);
        strcpy(x, "This is wrong");
}
int func1()
{
        func2();
}
int main()
{
        func1();
}
    \end{verbatim}
    先ulimit 設定要產生 core, 跑它之後產生core dump檔並用gdb來看它
    \begin{verbatim}
chuang@dsuen-lnx:~$ ulimit -c 1024
chuang@dsuen-lnx:~$ ./a.out
Segmentation fault (core dumped)
chuang@dsuen-lnx:~$ !gdb
gdb a.out core
GNU gdb (Debian 7.7.1+dfsg-5) 7.7.1
Copyright (C) 2014 Free Software Foundation, Inc.
License GPLv3+: GNU GPL version 3 or later <http://gnu.org/licenses/gpl.html>
This is free software: you are free to change and redistribute it.
There is NO WARRANTY, to the extent permitted by law.  Type "show copying"
and "show warranty" for details.
This GDB was configured as "x86_64-linux-gnu".
Type "show configuration" for configuration details.
For bug reporting instructions, please see:
<http://www.gnu.org/software/gdb/bugs/>.
Find the GDB manual and other documentation resources online at:
<http://www.gnu.org/software/gdb/documentation/>.
For help, type "help".
Type "apropos word" to search for commands related to "word"...
Reading symbols from a.out...done.
[New LWP 5707]
Core was generated by `./a.out'.
Program terminated with signal SIGSEGV, Segmentation fault.
#0  0x000000000040051a in func2 () at core.c:7
7   *x = 1;
(gdb) bt
#0  0x000000000040051a in func2 () at core.c:7
#1  0x0000000000400562 in func1 () at core.c:14
#2  0x0000000000400573 in main () at core.c:19
    \end{verbatim}
    當然這例子很簡單,不過不用自己一步步追到死掉的地方,
    gdb a.out core自動用file命令load了symbol table進來,所以我可以用
    print x來看它。通常捉到問題點就是用bt(backtrace)捉出之前到底叫了甚麼。

  \section{遠端除錯}
    一般device或者遠端機器上,如果沒有環境跑gdb來除錯,可以用gdb這項特異功能
    來除錯,或者我在i386機器上跑ddd 或gdb 除錯遠端是 arm64 embedded binary 
    環境,古時候,有點麻煩的要自己寫 gdb stub 連結 gdb 跟 terminal 的一些能力
    ,但現在只要 gdb 在local跑, 遠端機器跑 gdbserver 與要被除錯的程式就可以,
    gdbserver 是 gdb 現在會附上的 remote gdb stub 程式。
    gdb可以透過serial line或TCP/IP來對遠端除錯, 常用的serial line方法也可以
    透過terminal console server。假設使用 port 9999 attach process id 5312
    \begin{verbatim}
xyz $ gcc -g testing.c
xyz $ gdbserver :99999 a.out
    \end{verbatim}
    本地端
    \begin{verbatim}
$ gdb -q
(gdb) target remote xyz:99999
    \end{verbatim}
    通常有
    \begin{verbatim}
(gdb) target remote /dev/ttyA
(gdb) target remote machine:TCP_port (machine 是terminal server)
(gdb) target cisco machine:TCP_port
    \end{verbatim}
    可以直接用 ssh 叫起一個程式或者現在流行的 container
    \begin{verbatim}
(gdb) target remote | ssh -T xyz gdbserver - --attach 5312
(gdb) target remote | sudo docker exec -i e0c1afa81e1d gdbserver - /bin/sh
    \end{verbatim}
    傳統上這個要被除錯的可執行檔除了-g 以外還要 link 到一些特殊的sub-routine,
    這些subroutine是跟機器有關的程式叫stub, 用來在遠端執行然後跟本地的gdb溝
    通用的, stub跟target CPU有關才知道怎樣控制 target CPU 的 trace 功能,另
    外stub還要能解釋 gdb 送過來的 message 也就是 gdb 自己溝通用的protocol。
    所以是伴隨gdb package過來有的程式。
    \\\\
    現在有的stub i386-stub.c, sh-stub.c, VxWorks, sparc, m68k...,後來 arm
    等等,所以新的 gdb 把這些 stub 包成一個 gdbserver ,常用目前的 CPU,可以
    直接使用不需要自己處理這些了。
    \\\\
    這些是伴隨gdb來的.c source code, 例如 i386-stub.c, 這些xxx-stub.c
    裡面都含有
    \begin{verbatim}
set_debug_traps()
handle_exception()
breakpoint()
    \end{verbatim}
    但是gdb 只曉得CPU 而已,CPU 跟外界溝通的I/O chip gdb不管, 自己必需寫5個
    低階的函數來溝通。所以去看stub的code都會去呼叫下面函數, 不過這要自己
    implement。
    \begin{verbatim}	
getDebugChar()
putDebugChar()
flush_i_cache()
exceptionHandler()
memset()     這是ANSI C的標準應該都有的不用寫了
    \end{verbatim}	
    也就是set\_debug\_traps breakpoint 負責處理掉 target CPU 中的單步 trace 
    部份, 並且會處理由gdb送來的訊息,也就是remote protocol知道怎麼處理。像
    Cisco有他們自己的 protocol。ok gdb經由Host serial port到target,但是target
    系統中CPU還是要靠 I/O serial chip來跟Host溝通,這邊每家的I/O chip是gdb
    不知道的,不像Host這邊 OS已經處理掉serial port driver。所以target那邊的
    gdb stub會去呼叫自己要implement的getDebugChar putDebugChar...。另一種作法
    是用JTAG, 也就是target CPU跟外界溝通用JTAG,不用serial port,target CPU
    只跟JTAG 控制晶片溝通,JTAG message protocol算是一種介面,凡是想要給CPU的
    命令用JTAG的好處是不比針對某個特定CPU寫instruction,有比較大的general特性
    ,JTAG 會負責跟後面的 CPU 溝通,把該要的 Instruction 轉成這一個 CPU 懂的
    instruction。如果用這種方法則不需要stub在target CPU上compile,但必須要有
    個中介box作gdb remote protocol與 JTAG的轉換。(有人在賣這種盒子還貴的嚇人)
    \\\\  
    在要被debug的程式最前面需要呼叫
    \begin{verbatim}	
set_debug_trap();
breakpoint();
    \end{verbatim}	
    然後編譯時要把這些一起link起來最後丟到遠端機器跑,
    然後在gdb ddd 下
  	
    \begin{verbatim}	
target remote /dev/ttySx
or
target remote machine:TCP_port
    \end{verbatim}	
    就可以像正常的gdb用法來用了,不過要注意的是 image 是不是有symbol table,
    有些image的格式當初編譯時由於是embadded system,image大小很重要,所以拿掉了
    symbol table。記住gdb參照的symbol table image與source code是在local端,
    所以local host用gdb命令"path"指的image要有symbol table,這個要跟上一個章節
    的 release 後的 image 除錯合用。
