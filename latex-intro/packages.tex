\chapter{\LaTeX{} 與 packages 的使用}
參考資料講解可以從
\begin{itemize}
  \item \href{https://en.wikibooks.org/wiki/LaTeX}{wikibook \LaTeX{}}
  \item \href{https://en.wikibooks.org/wiki/TeX}{wikibook \TeX{}}
  \item \href{https://latexref.xyz}{非官方 \LaTeX{} reference}
  \item \href{https://texdoc.org}{texdoc 網站}
  \item \href{https://www.tug.org/texlive/devsrc/Master/doc.html}
    {TeXLive 收集的 package 文件}
  \item \href{https://www.overleaf.com/learn}{overleaf 學習網站}
\end{itemize}
都是標準 package ,剩下的都是屬於更細微的調整,所以最後只是去找 package
的詳細 reference 說明使用而已,而如果用 CTAN script 自己安裝的,用 tlmgr
裝 package 時,也都會把 pdf 文件安裝進來,在 TEXDIR 下搜尋 pdf 檔都能找到。
\\\\
{\LaTeX} 如果是多檔使用,可以用
\begin{verbatim}
\begin{docuemnt}
\include{chapter1}
 或者
\input{chapter1.tex}
\end{document}
\end{verbatim}
include 一定要是 .tex 副檔名,且不可寫副檔名,
input 則可以使用別的副檔名,副檔名可寫可不寫。

\section{書籍排版元素}
除了基本的章節段落 \verb=\chapter{} \section{} \subsection{} \paragraph{}=,
強迫分行使用兩個反斜線,\verb=\\=,所以空一行就要 \verb=\\\\= ,或者用
\verb=\\[1cm]= 指定換行空白大小。光這樣其實就可以編幾頁文章了。
\\\\
最重要的單位元素其實是 paragraph (par),很多排版的規矩,縮排,對齊,大小,留白
, 都是以 paragraph 為根本,它會在版面上形成一個小盒子,盒子內的文字太多才會自
動折行,沒有 paragraph 就不會自動折行,在後面像圖表對齊等使用,效果都需要這小
盒子幫忙才行。 
\\\\
幾個常用基本書籍文章邏輯元素使用,例如
\begin{itemize}
  \item 封面 title page
  \item 目錄 table of content TOC 每本書的目錄。
  \item 註腳 footnote 在書頁底下特別說明本頁中特定詞彙。
  \item 列舉 list 不同的列舉。
  \item 參照 cross reference 整本
  \item 索引 index 在書後面讓人好找到特別詞彙。
  \item 參考文獻 bibliography
\end{itemize}

  \subsection{Title and TOC}
  title 與 TOC 範例,主要是那個 maketitle 與 tableofcontents 要在 document 裡
  面。 如果只有一位作者,那就不需要那個 and 了。
  \begin{verbatim}
\documentclass[a4paper,10pt]{book}

\title{Linux Kernel Driver}
\author{
  Alibuda Huang \\
  alibuda\_huang@yahoo.com \\
  \and 
  Asaburu Liu \\
  asaburu\_liu@gmail.com \\
  Gyoza Publisher
}

\begin{document}
\maketitle
\tableofcontents
...
\end{document}
  \end{verbatim}
  TOC 要注意的就是要跑兩次引擎,第一次產生 .toc 檔,第二次才會真正在裡面產生
  漂亮的 TOC。

  \subsection{comment}
  在 \verb=%= 之後的字元都是 \LaTeX{} comment,要大片的 comment 用 comment
  package,不會出現在文章中
  \begin{verbatim}
\usepackage{comment}
\begin{comment}
  this is comment across
  multiple lines.
\end{comment}
  \end{verbatim}
  而想要對文件內容表達意見,可用 todonotes package,會在頁面旁邊空白處 (margin)
  出現意見內容
  \begin{verbatim}
\usepackage{todonotes}

...
The text in this line will be emphsized to give a comment by other people
\todo[linecolor=red,backgroundcolor=red!25,bordercolor=red]
{請改正拼錯字}
...
  \end{verbatim}
  The text in this line will be emphsized to give a comment by other people.
  \todo[linecolor=red,backgroundcolor=red!25,bordercolor=red]
  {請改正拼錯字}
  From this sentence, it's still part of the paragraph.
  \\\\
  在 todonotes 的 sty 裡面註解講到,如果版面的 marginparwidth 小於 2cm 則會跑
  出頁面,可以在 load todonotes 前設定 marginparwidth ,例如
  \begin{verbatim}
\setlength{\marginparwidth}{1.5cm}
\usepackage{todonotes}
  \end{verbatim}
  這裡面可以設定 author, date 等等,也能做出一個 note list 出來,如果原文章作者
  改正後,可以用 disable 把 notes 不顯示,但還是留在原有 tex 檔裡面,這可以跟版
  本控制系統做文件資訊管理,誰做了任何事都能保存紀錄。

  \subsection{footnote}
  footnote 範例
  \begin{verbatim}
宅男\footnote{一種胖胖的可愛動物}
  \end{verbatim}
  宅男\footnote{一種胖胖的可愛動物}

  \subsection{list}
  list 是最常用到元素,有三種
  \begin{verbatim}
\begin{itemize}
\item 最簡單前面圓點 list 1
\item 最簡單前面圓點 list 2
  \begin{enumerate}
  \item 不用寫1
  \item 不用寫2
    \begin{description}
    \item [item1] item1 會變粗體
    \item [item2] item2 會變粗體
    \end{description}
  \end{enumerate}
\item 最簡單前面圓點 list 3
\end{itemize}
  \end{verbatim}

  \subsection{cross reference}
  corss reference \label{sec:ref} 用在參考圖表或某段特定文字,例如在文章某處
  設下 mark 這個 label key
  \begin{verbatim}
\label{sec:mark}
....
  \end{verbatim}
  在文章別的地方用 \verb=ref= 或 \verb=\pageref=
  \begin{verbatim}
  .....
請參照 第\ref{sec:mark} 節
在第 \pageref{sec:mark} 頁中
  \end{verbatim}
  則會自動顯示該有的提示編號所在。 例如設定 ref 在這個 subsection,則本段參照
  (\ref{sec:ref}),會顯示這個 subsection 編號。
  \\\\
  使用 cross reference 也要跑兩次引擎才有用。 其中有個不成文規矩是 cross
  reference 內的 label 中,key 用
  \begin{itemize}
    \item sec:xxx 表示 section, page
    \item fig:xxx 表示圖形
    \item tab:xxx 表示 table
    \item equ:xxx 表示數學式子
  \end{itemize}
  表示不同的 cross reference id key。

  \subsection{index}
  index 索引製作必須要有 makeidx package,makeindex 命令必須放在 preamble ,
  然後 文章裡面用 index 命令加到索引資料庫 ,最後 printindex 印出
  \begin{verbatim}
\usepackage{makeidx}
\makeindex
.....
文字\index{索引key@文字印出格式}
....

\printindex
....
  \end{verbatim}
  這個 index\index{index@index} 裡面的長相可以由後面的格式設定決定,還瞞複雜沒
  規則的
  \begin{center}
  \begin{tabular}{ccp{4cm}}
    命令 & 意義 \\
    \hline
    \verb=哈囉\index{hello}= &
    用 hello \index{hello} 做索引 key 並印出 hello 頁數\\

    \verb=hello\index{hello@哈囉}= &
    在索引頁會印出哈囉\index{hello@哈囉}\\

    \verb=\index{hello!world}= &
    會在 hello 下面印出空格斜體的 world \index{hello!world} 表示是 sub index\\

    \verb=\index{hello|emph}= &
    頁數用 emph 方法印出 \index{hello|emph} \\
    \verb=\index{hello@\emph{hello}}= &
    hello 用emph印出\index{hello@\emph{hello}}\\
  \end{tabular}
  \end{center}
  \printindex
  它這跟TOC, reference 一樣,要跑兩次,而且要額外用 makeindex 程式處理之前產
  生的 idx 檔
  \begin{verbatim}
$ xelatex latex.tex; makeindex latex.idx; xelatex latex.tex
  \end{verbatim}

  \subsection{bibliography}
  bibliography 用在論文或書後面,使用thebibliography 環境與 cite 命令來引用,
  跟之前 lable 或 index 一樣有 key 值參照。以下範例,99 只是印出文獻時,前面
  數字編號的字體寬度,不知道為什麼當初是這樣設計的參數。
  \begin{verbatim}
\cite{kernel}


.....
\begin{thebibliography}{99}
  \bibitem{kernel}Linux Kernel Development Third Edition
        Addison Wesley, ISBN-13: 978-0-672-32946-3
  \bibitem{ldd3}Linux Device Drivers, 3rd Edition
        O'Reilly , ISBN-13: 978-0596005900
  \bibitem[docs]{docskernel} http://docs.kernel.org
  \bibitem[os]{osdev} http://wiki.osdev.org
  \bibitem{gitkernel}
        https://linux-kernel-labs.github.io/refs/heads/master/
\end{thebibliography}
  \end{verbatim}
  \begin{tabular}{cc}
    命令 & 效果 \\
    \hline
    \verb=\cite{kernel}=& \cite{kernel} \\
    \verb=\cite{docskernel}=& \cite{docskernel}\\
    \verb=\cite[alibuda,LDD3]{osdev,ldd3}=& \cite[alibuda,LDD3]{osdev,ldd3}\\
    \verb=\cite[docs.kernel.org]{docskernel}=& \cite[docs.kernel.org]{docskernel}\\
    \verb=\cite{osdev}=& \cite{osdev}\\
  \end{tabular}
  \begin{thebibliography}{99}
    \bibitem{kernel}Linux Kernel Development Third Edition
      Addison Wesley, ISBN-13: 978-0-672-32946-3
    \bibitem{ldd3}Linux Device Drivers, 3rd Edition
      O'Reilly , ISBN-13: 978-0596005900
    \bibitem[docs]{docskernel} http://docs.kernel.org
    \bibitem[os]{osdev} http://wiki.osdev.org
    \bibitem{gitkernel}
      https://linux-kernel-labs.github.io/refs/heads/master/
  \end{thebibliography}

\section{字形設定}
字型是讓整個文章排版成功的最重要設定,但原本的引擎plain \TeX{} 或 pdftex 
有很多限制
\begin{itemize}
  \item 內定用 Computer Modern 字型,metafont 格式,OT1 編碼向量字, 很多設定
    family, 斜體,大小等等其實都只是在設定 CM font 而已。並沒有很好統一的方法
    選取不同字型。
  \item 原本引擎使用 NFSS, 要有 font metric 檔還有相關map, fd 定義檔建造好才能
    使用新字型, 這些資訊檔,一般要靠工具擷取出來,是 distro 幫我們建造好的,因
    此沒有建造就無法使用新字型。除非對整個 \TeX{} 很熟的人,不然新手是叫苦連天。
\end{itemize}
在新的xetex, luatex 可以使用
\href{https://www.tug.org/texlive/devsrc/Master/texmf-dist/doc/latex/fontspec/fontspec.pdf}
{fontspec} 這個 macro 來 run time 時邊擷取出 font metric 等資訊,邊做排版,所
以用 CJK 相關套件時,其實後面都是用上 fontspec 這個套件。即使創造新命令也是相
近的 macro 名稱。
\begin{verbatim}
\usepackage{fontspec}
\setmainfont{Georgia}
\setsansfont{Arial}
\setmonofont{Arial}

\begin{document}
alibuda
....
\end{document}
\end{verbatim}
setmainfont, setsansfont, setmonofont 後面的參數可以直接是檔名,也可以是
fc-list 抓出來藏在字型檔裡面的字型名,fc-list 抓出的第一與第二欄位都可以。
而如果跟中文的 setCJKmainfont 一起用,則可以中文英文分開來。
\\\\
選定字型後才是字型參數變換, 就是那些斜體,大小等等設定,這些是原本 \LaTeX{}
就有的 macro 設定,一樣只想作用在某文章部份用 grouping 方式。
\\\\
家族,粗斜體設定
\begin{description}
  \item [Family] - \verb=\textrm{} \textsf{} \texttt{}=
  \item [Weight] - \verb=\textbf{} - bold, \textmd{} - medium=
  \item [Shape] - \verb=\textup{}, \textit{}, \textsl{}=
\end{description}
大小設定
\begin{center}
\begin{tabular}[t]{lll}
  命令 & 效果 & 在本文是10pt的相對大小 \\
  \hline
  \hline
  \verb=\tiny{tiny}= & \tiny{tiny} & 5pt \\
  \verb=\scriptsize{scriptsize}= & \scriptsize{scriptsize} & 7pt \\
  \verb=\footnotesize{footnotesize}= & \footnotesize{footnotesize} & 8pt \\
  \verb=\small{small}= & \small{small} & 9pt \\
  \verb=\normalsize{normalsize}= & \normalsize{normalsize} & 10pt \\
  \verb=\large{large}= & \large{large} & 12pt \\
  \verb=\Large{Large}= & \Large{Large} & 14.4pt \\
  \verb=\LARGE{LARGE}= & \LARGE{LARGE} & 17.28pt \\
  \verb=\huge{huge}= & \huge{huge} & 20.74pt \\
  \verb=\Huge{Huge}= & \Huge{Huge} & 24.88pt \\
\end{tabular}
\end{center}
上面都是一些已經定義好方便用的,也就是排版這麼多年有些約定成俗的規格,
大小設定命令格式是三種都有,所以它也有 environment begin, end 使用。
而直接選擇大小,只作用在我們想要的一部分文字可以用 grouping 的方法
\begin{verbatim}
{\fontsize{5cm}{5.5cm}\selectfont 想要作用的文字}
\end{verbatim}
5.5cm 是 line space 行距。前面 5cm 不寫單位就是用 pt,一個 pt 在
不同地方大小不太一樣,但大概是 0.35mm
\begin{itemize}
  \item American standard : 0.35136 mm
  \item Computer standard : 0.3527 mm
  \item European standard : 0.37597151 mm
\end{itemize}

最後其他一些效果
\begin{center}
\begin{tabular}[t]{lll}
  命令 & 效果 & 作用 \\
  \hline
  \hline
  \verb=\emph{emph}= & \emph{emph} & 強調 \\
  \verb=\underline{underline}= & \underline{underline} & 底線  \\
  \verb=\uppercase{uppercase}= & \uppercase{uppercase} & 大寫  \\
  \verb=\lowercase{LOWERCASE}= & \lowercase{LOWERCASE} & 小寫  \\
\end{tabular}
\end{center}
更複雜的都要自己寫 macro 了。

\section{一些設定範本}
\LaTeX{} 有內定標準 package ,但有的不是很常用, 一些簡單 package 的設定能讓呆
板醜醜的內定設定變得漂亮。
\begin{itemize}
  \item \href{https://www.tug.org/texlive/devsrc/Master/texmf-dist/doc/latex/xcolor/xcolor.pdf}{xcolor}
  能夠上顏色的 package。
  \item \href{https://www.tug.org/texlive/devsrc/Master/texmf-dist/doc/latex/hyperref/hyperref.pdf}{hyperref}
  能夠做像 html 跳來跳去連結 link 的功能。
  \item \href{https://www.tug.org/texlive/devsrc/Master/texmf-dist/doc/latex/fancyhdr/fancyhdr.pdf}{fancyhdr}
  能夠做頁眉上的提示。
\item \href{https://www.tug.org/texlive/devsrc/Master/texmf-dist/doc/latex/listings/listings.pdf}{listings}
  能夠用來寫程式語言的表現。
\end{itemize}

  \subsection{xcolor}
  設定顏色的 package ,很多其他 package 都以這個為基底加上顏色功能,為了使用一
  些好記的顏色名字,多加 dvipsnames driver
  \verb=\usepackage[dvipsnames]{xcolor}= 而且有的名字是大小寫有差別的。
  \begin{center}
    \includegraphics[width=0.8\textwidth]{images/xcolor-dvipsnames.png}
  \end{center}

  \begin{center}
    \begin{tabular}{ccc}
      命令 & 效果 \\
      \hline\hline
      \verb=\color{blue}{blue}= & \color{blue}{blue} & \\
      \verb=\textcolor{NavyBlue}{NavyBlue}= &
      \textcolor{NavyBlue}{NavyBlue} & \\

      \verb=\colorbox{green}{green background}= &
      \colorbox{green}{green background} & \\

      \verb=\fcolorbox{green}{yellow}{green frame}= &
      \fcolorbox{green}{yellow}{green frame} & \\

      \verb=\color[cmyk]{0,1,0,0}{red}= & \color[cmyk]{0,1,0,0}{red} \\
      \verb=\definecolor{myred}{rgb}{1,0,0}\color{myred}{myred} = &
      \definecolor{myred}{rgb}{1,0,0}\color{myred}{myred} \\
    \end{tabular}
  \end{center}
  定義自己的顏色使用 rgb, cmyk, gray 與 HTML 分類法,其中 RGB 跟 HTML 法都是 0
  - 255 的值,只是 HTML 用 hex 一定要大寫,rgb, cmyk, gray 是原本值的
  normalized 後 0 \~{} 1 的值,例如 rgb 是 0 \~{} 255 的相對 0 \~{} 1。
  \begin{itemize}
    \item \verb=\definecolor{mypink1}{rgb}{0.858, 0.188, 0.478}=
    \item \verb=\definecolor{mypink2}{RGB}{219, 48, 122}=
    \item \verb=\definecolor{mypink3}{cmyk}{0, 0.7808, 0.4429, 0.1412}=
    \item \verb=\definecolor{mygray}{gray}{0.6}=
    \item \verb=\definecolor{Mycolor2}{HTML}{00F9DE}=
    \item \verb=\colorlet{LightRubineRed}{RubineRed!70}=
    \item \verb=\colorlet{Mycolor1}{green!10!orange}=
  \end{itemize}
  而 green!10!orange 是說用綠色的 10\verb=%=,加上 90\verb=%= 的橘色,後面沒寫
  就自動是白色。

  \subsection{hyperref}
  這會讓 hyperlink 字包括目錄 TOC 是藍色的。在\verb=\begin{document}=前先設定
  \begin{verbatim}
\hypersetup{
  backref,
  unicode=true,
  bookmarks=true,
  pdfauthor=Alibuda Liu,
  colorlinks=true,
  breaklinks=true,
  hyperfigures=true,
  pdfstartview=FitH,
  linkcolor=blue
}
  \end{verbatim}
  在文章中使用
  \begin{verbatim}
\href{http://link.to.html}{hint text}
  \end{verbatim}

  \subsection{fancyhdr}
  頁眉在開始 document 前面 global 的 preamble 區設定
  \begin{verbatim}
\pagestyle{fancy}
\fancyhead[LE]{\small\bfseries\thepage\ \ \leftmark}
\fancyhead[RO]{\small\bfseries\rightmark\ \ \thepage}
\lfoot{餃子出品必屬佳作}
\cfoot{}
\rfoot{}
  \end{verbatim}

  \subsection{listings}
  這很像 verbatim ,但是用來針對程式語言的,不同程式語言保留字還能設定
  顏色等等設定。 主要是 lstset 在前面設定,用到的語言,是否有邊框等
  \begin{verbatim}
\lstloadlanguages{bash,[ANSI]C,[gnu]make}
\lstset{
    basicstyle=\ttfamily\small,
    keywordstyle=\color{blue}\bfseries,
    frame=single
}
  \end{verbatim}
  以下為使用 keyword comment 都有顏色,沒有 frame,有 line number 顯示的程式碼
  \begin{verbatim}
\lstset{
  language=c,
  basicstyle=\scriptsize,
  upquote=true,
  aboveskip={1.5\baselineskip},
  % keypoint for chinese
  extendedchars=false,
  breaklines=true,
  showtabs=false,
  showspaces=false,
  showstringspaces=false,
  identifierstyle=\ttfamily,
  keepspaces=true,
  %keywordstyle=\color{blue}\bfseries
  keywordstyle=\color[rgb]{0,0,1},
  commentstyle=\color[rgb]{0.133,0.545,0.133},
  stringstyle=\color[rgb]{0.627,0.126,0.941},
  % show line number
  numbers=left,
  stepnumber=1,
  breakatwhitespace=false
}
  \end{verbatim}
  在後面文章 用 lstinline, lstlisting, lstinputlisting 使用來表現程式 lstinline
  類似 \verb=\verb= 使用在文章中,兩邊用任意相同符號夾住就表示裡面文字不做任何
  轉換
  \begin{verbatim}
\lstinline=void my_func(int arg1, char* arg2)=
  \end{verbatim}
  lstlisting 用來直接寫 code
  \begin{verbatim}
\begin{lstlisting}[language={[gnu]make}]
EXTRA_CFLAGS = -g
obj-m = mysys.o
\end{lstlisting}
  \end{verbatim}
  lstinputlisting 用來引進一個外在程式檔,
  \begin{verbatim}
\lstinputlisting[language={C}]{src/sysfs/mysys.c}
  \end{verbatim}
  要小心他的 language 設定跟 lstset 的設定有點不一樣, 在 lstset 中, language
  直接=, 在 lstlisting 的設定要多加括號,而 language 有所謂方言, 像 make 有
  gnu 實做的特殊保留字,可以選擇特殊方言。

\lstset{
  language=c,
  basicstyle=\scriptsize,
  upquote=true,
  aboveskip={1.5\baselineskip},
  % columns=fullflexible,
  % keypoint for chinese
  extendedchars=false,
  breaklines=true,
  showtabs=false,
  showspaces=false,
  showstringspaces=false,
  identifierstyle=\ttfamily,
  keepspaces=true,
  %keywordstyle=\color{blue}\bfseries
  keywordstyle=\color[rgb]{0,0,1},
  commentstyle=\color[rgb]{0.133,0.545,0.133},
  stringstyle=\color[rgb]{0.627,0.126,0.941},
  % show line number
  numbers=left,
  stepnumber=1,
  breakatwhitespace=false
}
  範例使用 C 語言
  \begin{lstlisting}
#include <linux/init.h>
#include <linux/module.h>
#include <linux/device.h>
#include <linux/platform_device.h>

static struct platform_device *mysys_dev;
static ssize_t  mysys_private = 0;

static ssize_t show_mysys(struct device *dev, struct device_attribute *attr,
                                char *buf)
{
        sprintf(buf, "%ld\n", mysys_private);
        return mysys_private;
}
  \end{lstlisting}
  進階使用 define style
  \begin{verbatim}
\lstdefinestyle{Common}
{
    extendedchars=\true,
    language={[Visual]Basic},
    frame=single,
    %===========================================================
    framesep=3pt,%expand outward.
    framerule=0.4pt,%expand outward.
    xleftmargin=3.4pt,%make the frame fits in the text area. 
    xrightmargin=3.4pt,%make the frame fits in the text area.
    %=========================================================== 
    rulecolor=\color{Red}
}

\lstdefinestyle{A}
{
    style=Common,
    backgroundcolor=\color{Yellow!10},
    basicstyle=\scriptsize\color{Black}\ttfamily,
    keywordstyle=\color{Orange},
    identifierstyle=\color{Cyan},
    stringstyle=\color{Red},
    commentstyle=\color{Green}
}
\begin{lstlisting}[style=A]
....
\end{listlisting}
  \end{verbatim}

\section{插圖與表格使用}
插圖與表格是除了文字外最能表達觀念的元素,但跟原本本文文字排版的對齊,位置
等就傷腦筋了。重要觀念
\begin{enumerate}
  \item 圖表有固定的與 float 的。
  \item 排版時為了美觀,也就是位置都是排版軟體幫你計算後設定的, 如果圖表是
    一塊很大 object 物件, float 比較好額外計算空間的, 然後根據計算放在文字
    附近中的某處,通常在同頁的頂端或底部, 然後會設定一個 caption 表頭,與
    cross reference xx 在文中都會說參照 fig xx, table xx 。
  \item 圖表加入文字也只是大物件裡面。 在圖形中 includegraphics 是用
    \verb=\parbox= 並且使用 \verb=\textwidth= 整頁面寬度加入排版整頁排版的,
    裡面文字不會跟外面排版,tabular 也是設定整張 table 大小跟外面排版對齊。
  \item wrap text around figure 文繞圖表,由於是個大物件 object ,佔住一塊空
    間,因此真正本文排版時只是以為一個很大的字,文字內定是對齊底部而已,但
    因此出現一塊空空的空白,如果希望文字會很奇怪的繞著圖表填滿兩邊空白空間
    ,要有特別處理。  
\end{enumerate}

  \subsection{插圖}
  雖然現在用 jpeg, png 傳統點陣圖檔也能很漂亮的插進 {\LaTeX} 中了,但還是
  以向量圖形為主,可以直接在紙上用 {\LaTeX} 的 package macro 畫圖,或者先用
  其他畫圖工具做成 eps, svg, pdf 等向量圖再引進文章中。直接畫圖的 package
  有
  \begin{itemize}
    \item picture
    \item PGF/TikZ
  \end{itemize}
  但學 macro 命令實在太麻煩了,現在有很多繪圖工具,都可畫完再插圖。
  一般插圖可能是
  \begin{itemize}
    \item 色彩豐富的影像檔,這可以用轉檔工具轉成 eps, svg, pdf.
    \item 簡圖 ipe, inkscape
    \item 科學圖檔 如 x-y 座標, 統計圖
    \item 工程圖檔 這也很多工具,甚至 3D 的工具
    \item office 圖 各種 office,流程圖。
  \end{itemize}
  使用 \href{https://texdoc.org/serve/graphicx.pdf/0}{graphicx} 巨集,
  \verb=\usepackage{graphicx}= ,它其實會自動加入 color,
  graphics, graphicx, 能嵌入外面已有圖形。 graphicx 必須仰賴外部處理圖形
  driver ,尤其 dvips ,所以 driver 沒裝對使用時會出錯,會出現不知所以然的錯
  誤訊息
  \begin{verbatim}
(./latex.aux)
Runaway argument?
{\contentsline {subsection}{\numberline {3.4.3}浮動處理與
! File ended while scanning use of \@writefile.
<inserted text>
                \par
l.69 \begin{document}
  \end{verbatim}
  dvipdfmx, dvips, dvisvgm, luatex, pdftex, xetex 是目前可用 driver。 向量圖,
  大小可以任意設定也不會走鐘,不用特別在意原圖的大小, 但要在意原圖的解析度。
  總之用 eps, pdf 的圖檔就能餵給 {\LaTeX} 使用。
  \begin{verbatim}
\begin{center}
\includegraphics[width=10cm,height=2cm,keepaspectratio]{myimage.eps}
\end{center}
  \end{verbatim}
  例子
  \begin{center}
    \includegraphics[width=10cm,height=2cm,keepaspectratio]{images/riscv}
  \end{center}
  includegraphics 裡面的設定除了width height,還有放大跟旋轉
  \begin{itemize}
    \item scale=1.5
    \item angle=45
  \end{itemize}
  要引入 svg 檔,需要安裝 svg package,使用 \verb=\includesvg=

  \subsection{基本表格}
  表格也是很常使用的一種元素,基本表格用 tabular ,他的延伸使用是 tabular*, 但
  有些寬度高度是用一些技巧例如用 \verb=\\[-1pt]= 多加行列的控制不是很直觀, 加
  上 tabularx, array 兩個 package 可以控制更多的寬度高度線條粗細。而要表格有顏
  色,可以用 xcolor package 裡面有 table driver 選項, 或
  \href{https://texdoc.org/serve/colortbl/0}{colortbl} package 可以有簡單設定
  顏色,就會多新的 macro
  \begin{verbatim}
  \usepackage[table]{xcolor}
  \usepackage{colortbl}
  \end{verbatim}
  基本表格例子
  \begin{verbatim}
\begin{tabular}[b]{|p{3cm}@{\Large 大文字}rr||c}
  r1c1 & r1c2 & r1c3 \\
  \hline
  \hline
  r2c1 is a very long paragraph 
  that we can have new line automatically 
  with \verb=p{}= & r2c2 & r2c3 becomes longer & r2c4 becomes longer  \\
     & r3c2 & r3c3 & small \\
  r4c1 & r4c2 & r4c3 & a \\
  \cline{2-3}
  \multicolumn{2}{|c}{r5c1} & r5c2 & r5c3 \\
  \hline
\end{tabular}
  \end{verbatim}
  表現效果
  \begin{center}
    \begin{tabular}[b]{|p{3cm}@{\Large 大文字}rr||c}
    r1c1 & r1c2 & r1c3 \\
    \hline
    \hline
    r2c1 is a very long paragraph 
    that we can have new line automatically 
    with \verb=p{}= & r2c2 & r2c3 becomes longer & r2c4 becomes longer  \\
       & r3c2 & r3c3 & small \\
    r4c1 & r4c2 & r4c3 & a \\
    \cline{2-3}
    \multicolumn{2}{|c}{r5c1} & r5c2 & r5c3 \\
    \hline
    \end{tabular}
  \end{center}
  總之
  \begin{itemize}
    \item 畫整條橫線用 \verb=\hline= 合併欄位畫部份用 \verb=\cline{}=。
    \item 畫直線 必須在一開始 tabular 環境設定。合併欄位時會自動不畫。
    \item 新row 用換行 \verb=\\=
    \item 新column \verb=&=
    \item 合併 在 multicolumn 設定。
    \item 對齊 必須在 tabular 環境一開始設定,跨欄對齊在 multicolumn 設定。
  \end{itemize}
  其中 \verb=[]=是整個 table 與外面本文文字對齊上中下選項
  \begin{itemize}
    \item t, c, b 表示整張 table 與外面本文文字對齊 top, center, bottom。
  \end{itemize}
  \verb={}= 表示格子裡面文字欄位選項
  \begin{itemize}
    \item l, c, r 表示欄位內文字對齊 left, center, right,並沒有高低對齊。
    \item | 表示畫直線
    \item \verb=p{width}= 指定欄位真正寬度,p 是 paragraph,這會創造一個小盒子
      , 用 p 的功能才能讓文字多時能自動換行, 在 tablular 裡面要用 verbatim,
      listings 也需要用這個。 這無法再指定文字對齊, 內定是對齊 top,但加上
      array package 可以用 \verb=m{width}= 與 \verb=b{width}=,表示對齊 middle
      與對齊 bottom。
    \item \verb=@{font macro} 與 !{font macro}= 指定整欄位文字效果。 這也無法
      再指定文字對齊與寬度,就是多一欄,每一 row 的欄位都會出現相同 font macro
      的文字效果,然後上面的 lcr 跟 p 都跟這欄位沒有關係,它跟隔壁欄位間距設為
      0,如果要保留原有欄位間距離,用 \verb=!{font macro}=。
      \\\\
      而 \verb=>{font macro}c= 需要帶一個欄位 l,c,r,p,m,b,表示 font macro 會
      對所有欄位的內容都作用, 例如 \verb=>{\huge}c= 表示所有欄位裡面文字都會
      是 huge,這不像 @ 或 !  需要所有內容都一樣,而是所有設定都相同而已。
      \verb=c<{font macro}= 是一樣意思,不過是 font macro 加在後面,這可能比較
      少用,跟 @ 與 ! 比較有相同效果,> < ! 三種方式需要使用 array package,只
      要引入 array 就可使用。
    \item 當有多個重複的設定時,例如 \verb={c|c|}=,可以用 \verb=*{2}{c|}= 表示
    \item tabular 如果沒有用 p ,則是不會自動折行的,會一直跑出欄位, 因此最後
      一欄通常要使用 p 功能,來讓文字停在一定範圍內。或者用 tabularx 其他更好
      功能。
  \end{itemize}
  合併欄位
  \begin{itemize}
    \item \verb=\cline{2-3}= 指定劃直線只畫到何處 2-{}3 就是畫 2-{}3 兩欄。
      只有跨 column 沒有跨 row ,因為跨 row 其實是兩行 row 中某 row 欄位是空,
      然後不劃線而已。
    \item \verb=\multicolumn{count}{l|c|r}{xxx}= 從這個欄位跨 count 並且把文字
      xxx 靠 l c r 對齊,第二個 \verb={}= 跟上面的意思一樣的。
  \end{itemize}
  有些觀念可以讓對齊更得心應手
  \begin{itemize}
    \item 用 tabular 是可以巢狀下去的,一個 tabular 裡面可以再另一個 ,所以在
      對齊上面是很有用的。
    \item 原本 row 與 row,column 與 cloumn 間是有內定距離的,每產生新 row 與
      新 column 都有看不見的距離產生。row 間是命令 arraystretch 1,這是一個表
      格大小的 ratio 值,所以真正的間距是動態不是固定值,如果表格小就小,表格
      大就大, column 間則是固定 \verb=6pt=。
    \item 所以之前使用整欄位設定的 \verb=@{font macro}= 是會把 column 間距整個
      拿掉的,要保留要用 \verb=!{}=,設定row 間距可以重設
      \begin{verbatim}
\renewcommand{\arraystretch}{1.5}
      \end{verbatim}
      但這會把所有 table 都重設,所以單一個 table 內可以在 hline, cline 前後
      用\verb=\noalign{\smallskip}= 或者換行加上\verb=\\[1cm]=,增加一公分空白
      。 column 間距可以用
      \begin{verbatim}
\setlength{\tabcolsep}{5pt}
      \end{verbatim}
      或者用 \verb=@{\hspace{1cm}}= 先清掉 \verb=6pt= 再來做整欄位 \verb=1cm=
      的空白。
    \item 其他更多細微的對齊有很多其他 package, 例如
      \href{https://texdoc.org/serve/array/0}{array} package 來做更多的對齊,
      \href{https://texdoc.org/serve/dcolumn/0}{dcolumn} package 可用來做數字
      小數點對齊等等。
  \end{itemize}
  而 tabular 只能處理一頁的 table,要跨頁的大表格可用
  \href{https://texdoc.org/serve/longtable.pdf/0}{longtable} package,基本用法
  與 tabular 一樣,其他細微調整一樣去看文件
  \begin{verbatim}
  \usepackage{longtable}
  \begin{longtable}{c|c|c}
    \hline 
    a & b & c\\
    \hline
    d & e & f\\
    \hline
    \caption{大型表格}
  \end{longtable}
  \end{verbatim}
  \begin{longtable}{c|c|c}
    \hline 
    a & b & c\\
    \hline
    d & e & f\\
    \hline
    \caption{大型表格}
  \end{longtable}

  \subsection{浮動處理與文繞圖表}
  浮動 floating 是專業排版中的大圖表與圖表列表整理的重要處理,除了自動計算大
  圖表的距離空白等美觀與置放點外, 都會提供 caption label 選項,會自動幫你計
  算圖表數量,在書籍前後可以列出來參考。
  \\\\
  而文繞圖表很多在雜誌編排上會出現。

    \subsubsection{浮動處理}
    floating 的用法是要用
    \begin{itemize}
      \item
      \begin{verbatim}
\begin{figure}
  \caption{xx}\label{fig:myfig}
  \includegraphics{...}
\end{figure}
      \end{verbatim}
      \item
      \begin{verbatim}
\begin{table}
  \caption{xx}\label{tab:mytab}
  \begin{tabular}
    ...
  \end{tabular}
\end{table}
      \end{verbatim}
    \end{itemize}
    figure 使用範例

    \begin{figure}[h]
      \centering
      \caption{浮動圖形範例}
      \includegraphics[width=5cm,height=2cm]{images/riscv.png}
    \end{figure}

    table 使用範例

    \begin{table}[h]
      \centering
      \begin{tabular}{ccc}
        a & b & c \\
        \hline
        1 & 2 & 3
      \end{tabular}
      \caption{浮動表格範例}
    \end{table}

    table 的選項強制不要浮動選項,here 的選項
    \begin{itemize}
      \item \verb=\begin{figure}[h]=
      \item \verb=\begin{table}[h]=
    \end{itemize}
    不過這個 h 選項,很多時候是不盡人意,網路說法有改用 h! 或者 hb,或 hbt!,
    主要還是 table, figure 與文字還是要分開成不同 paragraph,可用空白行分開。
    但有的還是不行, 其實要用浮動 floating,則表示讓 figure 或 table 決定放置
    位置,所以如果真的需要 figure 的計算圖表數目與 caption 需求,然後又跑來跑
    去 的,可以不要用 figure, table 而單獨用
    \href{https://texdoc.org/serve/caption/0}{caption} package 的 
    \verb=\captionof{figure}= 與 \verb=\label= 來印出圖表編號。
    \begin{verbatim}
\usepackage{caption}

\begin{center}
  \includegraphics[scale=0.5]{van_gogh.jpg}
  \captionof{figure}{Image}
  \label{fig:VanGogh}
\end{center}
    \end{verbatim}
    也可單獨佔一頁 用 p 不用 h,但 p 通常會放在下一頁整頁,如果圖大小比例不對
    ,會有一大片空白就是。
    \\\\
    例如用上
    \begin{verbatim}
\begin{figure}[h]
  \caption{浮動圖形範例}
  \centering
  \includegraphics[width=5cm,height=2cm]{images/riscv.eps}
\end{figure}
  與
\begin{table}[h]
  \centering
  \caption{浮動表格範例}
  \begin{tabular}{ccc}
    a & b & c \\
    \hline
    1 & 2 & 3
  \end{tabular}
\end{table}
    \end{verbatim}
    即使沒有額外給 \verb=\label=,它還是自動幫忙做了 cross reference,
    Table 跟 Figure。這可以自己在table figure 環境裡面設定 label 讓自己在文章
    中用 ref pageref 來參照,只是 label 必須在 caption 之後使用。

    \subsubsection{文繞圖表}
    如果圖表是小小的只佔頁面寬度的某部份, 本文文字跟圖表大 object 看空間還夠
    不夠,放得下就放,這時圖表只是一個很大的字而已。 這時要對齊就要用 tabuler
    中的 tcb 對齊上中下選項。內定 tabular 是 center, 而 includegraphics 一律
    對齊 bottom 。這時並沒有 text-wrapping。
    \\\\
    例如這是小 table 
    \begin{tabular}{cc}
      title & item \\
      \hline\hline
      a & b
    \end{tabular}
    與後面文字。
    \\\\
    這是小圖 \includegraphics[width=5cm,height=5cm]{images/riscv} 跟它後面的
    文字。
    \\\\
    而之前用 includegraphics 與 tabular 時,為了好看都已經用上
    \verb=\begin{center}=,所以用 center 環境會佔滿整個頁面做 object,因此
    也就看不到文字與圖表並列排版與文繞圖表了。此時像 tabular 的可選擇參數的
    t,c,b 是沒有意義的。而 使用 table, figure floating 時,由於會有 caption,
    所以 caption 都是整頁中間文字,因此也沒有文字圖表這種對齊問題。
    \\\\
    如果想要圖表邊的空白也填滿文字,那就要文繞圖表,要用 
    \href{http://mirrors.ctan.org/macros/latex/contrib/wrapfig/wrapfig-doc.pdf}
    {wrapfig} package, 這個 wrapfig 也是浮動的,所以圖常常會跑來跑去的, 使用
    文繞表格用 wraptable
    \begin{verbatim}
\begin{wraptable}{r}{0pt}
  \begin{tabular}{cc}
  title & effect \\
  \hline\hline
  a & b \\
  \hline
  c & d \\
  \hline
  e & f 
  \end{tabular}
\end{wraptable}
    \end{verbatim}

\noindent
試著寫一大段文字,在圖表上下文旁邊看會長怎樣,如果長的醜,就不要用了,如果
長的漂亮,還值得學習這個 package ,不然就是浪費時間學習,閱讀,測試。

\begin{wraptable}{r}{0pt}
\centering
  \begin{tabular}{cc}
  title & effect \\
  \hline\hline
  a & b \\
  \hline
  c & d \\
  \hline
  e & f 
  \end{tabular}
  \caption{\label{wraptable}文繞圖表格}
\end{wraptable}

合格者には日本語能力認定書を送ります。また、日本国内での受験者全員に合否結果通
知書を送ります。 海外での受験者には2014年から合否結果通知書のかわりに証明書を
全員に送ります。 日本国内の場合、第1回(7月)試験の結果は9月上旬、第2回(12月)
試験の結果は2月上旬に送る予定です。 海外の場合は、受験地の試験実施機関を通じて
送りますので、 第1回(7月)試験の結果は10月上旬、第2回(12月)試験の結果は3月
上旬に受験者に届く予定です
\\\\

\noindent
繞圖形要用 wrapfigure
\begin{verbatim}
\begin{wrapfigure}[6]{l}{0.5\textwidth}
  \centering
  \includegraphics[width=0.48\textwidth]{images/riscv.eps}
\end{wrapfigure}
\end{verbatim}

\noindent
試著寫一大段文字,在圖表旁邊看會長怎樣,如果長的醜,就不要用了,如果長的漂亮,還
值得學習這個 package ,不然就是浪費時間學習,閱讀,測試。

\begin{wrapfigure}[6]{l}{0.5\textwidth}
  \includegraphics[width=0.48\textwidth]{images/riscv}
  \caption{\label{riscv}RISC-V Insturction Set format}
\end{wrapfigure}

合格者には日本語能力認定書を送ります。また、日本国内での受験者全員に合否結果通
知書を送ります。 海外での受験者には2014年から合否結果通知書のかわりに証明書を
全員に送ります。 日本国内の場合、第1回(7月)試験の結果は9月上旬、第2回(12月)
試験の結果は2月上旬に送る予定です。 海外の場合は、受験地の試験実施機関を通じて
送りますので、 第1回(7月)試験の結果は10月上旬、第2回(12月)試験の結果は3月
上旬に受験者に届く予定です
\\\\

選項
\begin{itemize}
  \item 第一個非必要選項是繞圖表文字行數,在圖表左右兩邊的行數,不是總行數。
  \item r, R, l, L,  位置選項,圖表在文字右邊左邊,大寫表示 float ,小寫表示
    就在原地。 所以通常用小寫。
  \item 圖表寬度。所以應該都是跟 includegraphics 或 tabular 的 p 設定差不多大
    小。但 wrap 的大小要稍微大一點,不然也可設 0pt。
\end{itemize}
測試的結果有點不如想的準確,主要還是字距與圖表大小設定的計算
\begin{itemize}
  \item 它是 floating 會自動計算。
  \item 不能用在 list 裡面。
  \item wrapfigure wraptable 前後要有字,也要空白一行,也就是必須是獨立的
    paragraph 才會動作。
  \item 基本上是 wrapfig 後面 paragraph 的文字才會繞。
  \item 也有看到有說第一個選項也要設定不然不會動或者走鐘去。
\end{itemize}

\section{排版注意事項}
有些排版特有的觀念與注意事項是我們之前一直在提醒的,另外還有的是 {\LaTeX} 處理
時的一些特別行為。

  \subsection{出版商的專業術語}
  \begin{itemize}
    \item break line 強迫斷行使用 \verb=\\ 或 \newline=, \verb=\\[2cm]=
      表示斷行後加個空白行兩公分。然後 每次的斷行不管自動或強迫的都會在兩行間補
      一個 空白。 但不是那麼剛好每個字的寬度排起來後可以填滿一行,因此在每行最
      後有可能英文字會被拆成兩半, 這也是有規矩的,英文不像中文是方塊字,而是
      會用 hyphen rule 把一個 word 拆兩半。
    \item space 空白,除了在換行時會加空白,在句子與句子中間,應該要多一點空白
      ,這是排版軟體看到標點符號 \verb= . ? :=, %stopzone
      自動幫我們做的。但有時一些縮寫像 \verb=etc. i.e Mr.=,這種就要強迫它不特
      別空白, 可以用 \verb=~=, 或者\verb=\ =,兩種都可以表示正常空白。而不同
      字體間的空白大小其實也都有嚴格規定只是排版軟體幫我們做掉而已。
    \item indent 縮排是預設每個章節的第一個段落的第一行是不縮排, 從第二個段落
      開始才會內縮。所謂的段落除了用\verb=\paragraph{}= 與 \verb=\par= 外,在
      文字編輯器上多加一個空白行就是新段落,因此比寫程式還嚴格。如果不要縮排就
      在空白行前加上 \verb=\noindent=,或者全域的設定 \verb|\parindent=0pt|。
    \item 引號 跟 shell 一樣,backquote, single quote, double quote 用法在排版
      界是有不同詮釋的,所有的引號都是需要有左右夾起來的。
      \begin{itemize}
        \item single quote 單引號必須用真的 backquote + 1 個真的 single quote。
        \item double quote 雙引號反而必須用兩個真的 backquote + 兩個真的 single
          quote
      \end{itemize}
      也就是原本鍵盤上的 backquote 變成左邊倒勾的引號符號,鍵盤上 single quote 
      變成右邊正勾的符號。 如果真想用原本的電腦上看到的樣子,只能用verbatim。
      使用 \verb=\verb 或者 begin{verbatim}= 也有效。\\\\
      `倒引號`, `排版單引號', '錯誤單引號', ``排版雙引號'', "錯誤雙引號"\\\\
      用 verbatim 原型變成\\\\
      \verb= `倒引號`=, \verb=`排版單引號'=, \verb='錯誤單引號'=,
      \verb=``排版雙引號''=, \verb="錯誤雙引號"=\\\\
    \item 三點 \dots 要用 \verb=\dots 或者 \ldots= ,不要直接用鍵盤的 ... 。
    \item 破折號與減號在英文排版上。 正式有分 hyphen, en-dash, em-dash 三種。
      hyphen 長度最短,en-dash 大概是 N 這個字長度, em-dash 是 M 這個字長度。
      hyphen 用來表示一個單一字彼此有關係,例如 son-in-law 或者在斷行時表示前
      後字元是一個字。en-dash 表示一段範圍,例如 1 -- 10, em-dash 表示句子的分
      開額外的意義說明,例如花好香 --- 帶有淡淡的牛奶香。
      \begin{itemize}
        \item Hyphen: 使用 \verb=-=
        \item En-dash: 使用 \verb=--= 或者 \verb= \textendash=
        \item Em-dash: 使用 \verb=---= 或者 \verb=\textemdash=
      \end{itemize}
      負號則須用數學的\verb=$-$=。
  \end{itemize}
  另外有引言的 quote 表現用 \verb=\begin{quote} ... \end{quote}=
  \begin{quote}
  引言,子曰三日不讀書,言語無味面目可憎。
  \end{quote}
  總之輸入的換行 newline,空白 space,空白行 empty line,轉成排版時有特殊轉換,
  而輸入的多個連續空白就只有一個空白,多個連續空白行就只有一個空白行, 還有鍵
  盤上的符號在正規嚴謹的排版要求下,是會長成不同樣子的,因此要看你要求嚴不嚴
  格,用排版軟體的特殊命令,用脫逃字元,用 verbatim,與單用鍵盤打出字元長相到
  最後排版都會不同,所以要求一致性長相時,要注意使用。

  \subsection{\TeX{}/\LaTeX{} 注意事項}
  \begin{itemize}
    \item table of content 要跑兩次引擎,第一次跑出 .toc 檔,才能產出漂亮的
      TOC。 cross reference label ref 也要。
    \item index 產生還要在兩次中間跑 makeindex 程式。
    \item 命令是從反斜線後第一個字母開始,到第一個非字母符號為止(包括空白、標
      點符號及數字)。因此:\verb={\LaTeX}= 與 \verb=\LaTeX{}= 與 \verb=\LaTeX=
      ,作用正常講會不一樣,也就是 \verb=\comand= 與\verb=\command   = 正常講
      是兩個不同命令。只是後面 macro 程式幫你作掉變一樣而已。排版時要小心尤其
      是 logo 形式的命令,還是要用\verb={\logo}=形式,不然會 logo 跟字間沒有空
      白。
    \item 註解符號(\verb=%=),可以放在一行的任何地方,\verb=%= 後的文字會被
      {\LaTeX} 忽略。 所以,如果是放在一行的最尾端,那麼 {\LaTeX} 會自動插入的
      斷行空白也將會被忽略, 有的 macro 是不准有空白跟著的,所以要寫多行的
      macro 也常看到在行尾是有註解符號的。另外在 vim 中的語法顏色有的會出錯,
      也是用 \verb=%=stopzone 想辦法把它改回來。
    \item \verb=\\= 斷行要小心,paragraph, end 環境等後面不能直接跟著它,會出
      錯。
    \item 特殊字元脫逃,\TeX{}/\LaTeX{} 是程式語言,有保留字等特殊意義,例如
      \begin{itemize}
        \item \verb=&= table 的對齊
        \item \verb=%= comment
        \item \verb=$= 數學 inline
        \item \verb=#= macro 的參數表示法
        \item \verb=^= 上標
        \item \verb=_= 下標
        \item \verb={= grouping 的開始
        \item \verb=}= grouping 的結束
        \item \verb=~= active 字元, 是一個單一字元在其他命令還沒執行前,就被
          展開的 macro,這有點像 shell script 的 expr 命令作用。
        \item \verb=\= 命令開始
      \end{itemize}
      要脫逃可以用
      \begin{itemize}
        \item 多加反斜線
        \item verbatim 環境或者 \verb=\verb=
        \item 數學環境
      \end{itemize}
  \end{itemize}
  除了上面那幾個外,+ - = * / . , : ; ( ) ! [ ] | < > ?  不用怕, 但單引號,雙
  引號等有之前說的 quote 定義,還有像連續兩三個 \verb=--=,會變成特殊破折號外
  ,也就還好,有些命令也可以印出特殊字元
  \begin{center}
    \begin{tabular}{c|c}
    \hline
    \verb=\textgreater= & \textgreater\\
    \hline
    \verb=\textbar= & \textbar\\
    \hline
    \verb=\textasciitilde= & \textasciitilde\\
    \hline
    \verb=\textasciicircum= & \textasciicircum\\
    \hline
    \verb=\textbackslash= & \textbackslash\\
    \hline
  \end{tabular}
  \end{center}
  只是其中有些反斜線後面加上符號又有其他意義,變成是一種命令,要小心,例如
  \begin{center}
    \begin{tabular}{c|c}
      \hline
      \verb=\.{u}= & \.{u} \\
      \hline
      \verb=\`{u}= & \`{u} \\
      \hline
      \verb=\"{u}= & \"{u} \\
      \hline
      \verb=\'{u}= & \'{u} \\
      \hline
      \verb=\\= & 換行\\
      \hline
    \end{tabular}
  \end{center}
  其中 \{ \} 其實可以是命令中的參數,或者其實是 grouping,有的是可以不給的有內
  定值的,使用
\begin{verbatim}
{ word }
\end{verbatim}
  跟
\begin{verbatim}
{       word }
\end{verbatim}
  是不一樣的
  \\\\
  所以結論是
  \begin{itemize}
    \item 不是所有反斜線後面都能亂加東西,就變成 esacpe,甚至會有奇怪效果的,
      反斜線在 \TeX{} 是程式要有處理才是 escape。
    \item 有的符號後面的參數可以是空白,所以像 \verb=\~{}=,是可以偷的,可是
      有的參數當時定義時就不准有空白,像\verb=\"=,就不行。
    \item 破折號用 \verb=--=,所以真要用時,可以用 \verb=-{}-=,這裡的
      \verb={}=,是等於 grouping 的意思,例如 \verb={\em text}= ,表示 em 到
      text 整個 grouping,所以如果裡面沒東西,沒問題,但卻隔開了連續兩個 -,
      產生\verb=--= 效果。
    \item 如果要真的像 terminal 螢幕看到的平常樣子, 用 verbatim 是比較好的。
      不然用反斜線脫逃會看到有點奇怪的樣子,例如百分比符號,還會有點大大的。
  \end{itemize}

\subsection{中文注意事項}
\begin{itemize}
\item 使用 \% 在尾巴消除一些自動英文空白。以前如果沒有加,就會每個中文字斷行地方
看起來多一個空白,怪怪的。新的 xeCJK package 做的好一點。
所以已經不需要像以前在屁股尾巴加上 \%,但這還是在很多地方很好用的技巧。
\item 斷行,空白,符號的使用規矩跟英文的會有點不一樣。但這也要看期刊,
單位的要求為何,或許有 {\LaTeX} 的特殊版本使用。
\end{itemize}

\subsection{其他}
vim 中打開 :syntax on,開 tex 檔會自動有基本語法顏色,可是用上特別的
 package 後,會走鐘掉。最基本的修正就是在亂掉地方加上註解 
\verb=%stopzone=。 但 package 太複雜跟其他的 macro 有衝突。例如 lstlistings
的 vim 的配合,用 \verb=%stopzone= 與多加一個檔案
\verb=$HOME/.vim/after/syntax/tex/lstlistings.vim=
\begin{footnotesize}
\begin{verbatim}
syn region texZone start="\\begin{lstlisting}" end="\\end{lstlisting}\|%stopzone\>"
syn region texZone  start="\\lstinputlisting" end="{\s*[a-zA-Z/.0-9_^]\+\s*}"
syn match texInputFile "\\lstinline\s*\(\[.*\]\)\={.\{-}}" contains=texStatement,texInputCurlies,texInputFileOpt
\end{verbatim}
\end{footnotesize}
