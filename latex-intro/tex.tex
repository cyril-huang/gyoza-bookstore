\chapter{{\TeX} 與 {\LaTeX} 介紹}
李果正先生的
\href{https://github.com/TeXtw/LaTeX123/releases/download/v1.0/latex123.pdf}
{大家來學 \LaTeX}
是很好的起點文章,當然所謂專業排版一些吹毛求疵的要求,空間距離的美感等等都要
再更進一步的了解才做得到。跟當年有些不同的是
\begin{itemize}
\item Unicode 已經普遍了,不像當年還在 Big5 轉換 Unicode 的時期。
\item 不同字型標準的支援更完整了。點陣字,type 1, ttf, open type 等都可以用。
\item double bytes 的支援一般性已經很成熟, 雖然還是有個別漂亮的支援%
macro 可以選擇,但內定標準的中日韓越東亞字處理已經很可以。
\end{itemize}
因此忘了以前那些奇怪的東西吧,現在就用標準工具就可以了。
另外其他作業系統平台的 \TeX 使用,我是沒有什麼經驗的,我既然有 Linux 了,所有資
源 packages 我都能在 Linux/vim 上完成,我就沒必要去玩其他的了,畢竟我的目的是
整理我所學,用我的語言彙整一遍以達教學相長,並留下自己參考的紀錄,所以我的文章
也都不是很嚴謹,很多規定排版也都很隨意沒有嚴格遵守。
\section{基本術語回顧}
總結一下, \TeX 是
\begin{itemize}
\item 一種程式語言,用來表達文字排版 (text formating, typesetting) 的各種可能。
例如對齊,字型,時間,counter (可用來算頁數等等) 等等。因為它是程式語言,
所以有 if else 這些語法,
又或者例如 \verb=\def= ,用來定義新函數的命令。 把它想成某種 script 也可以,
而引擎就是在解讀 \TeX 這種 script 的程式。因此它也必須像編譯或像 python, perl
一樣的處理 tex 檔案。
\item 是 stanford 的 computer science 老師 Donald Knuth 發表的程式。這起源於
  最早鼎鼎大名的Knuth教授不滿意他的數學示子被書商排的很醜,最早的排版都是使用
  傳統排字機(typsetting machine),但是他的書第一版跟第二版相隔了8年,打字機已
  經不一樣了,印出的品質不滿意下, 乾脆就自己開發了這個數位幕後排版語言。
  TeX 最早也只能輸出在固定品牌 xerox 的排字機上,後來就輸出成一個 dvi 檔格式
  ,從這個抽象排版的中介格式能再透過 driver 轉換世界其他的印表機,排字機,pdf
  ,ps, html, rft ... 等不同輸出。因此轉換過程中產出很多檔案,例如
  \begin{itemize}
    \item dvi device indepedent 的抽象檔,xdvi (extend dvi)。
    \item toc table of contents 目錄
    \item log 程式執行發生的狀況,錯誤時找原因的檔。
    \item idx index 索引檔。
    \item aux auxiliary 檔是包含所有參照資訊,在書中像很多圖,書後面索引等。
    \item lof list of figure 所有圖形資訊。
    \item olt list of table 所有表格資訊。
    \item ...
  \end{itemize}
  所有的程式與中介都是抽象再抽象的分類與涵蓋,不管你寫 C ,perl, python, \TeX{}
  ,舉例就是小時學的生物分類,界門綱目科屬種,動物,植物,胎生,卵生...的分類,
  每一種分類都會爭標準,例如圖書館分類學中,有杜威分類,美國國會分類等,
  排版寫文章一樣有分類,只是一般人或小孩子不知道規矩與歸納整理的分類,因此會產
  生上述檔案,dvi 只是一種中介分類標準,如果微軟的 doc 開放格式給大家知道,那
  他也可以來爭中介標準讓大家從這個標準下產生其他的輸出格式。 不過很多檔案先不
  用管他。
\item 版本號碼是 3.141592653....
\item 最根本的 primitive \TeX 總共有 325 control sequences,也可叫它 macro 或者
 command。使用
\begin{verbatim}
\command[optional parameter]{mandatory parameter}
\end{verbatim}
這樣形式。
\item literate programming 與 WEB 語言: 編譯程式其實是用 Pascal 寫的,但 Knuth
先生不滿意程式寫作語法而自創 WEB 程式系統,literate programming 意思就是像
寫作文那樣的寫出程式,而要達到這樣的語意分析,就會在程式語言解讀上動手腳,
舉例像物件導向程式中, 1 + 1 = 2, "string 1" + "string 2",為了能讓
數字相加跟字串相加有相同意義,就要在編譯程式解釋這些語意上加上很多判斷邏輯,
這增加的是寫編譯器的人的負擔,但盡量減少使用者的痛苦,尤其是什麼都不懂的初學者,
literate programming 比這更大膽的想法就是用人類自然語言就能寫出程式。WEB 還是
用 Pascal 的,只是像使用 script 解譯器,巨集等更多語意的解讀。在很多地方會看到
web 的工具,不要跟現在什麼 html web programming 搞混了。其實一堆的程式語言產生
很多部份就是在爭辯哪個語意比較好,這就是為什麼說真正要寫好程式還是
必須從 C 來學,才有真正了解系統與解決問題能力。搞這些就跟咬文嚼字一樣爭論不休。
\end{itemize}
其他術語
\begin{description}
\item [Macro] 或者有人叫 {\TeX} command 類似子函數呼叫的巨集。也就是基於一些
基本的 \TeX 工具呼叫又更高的功能呼叫。
像 C 的函數厙,printf 並不是 C 的基本功能,它只是基於 write() 的螢幕高功能呼叫。
在最早 Knuth 老師也有 release 他自己的 macro ,叫 plain \TeX 大約有 600 個新命令。
\item [package] 眾多 macro 集合的檔案就是 package,把它想成類似 C 的 library 檔。
例如美國數學協會用來寫數學式子,或者論文形式 2 columns 的排版效果。package 檔
以 .sty 為副檔名。
\item [engine] 所謂的引擎就是某種輸入經由變換,產出另一種輸出。像汽油燒一燒,
轉成車子動能輸出。 根據 \TeX 語言規則做輸入分析,輸出轉換的軟體就是 \TeX 引擎。
引擎有 HONDA,TOYOTA,同樣 \TeX 引擎也有多種實做。或者類似 C 編譯器,C 語言引擎
能解讀 C 檔案的語法,轉成某種作業系統的執行檔,有 ms visual c, gnu c, 以前有
borland c。目前{\TeX} 引擎實做有
  \begin{itemize}
  \item plain \TeX 最早由 Knuth 發表的引擎程式。輸入處理 7 bits,輸出原本
  只能輸出到特定打字機格式,但後來有加上所謂 dvi 格式(device independent),
  再由 dvi 再轉出可能的各種打字機或者 postscript, pdf, opendoc 等等不同格式。
  \item pdftex 後來大部分人使用的程式,但只能處理 8 bit,它跟 plain \TeX 最大
        差別在於它能直接輸出 pdf 檔,很多範例使用的 shell 命令 pdflatex 就是
        這引擎下的一個工具。
  \item luatex 除了能直接使用 unicode 輸入,最重要是可以使用 LUA 這個程式語言
	進一步處理的引擎。由於 \TeX
        當初在發展時就想到世界上不可能完全一樣邏輯處理不同語言,所以很多分類抽象化
	高階處理最後都是用 callback, hooking 呼叫,例如字型處理時,有這麼多
        種字型,ttf, opentype, bitmap 等,真正處理一定是呼叫相對應的函數,這些
        函數不是寫死的,而是動態可以設定將來呼叫 3rd party 的新函數來處理。
        \href{https://en.wikipedia.org/wiki/Lua_(programming_language)}{lua 語言}
        是類似一般 script 的語言。不過很多人也用它再去叫 python, perl,來當做
        處理的 hooking。也就是它下放新函數處理給一般使用者,不再侷限只是當初寫
        引擎的程式師而已。
  \item xetex 新的能直接處理 unicode 輸入的 tex 檔案引擎。輸出直接產生新的
        xdvi (extend dvi 格式) 與 pdf 格式。
  %stopzone
  \end{itemize}
\item [distribution] 跟 Linux 一樣意思,整個系統很龐大,macro 眾多,
這麼多 macro package 有相依性, 要怎樣參照,
怎樣收集,怎樣改版,升級,發布等等,目前很大的 {\TeX} distro 為 texlive。收集了
很多功能的 package ,很多的 sty 檔,還有各種工具程式,包括文件閱讀等等,這些檔
案寫作也有一定規矩還有檔案分佈也有 {\TeX} 組織所建議的 TeX Directory Structure.
\item [TUG] \href{http://tug.org}{{\TeX} users group} ,
是目前最大的 community ,主要發展討論還有標準的探討建立。
\item [CTAN] \href{http://www.ctan.org}{The Comprehensive TeX Archive Network}, 
是所有收集想到的 macro 工具,引擎等等網站。
\end{description}
過去中文環境或各國大學也都有出自己的 macro 收集,有些 macro 是為了解決亞洲語言
不是 8 bits 的問題等等,因此形成非常混亂的名詞,一堆 xxxTeX,
但說穿了,他們都只是 macro 的創造者與收集者,過去真正引擎還是 pdftex 用最多。
當然也有商用的引擎與 macro 收集(我猜還是很多是改 opensource 過來的)。
就把它想成一堆不同的 Linux 發行版本。
\\\\
基本上可以用 
\href{https://en.wikibooks.org/wiki/TeX}{primitive \TeX},
plain \TeX 跟一些 macro 就能完成排版的。可以參考
\href{https://ctan.math.utah.edu/ctan/tex-archive/info/impatient/book.pdf}{{\TeX} for the impatient}
,這大概有400頁的英文,先不要看,等將來有需要再回頭看怎麼寫自己 macro。
\\\\
而 \LaTeX
\begin{itemize}
\item 是專門用來寫作文章的 macro ,基於基本的 \TeX 與 macro 創造出的新 macro。
 例如 \verb=\chapter=, \verb=\section=,\verb=\begin{itemize} \item ... = 
 等用於文章邏輯結構的巨集命令。
\item 特別的 environment ,使用 \verb=\begin{xxx} ... \end{xxx}=
\item 命令形式有三種,standard command, group switch, environment,
並且提供後來的 macro 使用開發新 macro 。
\item 除了 \LaTeX 外,另外有一組寫文章的叫 ConTeXt 的 macro ,也在 TUG 的維護。
\item 目前版本為 \LaTeX 2e 在 1994 取代 \LaTeX 2.09。\LaTeX 3 在醞釀中。
\end{itemize}
\TeX 的原本命令中,也能做出像列舉 list ,原封不動的 verbatim 等元素,也有
所謂的 group 與可命名的 group 稱為 environment,environment 是表示這命令
的設定作用在另一群相關的命令。 而且是可以一個 environment 裡面包另一個 
environment nest 下去的, 但都要寫一堆複雜東西才得到效果,舉例像 \LaTeX 列舉
\begin{verbatim}
\begin{itemize}
\item item 1
  \begin{itemize}
  \item 1.1
  \item 1.2
  \end{itemize}
\item item 2
\end{itemize}
\end{verbatim}
itemize 是一個環境的名字,然後裡面有 \verb=\item= 這個命令,作用直到 end 結束
,而 item 這個命令也只在 itemize 這環境中有意義。但以往光用 \TeX 時不是這麼簡單。
\LaTeX 提供更 general 的 這樣 environment 語法,在後面寫 macro 中,
給後來者用這種方法加上新的 environment 的好方法。
\\\\
一般命令形式有三種形式
\begin{itemize}
\item standard \verb=\cmd[可選擇參數]{一定要給的參數}=,命令會影響整個後面文章。
\item group (switch) \verb={\cmd }=,包在括號裡面的樣子。命令只作用在括號中。
\item environment \verb=\begin{env} ... \end{env}=,用 begin, end 包起來。
\end{itemize}
例子
\begin{verbatim}
{\fontsize{50}{60} \selectfont \LaTeX}
\begin{center}
\begin{Huge}
Huge and Centralized Text !!
\end{Huge}
\begin{itemize}
\item item 1 is inside itemize environment
\end{itemize}
\end{center}
\end{verbatim}
最後命令與括號中如果有星星 * ,表示他有特別處理。每種命令有不同意義,要根據
文件說明才知道。例如
\begin{verbatim}
\section*{no section numbering}
\end{verbatim}表示這個 section 不會參與編號,在目錄上就看不出 1.1.3 \dots
這樣編號。而像 
\verb=\begin{tabular*}= 表示這個表格寬度會自動使用到整個頁面寬度。
\begin{verbatim}
\includegraphics*{my.eps}
\end{verbatim}表示會裁切掉圖片超出大小的部份。

\section{字型}
字型檔跟圖檔一樣,也有很多種標準,用來描述字型的方法有很多種,
但能任意放大縮小不會變醜變形還是要從數學來,數學方法中
主要是用曲線控制點方程式貝茲曲線 Bézier curves,
\begin{description}
\item [bitmap] 最笨的一點一點畫點方法的字型檔。
\item [\MF{}] Knuth 先生當年創立的向量字型描述。看到 xxxmf 的就是關於字型。
\MF{} 也是一種程式語言,由於背後數學曲線函數在那,所以也可拿來畫圖,
雖然後來 METAPOST and Asymptote 是比較推薦的畫數學圖形工具。
\item [postscript type 1] 由 adobe 公司在1985年制定的字型描述規格,副檔名是 pfb。
  這也是傳統 pdftex 引擎的使用字型。
\item [true type ttf] 由蘋果公司稍早制定的字型描述規格。檔名可以是 ttf 或者 ttc
,truetype collection。
\item [open type otf] 由microsoft adobe 在 2000 年左右制定的 truetype 後繼規格。
\item [TeX font metric] tfm 是一種字型檔案格式,跟其他向量字檔 pfb, ttf, ttc
等不同,ttf, ttc 這些檔有圖形資料,叫 outline 字檔,font metric
只提供寬度,高度,深度等需要丈量的資訊,並不提供實際的圖形,實際的 glyphs
是存在 outline 的。metric font 是排版一定需要的,不然無法計算距離空白等等,adobe
的排版 font metric 格式叫 afm 。Windows 的叫 pfm。因此會有工具
從字型檔擷取出 font metric,然後會有一個 map 檔來 map manufacturer name 跟
font name 在兩者間關係。一種字型可能是好多個 outline 檔組成,所以 map 檔內容
就是很多 tfm 檔跟 pfb 的對應。
\item [font definition] fd 檔,這裡面是 font 的文字定義檔,一個字型,有名字
有編碼,有分類等等,由於 \TeX 能處理多種字型,所以必須把不同字型的資訊做
統整統一的抽象化。 fd 檔是 \TeX 程式存取的資訊檔。 舊的引擎設定字型名,必須
從這個 fd 檔來。一個 .fd 檔列出了一個 encode 一個 family 所有可能的 fonts
,例如 ot1ptm.fd 紀錄了所有屬於ptm (Times Roman) 與 OT1 編碼的字型。
\end{description}
type 1 跟 truetype 在當年算是競爭對手,只是當年 Microsoft 在 Windows 選擇了
Truetype ,type 1 其實用 cubic Bézier curves, 用了四個控制點來畫曲線,truetype
只用三個控制點的 quadratic curves, 主要是當初 Windows 目標是螢幕畫圖,需要
快速運算又作用在低解析度的螢幕上就好,type 1 目標是印表機高解析度的紙張上。
opentype 檔案裡面其實是可以包含 type 1 ,truetype 的資訊的,所以 opentype
裡面如果其實是 type 1 或 truetype 字型, 那麼優缺點跟原本的是一樣的。
\\\\
引擎中 luatex, xetex 都能處理以上字型,並且處理程式會自己找一些內定
目錄跟視窗系統,不管 是 Microsoft, X (包含蘋果跟類Unix系統)的字型使用。
\\\\
但是 pdftex 有兩件事是比較落伍的
\begin{enumerate}
\item 需要在處理前,所有的 tfm 檔與 map 檔都建造完成,不像新的 luatex
與 xetex 是能一邊抓到字型檔,一邊建造 tfm, map 檔,因此 pdflatex 的字型引用
很奇怪,必須是 fd 檔裡面設定的字型名。
\item 它一次每個 tfm 只能處理 256 個 glyphs 外型,由於中文字很多,所以
要多建立不同 tfm 檔,所以在 fonts/type1 下看到中文字型的 pfb 檔很多個。
\end{enumerate}
如果沒有建立好 tfm 檔,則 pdflatex 是無法使用這個字型的,也因此能用的字
比較受限,除非自己很懂,用工具創造出 distro 外的新字型。
\\\\
字型跟設定有關的attribute,這些資訊也是抽象化的資料型態。
\begin{description}
\item [encode] 就是這字型編碼,最早由於電腦資源寶貴,很多標準都是考慮英
文字母而已,因此最早的 \MF{} 字型編碼是 7bits 的 OT1 編碼,這是 Knuth
 先生的發明, 這也導致原本的 \TeX 引擎其實是只能處理 7bits 的,當然
現在盡可能都不用了。
\item [family] 或叫 typeface, 有三大種 roman, sans serif, teletype(typewriter typeface)
\item [series] medium , bold 粗體
\item [shape] italic 義大利斜體, slant 斜體, small caps (sc)
\item [size] 就是大小,要注意的應該是大小所用單位。
\end{description}
字型 family (typeface) 有成千上萬例如 Times, Courier, Helvetica ...,但會
落在三大分類, serif, Sans serif 跟 monospaced ,{\LaTeX} 用 roman (rm),
sans serif (sf), 還有 teletype (tt) 來分別表示。看到 serif 就是羅馬字,
sans serif 特點是筆畫厚度均勻,
而 tt 就是字與字間隔固定的字型,用在以往打字機與現代 terminal。
\\\\
\LaTeX 內定是用一個叫 Computer Modern 的字型,是 Knuth 老師設計的,也包含上面
三種分類。看到 cmxxxx 什麼的就是關於它的字型。
\LaTeX 內文設定用 Roman(serif) 字,羅馬字規定在很多排版術語中都是內定字,
或者電腦圖形界面的內定使用字型,網頁版面內定字型等等。
\\\\
中文的宋體、黑體、仿宋體和楷體為漢字印刷的主要四種字體,歷史上宋體是
中國明代木版印刷中出現的字體,原形為宋代模仿楷書基本筆劃,但由於木板很硬,
結果變成不是楷書的直劃新字體。傳進日本稱為明朝體,由日本傳回台灣叫明體,
中國大陸一直叫宋體\footnote{康熙時,官方規定叫做宋體} ,
台灣則兩者都有人稱呼。明體為正統排版內文使用,為
roman (serif),在 google noto 字型,或者 adobe 的字型中,分別為
宋體(rm/serif) 黑體(sf/sans serif)。
\\\\
NFSS New Font Selection Scheme 是在 \LaTeX 2e 版本出現的選字,
plain \TeX 並無法同時選用斜體與粗體,當用\verb=\bf\it= 時,只得到斜體
當用\verb=\it\bf=時只得到粗體,為了解決這問題,\LaTeX 2e 提出了
新的選字 macro 。
