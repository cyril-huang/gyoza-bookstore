\chapter{進階使用}
上面的基本介紹應該足夠知識了解粗淺的{\LaTeX} 系統,但還是有很多細微龜毛調整是
要知道的,尤其前面完全不管一些粗細長寬等設定,只專注在文章邏輯結構,章節目錄
等等,可是它的長相都是 {\LaTeX} 內定的,當想讓本文寬度更大些就要來改變內定設
定。
\\\\
要先了知的是整個版面的相關術語與知識,然後其實電子排版就是在一張虛擬畫布上,
把物件擺來擺去,所以設定各物件空白的命令 macro 很重要。
\\\\
最後要來知道 macro 是怎麼寫出來,將來我們要自己增加功能, 用 \TeX{} 原始命令
與 \LaTeX{} 新macro 增加自己的新命令,並且希望能提供 package 使用。

\section{參數設定}
包括一些內定counter,點線面的設定。

  \subsection{幾何設定}
  排版對於字型的定義是很嚴格的,以下為一個字的定義
  \begin{center}
    \includegraphics[width=0.5\textwidth,keepaspectratio]{images/sphinx.png}
  \end{center}
  大寫字一律要站在 baseline,很多的對齊參考基準都是根據這些定義而來丈量的。
  \\\\
  設定直線或黑盒子,直線就是一塊細長的黑盒子
  \begin{center}
    \begin{tabular}{lll}
      命令 & 效果 & 意義 \\
      \hline
      \verb=\rule[-1mm]{5mm}{3mm}= & \rule[-1mm]{5mm}{3mm} Sphinx & 
      從 baseline 下1mm 升起 5mm 寬 3mm 厚的黑盒子\\
      \verb={\color{red}\rule{5mm}{.1pt}}= &
      {\color{red}\rule{5mm}{.1pt} Sphinx} &
      從 baseline 升起 5mm 寬 .1pt 厚的紅盒子\\
    \end{tabular}
  \end{center}
  設定文字盒子
  \begin{center}
    \begin{tabular}{p{5.5cm}m{4cm}m{4cm}}
      命令 & 效果 & 意義 \\
      \hline
      \verb=\parbox{4cm}{=paragraph 盒子
      如果文字長度太長,才會自動折行 \verb=}= &

      \parbox{4cm}{paragraph 盒子
      如果文字長度太長,才會自動折行} & 段落盒子 \\

      \verb=\mbox{mbox}= & \mbox{mbox} & 無框盒子 \\

      \verb=\makebox[3cm][s]{make a box}= &
      \makebox[3cm][s]{make a box} & 寬 3cm,文字 spread 盒子 \\

      \verb=\fbox{fbox}= & \fbox{fbox} & 有框盒子 \\

      \verb=\framebox[3cm][r]{frame a box}= &
      \framebox[3cm][r]{frame a box} & 寬3cm,文字從右邊寫起 \\
    \end{tabular}
  \end{center}
  其中 parbox, makebox, framebox 才有對齊選項,也是文字物件對齊折行的主要底層
  物件
  \begin{verbatim}
\parbox[<alignment>][<height>][<inner arrangement>]{<width>}{<text>}
\makebox[<widht>][<alignment>]{<text>}
  \end{verbatim}
  inner arrangement 是高度的垂直對齊, 對齊除了過去 table 的水平對齊 l, c, r, 
  垂直對齊 t, c, b 外,有個新的 s 能做到整行分佈對齊。 除了這些 box 命令外,
  也可以使用 minipage 這個 environment,創造一小段 parbox 的效果。
  \begin{verbatim}
\begin{minipage}[b]{3cm}
測試小盒子的 minipage 環境設定,使他超過三公分的大小看會怎樣!
\end{minipage}
  \end{verbatim}
  \begin{minipage}[b]{3cm}
測試小盒子的 minipage 環境設定,使他超過三公分的大小看會怎樣!
  \end{minipage}
  minipage 環境有很多選項,通式為
  \begin{verbatim}
\begin{minipage}[position][height][inner-pos]{width}
  contents
\end{minipage}
  \end{verbatim}
  position 一樣是 c, t, b 跟外面文字對齊 center, top, bottom,內定是 center
  ,而 inner-pos 則是內部文字對齊,一樣是 c, t, b 跟 s,height 則是盒子高度
  ,但通常不必設。 由於這是一個盒子,所以把 tabular, figure 放進盒子,可以用
  來做文繞盒子的方式產生文繞圖表的效果。
  \\\\
  設定某參數長度在 preamble 區使用 setlength,它會影響全部,可以多使用 calc
  package,可以在長度裡面直接做加減乘除,例如
  \begin{verbatim}
\documentclass{book}

\usepackage{calc}

\setlength{\hoffset}{0mm}
\setlength{\oddsidemargin}{20mm-1in}
\setlength{\paperwidth}{210mm}
\setlength{\textwidth}{\paperwidth-40mm}
\begin{document}
...
\end{document}
  \end{verbatim}
  {\LaTeX}設新長度名字方法,其中可以用一群文字來把
  新變數的寬度高度深度設成後面文字的寬度高度深度。
  \begin{verbatim}
\newlength{\mylength}
\setlength{\mylength}{100cm}
\settowidth{\mylength}{大頭 大頭你好長}
\settoheight{\mylength}{大雄 大雄你好高}
\settodepth{\mylength}{大家 大家你好深}
  \end{verbatim}
  原本 {\TeX} 的方法
  \begin{verbatim}
\newdimen\mylength
mylength=1.5pt
\end{verbatim}
  印出長度用the 會印出。
  \begin{center}
    \begin{tabular}{ccc}
      命令 & 效果 & 意義\\
      \hline
      \verb=\the\textwidth= & \the\textwidth & 文字區域寬度 \\
      \verb=\the\paperwidth= & \the\paperwidth & 紙張真正寬度 \\
      \verb=\the\paperheight= & \the\paperheight & 紙張真正高度 \\
      \verb=\the\tabcolsep= & \the\tabcolsep & 表格內欄位間距離 \\
    \end{tabular}
  \end{center}
  這裡面的單位除了簡單的公英制單位外
  \begin{itemize}
    \item pt 前面說過大約是0.35mm
    \item em 這是目前使用字型M的寬度
    \item ex 這是目前使用字型x的高度
    \item mu math unit,大約是 1/18 的 em
  \end{itemize}
  在網路上看到有大學老師在教 Word, 不管怎樣,所需的表格對齊用 \LaTeX{} 只要用
  tabular 加上上述設定就完成,例子如下
  \begin{verbatim}
\begin{center}
\begin{tabular}{
    p{4em}@{ : }@{\rule{10em}{.1pt}}
    @{\hspace{1em}}
    p{4em}@{ : }@{\rule{10em}{.1pt}}
  }
  \makebox[4em][s]{甲方} & \makebox[4em][s]{乙方}\\
  \makebox[4em][s]{電話} & \makebox[4em][s]{電話}\\
  \makebox[4em][s]{代表人} & \makebox[4em][s]{代表人}\\
  \makebox[4em][s]{聯絡地址} & \makebox[4em][s]{聯絡地址}\\
\end{tabular}
\end{center}
  \end{verbatim}
  \begin{center}
  \begin{tabular}{
      p{4em}@{ : }@{\rule{10em}{.1pt}}
      @{\hspace{1em}}
      p{4em}@{ : }@{\rule{10em}{.1pt}}
    }
    \makebox[4em][s]{甲方} & \makebox[4em][s]{乙方}\\
    \makebox[4em][s]{電話} & \makebox[4em][s]{電話}\\
    \makebox[4em][s]{代表人} & \makebox[4em][s]{代表人}\\
    \makebox[4em][s]{聯絡地址} & \makebox[4em][s]{聯絡地址}\\
  \end{tabular}
  \end{center}
  也就是說用 array, tabular, parbox, minipage 等手段交互運用,可以創造出很
  多對齊的方式。

  \subsection{版面設定}
  在 documentclass 一開始,通常會設定一個紙張大小, a4, letter, b5 ... 等等一些
  預定的紙張大小標準, layout 是整個版面的定義,長寬高各種術語與設定。 可用
  layout package 的 \verb=\layout{}= 印出來目前設定如下
  \clearpage
  \layout{}
  一頁是奇數,一頁是偶數,layout package 主要設定目前的版面而已,要完整重新定義
  版面紙張大小等設定, 可用
  \href{https://texdoc.org/serve/geometry.pdf/0}{geometry} package 來設定,裡面
  一樣定義了。 例如
  \begin{verbatim}
\usepackage[a4paper]{geometry}
  \end{verbatim}
  簡單的設定四個距離
  \begin{verbatim}
\usepackage[
  top=1in,
  bottom=1.25in,
  left=1.25in,
  right=1.25in]
{geometry}
  \end{verbatim}
  或者快速設定四邊 margin
  \begin{verbatim}
\usepackage[margin=1in]{geometry}
  \end{verbatim}
  表示設定四周紙張到文字的距離,用 margin=2cm 是比較快速設定四周距離。
  如果要重新設定自訂紙張大小
  \begin{verbatim}
\documentclass{article}
\usepackage[
  paperheight=10.75in,
  paperwidth=8.25in,
  margin=1in,
  heightrounded,showframe
]{geometry}

\begin{document}
....
\end{document}
  \end{verbatim}

  \subsection{空白對齊設定}
  內定空白
  \begin{itemize}
    \item 字元與字元,字與字間距是由字的大小動態調整的,default 通常很好不用設定
      了。
    \item 行與行 \verb=\baselineskip= \the\baselineskip,\verb=lineskip=
      \the\lineskip 。
    \item paragraph 與 paragraph \verb=\parskip= \the\parskip。
    \item list 與 list \verb=\itemsep= \the\itemsep。
    \item table row 與 row 是根據 table 大小自動調整的 \verb=\arraystretch=
      ratio 1, column 與 column 是 \verb=\tabcolsep= \the\tabcolsep。
  \end{itemize}
  設定用\verb=\setlength=,例如\verb=\setlength\tabcolsep{10pt}=,基本小空白設定
  \begin{itemize}
    \item \verb=~= 多出字與字間小空白
    \item \verb=\\[1cm]= 換行時使用 1cm 空白。
    \item \verb=\hspace{1cm}= 向右多出1cm空白,負數向左。
    \item \verb=\vspace{1cm}= 向下空出1cm空白,負數向上。
    \item \verb=\quad \qquad= 向右空出1em 與 2em 的空白。
    \item \verb=\,= 字元與字元間產生 3/18 \verb=\quad=
  \end{itemize}
  行間的空白, 要注意的是 big, med, small 這些 skip 在隔開文字時,只有在空白
  一行段落後面才是對的。 因為它們是垂直的,所以只能是垂直物件間的距離。
  \begin{itemize}
    \item \verb=\bigskip= 產生 12pt 的垂直空白。
    \item \verb=\medskip= 產生 6pt 的垂直空白。
    \item \verb=\smallskip= 產生 3pt 的垂直空白。
  \end{itemize}
  多空白的產生
  \begin{itemize}
    \item \verb=\stretch{1.5}= 產生 1.5 倍原有的空白,例如
      \verb=\hspace{\stretch{3}}=
    \item \verb=\arraystretch{1.5}= 這在 table 時見過,是設定 row 與 row 間空
      白。
    \item \verb=\linespread{1.5}= 產生原本 1.5 倍的行距空白。
    \item \verb=\hfill= 如果這段文字寬度不夠充滿整行,就從 hfill 開始填空白使
      得這段文字充滿一行,這跟前面說的用 makebox 的 spread 不一樣,只是單純分
      開文字。
    \item \verb=\vfill= 類似 hfill,但 vfill 兩邊的字填上空白直到佔滿整頁。
  \end{itemize} 
  \LaTeX{} 的空白其實不是那麼死的,而是有很多計算的,叫做 \emph{glue},表示一種
  介於不同 box 中的 flexible space, \TeX{} 其實是所有東西都是 box,除了之前說的
  parbox,連字元 都是 box 物件, glue 其實是所有物件黏在紙張上動態位置空白的計算
  , 例如最基本的行距,用 \verb=\the\parskip= 會發現他回傳 3 個值,
  \begin{itemize}
    \item fixed 固定一個開始值
    \item plus 根據大小動態長大比例時,自動增加 size
    \item minus 根據大小動態縮小比例時,自動縮減的 size
  \end{itemize}
  所以真正空白大小是 fixed amount plus amount to stretch minus amount to
  shrink, fixed amount 是所有 glue 物件都需要的,plus (stretch) 與 minus
  (shrink) 則是 optional,所以 \verb=\baselineskip= 只有一個 fixed 值。
  \\\\
  有興趣可以看 \href{https://en.wikibooks.org/wiki/LaTeX/Boxes}{box} 與複雜
  的計算
  \href{https://www.overleaf.com/learn/latex/Articles/How_to_change_paragraph_spacing_in_LaTeX#A_few_notes_on_glue}{overleaf 對於 glue 解說},
  也有其他 setspace 與 parskip package 做更多的 space 簡單空白控制。

    \subsubsection{空白與對齊}
    雖說根本基底對齊單位是 box,每個 box 就是個小物件,根據一些計算排列整齊,
    只是 box 更高對於人類的物件分類使用像 chapter, section, table, array 等等
    文件元素,通常文章中最常見的除了使用 tabular 可對齊,還有 paragraph 段落對
    齊設定, 這就是在 一般 word 等軟體看到的三種本文對齊,paragraph box pbox
    也才會自動折行, 如果文字不夠一行, 這會自動補上本文寬度的前後空白。例如
    \begin{verbatim}
\begin{flushright}
哈哈 MS Word: 我可以對齊右邊耶
\end{flushright}
    \end{verbatim}
    \begin{flushright}
    哈哈 MS Word: 我可以對齊右邊耶
    \end{flushright}
    使用以下環境或命令
    \begin{center}
      \begin{tabular}{ccc}
        對齊 & 環境命令  & 基本命令\\
        \hline
        左對齊 & flushleft & \verb=\raggedright= \\
        右對齊 & flushright & \verb=\raggedleft= \\
        中對齊 & center & \verb=\centering= \\
      \end{tabular}
    \end{center}
    段落空白
    \begin{itemize}
      \item \verb=\parindent= 段落縮排時的空白。
      \item \verb=\parskip= 每個段落間距離。
      \item \verb=\indent= 恢復縮排 indent
      \item \verb=\noindent= 不要縮排 indent
    \end{itemize}
    整頁空白,有時在大文件時或者有些 floating 物件讓頁面看起來排的不好
    \begin{itemize}
      \item \verb=\newpage= 結束這一頁,另起新頁。
      \item \verb=\pagebreak[num]= 在這命令之後  break 這一頁。num 是 0-4 的
        priorty 。
      \item \verb=\clearpage= 結束這一頁,而且任何在這頁前的 float 物件印出。
        之後的新 float 物件都重新計算,從新頁開始,
        這個對付一些跑來跑去的大物件很有效。前面兩個對大圖表都沒有重計算影響,
        必須用這個才行。
    \end{itemize}

  \subsection{counter 設定}
  排版中還有一些數字常常要加總,頁數,圖表數目,參照數目等等。
  \begin{itemize}
    \item \verb=\newcounter{mycounter}[section]= 命名一個 mycounter ,然後每次
      section counter 增加時,就會變成0
    \item \verb=\setcounter{mycounter}{7}= 設 mycounter 7
    \item \verb=\addtocounter{mycounter}{8}= mycounter 加8
    \item \verb=\stepcounter{mycounter}= mycounter 加 1
  \end{itemize}
  counter 取值
  \begin{center}
    \begin{tabular}{ccc}
      命令 & 效果 & 意義 \\
      \hline
      \verb=\thesection= & \thesection & section counter 值\\
      \verb=\thepage= & \thepage &  page counter 值\\
      \verb=\value{section}= & x & value 回傳不是字串,不能印 \\
      \verb=\arabic{chapter}= & \arabic{chapter} &
        印出 chapter counter 的阿拉伯符號\\
      \verb=\roman{page}= & \roman{page} & 印出 page counter 的羅馬符號\\
    \end{tabular}
  \end{center}

\section{寫新 macro}
在前面設定新名字長度時,就發現有兩種, 可以使用 \LaTeX{} 提供的工具,也能使用
\TeX{} 最基本原始的命令再新定命令 macro ,兩者在新 macro 上也常常混用,一般也
看不出來,寫法上有些注意事項
\begin{itemize}
  \item 有的語法中不能換行,中間必須不能空白
  \item 可以使用 \verb=%= 在行尾,表示接續下行
\end{itemize}

  \subsection{\LaTeX{} 方法}
  包括了
    \subsubsection{newcommand/renewcommand/providecommnd}
    \LaTeX{} 使用 newcommand 來定義新函數。參數用 \#1 \#2 \#3 ...這樣
    \begin{verbatim}
\newcommand {\name} [參數數目][default] { this is new command with #1 #2 ...}
    \end{verbatim}
    參數數目就是包含選擇性與必要性的參數數目總和, default 就是內定值,就是之
    前用命令時,選擇性的命令,可給可不給, 一個例子:
    \begin{flushleft}
      \begin{tabular}{m{0.1\textwidth}|m{0.7\textwidth}}
      命令 & 
      \begin{verbatim}
      \newcommand{\bighead}[2][大頭]{{#1}{#1}下雨不愁,{#2}有傘你有{#1}}
      \bighead{\color{blue}人家}\newline\bighead{大雄}{小叮噹}
      \end{verbatim}  \\
      \hline \\[0.2cm]
      效果 &
      \newcommand{\bighead}[2][大頭]{{#1}{#1}下雨不愁,{#2}有傘你有{#1}}
      \bighead{\color{blue}人家}\newline\bighead[大雄]{小叮噹} \\[0.2cm]
      \hline \\[0.2cm]
      &
      由上面可以知道,它只是很像 C 的 macro ,字串完全代換而已,然後看得出來
      使用\{\} 像是 shell, perl 可以解決很多事情。
      \end{tabular}
    \end{flushleft}
    想要更改原有命令,則用 renewcommand 重新定義,使用通式跟 newcommand 一樣
    \begin{verbatim}
\renewcommand\emph[1]{%
  {\my@emphstyle #1}
}
    \end{verbatim}
    也有 providecommand,差別是 providecommand 發現已經有同樣 command 定義了
    ,只是單純的不理會,不會出現 error 死掉。

    \subsubsection{newenvironment/renewenvironment}
    而新環境, num 跟 newcommand 一樣是參數數目,default 也一樣是內定值,
    \#1 \#2 ... 也是參數使用, 在 default 後面有兩個中括號,before 裡面是會先
    執行的東西, 然後就是夾在 begin end 中間的東西,最後執行 after 裡面的東西。
    \begin{verbatim}
\newenvironment{name}[num][default]{before}{after}
    \end{verbatim}
    例子是
    \begin{flushleft}
      \begin{tabular}{m{0.1\textwidth}|m{0.7\textwidth}}
      命令 & 
      \begin{verbatim}
\newenvironment{mybox}
{\rule{1ex}{1ex}\hspace{\stretch{1}}}
{\hspace{\stretch{1}}\rule{1ex}{1ex}}
\begin{mybox}我的新盒子\end{mybox}
      \end{verbatim}  \\
      \hline \\[0.2cm]
      效果 &
      \newenvironment{mybox}
      {\rule{1ex}{1ex}\hspace{\stretch{1}}}
      {\hspace{\stretch{1}}\rule{1ex}{1ex}}
      \begin{mybox}我的新盒子\end{mybox}
      \end{tabular}
    \end{flushleft}

    \subsubsection{makeatletter/makeatother}
    在看 \LaTeX{} 文件時,會發現文件內常直接把如何實做命令的程式碼寫出,而且
    很多命令名字內有 @ 這個字元,這種命令表示所有不應該公開的 internal macros
    命名的 \LaTeX{} 特有規定。
    \\\\
    當底層 \TeX{} 讀取字元時,它會指定每個字元一個 categroy code 稱為 catcode
    ,例如 backslash \verb=\= 的 catcode 是 0,任意的英文字母是 11,如果一段
    字串含有 catcode 0 的話,那麼後面必須是 catcode 11 的字元,這是程式邏輯的
    規定。
    \\\\
    \LaTeX{} 模仿這種字元規定,本來 @ 在 \TeX{} 是 11,但 \LaTeX{} 的 @ 內定變
    成是 catcode 12,因此只要命令內有 @,就是內部 API, 例如設定 documentclass
    的紙張大小程式碼
    \begin{verbatim}
\if@compatibility\else
 \DeclareOption{a4paper}
 {\setlength\paperheight {297mm}%
 \setlength\paperwidth {210mm}}
 .....
    \end{verbatim}
    \href{https://texdoc.org/serve/macro2e/0}{\LaTeX{}2e 提供的內部 macro 工具}
    有 \verb=\if@compatibility\else= 這個 @compatibility 就是 \LaTeX{} 內部的
    自訂的不公開 macro,其實應該是 \verb=\compatibility=,只是 \verb=\= 變成
    @。
    \\\\
    由於內定變成 12,所以當有自訂 @ 命令時,要先轉回 11,然後做完自訂 macro,
    需要再轉回 12 , makeatletter makeatother 中間應該要定義一個新命令,新命令
    裡面要有 @ 字元,且照規矩是內部使用而已
    \begin{verbatim}
\makeatletter
  ... definition of commands with @ in their name ..
\makeatother
    \end{verbatim}
    而且不要用在 .sty .cls 檔中,而是用在 .tex 檔的 preamble 區。

    \subsubsection{hooks}
    hooks 是撞到一個條件時,去呼叫之前特別定義的函數,\LaTeX{} 提供有
    \begin{itemize}
      \item \verb=\AtBeginDocument{macro}= 在碰到 \verb=\begin{document}= 時
        執行 macro。
      \item \verb=\AtEndDocument{macro}= 同理碰到 \verb=\end{document}= 時,執
        行 macro。
      \item \verb=\AtEndOfPackage{macro}= 在 usepackage 後。
      \item \verb=\AtEndOfClass{macro}= 在 documentclass 後。
    \end{itemize}

    \subsubsection{控制語法}
    \LaTeX{} 有提供一個 ifthen package 使用,可以用來執行大小比較等運算
    \begin{verbatim}
\usepackage{ifthen}

\ifthenelse{condition}{A}{B}
\ifthenelse{\value{num}>3\AND\(1<0 \OR \value{num}=10\)}{True.}{False.}
\end{verbatim}
    loop 迴圈,使用 loop, repeat, \verb=\loop= 是 \TeX{} plain API,只是
    \LaTeX{} 多定義了 \verb=\repeat=,用法是
    \begin{verbatim}
    \end{verbatim}
    \begin{verbatim}
\count255 = 0
\loop
  [\number\count255 =\char\number\count255]
  \ifnum\count255 < 127
    \advance\count255 by 1
\repeat
    \end{verbatim}
    \verb=\ifnum= 是 \TeX{} 提供的,所以其實兩者在使用上會互相使用就是。

  \subsection{plain \TeX{}}
  用 pascal 程式寫死在 \TeX{} 程式的命令稱為 primitive macro,基於原始 macro,
  Kunth 先生提供更上一層的 plain 命令 macro,要先了解一些 \TeX{} 的想法術語
  \begin{itemize}
    \item token\\
      任何的box, 字元,command sequence, group ...
    \item command sequence, command, primitive\\
      用反斜線開始的字串為 command sequence,能展開的 command sequence 或重新
      定義的為 command,primitive 是寫死在原 pascal 的 \TeX{} 命令,他不是後
      來創造的 plain \TeX{} API。
    \item register\\
      程式處理變數宣告與定義,有六種
      \begin{itemize}
        \item box
        \item count
        \item dimen
        \item muskip
        \item skip
        \item toks
      \end{itemize}
      可以直接使用 \verb=\box100= 表示一個 box register,最多可使用 0\~{}255
      宣告,而有些不成文的建議規定是
      \begin{itemize}
        \item 雙數像 \verb=\count22= 用在 local 變數,單數 \verb=\count25= 用在
          全域變數。
        \item 暫時使用的變數,應該用 255 像 \verb=\count255=, \verb=\dimen255=
          ...。
        \item 0 \~{} 9 在 plain \TeX{} 下是保留的。
      \end{itemize}
    \item box\\
      最基本通用物件,除了認知的 paragraph, 圖形,表格等等,連個字元都是 box,
      一個 box 有 height, width, baseline, depth, reference point 定義,如下
      \setlength{\unitlength}{1cm}
      \begin{figure}[h]
        \begin{center}
        \begin{picture}(5,5)
          \put(0,0){\framebox(4,1.8)[b]{width}}
          \put(0,1.8){\framebox(4,3)[b]{baseline}}
          % text
          \put(-3,1.8){reference point}
          \put(5,3){height}
          \put(5,0.9){depth}
          % vertical line
          \put(5.5,3.5){\vector(0,1){1.3}}
          \put(5.5,2.6){\vector(0,-1){0.8}}
          \put(5.5,1.2){\vector(0,1){0.6}}
          \put(5.5,0.6){\vector(0,-1){0.6}}
          % horonzantal 
          \put(-0.5,1.8){\vector(1,0){0.3}}
          \put(1.4,0.1){\vector(-1,0){1.4}}
          \put(2.6,0.1){\vector(1,0){1.4}}
          % dot
          \put(0,1.8){\circle*{0.1}}
        \end{picture}
        \end{center}
        \caption{box 的細部定義}
      \end{figure}
      這即使在 paragraph 圖形表格一樣有這些定義。
    \item glue\\
      將 box 黏在紙上需要根據前後左右的物件特性計算的空白尺寸。
  \end{itemize}
  \TeX{} 是語言也有特殊字元,就是那些有特殊意義,正常不會印出的字元,為了區別
  不同特殊字元, 這在以前,只是單純認為是個程式語言特殊脫逃字元,但在 \TeX{}
  會給每個字元一個 category code 定義稱為 catcode
  \begin{center}
  \begin{tabular}{rrr}
    catcode & 用途 & 字元集合 \\
    \hline
    0 & 脫逃字元與命令開始 & \verb=\= \\
    1 & grouping 的開始 & \verb={= \\
    2 & grouping 的結束 & \verb=}= \\
    3 & 數學 inline 模式 & \verb=$= \\
    4 & tabular 內的欄位對齊 & \verb=&= \\
    5 & 表示一行結束 EOL, carriage return  & \verb=^^M= (ASCII 13 return) \\
    6 & 用在 macro 的參數表示 & \verb=#= \\
    7 & 上標 & \verb=^ 與 ^^K= \\
    8 & 下標 & \verb=_= 與 \verb=^^A= \\
    9 & null 字元 & \verb=^^@= (ASCII 0 null) \\
    10 & 空白字元與 tab 字元 & ␣ (ASCII 32) 與 \verb=^^I= (ASCII 9 tab) \\
    11 & 任一 A..Z 字元 & A...Z and a...z \\
    12 & 其他字元, 沒有在其他 catcode 分類字元 & 在 \LaTeX{} 中有設定 @。 \\
    13 & active 字元 & \verb=~ 與^^L= (ASCII 12 form feed) \\
    14 & 註解字元 & \verb=%= \\
    15 & 不合法字元 & \verb=^^?= (ASCII 127 delete)\\
  \end{tabular}
  \end{center}
  其中 active 字元是一個單一字元在其他命令還沒執行前,就被展開的 macro,這有點
  像 shell script 的 expr,或寫 C 程式 macro 的先行展開意思。
  \\\\
  定義 catcode 與用 active 字元代換例子
  \begin{verbatim}
\catcode`| = 13
\def|{\TeX}
  \end{verbatim}
  先定義 | 變成 catcode 13,然後定義 | 會代換成 \verb={\TeX}= 字串,則在文章中
  \begin{verbatim}
This is | example
  \end{verbatim}
  就會變成
  \begin{verbatim}
This is {\TeX} example
  \end{verbatim}
  所以上述的 \LaTeX{} @ 命令也是一樣做法,只是變成 catcode 11 而已,所以其實
  \LaTeX{} 的 makeat... macro 實際定義是
  \begin{verbatim}
\def\makeatletter{\catcode`@ = 11}
\def\makeatother{\catcode`@ = 12}
  \end{verbatim}
  這在 GNU Texinfo,或有些 pdf 轉換工具其實是用這類似方法請 \LaTeX{} 代勞的轉
  換。

    \subsubsection{資料型別}
    設 register 值,可直接用 = 設定數值
    \begin{verbatim}
\count0=100
    \end{verbatim}
    因為系統有些已經拿來使用的 register 0 \~{} 9,還有為了避免跟已被使用的衝突,
    但最好是用 \verb=\newbox=, \verb=\newcount=, \verb=\newdimen=,
    \verb=\newmuskip=, \verb=\newskip=, \verb=\newtoks= 動態使用 register。
    \\\\
    對於 register 取值可以
    \begin{itemize}
      \item \verb=\the= 後面接數值 register
      \item \verb=\number= 後面接 counter register
      \item \verb=\box= 後面接 box register
    \end{itemize}
    印出 register 值。而用 \verb=\char`<char>= 跟用 \verb=\catcode`= 一樣,
    後面跟著字元,就可以定義字元。使用
    \begin{itemize}
      \item \verb=\advance <register> by <number>= 做加減
      \item \verb=\multiply <register> by <number>= 做乘法
      \item \verb=\divide <register> by <number>= 做除法
    \end{itemize}

    \subsubsection{def 與 let}
    副程式定義,這像一般的 macro 定義
    \begin{verbatim}
    \def <command> <parameter-text>{<replacement-text>}

    \def \NAS {National Academy of Science}
    \def \foo [#1]#2{The first argument is ``#1'', the second one is ``#2''}
    \end{verbatim}
    定義參數 \#1, \#2 ... 最多有 9 個 \#9,然後第一個參數要用中括號,參數跟
    大括號中間不能有空白。 跟 \LaTeX{} 的 \verb=\newcommand= 比起來,
    newcommand 會幫忙檢查。
    \\\\
    \verb=\let= 能 copy 舊命令變成新命令,例如 endgraf 等於 par
    \begin{verbatim}
\let<new-command>[[<spaces>]=]<original-command>

\let\endgraf=\par
    \end{verbatim}
    與 renewcommand 相比是他只有改變名字,創造別名而已,沒有參數改變,當使用
    \begin{verbatim}
\renewcommand{\foo}{\bar\foo}
    \end{verbatim}
    是會造成無窮 loop 的,因此必須用
    \begin{verbatim}
\let\newfoo\foo
\renewcommand{\foo}{\bar\newfoo}
    \end{verbatim}
    edef 能展開別的 macro 但不執行,成為新定義
    \begin{verbatim}
\edef<macroname><argumentslist>{<expanded content>}
    \end{verbatim}
    long def 才能使用跨 paragraph 的命令,像 \verb=\par=,或者 line break
    \verb=\\=,這些在之前都不能用在內文,例如
    \begin{verbatim}
\long\def\dummy#1{#1}
\dummy{First paragraph\par Second paragraph}
    \end{verbatim}
    使用 grouping 能創造一個像 C 的 local scope,如果要全域設定則要使用
    \begin{itemize}
      \item \verb=\global\def= 或 \verb=\gdef=
      \item \verb=\global\edef= 或 \verb=xdef=
    \end{itemize}

    \subsubsection{macro 展開}
    當使用 macro 時,如果後面有 argument,那會假設 argument 是在 [ ] 括號內
    ,這樣 argument 只能是變數,不能是 macro
    \begin{verbatim}
    \def\a[#1]{A's argument was `#1'}
    \def\arg{[FOO]}
    \a\arg
    \end{verbatim}
    這樣是行不通的,因為 a 後面必須是中括號的變數,不能是 macro,所以
    expandafter 是個將兩個 macro 先展開後面的再展開全部 macro 的方法
    \begin{verbatim}
    \expandafter{token1}{token2}
    \end{verbatim}
    等於
    \begin{verbatim}
{token1}{token2 expansion}
    \end{verbatim}
    跟 edef 差別是 edef 會完全 expand,noexpand 會抑制 expand。

    \subsubsection{控制語法}
    一般程式都會有判斷,迴圈語法,\TeX{} 自然也提供
    \begin{itemize}
      \item \verb=\if=
      \item \verb=\ifcase=
      \item \verb=\ifcat=
      \item \verb=\ifdim=
      \item \verb=\ifeof=
      \item \verb=\iffalse=
      \item \verb=\ifhbox=
      \item \verb=\ifhmode=
      \item \verb=\ifinner=
      \item \verb=\ifmmode=
      \item \verb=\ifnum=
      \item \verb=\ifodd=
      \item \verb=\iftrue=
      \item \verb=\ifvbox=
      \item \verb=\ifvmode=
      \item \verb=\ifvoid=
      \item \verb=\ifx=
      \item \verb=\else=
      \item \verb=\fi=
    \end{itemize}
    例子
    \begin{verbatim}
\ifnum\value{num}>n {A} \else {B}\fi
\ifodd\value{num} {A}\else {B}\fi
    \end{verbatim}

  \subsection{\LaTeX{} package 提供者}
  要寫作 \LaTeX{} package,並需提供 dtx 檔案,將 \TeX{} source code 與 文件
  放在一起的檔案,文件跟其他文件系統一樣,在 comment 裡面,會伴隨著一個 ins 檔
  ,install 檔,用 tex/latex/pdflatex 命令工具生成 sty 檔與文件檔 pdf。
  \\\\
  在 texlive 的系統下,尋找 classlist.dtx,裡面寫的格式與命令可以做為練習用,
  例如定義一個 custome package 有三個選項,sans, roman 跟 neverindent
  \begin{verbatim}
\NeedsTeXFormat{LaTeX2e}[1994/06/01]
\ProvidesPackage{custom}[2013/01/13 Custom Package]
\RequirePackage{lmodern}

%% 'sans serif' option
\DeclareOption{sans}{
  \renewcommand{\familydefault}{\sfdefault}
}
%% 'roman' option
\DeclareOption{roman}{
  \renewcommand{\familydefault}{\rmdefault}
}

%% Global indentation option
\newif\if@neverindent\@neverindentfalse
\DeclareOption{neverindent}{
  \@neverindenttrue
}

...
\ExecuteOptions{roman}
\ProcessOptions\relax
....


\endinput
  \end{verbatim}
  放到
  \begin{verbatim}
$TEXMFHOME/tex/latex/custom/custom.sty
  \end{verbatim}
  並執行
  \begin{verbatim}
$ texhash
  \end{verbatim}
  其中
  \begin{itemize}
    \item \verb=\NeedsTeXFormat{LaTeX2e}[1994/06/01]= 定義 LaTeX 的版本需求
    \item \verb=\ProvidePackage= 定義 package 名稱
    \item \verb=\RequirePackage= 是 dependancy
    \item \verb=\DeclareOption= 定義 package 的選項。
    \item \verb=\ExecuteOption= 是內定選項
    \item \verb=\ProcessOptions\relax= 表示停止選項處理
    \item \verb=\endinput= 結束 package
  \end{itemize}
  提供 documentclass 也是類似
  \begin{verbatim}
\NeedsTeXFormat{LaTeX2e}
\ProvidesClass{myclass}[2011/12/23 My Class]

%% Article options
\DeclareOption{10pt}{
  \PassOptionsToClass{\CurrentOption}{article}
}

%% Custom package options
\DeclareOption{sans}{
  \PassOptionsToPackage{\CurrentOption}{custom}
}
\DeclareOption{neverindent}{
  \PassOptionsToPackage{\CurrentOption}{custom}
}

%% Fallback
\DeclareOption*{
  \ClassWarning{myclass}{Unknown option '\CurrentOption'}
}

%% Execute default options
\ExecuteOptions{10pt}

%% Process given options
\ProcessOptions\relax

%% Load base
\LoadClass[a4paper]{article}

%% Load additional packages and commands.
\RequirePackage{custom}

...
\endinput
  \end{verbatim}
  \begin{itemize}
    \item \verb=\ProvidesClass= 取代 \verb=\ProvidePackage=。
    \item \verb=\DeclareOption*{}= 是如果沒有定義的 option 就到這邊執行。
    \item \verb=\ClassWarning= 是定義 class warning 時,會出現的 warning 字串。
    \item \verb=\LoadClass= 是載入需要的 parent class。
  \end{itemize}
