\chapter{數學式子}
基本上有了以上的觀念,不管是排版,還是 {\TeX} 的運作機制後,剩下的都只是
參考小抄而已。數學也是一樣,只不過數學式子在 {\TeX} 中有一個重要觀念就是它有
inline mode 與 equation mode 就是夾在文字中 與單獨一塊大 object 形式而已。
\begin{center}
\begin{tabular}{m{3cm}m{4cm}m{3cm}}
mode 例子 & 呈現 & 說明\\
\hline
\verb|$ x^2+2xy+y^2=0 $|&$x^2+2xy+y^2=0 $& 用 \$ 夾住,inline mode\\
\verb|\[x^2+2xy+y^2=0\]|&\[x^2+2xy+y^2=0\]& 用 \verb=\[ \]= 夾住, equation mode\\
\verb|\begin{equation}|
\verb|x^2+2xy+y^2=0|
\verb|\end{equation}|&\begin{equation}x^2+2xy+y^2=0\end{equation}&
	 用環境, equation mode with label \\
\end{tabular}
\end{center}

\section{注意事項}
\begin{itemize}
\item 由於應用情況不同,所以佔據空間大小排版對齊會不同。
\item 用環境會自動編號 label 。counter 是 equation。
\item 在 tabular 裡面使用 equation mode 時, 由於它是一塊物件,
      只能用 paragraph mode, \verb|p{3cm}| 這樣,是用 parbox 做到容納物件的。
\end{itemize}
關於編號 label ,基本的還有 \AmS-\LaTeX 的 amsmath 提供了 
\verb| \eqref{mark} \tag{5} \multitag{6}| 連原本 euqation 環境的編號,都可以
強迫改掉後面的編號成自己隨心所欲的樣子。 例如
\begin{verbatim}
\begin{equation}
E=mc^2 \label{eq:einstein} \tag{\ding{37}}
\end{quation}
從 \eqref{eq:einstein}\ldots 看來,這是隻光速電話
\end{verbatim}
\begin{equation}
E=mc^2 \label{eq:einstein} \tag{\ding{37}}
\end{equation}
從 \eqref{eq:einstein}\ldots 看來,這是隻光速電話

\section{簡單數學}
剩下的就是小抄 cheetsheet 與一些排版微調。網路上有很多人把常用的
  \begin{itemize}
    \item 上標
      \begin{verbatim}
{a}^{2}
      \end{verbatim}
      ${a}^{2}$
    \item 下標
      \begin{verbatim}
{a}_{2}
      \end{verbatim}
      ${a}_{2}$
    \item 分號
      \begin{verbatim}
\frac{2}{5}
      \end{verbatim}
      $\frac{2}{5}$
    \item 根號
      \begin{verbatim}
\sqrt[n]{5}
      \end{verbatim}
      $\sqrt[n]{5}$
    \item 向量
      \begin{verbatim}
\vec{c}
      \end{verbatim}
      $\vec{c}$
    \item 加總
      \begin{verbatim}
\sum_{k=1}^{N} k^2
      \end{verbatim}
      $\sum_{k=1}^{N} k^2$
    \item 矩陣
      \begin{verbatim}
\begin{Bmatrix}
x & y \\
z & v
\end{Bmatrix}
      \end{verbatim}
      $\begin{Bmatrix}
      x & y \\
      z & v
      \end{Bmatrix}$
    \item 極限
      \begin{verbatim}
\lim_{n \to \infty}x_n
      \end{verbatim}
      $\lim_{n \to \infty}x_n$
    \item 積分
      \begin{verbatim}
\int_{-N}^{N} e^x\, dx
      \end{verbatim}
      $\int_{-N}^{N} e^x\, dx$
    \item 希臘符號
      \begin{verbatim}
\theta
      \end{verbatim}
      $\theta$
  \end{itemize}
都做了整理,主要就是根據命令格式, 通用 \{\} 包住一個想要的表示範圍,
其他就自行 google 就有了。
\\\\
\href{https://www.ams.org/arc/tex/amsmath/amsldoc.pdf}{美國數學協會\AmS-\LaTeX}
的數學式子 macro 提供一些比標準 macro 還好的功能。有些學術單位或者出版商是有他
們的個別要求或者更進一步的加強,所以要看投稿單位與要求做設定。\LaTeX 的好處之一就是
套上出版商或投稿單位的設定後,同樣文章內容不變,版型就改變,一稿多投很方便。
