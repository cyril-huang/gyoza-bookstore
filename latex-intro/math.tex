\chapter{數學式子}
基本上有了以上的觀念,不管是排版,還是 {\TeX} 的運作機制後,剩下的都只是
參考小抄而已。數學也是一樣,只不過數學式子在 {\TeX} 中有一個重要觀念就是它有
inline mode 與 equation mode 就是夾在文字中 與單獨一塊大 object 形式而已。
\begin{center}
\begin{tabular}{m{3cm}m{4cm}m{3cm}}
mode 例子 & 呈現 & 說明\\
\hline
\verb|$ x^2+2xy+y^2=0 $|&$x^2+2xy+y^2=0 $& 用 \$ 夾住,inline mode\\
\verb|\[x^2+2xy+y^2=0\]|&\[x^2+2xy+y^2=0\]& 用 \verb=\[ \]= 夾住, equation mode\\
\verb|\begin{equation}|
\verb|x^2+2xy+y^2=0|
\verb|\end{equation}|&\begin{equation}x^2+2xy+y^2=0\end{equation}&
	 用環境, equation mode with label \\
\end{tabular}
\end{center}

\section{注意事項}
\begin{itemize}
  \item 由於應用情況不同,所以佔據空間大小排版對齊會不同,用中括號等於是
    equation mode 沒有 label,也等於用 \verb=\begin{equation*}=。
  \item 用環境會自動編號 label ,counter 是 equation。
  \item 在 tabular 裡面使用 equation mode 時, 由於它是一塊物件, 只能用
    paragraph mode, \verb|p{3cm}| 這樣,是用 parbox 做到容納物件的。
  \item 對齊就用 array 環境,或者其他 package 提供相對容易使用的環境。
\end{itemize}
關於編號 label ,基本的還有 \AmS-\LaTeX 的 amsmath 提供了 
\verb| \eqref{mark} \tag{5} \multitag{6}| 連原本 euqation 環境的編號,都可以
強迫改掉後面的編號成自己隨心所欲的樣子。 例如
\begin{verbatim}
\begin{equation}
E=mc^2 \label{eq:einstein} \tag{\ding{37}}
\end{quation}
從 \eqref{eq:einstein}\ldots 看來,這是隻光速電話
\end{verbatim}
\begin{equation}
E=mc^2 \label{eq:einstein} \tag{\ding{37}}
\end{equation}
從 \eqref{eq:einstein}\ldots 看來,這是隻光速電話

\section{簡單數學}
剩下的就是小抄 cheetsheet 與一些排版微調。網路上有很多人把常用的都做了整理,
例如
  \begin{itemize}
    \item 上標
      \begin{verbatim}
{a}^{2}
      \end{verbatim}
      ${a}^{2}$
    \item 下標
      \begin{verbatim}
{a}_{2}
      \end{verbatim}
      ${a}_{2}$
    \item 分號
      \begin{verbatim}
\frac{2}{5}
      \end{verbatim}
      $\frac{2}{5}$
    \item 根號
      \begin{verbatim}
\sqrt[n]{5}
      \end{verbatim}
      $\sqrt[n]{5}$
    \item 向量
      \begin{verbatim}
\vec{c}
      \end{verbatim}
      $\vec{c}$
    \item 加總
      \begin{verbatim}
\sum_{k=1}^{N} k^2
      \end{verbatim}
      $\sum_{k=1}^{N} k^2$
    \item 極限
      \begin{verbatim}
\lim_{n \to \infty}x_n
      \end{verbatim}
      $\lim_{n \to \infty}x_n$
    \item 積分
      \begin{verbatim}
\int_{-N}^{N} e^x\, dx
      \end{verbatim}
      $\int_{-N}^{N} e^x\, dx$
    \item 希臘符號
      \begin{verbatim}
\theta
      \end{verbatim}
      $\theta$
  \end{itemize}
主要就是根據命令格式, 通用 \{\} 包住一個想要的表示範圍,另外不同mode 會產生
長相不太同的符號,例如 $\sum_{i=0}^{n}$,在 inline 與 equation mode
\begin{equation*}
  \sum_{i=0}^{n}
\end{equation*}
會不同,想要在 inline mode 也是大隻長相, $\displaystyle\sum_{i=0}^{n}$,用
displaystyle
\begin{verbatim}
$\displaystyle\sum_{i=0}^{n}$
\end{verbatim}
如果不想每次都打 displaystyle,在 preamble 處用 \verb=\everymath{\displaystyle}=
,就自動全部用大隻長相, 其他就自行 google 就有了。
\\\\
除了基本的小抄外,排版中重要的好看關鍵就是對齊,數學式子中,有很多對齊,如果
用傳統的 tabular, array, box ... 也能做到。
\begin{verbatim}
\[
  \begin{array}{rl}
    \displaystyle\sum_{n=3}^{5}\frac{1}{n(n+1)}
          &= \displaystyle\sum_{n=3}^{5}\frac{1}{n} - \frac{1}{n+1}\\
          &= \frac{1}{3} - \frac{1}{4} + \frac{1}{4} - \frac{1}{5} + \dots \\
          &= \frac{1}{3} - \frac{1}{6}
  \end{align}
\]
\end{verbatim}
\begin{equation*}
  \begin{array}{rl}
    \displaystyle\sum_{n=3}^{5}\frac{1}{n(n+1)}
          &= \displaystyle\sum_{n=3}^{5}\frac{1}{n} - \frac{1}{n+1}\\
          &= \frac{1}{3} - \frac{1}{4} + \frac{1}{4} - \frac{1}{5} + \dots \\
          &= \frac{1}{3} - \frac{1}{6}
  \end{array}
\end{equation*}
與
\begin{verbatim}
  \begin{array}{*5c}
      a_{11} & a_{12} & a_{13} & \cdots & a_{1n} \\
      a_{21} & a_{22} & a_{23} && a_{2n} \\
      a_{31} & a_{32} & a_{33} && a_{3n} 
  \end{array}
\end{verbatim}
\[
  \begin{array}{*5c}
      a_{11} & a_{12} & a_{13} & \cdots & a_{1n} \\
      a_{21} & a_{22} & a_{23} && a_{2n} \\
      a_{31} & a_{32} & a_{33} && a_{3n} 
  \end{array}
\]
特別的 matrix 矩陣符號可以用 \verb=\left= 與 \verb=\right= 產生。
\begin{verbatim}
\begin{equation*}
  A_{m \times n} =
  \left[
    \begin{array}{*5c}
      a_{11} & a_{12} & a_{13} & \cdots & a_{1n} \\
      a_{21} & a_{22} & a_{23} && a_{2n} \\
      a_{31} & a_{32} & a_{33} && a_{3n} \\
      \vdots & 
      &
      & \ddots & \vdots \\
      a_{m1} & a_{m2} & a_{m3} & \cdots & a_{mn}
    \end{array}
  \right]
\end{equation*}
\end{verbatim}
\begin{equation*}
  A_{m \times n} =
  \left [
    \begin{array}{*5c}
      a_{11} & a_{12} & a_{13} & \cdots & a_{1n} \\
      a_{21} & a_{22} & a_{23} && a_{2n} \\
      a_{31} & a_{32} & a_{33} && a_{3n} \\
      \vdots & 
      &
      & \ddots & \vdots \\
      a_{m1} & a_{m2} & a_{m3} & \cdots & a_{mn}
    \end{array}
  \right]
\end{equation*}
binomial coefficient 二項式係數,用在像排列組合統計上
\begin{verbatim}
\[
  \frac{n!}{k!(n-k)!} = 
  \left(
    \begin{array}{c}
      n \\ k
    \end{array}
  \right)
\]
\end{verbatim}
\[
  \frac{n!}{k!(n-k)!} = 
  \left(
    \begin{array}{c}
      n \\ k
    \end{array}
  \right)
\]
cases 條件,由於 \verb=\left= \verb=\right= 是一定要成對,所以故意在 right
用小數點
\begin{verbatim}
\[
  f(x)=
  \left\{
  \begin{array}{cl}
    \frac{x^2-x}{x},& \text{if } x\geq 1\\
    0,              & \text{otherwise}
  \end{array}
  \right.
\]
\end{verbatim}
\[
  f(x)=
  \left\{
  \begin{array}{cl}
    \frac{x^2-x}{x},& \text{if } x\geq 1\\
    0,              & \text{otherwise}
  \end{array}
  \right.
\]
\verb=\left= 與 \verb=\right= 也是用在某個大 group 的數學式,例如
\begin{verbatim}
\[
  \left[p + a\left(\frac{n}{V}\right)^2\right](V-nb) = nRT
\]
\end{verbatim}
\[
  \left[p + a\left(\frac{n}{V}\right)^2\right](V-nb) = nRT
\]

\section{\AmS-\LaTeX}
\href{https://www.ams.org/arc/tex/amsmath/amsldoc.pdf}{美國數學協會\AmS-\LaTeX}
的數學式子 macro 提供一些比標準 macro 還好的功能,有些符號畫出的長相也比較好看
。 \AmS-\LaTeX 數學中有 matrix, align, split 環境等功能可以用在數學對齊,也很像
tabular, array 一樣用 \verb=&= 來做對齊。 例如用 bmatrix 環境做矩陣,不需要用
\verb=\left= 與 \verb=\righ=。例如
\begin{verbatim}
\usepackage{amsmath}

\begin{equation*}
A_{m \times n} =
\begin{bmatrix}
a_{11} & a_{12} & a_{13} & \cdots & a_{1n} \\
a_{21} & a_{22} & a_{23} && a_{2n} \\
a_{31} & a_{32} & a_{33} && a_{3n} \\
\vdots &
&
& \ddots & \vdots \\
a_{m1} & a_{m2} & a_{m3} & \cdots & a_{mn}
\end{bmatrix}
\end{equation*}
\end{verbatim}
\begin{equation*}
A_{m \times n} =
\begin{bmatrix}
a_{11} & a_{12} & a_{13} & \cdots & a_{1n} \\
a_{21} & a_{22} & a_{23} && a_{2n} \\
a_{31} & a_{32} & a_{33} && a_{3n} \\
\vdots &
&
& \ddots & \vdots \\
a_{m1} & a_{m2} & a_{m3} & \cdots & a_{mn}
\end{bmatrix}
\end{equation*}
用 binom 做二項係數
\begin{verbatim}
\frac{n!}{k!(n-k)!} = \binom{n}{k}
\end{verbatim}
\[
  \frac{n!}{k!(n-k)!} = \binom{n}{k}
\]
case 型態,因為在基本數學用 \verb=\left= 一定要跟著個 \verb=\right=
,有點彆扭,所以可以用 cases
\begin{verbatim}
\[
  f(x)=
  \begin{cases}
    \frac{x^2-x}{x},& \text{if } x\geq 1\\
    0,              & \text{otherwise}
  \end{cases}
\]
\end{verbatim}
\[
  f(x)=
  \begin{cases}
    \frac{x^2-x}{x},& \text{if } x\geq 1\\
    0,              & \text{otherwise}
  \end{cases}
\]

有些學術單位或者出版商是有他們的個別要求或者更進一步的加強,所以要看投稿單位
與要求做設定。 \LaTeX 的好處之一就是套上出版商或投稿單位的設定後,同樣文章內
容不變,版型就改變,一稿多投很方便。
